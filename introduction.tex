\chapter{Introduction}

\section*{Preface}

In this Chapter, I introduce the basic concepts and methodologies necessary to understand this Thesis.

\section{The purpose of life}

``Long, long ago there was a time when nothing but mere matter existed in this world. In the teeming ooze, forms of a certain \textit{something} appeared, disappeared, and appeared again and one of them eventually survived. We know it as \textit{life}. The reason that \textit{life} ultimately survived was because it was in its nature \textit{to multiply}. \textit{Life} took new forms in order to multiply, adapting to every kind of environment, and culminating in \textit{us} today. Greater numbers, greater diversity, greater abundance. This is why we say that the purpose of \textit{life} is \textit{to multiply}.'' \cite{ISAYAMA_2021}

\section{Central dogma of molecular biology}

It is estimated based on geological \cite{SCHIDLOWSKI_1979_LIFE}, fossil \cite{SCHOPF_2007_LIFE} and phylogenetic \cite{BETTS_2018_LIFE} analyses that the origin of life in our planet Earth dates back to 3.7-4.0 billion years ago. Since then, \textit{life} has not just survived, but adapted and evolved to give raise to a gargantuan estimated biodiversity of 8.7 million eukaryotic species \cite{MORA_2011_SPECIES} and upward of 1 trillion microbial species \cite{HUG_2016_SPECIES, LOCEY_2016_SPECIES} with the vast majority of these ($>$90\%) still to be described \cite{COSTELLO_2013_SPECIES}. Despite the immense variation across species in terms of reproductive strategies, morphological, metabolic, behavioural traits or ecological niche, there is one thing \textit{all} species have in common: nucleic acids \cite{KOONIN_2011_LIFE}. All living species rely on nucleic acids, mostly deoxyribonucleic acid (DNA) except for some viruses \cite{KOONIN_2006_VIRUS} and viroids \cite{NAVARRO_2021_VIROIDS} that use ribonucleic acid (RNA), to store their genetic information. This information flows sequentially from DNA to RNA through the process of transcription and from RNA to protein through translation. This flow of molecular information is known as the \textit{Central Dogma of Molecular Biology} \cite{CRICK_1958_DOGMA, CRICK_1970_DOGMA}.

\section{The genetic code}

DNA is a polymer composed of two polynucleotide chains that coil around each other to form a double helix \cite{WATSON_1953_DNA}. These polymer chains are formed by simpler units called nucleotides. Each nucleotide presents a common scaffold formed of a deoxyribose sugar and a phosphate and a variable nitrogen-containing nucleobase. There are four different bases: adenine (A), thymine (T), cytosine (C) and guanine (G). The information stored in DNA gets transferred to RNA through the process of transcription. RNA tends to adopt a single-stranded conformation and is also formed by nucleotides. These nucleotides differ from DNA ones in that they present a ribose sugar, instead of deoxyribose, and an alternative uracil (U) nucleobase instead of thymine \cite{LEVENE_1909_NUCLEICS}. Messenger RNA (mRNA) corresponds to the sequence of a gene and is read by the ribosomal macromolecular machinery in the process of protein synthesis, or translation. In this process, a peptide chain is formed by linking amino acids in the order specified by the codons in the mRNA \cite{CRICK_1957_CODE}. A codon is a set of three nucleotides that corresponds to one of the twenty canonical amino acids. The equivalence between these codons and the amino acid they encode is known as the \textit{Genetic code} \cite{GAMOW_1954_CODE}.

\section{Genetic variation}

XXX.

\subsection{Genomic location}

XXX.

\subsubsection{Non-coding variation}

XXX.

\subsubsection{Coding variation}

XXX.

\paragraph{Synonymous variation}

XXX.

\paragraph{Missense variants}

XXX.

\paragraph{Nonsense variation}

XXX.

\paragraph{Frameshift variation}

XXX.

\subsection{Impact on phenotype}

XXX.

\subsubsection{Neutral variation}

XXX.

\subsubsection{Pathogenic variation}

XXX.

\subsubsection{Unknown significance variation}

XXX.

\subsection{Variation and function}

XXX.

\section{Proteins}

XXX.

\subsection{Protein sequence}

XXX.

\subsubsection{Multiple sequence alignment}

XXX.

\subsubsection{Amino acid conservation}

XXX.

\subsubsection{Conservation and function}

XXX.

\subsection{Protein structure}

XXX.

\subsubsection{Primary structure}

XXX.

\subsubsection{Secondary structure}

XXX.

\subsubsection{Tertiary structure}

XXX.

\subsubsection{Quaternary structure}

XXX.

\subsection{Protein structure determination}

XXX.

\subsubsection{Nuclear magnetic resonance}

XXX.

\subsubsection{Cryo-electron microscopy}

XXX.

\subsubsection{X-ray crystallography}

XXX.

\subsection{Protein structure prediction}

XXX.

\subsubsection{Homology-based}

XXX.

\subsubsection{De novo}

XXX.

\subsection{Protein structure characterisation}

XXX.

\subsubsection{Secondary structure}

XXX.

\subsubsection{Solvent accessibility}

XXX.

\subsubsection{Hydrophobicity}

XXX.

\subsubsection{Charge}

XXX.

\subsubsection{Binding pockets}

XXX.

\section{The conservation plane}

XXX.

\section{Drug discovery}

XXX.

\subsection{Drug discovery pipeline}

XXX.

\subsubsection{Target identification}

XXX.

\subsubsection{Target validation}

XXX.

\subsubsection{Lead identification}

XXX.

\subsubsection{Lead optimisation}

XXX.

\subsubsection{Pre-clinical development}

XXX.

\subsubsection{Clinical studies}

XXX.

\subsubsection{Drug approval}

XXX.

\subsection{Fragment screening}

XXX.

\section{Machine learning}

XXX.

\section{Databases}

XXX.

\subsection{UniProt}

XXX.

\subsection{PDB}

XXX.

\subsection{gnomAD}

XXX.

\section{Thesis scope}

XXX.


% WHAT IS LEFT?