\chapter*{Agradecimientos} % means there is no chapter number for this

Este doctorado ha sido todo un viaje. Cuatro años y medio que han pasado volando, pero que al mismo tiempo, han parecido tan largos y en los que han ocurrido tantas cosas tanto en mi vida personal como profesional. Tengo muchísimas personas a las que agradecer su apoyo, sin las cuales no estaría donde estoy hoy. Empezaré por el principio.

Mi interés por las ciencias de la vida comenzó en mi infancia, alrededor de los seis años, cuando vivía en un pequeño pueblo llamado Almedinilla, en la provincia de Córdoba, Andalucía, España. Un biólogo, cuyo nombre no recuerdo, solía llevarme a mí, a compañeros de escuela y a nuestros padres a excursiones para explorar la naturaleza local. Nos hablaba sobre las plantas, líquenes, aves, mamíferos y minerales que podíamos encontrar en estas salidas. Me fascinaba la biodiversidad de la zona y las diferencias entre los distintos organismos, su comportamiento y adaptación. Después de mudarme a Terrassa, una ciudad cerca de Barcelona, Pepita Penalba, mi maestra de biología en la escuela \textit{Cultura Pràctica}, solía formar una fila con algunos alumnos y hacernos preguntas sobre lo que habíamos aprendido. Dependiendo de la respuesta, avanzábamos o nos manteníamos en la fila. Las respuestas correctas te acercaban a la parte delantera, indicando un buen dominio del tema. Creo que esto alimentó mi espíritu competitivo y me animó a aprender y estudiar más para estar entre los mejores estudiantes de la clase. Más tarde, en secundaria, mis profesores el Dr. Joel Pascual y Carme Hernández me enseñaron más sobre física, química, biología y geología, aumentando así mi interés por estas áreas. Me parecían especialmente interesantes las leyes de la genética mendeliana, lo que me llevaría posteriormente a estudiar un grado en Genética en la Universidad Autónoma de Barcelona (UAB). Durante el bachillerato en el Instituto Montserrat Roig, mi compañero de clase el Dr. Marc Botifoll obtuvo una beca para el taller \textit{Crazy about Biomedicine} organizado por el IRBB. Gracias a esto, junto con otra compañera, Cristina Ortiga, llevamos a cabo un proyecto para evaluar posibles fármacos contra el VIH utilizando métodos computacionales, bajo la supervisión de la Dra. Michela Candotti. Este fue mi primer contacto con la bioinformática. Durante los tres años que estudié en la UAB, tuve la suerte de asistir a las clases de grandes docentes. Entre quienes influyeron en mi decisión de seguir estudiando después del grado, quiero destacar a los Profes. Antonio Barbadilla, Vicente Martínez, Hafid Laayouni, Alfredo Ruíz, Isaac Salazar, Jesús Piedrafita y la Dra. Raquel Egea. Me enseñaron sobre las bases de la genética, fisiología animal, bioestadística, genética de poblaciones, del desarrollo y cuantitativa, además de programación y bioinformática. Fue en el tercer año de mi carrera cuando me di cuenta de que programar era ``lo mío'' y que los experimentos no eran para mí. Ese verano, me uní al grupo del Prof. Francesc Calafell en el PRBB para un proyecto de tres meses sobre genética de poblaciones y forense, trabajando con R. Justo después, realicé un intercambio de estudiantes Erasmus a Dundee durante el cuarto y último año de la carrera. Allí, cursé Estructura Molecular e Interacciones con el Prof. Bill Hunter y Bioinformática Aplicada con el Dr. David Martin, lo que reafirmó mi pasión por la bioinformática y despertó mi interés por su lado estructural. Al año siguiente, comencé un máster en Bioinformática para las Ciencias de la Salud en la Universidad Pompeu Fabra (UPF), en Barcelona. Las clases del Dr. Javier García sobre programación en Python y del Prof. Baldo Oliva sobre Bioinformática Estructural me cautivaron, por lo que solicité una estancia de un año en el \textit{Barton Group} como parte del máster, que realicé de septiembre de 2019 a julio de 2020. Durante ese tiempo, obtuve una beca EASTBIO DTP para llevar a cabo mi doctorado bajo la supervisión del Prof. Geoff Barton. Me gustaría agradecer a todas las personas que contribuyeron a mi formación antes de comenzar este programa de doctorado, que he llevado a cabo de octubre de 2020 a marzo de 2025.

Quiero agradecer especialmente al Prof. Geoff Barton, quien, me gusta pensar, vio algo en mí en 2019, cuando vine a la entrevista sin saber siquiera que STAMP provenía del su grupo, \textit{jajaja}. Muchas gracias por la oportunidad de unirme al \textit{Barton Group} para aquella estancia y, posteriormente, para este doctorado. Gracias por la confianza, la paciencia y la flexibilidad ante las diversas circunstancias que han surgido durante el transcurso de este proyecto. Gracias por todo lo que me has enseñado en cuanto a redacción científica, comunicación, análisis de datos y otros conocimientos generales, como expresiones en inglés, aplicaciones tecnológicas, cómo configurar una red doméstica, una estación meteorológica, restaurar ventanas o suelos de madera, además de tu talento musical y las incontables charlas y anécdotas sobre cómo se hacía ciencia en los \textit{días oscuros}, cuando había que dibujar los gráficos a mano, usar papel carbón y máquinas de escribir, y los manuscritos se enviaban en papel por correo postal -- sí, postal, no \textit{electrónico}. Sin duda alguna, has sido el mejor director de tesis que podría haber tenido. Gracias, Geoff.

También quiero expresar mi gratitud a otra persona que ha tenido una gran influencia durante mi doctorado: el Dr. Stuart MacGowan. Stuart fue mi supervisor en la estancia, en la que aplicamos su idea de combinar la divergencia evolutiva con la variación genética en la familia de repeticiones de anquirinas. Tanto durante esa estancia como a lo largo de mi doctorado, Stuart ha sido un gran mentor y ha compartido consejos sobre buenas prácticas en análisis de datos, programación y reproducibilidad. Admiro su inagotable entusiasmo por la ciencia, su visión, sus brillantes ideas de investigación y su disponibilidad para ayudar siempre que lo he necesitado. Gracias, Stuart. Gracias también a los demás miembros, pasados y presentes, del \textit{Barton Group}, DAG y el equipo de Jalview: Mateusz Warowny, Renia Correya, los Dres. Ben Soares, Carey Metheringham, James Abbot, Jim Procter, Khadija Jabeen, Marek Gierlinski, Matt Parker, Maxim Tsenkov, Michele Tinti, Pete Thorpe y otros. Ha sido un placer trabajar con todos vosotros y disfrutar de numerosas charlas, sesiones de \textit{Journal Club} y comidas de celebración. Gracias también a mi supervisor secundario, el Prof. Ulrich Zachariae, por su apoyo y por revisar uno de los capítulos de esta Tesis, junto a los Dres. Ben Soares, Radoslav Krivák, Stuart MacGowan y el Prof. Geoff Barton.

A mis dos comités de tesis: los Profes. Daan van Aalten, Satpal Virdee, Vicky Cowling y la Dra. Jorunn Bos por su orientación y comentarios durante todo mi doctorado. A mis examinadores, los Profes. Alessio Ciuli y David Hoksza, que amablemente accedieron a leer y evaluar esta Tesis. Al Prof. Rastko Sknepnek por ser el organizador de mi defensa \textit{viva voce} y a todos los investigadores principales de la división de Biología Computacional: los Profes. Andrei Pisliakov, Ulrich Zachariae y los Dres. Gabriele Schweikert, Hajk Drost y Maxim Igaev por llevar la ciencia a un nivel tan excelente, elevando así esta División, el Instituto y la Universidad a nuevas alturas, convirtiéndolas en un destino ideal para la investigación de vanguardia. Gracias también al servicio de informática de la universidad por su apoyo a la infraestructura en la que se llevó a cabo este trabajo.

A mis antiguos colegas de doctorado en la división, los Dres. Callum Ives, Dom Gurvik, Marcus Bage, Maxim Tsenkov y Neil Thomson, y a mis compañeros actuales: Alp Tegin, Euan MacKay, Peter Ezzat, Rosie Gallagher, Stefan Manolache, Tanmayee Narendra y Yijia Qiang, por hacer que el día a día y la rutina en la oficina sean tan cómodos y agradables. Un agradecimiento especial para Rosie y Carey por sus valiosos comentarios, pasteles y cenas; Maxim por ser un gran supervisor, colega y amigo, y por haber progresado y crecido juntos; Peter por tantas tardes en la oficina y por seguir adelante juntos. A los compañeros del instittuto, PiCLS, mi cohorte de EASTBIO y los estudiantes a los que he supervisado. A los posdoctorados y al resto del personal de la división e instituto, y al maravilloso equipo administrativo y de secretaría, tanto actual como pasado: Sara Salvaterra, Kirsty Forbes, Jenna Lyons, Paige Nell y Ulla Gingule, por su impecable y eficiente trabajo. A la jefa de estudios de posgrado, la Profa. Carol MacKintosh, siempre dispuesta a ayudar con una sonrisa. A la administradora de EASTBIO, la Dra. Maria Filippakopoulou, por su amabilidad y excelente gestión del programa de doctorado. Y, por supuesto, a la facultad de Biociencias de la Universidad de Dundee, a EASTBIO, BBSRC y UKRI por financiar esta beca.

Mi agradecimiento más profundo también a los increíbles músicos y compositores Go Shiina, Hans Zimmer, Hiroyuki Sawano, Kohta Yamamoto, Ramin Djawadi, Sofiane Pamart y Yuki Kajiura. Vuestra música, tanto bandas sonoras originales como clásica, me ha acompañado durante miles de horas de emocionante trabajo de investigación en los últimos seis años. Me ha llenado de emoción, fuerza, determinación, tristeza, esperanza y otras sensaciones que transmitís con vuestro hermoso arte. Gracias por vuestra magia.

\newpage

También quiero agradecer a mis queridos amigos de la escuela Ana, Anabella, Cristina, Eric, Lorena y Paulino, a mi vecina y amiga Rigo y a mis amigos de Dundee Alex, Ethan, Karo, Katie, Matt, Maxim, Niamh, Nikita y Sam. Vuestra amistad y apoyo emocional han sido pilares imprescindibles durante estos últimos seis años. Habéis sido los mejores amigos que uno podría desear y me habéis levantado en los momentos más oscuros. No podría estar más agradecido. A mis amigos de la UPF: Aina, Altaïr, Luisa y los Dres. Alexander Gmeiner, Carla Castignani y Pau Badia. Sois grandes amigos y científicos. Hemos recorrido un largo camino desde aquellas tardes en las ruinas de la UB y nuestras \textit{maravillosas} clases de diseño web y algorítmica. En un par de meses, todos seremos doctores. Tengo muchas ganas de celebrarlo juntos.

A mis amigos de la UAB: Xavi, la Dra. Nerea Moreno y mis queridas \textit{Piñas}: los Dres. Ferran Garcia, Guillermo Palou, Núria Serna y Sergio Marco. Ha sido un honor compartir la última década en la academia con vosotros. Desde el primer día del grado en Bellaterra en 2014 hasta la última defensa de doctorado en Dundee en 2025. Hemos crecido y aprendido tanto, y lo hemos hecho juntos. No podría haber elegido mejores compañeros de viaje. Estoy tan orgulloso y os admiro a cada uno de vosotros. Estoy ansioso por ver qué nos depara el futuro. Gracias por estar en mi vida. Os quiero.

A mi padre, Alfonso, que en paz descanse, a mi madre, Alba, y a mis hermanos Héctor y Carlos. Gracias por haberme educado como lo habéis hecho, inculcándome valores de humildad, respeto, generosidad y constancia, y por estar siempre ahí para mí. Al resto de mi familia: abuelos, tíos y primos, por vuestro cariño y por los recuerdos imborrables que creamos juntos. Especialmente, a mis Titos Alfonso y Paco, mis Titas María Elena y Merchi, y mis primos Alfonso, Blanca, Carmen, Elena y Luis, por haber sido los mejores anfitriones mientras teletrabajaba desde vuestras casas. Siempre estáis en mi corazón.

Gracias a Darshan, Hina y Dhyan por acogerme en vuestra preciosa familia y cultura, y también por preocuparos por mí y cuidarme. Por último, quiero dar gracias a mi maravillosa pareja, Prarthna. Estoy tan agradecido de haberte encontrado. Gracias por estos tres años de amor, apoyo, consejo, paciencia, fe, ánimo, alegría y pura felicidad que has traído a mi vida. Deseo vivir lo que venga \textit{juntos}. Te amo con todo mi corazón.