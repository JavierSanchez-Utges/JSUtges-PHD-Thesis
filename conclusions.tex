\chapter{Conclusions}

\section*{Preface}

This Chapter brings together the main findings presented in the four previous Chapters of results of this Thesis and contextualises them with the current state of the art within the field. Additionally, it covers the direction in which this research is heading by describing the next steps that could be taken to take the work presented in this Thesis further.

\section{Introduction}

The work presented in this Thesis sheds light into the nature of ligand binding sites by developing methods to define and characterise them. A pipeline is described to characterise experimentally determined fragment screening and biologically relevant ligand binding sites in the PDBe resulting in the LIGYSIS dataset. A web resource is presented for the exploration of 64,782 ligand sites across 25,003 proteins, as well as for the analysis of user sets of protein-ligand complexes. Finally, the human component of the LIGYSIS set is employed to carry out the largest independent critical comparative assessment of ligand binding site prediction tools to date.

\section{Fragment screening sites analysis}

\autoref{chap:FRAGSYS} describes novel methods for the definition and characterisation of 293 ligand binding sites derived from 37 publicly available fragment screening experiments \cite{UTGES_2024_FRAGSYS}. In this Chapter, sites are defined by grouping ligands based on their interactions with the protein, therefore avoiding completely the need for structural superposition. Ligand binding sites were later grouped into four robust clusters (C1-C4) by their relative solvent accessibility profile. The four clusters differed in solvent accessibility, evolutionary divergence across homologues, relative enrichment in human missense variation, and most importantly, enrichment in known functional sites. C1 sites were the most buried, conserved across homologues, depleted in missense variation and enriched in functional sites, whereas C4 sites were the complete opposite.

This information could be used in early-stage drug discovery to prioritise between targets and between sites within the same target protein, potentially leading to a reduction in attrition in clinical trials. Cluster 1 sites are $\approx$28 times more likely to be functional than sites with a C4 label. Consequently, targeting a C1 site is much more likely to have an effect on the target protein function. Considering the evolutionary divergence profile of the site can also provide insight into off-target effects, i.e., targeting a conserved site might also affect other members of the same protein family. In a similar manner, the missense enrichment profile of the site offers insight into potential pharmacogenetic effects. For example, a site with a high enrichment in genetic variation might not be desirable, as individuals with distinct alleles might respond differently to the drug targeting the site. This information could be leveraged in a different way if looking for a site for a molecular glue \cite{SCHREIBER_2021_GLUES}, targeted modification \cite{BREWER_2024_ATLAS, BREWER_2024_SMAD3, ZHAO_2024_TFEB} or degradation \cite{ZENGERLE_2015_BRD4, GADD_2017_PROTAC} of a protein. In this case, easily accessible sites that are divergent across homologues and do not alter the protein function in any way might be of interest, i.e., C3-C4 sites.

\section{The LIGYSIS dataset and web resource}

\autoref{chap:LIGYSIS_WEB} extends the ligand site definition and characterisation methodology introduced in \autoref{chap:FRAGSYS} and applies it to the entire PDBe, resulting in the LIGYSIS dataset. LIGYSIS comprises 64,782 ligand binding sites from biologically relevant protein-ligand complexes across 25,003 proteins. Sites are characterised in terms of conservation, variation and accessibility in the same manner as the fragment screening sites. In addition, a score is defined that is indicative of the likelihood of function for a given ligand binding site. 

\autoref{chap:LIGYSIS_WEB} also describes the architecture and implementation of LIGYSIS-web, a resource for the analysis of protein-ligand binding sites. LIGYSIS-web is a Python Flask web application to dynamically explore the full LIGYSIS dataset. Furthermore, users can submit their set of protein-ligand complexes for analysis and subsequent visualisation and download of results. The LIGYSIS dataset can be browsed by UniProt accession and provides rich output with hyperlinks to the relevant databases, downloadable tabular data, alignments, images and structural data in PyMOL or ChimeraX format. LIGYSIS-web can be found here: \url{https://www.compbio.dundee.ac.uk/ligysis/}.

\section{Assessing ligand binding site prediction tools}

\autoref{chap:LBS_COMP} uses the human component of the LIGYSIS dataset to carry out the largest independent comparative assessment of ligand binding site prediction tools to date \cite{UTGES_2024_LBSCOMP}. Thirteen canonical methods are evaluated at the pocket and residue level employing fourteen informative metrics. LIGYSIS defines ligand binding sites by aggregating unique biologically relevant protein-ligand interfaces across the biological assemblies of multiple structures available for a given protein. Other datasets, previously used to train and test ligand binding site prediction tools, consider redundant protein-ligand interfaces, single ligand-protein complexes and asymmetric instead of biological units. For these reasons, LIGYSIS presents and advantage over these other sets and is therefore proposed as a new reference set to test on for future methods or benchmarks.

Re-scored fpocket predictions by PRANK (60.4\%) and DeepPocket (58.1\%) present the highest recall considering a threshold of DCC = 12 \AA{} and predictions within the top-\textit{N}+2, demonstrating the paramount importance of a robust pocket scoring scheme. VN-EGNN, IF-SitePred and other machine learning-based methods are very precise ($>$90\%). However, their recall is low ($<$40\%) due to the small number of predicted pockets (PUResNet), the redundancy of their predictions (VN-EGNN), sub-optimal clustering of points (IF-SitePred) or score thresholding (GrASP). \autoref{chap:LBS_COMP} presents clear evidence to pinpoint the areas in which the ligand site prediction community should focus on in order to improve the quality of the methods as well as the rigorosity with which these are evaluated.

\section{Improving ligand binding site prediction tools}

\autoref{chap:LBS_IMPROV} carries on the work in \autoref{chap:LBS_COMP} by exploring two factors identified to have an effect on ligand binding site prediction performance: pocket scoring and prediction redundancy \cite{UTGES_2024_LBSCOMP}. Fifteen non-redundant and scoring variants are analysed for methods that result in redundant predictions: VN-EGNN, IF-SitePred and DeepPocket\textsubscript{SEG} and methods that do not score their pockets: PUResNet, PocketFinder\textsuperscript{+}, Ligsite\textsuperscript{+} and Surfnet\textsuperscript{+}. This Chapter demonstrates the detrimental effect of redundancy in pocket prediction by showing improvements in recall of $>$5\% for VN-EGNN and DeepPocket\textsubscript{SEG} and up to 13\% for IF-SitePred after re-scoring its predictions too. Additionally, pocket scoring can result in improvements of up to 29\% in Precision\textsubscript{1K} (Surfnet\textsuperscript{+}) and $>$500 \# TP\textsubscript{100 FP} (PUResNet).

The improvement in performance accomplished in this Chapter is limited by the fact that it takes place post-prediction. It makes sense to think that approaching these issues prior to prediction, in the source code of the methods, would have an even higher positive impact on their performance. For example, making use of the probabilities returned by the 40 IF-SitePred models and improving their clustering implementation, ensuring non-redundant predictions for VN-EGNN, lowering the atom ligandability threshold on GrASP, or returning more than a single prediction for PUResNet. Authors are encouraged to consider these aspects for the benefit of their tools and the community.

\section{Future steps}

Future work on LIGYSIS-web would focus on the analysis of heterometric protein-ligand complexes and the implementation of functionalities to improve the accessibility and usability of the database. Examples of these improvements are: a function to search by ligand name or type, the selection of multiple sites or residues at once, the export of the current structure view -- instead of a default one -- or the alignment of all structures of a protein on the same coordinate reference. Additionally, an integration of LIGYSIS-web with Jalview would be really enriching, as it would boost access to the LIGYSIS resource and allow for enhanced integrative analysis with the many features, databases and tools that Jalview already offers.

Another area of interest would be the ligand site functional score. The current score, reported on LIGYSIS-web, is purely based on the solvent accessibility profile of a given ligand site. Exploring the relationship between the evolutionary divergence, missense enrichment profiles and known functional sites would be of interest and potentially provide further insight into the likelihood of function of a given ligand binding site. For example, a new functional score could be developed that integrates structural, protein sequence and genetic variation data, amongst other features, to predict likelihood of function.

An in-depth characterisation of the experimentally determined \textit{pocketome}, i.e., LIGYSIS, would also be of value. This would extend the characterisation introduced in \autoref{chap:FRAGSYS} to include features such as pocket surface area, volume, hydrophobicity, charge, orientation relative to the main axis or centroid, interaction type and so on. The thorough evaluation of ligand site prediction methods in \autoref{chap:LBS_COMP} and \autoref{chap:LBS_IMPROV} strongly suggests fpocket, PRANK and P2Rank as the best predictors. A combination of these methods could be used to predict on all the human proteome making use of 3D structure models as provided by AlphaFold DB. The same characterisation pipeline could then be employed, thus providing additional information into the biological, evolutionary and structural context of the human pocketome, reaching beyond the confines of experimentally determined protein-ligand complexes.

\section{Concluding remarks}

Overall, the work presented in this Thesis aims to advance the understanding of ligand binding sites. This is done by carrying out a binding site characterisation integrating structural, divergence and variation data as well as a comprehensive evaluation of methods for binding site prediction. This characterisation could be integrated with other features or scores, such as the \textit{P} and \textit{FP} scores recently defined by Ibrahim \textit{et al.} \cite{IBRAHIM_2024_PSCORE, IBRAHIM_2024_FMOPHORE}. These scores make use of molecular dynamics to estimate the residence time of a ligand in a protein pocket and the interaction energy of the complex. This information is in turn leveraged to identify hotspot residues with a greater contribution to the energetic stability and affinity of the complex. The combination of these two approaches would merge structural and evolutionary data with chemical and energetic information, which could be employed to screen structure databases like the PDBe or AFDB for those pockets, and residues within them, with more favourable features for therapeutic targeting.

The synthesis of these diverse perspectives provides a promising avenue of research for developing a more holistic and complete understanding of ligand binding sites. These insights are pivotal to modern drug discovery, where they could accelerate target identification, lead optimisation and reduce attrition and off-target effects. Furthermore, this research has broader implications, potentially extending beyond drug discovery to applications such as protein function annotation and evolution or enzyme engineering.