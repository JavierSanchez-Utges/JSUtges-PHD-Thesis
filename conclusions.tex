\chapter{Conclusions}

\section*{Preface}

This Chapter brings together the main findings presented in the four previous Chapters of results of this Thesis and contextualises them with the current state of the art within the field. Additionally, it also covers the direction in which this research is heading by describing the next steps that should be taken to take the work presented in this Thesis further.

\section{Introduction}

Proteins are the functional and structural blocks upon which life is built. They carry out or are involved in all biological processes that take place within cells. The amount, localisation, state and interaction with other molecules of proteins is strictly controlled by complex gene regulation networks, signalling cascades, and environment-dependent conformational changes. Any of these mechanisms failing to work correctly can result in under- or overexpression, hypo- or hypermorphism, truncated, mutated or unfolded proteins, amongst other issues, which eventually can lead to disease. Proteins do not carry their function in isolation, but usually interact with other proteins, nucleic acids, ions or small molecules. Moreover, chemical compounds can be used as drugs to modulate or inhibit protein function. Identifying or predicting the location of where these molecules bind to proteins and the way in which they do so is therefore critical to understand better and modulate protein function.

The work presented in this Thesis sheds light into the nature of ligand binding sites by developing methods to define and characterise ligand binding sites using protein sequence data from evolutionarily related homologues, missense genetic variation from the human species and structural features such as solvent accessibility \cite{UTGES_2024_FRAGSYS}. A pipeline is described to characterise all experimentally determined ligand binding sites deposited in the PDBe \cite{ARMSTRONG_2020_PDBE} resulting in the LIGYSIS dataset and a web resource presented for the exploration of 65,000 ligand binding sites across 25,000 proteins as well as for the analysis of custom user sets of protein-ligand complexes. Finally, the human component of the LIGYSIS set is employed to carry out the largest critical comparative assessment of ligand binding site prediction tools. This analysis objectively evaluates the performance of thirteen canonical methods and fifteen novel variants first introduced in this work using the LIGYSIS dataset as a benchmark and more than ten assessment metrics \cite{UTGES_2024_LBSCOMP}.

\section{Fragment screening sites analysis}

Fragment screening is a widely used technique to find lead molecules in early-stage drug discovery \cite{MURRAY_2009_FBDD}. It usually employs X-ray crystallography to obtain high-quality and throughput data about the binding mode of small molecule fragments to a target protein. \autoref{chap:FRAGSYS} describes novel methods for the definition and characterisation of 293 ligand binding sites derived from 37 publicly available fragment screening experiments \cite{UTGES_2024_FRAGSYS}. Unlike previous methods that rely on the structural superposition of protein chains or ligands \cite{SHIN_2005_PDBLIGAND, KOZAKOV_2005_CLUSTERING, WASS_2010_3DLIGANDSITE}, the method proposed in \autoref{chap:FRAGSYS} defines sites by grouping ligands based on their interactions with the protein. This approach avoids completely the structural alignment of protein chains, which can be difficult when dealing with a high number of protein chains or different conformational states. The defined ligand sites therefore represent a location on the protein surface to which ligands with a similar interaction fingerprint bind.

Ligand binding sites were later grouped into four robust clusters (C1-C4) by their relative solvent accessibility profile. Vectorial representations of the RSA \cite{TIEN_2013_RSA} of each residue within the site were employed, as this captured better the solvent accessibility distribution of the site than a simple average of the RSA of the residues. The four clusters did not only differ in terms of accessibility, but also in their evolutionary divergence across homologues \cite{UTGES_2021_ANKS}, relative enrichment in human missense variation \cite{MACGOWAN_2024_VARIANTS}, and most importantly enrichment in known functional sites annotated in UniProt \cite{UNIPROT_2018_UNIPROT}. C1 sites were the most buried, conserved across homologues, depleted in missense variation and enriched in functional sites, whereas C4 sites were the complete opposite. Through a literature search, seventeen examples were identified of sites yet to be annotated in UniProt, but with literature evidence. These findings, further supported the hypothesis that these RSA-defined cluster labels recapitulate known functional sites and can be employed to rank the sites on a protein based on likelihood of function.

This information could be used in early-stage drug discovery to prioritise between targets and even between sites within the same target protein. Cluster 1 sites are $\approx$28 times more likely to be functional than sites with a C4 label. Consequently, targeting a C1 site is $\approx$28 more likely to have an effect on the target protein function. Considering the evolutionary divergence profile of the site can also provide insight into off-target effects, i.e., targeting a conserved site might also affect other members of the same protein family. In a similar manner, the missense enrichment profile of the site offers insight into potential pharmacogenetic effects. For example, a site with a high enrichment in genetic variation might not be desirable, as individuals with distinct alleles might respond differently to the drug targeting the site. This information could be leveraged in a different way if looking for a site for a molecular glue \cite{SCHREIBER_2021_GLUES}, targeted modification \cite{BREWER_2024_ATLAS, BREWER_2024_SMAD3, ZHAO_2024_TFEB} or degradation \cite{ZENGERLE_2015_BRD4, GADD_2017_PROTAC} of a protein. In this case, easily accessible sites which do not alter the protein function in any way might be of interest, and so highly accessible, divergent across homologues sites might be more attractive, i.e. C3-C4 sites.

\section{The LIGYSIS dataset and web resource}

\autoref{chap:LIGYSIS_WEB} extends the ligand site definition and characterisation methodology introduced in \autoref{chap:FRAGSYS} and applies it to the entire PDBe, resulting in the LIGYSIS dataset. LIGYSIS comprises $\approx$65,000 ligand binding sites from $\approx$25,000 proteins with biologically relevant protein-ligand complexes as defined by BioLiP \cite{YANG_2013_BIOLIP}. These sites are characterised in terms of divergence, missense enrichment and solvent accessibility in the same manner as the fragment screening sites. In addition, a functional score (\autoref{eq:func_score}) is defined in this Chapter. This score results from combining the probabilities returned by the MLP to predict RSA cluster labels and the enrichment of known functional sites in UniProt \cite{NIGHTINGALE_2017_API} within each cluster. The use of this numerical score offers an advantage relative to the categorical C1-C4 RSA labels as it maximises the information extracted from the solvent accessibility profile of a given binding site. In this manner, sites with the same cluster label could be further stratified by a functional score gradient.

Additionally, \autoref{chap:LIGYSIS_WEB} describes the architecture and implementation of LIGYSIS-web, a resource for the analysis of protein-ligand binding sites. LIGYSIS-web is a Python Flask \cite{GRINBERG_2018_FLASK} web application to dynamically explore the full LIGYSIS dataset. Furthermore, users can submit their set of protein-ligand complexes in either PDB or mmCIF formats for analysis and subsequent visualisation and download of results. The LIGYSIS dataset can be browsed by UniProt accession identifier, entry or protein name. The results page id divided in three panels, each offering different relevant information for the protein of interest: Binding Sites Panel (left), Structure Panel (centre) and Binding Site Residues Panel (right). The Binding Sites and Residues panels are structured in the same fashion and include an interactive graph, implemented with Chart.js \cite{CHARTJS}, and dynamic table displaying the same data. These charts and tables present average site features, e.g., size, divergence, missense enrichment, RSA or functional score for the Binding Sites Panel whereas they represent individual residue-level features in the Binding Site Residues Panel. Both of these panels are linked to the central Structure Panel through hover and click events. This Structure Panel includes a 3Dmol.js \cite{REGO_2014_3DMOL} structure viewer which can be used to explore a superposition of all ligands binding to the protein of interest as well as the individual protein-ligand complexes and the atomic interactions within them. This page is rich in output with hyperlinks to the PDBe ligand pages \cite{CHOUDHARY_2024_PDBETOOLS}, UniProt and with tabular data downloadable in CSV format, MSA in STO, images or screenshots in PNG and finally, superposition and individual assemblies in PyMOL (PML) or ChimeraX (CXC) format. LIGYSIS-web can be found here: \url{https://www.compbio.dundee.ac.uk/ligysis/}.

\section{Assessing ligand binding site prediction tools}

\autoref{chap:LBS_COMP} uses the human component of the LIGYSIS dataset to carry out the largest independent comparative assessment of ligand binding site prediction tools analysing thirteen canonical methods at the pocket and residue level employing more than ten informative metrics. LIGYSIS defines ligand binding sites by aggregating unique biologically relevant protein-ligand interfaces across the biological assemblies of multiple structures available for a given protein. Other datasets, previously used to train and test ligand binding site prediction tools, consider redundant protein-ligand interfaces, single ligand-protein complexes and asymmetric instead of biological units. For these reasons, LIGYSIS presents and advantage respect these other sets and should be the new reference set for future methods or benchmarks to test on.

Re-scored fpocket \cite{GUILLOUX_2009_FPOCKET} predictions by PRANK \cite{KRIVAK_2015_PRANK} (60.4\%) and DeepPocket \cite{AGGARWAL_2022_DEEPPOCKET} (58.1\%) present the highest recall considering a threshold of DCC = 12 \AA{} and predictions within the top-$N$+2 demonstrating the paramount importance of a robust pocket scoring scheme. VN-EGNN \cite{SESTAK_2024_VNEGNN}, IF-SitePred \cite{CARBERY_2024_IFSP} and other machine learning-based methods are very precise ($>$90\%), however lack in recall ($<$40\%) due to the small number of predicted pockets (PUResNet \cite{KANDEL_2024_PURESNET}), the redundancy of their predictions (VN-EGNN), sub-optimal clustering of points (IF-SitePred) or score thresholding (GrASP \cite{SMITH_2024_GrASP}). \autoref{chap:LBS_COMP} presents clear evidence to pinpoint the areas in which the ligand site prediction community should focus on in order to improve the quality of the methods as well as the rigorosity with which these are evaluated.

\section{Improving ligand binding site prediction tools}

\autoref{chap:LBS_IMPROV} carries on the work in \autoref{chap:LBS_COMP} by exploring two factors identified to have an effect on ligand binding site prediction performance: pocket scoring and prediction redundancy. Fifteen non-redundant and scoring variants are analysed for methods that result in redundant predictions: VN-EGNN, IF-SitePred and DeepPocket\textsubscript{SEG} and methods that do not score their pockets: PUResNet, PocketFinder\textsuperscript{+} \cite{AN_2005_POCKETFINDER}, Ligsite\textsuperscript{+} \cite{HENDLICH_1997_LIGSITE} and Surfnet\textsuperscript{+} \cite{LASKOWSKI_1995_SURFNET}. This Chapter demonstrates the detrimental effect of redundancy in pocket prediction by showing improvements in recall of $>$5\% for VN-EGNN and DeepPocket\textsubscript{SEG} and up to 13\% for IF-SitePred after re-scoring its predictions too. Additionally, pocket scoring can result in improvements of up to 29\% in Precision\textsubscript{1K} (Surfnet\textsuperscript{+}) and $>$500 \# TP\textsubscript{100 FP}.

The improvement in performance accomplished in this Chapter, while considerable, is limited by the fact that it takes place post-prediction. It makes sense to think that approaching these issuesprior to prediction in the source code of the methods, e.g., making use of the probabilities returned by the 40 IF-SitePred models and improving their clustering implementation, ensuring predictions are not redundant in VN-EGNN, lowering the atom ligandability threshold on GrASP, or returning more than a single prediction on PUResNet, would have an even higher positive impact on the performance of these methods, and so authors are encouraged to consider these aspects for the benefit of their tools and the wider community.

\section{Future steps}

The future steps to be taken in this research venue would focus on the analysis and updating of the LIGYSIS dataset as well as on the improvement and addition of features to the LIGYSIS web server. The current functional score reported on LIGYSIS-web is purely based on the solvent accessibility profile of a given ligand site. Exploring the relationship between the evolutionary divergence, missense enrichment profiles and known functional sites would be interesting and potentially provide further insight into the likelihood of function for a given ligand binding site. For example, a functional score could be developed that integrates structural, protein sequence and genetic variation data, amongst other features, to predict likelihood of function. An in-depth characterisation of the existing \textit{pocketome}, i.e., LIGYSIS, would also be of interest. This would extend the characterisation introduced in \autoref{chap:FRAGSYS} to include features such as pocket surface area, volume, hydrophobicity, charge, orientation relative to the main axis or centroid, interaction type, etc. The application of this pocket characterisation pipeline, could also be applied to predicted pockets, as by the methods explored in \autoref{chap:LBS_COMP} and \autoref{chap:LBS_IMPROV}, providing additional information into the biological, evolutionary and structural context of the predicted ligand binding site.

Regarding LIGYSIS-web, future work would allow the analysis of heterometric protein-ligand complexes, i.e., a ligand interacting with different proteins, the characterisation of predicted pockets, as mentioned above, other functionalities to improve the accessibility of the database, i.e., a search by ligand name or type, and general improvements to the usability, such as selecting multiple sites or residues at once, exporting the current view, and not a default one, to PyMOL \cite{SCHRODINGER_2015_PYMOL} or ChimeraX \cite{PETTERSEN_2021_CHIMERAX} or having all structures for the same protein aligned on the same reference.

\section{Concluding remarks}

XXX.