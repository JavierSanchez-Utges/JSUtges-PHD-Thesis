\chapter*{Acknowledgements} % means there is no chapter number for this

This PhD has been quite the journey. Four and a half years that have flown by, but at the same time feel like such a long time, in which so much happened in both my personal and professional life. I have so many people to thank for their support, without whom I would not be where I am today. I shall start from the beginning.

I first got interested in life sciences during my childhood, at around six years old, whilst living in a small village, called Almedinilla, in the province of Córdoba, Andalucía, Spain. A biologist, whose name I cannot remember, used to take me, school friends and parents on beautiful day trips to explore the local nature. He would tell us all about the plants, lichens, birds, mammals and minerals we could find on these excursions. I was fascinated by the biodiversity of the area and the differences between different organisms, their behaviour and adaptation. After moving to Terrassa, a city near Barcelona, Pepita Penalba, my biology school teacher, would take some students, line us up and ask us questions about what we had been learning about. Students would move along the line depending on the answer to her questions. Accurate answers would move you towards the top of the line, indicating a good notion of the studied subject. I believe this fuelled my competitive spirit and encouraged me to learn and study more, so I could be amongst the top students of the class. Later, in secondary school, my teachers Dr Joel Pascual and Carme Hernández taught me more about physics, chemistry, biology and geology, further increasing my interest in these areas. I found Mendelian genetics particularly truly interesting and fun, which would lead me later on to study a BSc degree in \textit{Genetics} at the Universitat Autònoma de Barcelona (UAB). During high school, my class mate Dr Marc Botifoll got awarded a grant for the \textit{Crazy about Biomedicine} workshop organised by IRBB. Thanks to this, and under Dr Michela Candotti's supervision, together with another class mate, Cristina Ortiga, we carried out a project to evaluate potential HIV drugs using computational methods. This was my first exposure to bioinformatics. During the three years I studied at UAB, I was fortunate to enjoy the lectures of many great readers. Here are some of the ones that really had an impact on my decision to keep on studying after the degree: Profs Antonio Barbadilla, Vicente Martínez, Hafid Laayouni, Alfredo Ruíz, Isaac Salazar, Jesús Piedrafita and Dr Raquel Egea. They taught me about the basis of genetics, animal physiology, biostatistics, population, developmental and quantitative genetics, as well as coding and bioinformatics. It was on the third year of my degree when I realised coding was ``my thing'' and wet lab was over for me. On that summer, I joined Prof Francesc Calafell's group at PRBB to undertake a 3-month project revolving around population and forensic genetics and R programming. Straight after that, I went on an Erasmus student exchange to Dundee, on the fourth and last year of my degree. During the exchange, I took lectures on \textit{Molecular Structure \& Interactions} with Prof Bill Hunter and \textit{Applied Bioinformatics} with Dr David Martin, further confirming that bioinformatics was my passion and starting to develop an interest for the structural side of it. The next year, I started an MSc in \textit{Bioinformatics for Health Sciences} at Universitat Pompeu Fabra, in Barcelona. Dr Javier García's lectures on Python programming and Prof Baldo Oliva's on \textit{Structural Bioinformatics} captivated me, and so I applied for a 1-year internship, part of the MSc, here in the Barton Group, which took place from September 2019 -- July 2020. During that time, I secured an EASTBIO DTP studentship to carry out my PhD under Prof Geoff Barton's supervision. I would like to thank all these people that contributed to my education before the beginning of my doctoral training programme, which I have carried out from October 2020 to March 2025.
















%Prof Geoff Barton (+ Ulrich Zachariae), Dr Stuart MacGowan

%Drs. James Abbot, Jim Procter, Khadija Jabeen, Ben Soares, Pete Thorpe, Carey Metheringham, Matt Parker, Michele Tinti, 

%Peter Ezzat, Alp Tegin, Rosie Gallagher, Stefan Manolache, Euan MacKay, Drs Callum Ives, Neil Thomson, Dom Gurvik.

%Supervised students, SLS and EASTBIO cohort colleagues,

%TCM: Vicky Cowling, Dan Van Aalten, Satpal Virdee, Jorunn Bos. Examiners: Profs Alessio Ciuli, David Hoksza.

%Rastko as convener, other professors, postdocs, division secretaries, Sara Salvaterra, Kirsty Forbes, Jenna Lyons, Ulla. SLS admin team and particularly Carol MacIntosh.

%Drs. Ferran, Guille, Sergio, Núria.

%Xavi.

%Nerea, Pau, Altaïr, Carla, 

%Alex, Ethan, Katie, Maxim, Nikita, Matt, Sam.

%Ana, Anabella, Cristina, Eric, Lorena, Paulino, Rigo.

%Mom and brothers. Grandparents, aunts, uncles, cousins.

%PRARTHNA and her family