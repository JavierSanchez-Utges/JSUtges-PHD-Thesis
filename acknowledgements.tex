\chapter*{Acknowledgements} % means there is no chapter number for this

This PhD has been quite the journey. Four and a half years that have flown by, but equally feel like such a long time, in which so much happened in both my personal and professional life. I have so many people to thank for their support, without whom I would not be where I am today. I shall start from the beginning.

I first got interested in life sciences during my childhood, at around six years old, whilst living in a small village, called Almedinilla, in the province of Córdoba, Andalucía, Spain. A biologist, whose name I cannot remember, used to take me, school friends and parents on beautiful day trips to explore the local nature. He would tell us all about the plants, lichens, birds, mammals and minerals we could find on these excursions. I was fascinated by the biodiversity of the area and the differences between distinct organisms, their behaviour and adaptation. After moving to Terrassa, a city near Barcelona, Pepita Penalba, my biology school teacher at \textit{Cultura Pràctica} school, would take some students, line us up and ask us questions about what we had been learning about. Students would move along the line depending on the answer to her questions. Accurate answers would move you towards the top of the line, indicating a good notion of the studied subject. I believe this fuelled my competitive spirit and encouraged me to learn and study more, so I could be amongst the top students of the class. Later, in secondary school, my teachers Dr Joel Pascual and Carme Hernández taught me more about physics, chemistry, biology and geology, further increasing my interest in these areas. I found Mendelian genetics particularly interesting and fun, which would lead me later on to study a BSc degree in Genetics at the \textit{Universitat Autònoma de Barcelona} (UAB). During high school at \textit{Institut Montserrat Roig}, my class mate Dr Marc Botifoll got awarded a grant for the \textit{Crazy about Biomedicine} workshop organised by IRBB. Thanks to this, together with another class mate, Cristina Ortiga, we carried out a project to evaluate potential HIV drugs using computational methods under the supervision of Dr Michela Candotti. This was my first exposure to bioinformatics. During the three years I studied at UAB, I was fortunate to enjoy the lectures of many great readers. Here are some of the ones that really had an impact on my decision to keep on studying after the degree: Profs Antonio Barbadilla, Vicente Martínez, Hafid Laayouni, Alfredo Ruíz, Isaac Salazar, Jesús Piedrafita and Dr Raquel Egea. They taught me about the basis of genetics, animal physiology, biostatistics, population, developmental and quantitative genetics, as well as coding and bioinformatics. It was on the third year of my degree that I realised coding was ``my thing'' and wet lab was over for me. On that summer, I joined Prof Francesc Calafell's group at PRBB to undertake a 3-month project revolving around population and forensic genetics and R programming. Straight after that, I went on an Erasmus student exchange to Dundee, on the fourth and last year of my degree. During the exchange, I took lectures on Molecular Structure \& Interactions with Prof Bill Hunter and Applied Bioinformatics with Dr David Martin, further confirming that bioinformatics was my passion and starting to develop an interest for the structural side of it. The next year, I started an MSc in Bioinformatics for Health Sciences at \textit{Universitat Pompeu Fabra}, in Barcelona. Dr Javier García's lectures on Python programming and Prof Baldo Oliva's on Structural Bioinformatics captivated me, and so I applied for a 1-year internship, part of the MSc, here in the Barton Group, which took place from September 2019 -- July 2020. During that time, I secured an EASTBIO DTP studentship to carry out my PhD under Prof Geoff Barton's supervision. I would like to thank all these people that contributed to my education before the beginning of my doctoral training programme, which I have carried out from October 2020 to March 2025.

I would like to thank Prof Geoff Barton, who, I like to believe, saw something in me back in 2019, when I came for an interview without even knowing that STAMP came from the Barton Group, \textit{hahaha}. Thank you so much for the opportunity of joining the Barton Group for that internship project and later on for this doctoral programme. Thanks for the confidence, trust, understanding, patience and flexibility that you have had due to the various circumstances that have arisen during the length of this project. Thanks for everything that you have taught me in terms of scientific writing, communication, data analysis, project management and other general knowledge such as English phrases, technological applications, broadband network set-up, weather station, how to refurbish windows or wooden floors, as well as your music talent, and countless chats and anecdotes about how science used to be done in the \textit{dark} days when one had to draw graphs by hand, use carbon paper copies and typewriters and paper manuscripts were sent via mail -- yes, mail, not \textit{e-mail}. I can say without hesitation that you are the best supervisor I could have had for this PhD. Thanks, Geoff.

I would like to extend my gratitude to another person that has had a massive influence during my PhD: Dr Stuart MacGowan. Stuart first supervised me during my internship, which applied his idea of combining evolutionary divergence with genetic variation to the ankyrin repeat family. Both during the internship and throughout this PhD, Stuart has been a great mentor and shared advice on best practices in data analysis, coding and reproducibility. I also admire his never-ending enthusiasm about science, which is truly contagious, his vision, brilliant research ideas and his readiness to help whenever I have needed it. Thanks, Stuart. Thanks also to the other members, past and present, of the Barton Group, DAG and Jalview team: Mateusz Warowny, Renia Correya, Drs Ben Soares, Carey Metheringham, James Abbot, Jim Procter, Khadija Jabeen, Marek Gierlinski, Matt Parker, Maxim Tsenkov, Michele Tinti, Pete Thorpe and others. It has been a pleasure working with you and having many enjoyable Group Talks, Journal Club sessions and celebration lunches. Thanks also to my secondary supervisor Prof Ulrich Zachariae for his support and for proof-reading a Chapter of this Thesis, along with Drs Ben Soares, Radoslav Krivák, Stuart MacGowan, and Prof Geoff Barton.

To my two Thesis committees: Profs Daan van Aalten, Satpal Virdee, Vicky Cowling and Dr Jorunn Bos for their guidance and feedback during the length of my doctorate. My examiners Profs Alessio Ciuli and David Hoksza, who kindly agreed to read and evaluate this Thesis. Prof Rastko Sknepnek for being convenor of my \textit{viva voce} defence and all other principal investigators in the Computational Biology division: Profs Andrei Pisliakov, Ulrich Zachariae, and Drs Gabriele Schweikert, Hajk Drost and Maxim Igaev for making science at such an excellent level, thus elevating this Division, School and University to new heights, making it an ideal destination for cutting-edge research. Thanks also to the IT service of the university for their support of the HPC infrastructure, the work presented in this Thesis was carried out on.

To my past CB PhD colleagues Drs Callum Ives, Dom Gurvik, Marcus Bage, Maxim Tsenkov and Neil Thomson, and present ones: Alp Tegin, Euan MacKay, Peter Ezzat, Rosie Gallagher, Stefan Manolache, Tanmayee Narendra and Yijia Qiang for making the day-to-day work and office routine so comfortable and enjoyable. Special thanks to: Rosie and Carey for their most valued feedback, cakes and dinner parties; Maxim for being such a great supervisor, colleague and friend and having progressed and grown together; Peter for many late evenings in the office and pushing through together. To the wider PhD cohort in the school, PiCLS, my EASTBIO cohort and supervised students. To the postdocs and other staff in the division and school, and the amazing administrative and secretaries team, past and present: Sara Salvaterra, Kirsty Forbes, Jenna Lyons, Paige Nell and Ulla Gingule for their impeccable and efficient work. To the head of postgraduate studies, Prof Carol MacKintosh, who has always been eager and ready to help with a smile on her face. To the EASTBIO DTP administrator, Dr Maria Filippakopoulou, for her kindness and excellent management of the DTP. And, of course to the SLS of the University of Dundee, EASTBIO, BBSRC and UKRI for funding this scholarship.

My deepest gratitude goes also to the amazing musicians and composers Go Shiina, Hans Zimmer, Hiroyuki Sawano, Kohta Yamamoto, Ramin Djawadi, Sofiane Pamart and Yuki Kajiura. Your original sound track and classical music has accompanied me through thousands of hours of exciting work on my research during the last six years. It has filled me with the excitement, strength, determination, sadness, hope and other emotions that you so strongly profess with your beautiful art. Thank you for your magic.

I would too like to thank my dear school friends Ana, Anabella, Cristina, Eric, Lorena and Paulino, my neighbour and friend Rigo and my Dundee friends Alex, Ethan, Karo, Katie, Matt, Maxim, Niamh, Nikita and Sam. Your friendship and emotional support have been an indispensable pillar during the last six years. You have been the best friends one could wish for and have pulled me up in the darkest of times. I could not be more grateful. To my UPF friends Aina, Altaïr, Luisa and Drs Alexander Gmeiner, Carla Castignani and Pau Badia. You guys are great friends and scientists. We have come a long way since our afternoons at the ruins of UB and our \textit{great} lectures about web design and algorithmics. In a couple of months we will hopefully all be doctors. I am really looking forward to celebrating it together.

To my UAB friends Xavi, Dr Nerea Moreno and my dear \textit{Piñas}: Drs Ferran Garcia, Guillermo Palou, Núria Serna and Sergio Marco. It has been such an honour to share the last decade in academia with you. From the first day of undergrad in Bellaterra in 2014 to the last PhD defence in Dundee in 2025. We have grown and learned so much, and we have done so together. I could not have chosen better companions for this journey. I am so proud and admire every single one of you. I cannot wait to see what the future holds for us. Thanks for being in my life. I love you, guys.

To my Dad, Alfonso, may he rest in peace, my Mother, Alba, and my brothers Héctor and Carlos. Thank you for having raised me as you have, imprinting the values of humility, respect, generosity and perseverance in me, and for always being there for me. To the rest of my family: grandparents, aunts, uncles and cousins for your support, love and the everlasting memories we make when we are together. Specially, to my uncles Alfonso and Paco, my aunts María Elena and Merchi and cousins Alfonso, Blanca, Carmen, Elena and Luis for being the best hosts whilst I worked remotely from your homes. You are always in my heart.

Thanks to Darshan, Hina and Dhyan for welcoming me into your beautiful family and culture, and also for worrying about me and taking care of me. Last, but definitely not least, I would like to thank my amazing partner, Prarthna. I am so thankful to have found you. Thanks for these three years of love, support, advice, patience, faith, encouragement, joy and pure happiness and bliss that you have brought into my life. I can't wait to live whatever comes next \textit{together}. I love you with all my heart.