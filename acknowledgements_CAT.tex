\chapter*{Agraïments} % means there is no chapter number for this

Aquest doctorat ha estat tot un viatge. Quatre anys i mig que han passat volant, però que, alhora, han semblat tan llargs i durant els quals han succeït tantes coses tant en la meva vida personal com professional. Tinc moltes persones a qui agrair el seu suport, sense les quals no estaria on sóc avui. Començaré pel principi.

El meu interès per les ciències de la vida va començar durant la infantesa, al voltant dels sis anys, quan vivia en un petit poble anomenat Almedinilla, a la província de Còrdova, Andalusia, Espanya. Un biòleg, del qual no recordo el nom, solia portar-me a mi, a companys d'escola i als nostres pares d'excursió per explorar la natura local. Ens parlava sobre les plantes, els liquens, ocells, els mamífers i els minerals que podíem trobar en aquelles sortides. Em fascinava la biodiversitat de la zona i les diferències entre els diversos organismes, el seu comportament i adaptació. Després de mudar-me a Terrassa, una ciutat a prop de Barcelona, la Pepita Penalba, mestra de biologia a l'escola Cultura Pràctica, acostumava a formar una fila amb alguns alumnes i a fer-nos preguntes sobre el que havíem après. Segons la resposta, avançàvem o ens manteníem a la fila. Les respostes correctes et feien avançar cap a la part davantera, indicant un bon domini del tema. Crec que això va fomentar el meu esperit competitiu i em va animar a aprendre i estudiar més per estar entre els millors estudiants de la classe. Més endavant, a secundària, els meus professors el Dr. Joel Pascual i la Carme Hernández em van ensenyar més sobre física, química, biologia i geologia, incrementant encara més el meu interès per aquestes àrees. Em resultaven especialment interessants les lleis de la genètica mendeliana, cosa que em portaria posteriorment a estudiar un grau en Genètica a la Universitat Autònoma de Barcelona (UAB). Durant el batxillerat a l'Institut Montserrat Roig, el meu company de classe el Dr. Marc Botifoll va obtenir una beca per al taller \textit{Crazy about Biomedicine} organitzat per l'IRBB. Gràcies a això, juntament amb una altra companya, la Cristina Ortiga, vam dur a terme un projecte per avaluar possibles fàrmacs contra el VIH fent servir mètodes computacionals, sota la supervisió de la Dra. Michela Candotti. Aquesta va ser la meva primera aproximació a la bioinformàtica. Durant els tres anys que vaig estudiar a la UAB, vaig tenir la sort d'assistir a les classes de grans docents. Entre els que van influir de manera decisiva en la meva tria de continuar estudiant després del grau, vull destacar els Profs. Antonio Barbadilla, Vicente Martínez, Hafid Laayouni, Alfredo Ruíz, Isaac Salazar, Jesús Piedrafita i la Dra. Raquel Egea. Em van ensenyar les bases de la genètica, fisiologia animal, bioestadística, genètica de poblacions, del desenvolupament i quantitativa, a més de programació i bioinformàtica. Va ser al tercer any de la carrera quan em vaig adonar que programar era ``el meu camí'' i que els experiments al laboratori no eren per a mi. Aquell estiu, em vaig unir al grup del Prof. Francesc Calafell al PRBB per fer un projecte de tres mesos sobre genètica de poblacions i forense, treballant amb R. Tot seguit, vaig fer un intercanvi estudintil d'Erasmus a Dundee durant el quart i darrer any de la carrera. Allà, vaig cursar Estructura Molecular i Interaccions amb el Prof. Bill Hunter i Bioinformàtica Aplicada amb el Dr. David Martin, cosa que va reafirmar la meva passió per la bioinformàtica i va fer créixer el meu interès pel seu vessant estructural. L'any següent, vaig començar un màster en Bioinformàtica per a les Ciències de la Salut a la Universitat Pompeu Fabra (UPF), a Barcelona. Les classes del Dr. Javier García sobre programació en Python i del Prof. Baldo Oliva sobre Bioinformàtica Estructural em van captivar, i per això vaig sol·licitar una estada d'un any al \textit{Barton Group} com a part del màster, que vaig dur a terme de setembre de 2019 a juliol de 2020. Durant aquell període, vaig obtenir una beca EASTBIO DTP per realitzar el meu doctorat sota la supervisió del Prof. Geoff Barton. M'agradaria agrair a totes les persones que van contribuir a la meva formació abans de començar aquest programa de doctorat, que he portat a terme d'octubre de 2020 a març de 2025.

Vull agrair especialment al Prof. Geoff Barton, qui, m'agrada pensar, va veure quelcom en mi al 2019, quan vaig anar a l'entrevista sense ni tan sols saber que STAMP provenia del seu grup, \textit{jajaja}. Moltes gràcies per l'oportunitat d'unir-me al \textit{Barton Group} per aquella estada i, posteriorment, per aquest doctorat. Gràcies per la confiança, la paciència i la flexibilitat davant les diverses circumstàncies que han sorgit durant el transcurs d'aquest projecte. Gràcies per tot el que m'has ensenyat pel que fa a redacció científica, comunicació, anàlisi de dades i altres coneixements generals, com expressions en anglès, aplicacions tecnològiques, com configurar una xarxa domèstica, una estació meteorològica, restaurar finestres o terres de fusta, a més del teu talent musical i les incomptables xerrades i anècdotes sobre com es feia ciència en els \textit{dies foscos}, quan s'havien de dibuixar els gràfics a mà, utilitzar paper carbó i màquines d'escriure, i els manuscrits s'enviaven en paper per correu postal -- sí, postal, no \textit{electrònic}. Sens dubte, has estat el millor director de tesi que podria haver tingut. Gràcies, Geoff.

També vull expressar la meva gratitud a una altra persona que ha tingut una gran influència durant el meu doctorat: el Dr. Stuart MacGowan. Durant la meva estada, en Stuart em va supervisar en la tasca d'aplicar la seva idea de combinar la divergència evolutiva amb la variació genètica en la família de repeticions d'ankirines. Tant durant aquella estada com al llarg del doctorat, en Stuart ha estat un gran mentor i ha compartit consells sobre bones pràctiques en l'anàlisi de dades, programació i reproductibilitat. Admiro el seu entusiasme inesgotable per la ciència, la seva visió, les seves idees de recerca brillants i la seva disponibilitat per ajudar sempre que ho he necessitat. Gràcies, Stuart. Gràcies també als altres membres, passats i presents, del \textit{Barton Group}, DAG i l'equip de Jalview: en Mateusz Warowny, la Renia Correya, els Drs. Ben Soares, Carey Metheringham, James Abbot, Jim Procter, Khadija Jabeen, Marek Gierlinski, Matt Parker, Maxim Tsenkov, Michele Tinti, Pete Thorpe i d'altres. Ha estat un plaer treballar amb tots vosaltres i gaudir de nombroses xerrades, sessions de \textit{Journal Club} i dinars de celebració. Gràcies també al meu supervisor secundari, el Prof. Ulrich Zachariae, pel seu suport i per revisar un dels capítols d'aquesta Tesi, juntament amb els Drs. Ben Soares, Radoslav Krivák, Stuart MacGowan i el Prof. Geoff Barton.

Als meus dos comitès de tesi: els Profs. Daan van Aalten, Satpal Virdee, Vicky Cowling i la Dra. Jorunn Bos per la seva orientació i comentaris durant tot el meu doctorat. Als meus examinadors, els Profs. Alessio Ciuli i David Hoksza, que van acceptar amablement llegir i avaluar aquesta Tesi. Al Prof. Rastko Sknepnek per ser l'organitzador de la meva defensa \textit{viva voce} i a tots els investigadors principals de la divisió de Biologia Computacional: els Profs. Andrei Pisliakov, Ulrich Zachariae i els Drs. Gabriele Schweikert, Hajk Drost i Maxim Igaev per portar la ciència a un nivell excel·lent, elevant així aquesta Divisió, l'Institut i la Universitat a noves altures, convertint-les en una destinació ideal per a la recerca capdavantera. Gràcies també al servei d'informàtica de la universitat pel seu suport a la infraestructura computacional sobre la qual s'ha dut a terme aquest treball.

Als meus antics companys de doctorat en la divisió, els Drs. Callum Ives, Dom Gurvik, Marcus Bage, Maxim Tsenkov i Neil Thomson, i als actuals: l'Alp Tegin, l'Euan MacKay, en Peter Ezzat, la Rosie Gallagher, l'Stefan Manolache, la Tanmayee Narendra i la Yijia Qiang, per fer que el dia a dia i la rutina a l'oficina siguin tan còmodes i agradables. Un agraïment especial per a la Rosie i la Carey pels seus valuosos comentaris, pastissos i sopars; en Maxim per ser un gran supervisor, company i amic, i per haver progressat i crescut junts; en Peter per tantes tardes a l'oficina i seguir endavant plegats. Als companys de l'institut, PiCLS, la meva cohort d'EASTBIO i els estudiants que he supervisat. Als postdoctorands i a la resta de personal de la divisió i l'institut i al meravellós equip administratiu i de secretaria, tant actual com passat: la Sara Salvaterra, la Kirsty Forbes, la Jenna Lyons, la Paige Nell i l'Ulla Gingule, per la seva feina impecable i eficient. A la cap d'estudis de postgrau, la Prof. Carol MacKintosh, sempre disposada a ajudar amb un somriure. A l'administradora d'EASTBIO, la Dra. Maria Filippakopoulou, per la seva amabilitat i excel·lent gestió del programa. I, per descomptat, a la facultat de Biociències de la Universitat de Dundee, a EASTBIO, BBSRC i UKRI per finançar aquesta beca.

El meu agraïment més profund també als increïbles músics i compositors Go Shiina, Hans Zimmer, Hiroyuki Sawano, Kohta Yamamoto, Ramin Djawadi, Sofiane Pamart i Yuki Kajiura. La vostra música, tant bandes sonores originals com clàssica, m'ha acompanyat durant milers d'hores d'apassionant feina d'investigació en els darrers sis anys. M'ha omplert d'emoció, força, determinació, tristesa, esperança i altres sensacions que transmeteu amb el vostre art meravellós. Gràcies per la vostra màgia.

També vull agrair als meus estimats amics de l'escola: l'Ana, l'Anabella, la Cristina, l'Eric, la Lorena i en Paulino, a la meva veïna i amiga Rigo i als meus amics de Dundee: l'Alex, l'Ethan, la Karo, la Katie, en Matt, en Maxim, la Niamh, en Nikita i en Sam. La vostra amistat i suport emocional han estat un pilar indispensable durant aquests últims sis anys. Heu estat els millors amics que algú podria desitjar i m'heu ajudat en els moments més foscos. No podria estar més agraït. Als meus amics de la UPF: l'Aina, l'Altaïr, la Luisa i els Drs. Alexander Gmeiner, Carla Castignani i Pau Badia. Sou grans amics i científics. Hem recorregut un llarg camí des d'aquelles tardes a les runes de la UB i les nostres \textit{meravelloses} classes de disseny web i algorítmica. En un parell de mesos, tots serem doctors. Tinc moltes ganes de celebrar-ho plegats.

Als meus amics de la UAB: en Xavi, la Dra. Nerea Moreno i els meus estimats \textit{Piñas}: els Drs. Ferran Garcia, Guillermo Palou, Núria Serna i Sergio Marco. Ha estat un honor compartir l'última dècada en l'àmbit acadèmic amb vosaltres. Des del primer dia de grau a Bellaterra l'any 2014 fins a l'última defensa de doctorat a Dundee l'any 2025. Hem crescut molt i après molt i ho hem fet junts. No podria haver escollit millors companys de viatge. Estic molt orgullós de vosaltres i us admiro a cadascun. Tinc moltes ganes de veure què ens depara el futur. Gràcies per ser a la meva vida. Us estimo.

Al meu pare, l'Alfonso, que en pau descansi, a la meva mare, l'Alba, i als meus germans, l'Héctor i el Carlos. Gràcies per haver-me educat tal com ho heu fet, inculcant-me valors d'humilitat, respecte, generositat i constància, i per ser sempre al meu costat. A la resta de la meva família: avis, tietes, oncles i cosins, pel vostre suport, estima i pels records inesborrables que creem quan estem junts. Especialment, als meus oncles Alfonso i Paco, a les meves tietes María Elena i Merchi, i als meus cosins l'Alfonso, la Blanca, la Carmen, l'Elena i en Luis per haver estat els millors amfitrions mentre treballava en remot des de casa vostra. Sempre sou al meu cor.

Gràcies a en Darshan, la Hina i la Dhyan per acollir-me a la vostra bonica família i cultura, i també per preocupar-vos per mi i tenir cura de mi. Finalment, però no menys important, vull agrair a la meva meravellosa parella, la Prarthna. Estic molt agraït d'haver-te trobat. Gràcies per aquests tres anys d'amor, suport, consell, paciència, fe, ànim, alegria i felicitat pura que has portat a la meva vida. Desitjo amb totes les meves forces viure el que vingui \textit{junts}. T'estimo amb tot el meu cor.