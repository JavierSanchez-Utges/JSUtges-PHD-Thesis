\documentclass[12pt,a4paper]{report}
\usepackage[utf8]{inputenc}
\usepackage{geometry}
\usepackage{setspace}
\usepackage{graphicx}
\usepackage{fancyhdr}
\usepackage{caption}
\usepackage{capt-of} % to allow for multi-page caption
\usepackage{float} % for the location of figures
\usepackage{amsmath} % for equations
\usepackage{times} % to set font
\usepackage{hyperref} % Add this line to include the hyperref package
\usepackage{nameref}
\usepackage{textgreek}
%\usepackage[compress]{cite} % Add this line to include the cite package with compression
\usepackage{xcolor} % For text colors
\usepackage{xurl}
\usepackage{tocbibind} % for table of contents
\usepackage{tocloft}  % for table of contents
\usepackage{booktabs}  % For professional looking tables 
\usepackage{placeins} % More control on float placing
\usepackage{pdflscape}
\usepackage{booktabs}
\usepackage{longtable} % For specifying cell format
\usepackage{array} % For tables
\usepackage{adjustbox} % For signatures
\usepackage{acronym} % For acronyms
\usepackage{amssymb}% http://ctan.org/pkg/amssymb for check and cross marks
\usepackage{pifont}% http://ctan.org/pkg/pifont for check and cross marks
\usepackage{lipsum}  % For example text
\usepackage{listings} % For code snippet embedding
\usepackage{upquote}  % Ensures straight quotes in verbatim environments
\usepackage{afterpage}
\usepackage{fontspec}

\usepackage[backend=biber, style=numeric-comp, sorting=none, giveninits=true, maxnames=2]{biblatex}
\addbibresource{references_short.bib} % Link to your .bib file

\AtEveryBibitem{
  \ifentrytype{article}{
    \clearfield{url}
  }{}
}

\AtEveryBibitem{
  \ifentrytype{book}{
    \clearfield{url}
  }{}
}

\AtEveryBibitem{
  \ifentrytype{inbook}{
    \clearfield{url}
  }{}
}

\AtEveryBibitem{
  \clearfield{doi}
  \clearfield{issn}
  %\clearfield{pages}
}

\renewbibmacro*{note+pages}{%
  \printfield{note}}

\newcommand*{\mkandothers}{\mkbibemph}
\renewbibmacro*{name:andothers}{%
  \ifboolexpr{
    test {\ifnumequal{\value{listcount}}{\value{liststop}}}
    and
    test \ifmorenames
  }
    {\ifnumgreater{\value{liststop}}{1}
       {\finalandcomma}
       {}%
     \andothersdelim\bibstring[\mkandothers]{andothers}}
    {}}
  
% Customize the bibliography format
\DeclareNameAlias{author}{last-first} % Last name first
\renewcommand{\bibinitdelim}{} % No space between initials
%\renewcommand{\bibinitperiod}{} % No dots after initials
\DeclareFieldFormat[article]{title}{#1} % Title without quotes
\renewbibmacro{in:}{} % Remove "In:" before journal names
\DeclareFieldFormat[article]{volume}{\textbf{#1},} % Bold volume number
\DeclareFieldFormat[article]{pages}{\addcomma\space#1} % Pages after volume and number
\DeclareFieldFormat[article]{number}{} % Issue without parentheses
\DeclareFieldFormat{year}{\addspace(#1)} % Add space before and wrap year in parentheses

\DeclareFieldFormat{labelnumber}{%
  \ifbibliography
    {\iffieldundef{url}
       {#1}                                   % If no 'url' field, just print the number
       {\href{\thefield{url}}{#1}}}           % If 'url' field, hyperlink the number
    {#1}} 

\renewbibmacro*{journal+issuetitle}{%
  \usebibmacro{journal}%
  \setunit{\addspace}% Space after journal name
  \printfield{volume}%
  \setunit{\addcomma\space}% Separator after volume
  \printfield{pages}% Print pages next
  \setunit{\addspace}% Space after pages
  \printfield{year}% Add year in parentheses
}

\DeclareFieldFormat[misc]{title}{#1} % Remove italics from the title for misc
\DeclareFieldFormat[misc]{year}{\addspace(#1)} % Add parentheses around the year
\DeclareFieldFormat[misc]{url}{\url{#1}} % Keep URL as hyperlink without "URL:"

% Apply changes only to misc entries
\AtEveryBibitem{
  \ifentrytype{misc}{%
    \renewcommand{\newunitpunct}{\addspace} % Remove punctuation between fields for misc
    %\renewcommand{\finentrypunct}{}        % Remove final period for misc
  }{}
}

%\setmonofont{Menlo} % Set Menlo as the monospaced font
\setmainfont{Times New Roman} % Set Times New Roman as the main font
%\setmainfont{Verdana}        % Uncomment to use Verdana instead
%\setmainfont{Arial}          % Uncomment to use Arial instead
%\setmainfont{Georgia}        % Uncomment to use Calibri instead
\newfontfamily\menlo{Menlo}

\definecolor{royalblue}{rgb}{0.25, 0.41, 0.88}
\definecolor{forestgreen}{rgb}{0.13, 0.55, 0.13}
\definecolor{firebrick}{rgb}{0.7, 0.13, 0.13}
\definecolor{purple}{rgb}{0.6, 0.18, 0.82} %(0.604, 0.176, 0.812)
\definecolor{orange}{rgb}{0.89, 0.5, 0.02} %(0.89, 0.498, 0.02)
\definecolor{listing-identifier}{HTML}{435489}

\def\beginlstdelim#1#2#3#4%
{
    \def\endlstdelim{\textcolor{#4}{#2}\egroup}  % Define the end delimiter with color
    \bgroup\hspace*{-1.2em}\textcolor{#4}{#1}\color{#3}\aftergroup\endlstdelim  % Use negative hspace to remove leading space
}

\definecolor{kwColor}{RGB}{186, 136, 175} % Pink for keywords
\definecolor{funcColor}{RGB}{108, 148, 189} % Blue for function names
\definecolor{stringColor}{RGB}{54, 187, 56} % Green for strings
\definecolor{orangecolor}{RGB}{245, 113, 87} % Orange-like color
\definecolor{lightorangecolor}{RGB}{255, 170, 67} % light orange color
\definecolor{browncolor}{RGB}{165, 105, 79} % Brown-like color
\definecolor{CBOrange}{RGB}{230,159,0} % Orange for "wrong"
\definecolor{CBBlue}{RGB}{86,180,233}  % Blue for "right"


\lstdefinelanguage{MyPython}{
    keywords=[1]{def, lambda, for, in, if, False, None, return, transform\_dict},
    otherkeywords={!=, ==, <=, >=, =, !, <, >, +=},
    keywords=[2]{!=, ==, <=, >=, =, !, <, >, +=},         % Operators treated as keywords
    keywords=[3]{route, json, apply, items, itertuples, read\_pickle, to\_json, startswith, drop\_duplicates, jsonify, list},
    keywords=[4]{get\_contacts},
    keywordstyle=[1]\color{kwColor},                      % Style for standard Python keywords
    keywordstyle=[2]\color{orangecolor},                         % Style for operators
    keywordstyle=[3]\color{funcColor},
    keywordstyle=[4]\color{teal},
    sensitive=true,
    showstringspaces=false,
    stringstyle=\color{red},
    moredelim = **[is][\beginlstdelim{'}{'}{stringColor}{teal}]{'}{'},
    literate=%{[}{{\textcolor{forestgreen}{[}}}1    % Colorize '[' in orange
             %{]}{{\textcolor{forestgreen}{]}}}1   % Colorize ']' in orange
             {.}{{\textcolor{orange}{.}}}1    % Colorize '[' in orange
    		 {,}{{\textcolor{browncolor}{,}}}1   % Colorize ']' in orange
    		 {:}{{\textcolor{browncolor}{:}}}1
    		 {@}{{\textcolor{teal}{@}}}1
    		 {\&}{{\textcolor{orange}{\&}}}1
}
\lstdefinestyle{mystyle}{
    language=MyPython,
    basicstyle=\menlo\scriptsize,
    commentstyle=\color{codegreen},       % Style for comments
    breakatwhitespace=false,
    breaklines=true,
    captionpos=b,
    aboveskip=1.5\abovecaptionskip,
	belowskip=1.5\belowcaptionskip,
    keepspaces=true,
    numbers=left,
    numbersep=10pt,
    numberstyle=\menlo\tiny,
    showspaces=false,
    showstringspaces=false,
    showtabs=false,
    tabsize=2,
    frame=single,
    framerule=1pt,
    frameround=tttt,
    framesep=10pt,
    xleftmargin=25pt,
    framexleftmargin=14pt,
    framexrightmargin=-11pt,
    linewidth=\textwidth,
    escapeinside={(*}{*)}
}

\lstdefinelanguage{MyJavaScript}{
    keywords=[1]{function, var, let, const, if, else, for, while, do, switch, case, break, continue, new, this, typeof, in, of, class, extends, constructor, static, get, set, import, from, as, export, default, try, catch, finally, throw, null, undefined, true, false, return},
    % Treat JS operators as “keywords” so they can share the same highlighting
    otherkeywords={!=, ==, <=, >=, =, !, <, >, +=, -=, =>, \${, }, `},
    keywords=[2]{canvasId, filename},
    % Function-like or DOM methods might go here
    keywords=[3]{document, createElement, getElementById, toDataURL, click},
    % Add any extra special identifiers if desired
    keywords=[4]{saveImage},
    keywordstyle=[1]\color{kwColor},
    keywordstyle=[2]\color{lightorangecolor},
    keywordstyle=[3]\color{funcColor},
    keywordstyle=[4]\color{teal},
    sensitive=true,
    showstringspaces=false,
    stringstyle=\color{red}\itshape,  % Make strings italic as well
    % Emphasise single-quoted strings similarly
    moredelim = **[is][\beginlstdelim{'}{'}{stringColor}{teal}]{'}{'},
    %moredelim = **[is][\beginlstdelim{`}{`}{stringColor}{teal}]{`}{`},
    literate=
      {.}{{\textcolor{orangecolor}{.}}}1
      {,}{{\textcolor{browncolor}{,}}}1
      {;}{{\textcolor{browncolor}{;}}}1
      {:}{{\textcolor{browncolor}{:}}}1
      {@}{{\textcolor{teal}{@}}}1
      {\&}{{\textcolor{orangecolor}{\&}}}1
      {1}{{\textcolor{lightorangecolor}{1}}}1
      {`}{{\textcolor{teal}{`}}}1
}

\lstdefinelanguage{MyHTML}{
    keywords=[1]{class, style, px, id, solid, black, flex, center, onclick, src, alt, rem, nbsp},
    % Treat JS operators as “keywords” so they can share the same highlighting
    otherkeywords={!=, ==, <=, >=, !, >, +=, -=, =>, \${, }, `},
    keywords=[2]{div, button, img},
    % Function-like or DOM methods might go here
    keywords=[3]{saveAllAssembliesContactData},
    % Add any extra special identifiers if desired
    keywords=[4]{saveImage},
    keywordstyle=[1]\color{kwColor},
    keywordstyle=[2]\color{red},
    keywordstyle=[3]\color{funcColor},
    keywordstyle=[4]\color{teal},
    sensitive=true,
    showstringspaces=false,
    stringstyle=\color{red}\itshape,  % Make strings italic as well
    % Emphasise single-quoted strings similarly
    moredelim = **[is][\beginlstdelim{'}{'}{stringColor}{teal}]{'}{'},
    %moredelim = **[is][\beginlstdelim{`}{`}{stringColor}{teal}]{`}{`},
    literate=
      {.}{{\textcolor{browncolor}{.}}}1
      {,}{{\textcolor{browncolor}{,}}}1
      {;}{{\textcolor{browncolor}{;}}}1
      {:}{{\textcolor{browncolor}{:}}}1
      {(}{{\textcolor{black}{(}}}1
      {)}{{\textcolor{black}{)}}}1
      {=}{{\textcolor{browncolor}{=}}}1
      {<}{{\textcolor{teal}{<}}}1
      {>}{{\textcolor{teal}{>}}}1
      {/}{{\textcolor{teal}{/}}}1
      {\&}{{\textcolor{teal}{\&}}}1
      {1}{{\textcolor{lightorangecolor}{1}}}1
      {5}{{\textcolor{lightorangecolor}{5}}}1
      {0}{{\textcolor{lightorangecolor}{0}}}1
      {8}{{\textcolor{lightorangecolor}{8}}}1
      {`}{{\textcolor{teal}{`}}}1
      {"}{{\textcolor{teal}{"}}}1
      {'}{{\textcolor{teal}{'}}}1
}

\lstdefinelanguage{MyCSS}{
    keywords=[1]{spinner-overlay},
    otherkeywords={!=, ==, <=, >=, !, >, +=, -=, =>, \${, }, `},
    keywordstyle=[1]\color{kwColor},
    sensitive=true,
    showstringspaces=false,
    literate=
      {.}{{\textcolor{teal}{.}}}1
      {,}{{\textcolor{browncolor}{,}}}1
      {;}{{\textcolor{browncolor}{;}}}1
      {:}{{\textcolor{browncolor}{:}}}1
      {\%}{{\textcolor{kwColor}{\%}}}1
      {1}{{\textcolor{lightorangecolor}{1}}}1
      {2}{{\textcolor{lightorangecolor}{2}}}1
      {5}{{\textcolor{lightorangecolor}{5}}}1
      {0}{{\textcolor{lightorangecolor}{0}}}1
      {8}{{\textcolor{lightorangecolor}{8}}}1
}

\setcounter{tocdepth}{4} % this is to include subsubsections
\setcounter{secnumdepth}{3}  % this is to number subsubsections

\renewcommand{\chapterautorefname}{Chapter} % For chapters
\renewcommand{\sectionautorefname}{Section} % For sections
\renewcommand{\subsectionautorefname}{Section} % For subsections
\renewcommand{\subsubsectionautorefname}{Section} % For subsubsections


% Set up the margins
\geometry{
    a4paper,
    left=40mm,
    right=20mm,
    top=35mm,
    bottom=20mm
}

% Set up line spacing
\doublespacing

% Set up the header and footer
\pagestyle{fancy}
\fancyhf{}
\fancyhead[C]{\thepage}

% Redefine the plain pagestyle for chapter pages
\fancypagestyle{plain}{
  \fancyhf{} % clear all header and footer fields
  \fancyhead[C]{\thepage}
}

\setlength{\headheight}{15pt}

% Command to handle smaller fonts for charts, captions, and footnotes
\captionsetup[figure]{
    font={footnotesize,stretch=1.25}, % Change the font size of the caption
    labelfont={bf,color=black}, % Make the label (e.g., "Figure 8") bold
    textfont={color=black}, % Make the text colored
    labelsep=period % Use period after the label (e.g., "Figure 8.")
}

% Command to handle smaller fonts for charts, captions, and footnotes
\captionsetup[table]{
    font={footnotesize,stretch=1.25}, % Change the font size of the caption
    labelfont={bf,color=darkgray}, % Make the label (e.g., "Figure 8") bold
    textfont={color=darkgray}, % Make the text colored
    labelsep=period % Use period after the label (e.g., "Figure 8.")
}

% Command to handle smaller fonts for charts, captions, and footnotes
\captionsetup[lstlisting]{
    font={footnotesize,stretch=1.25}, % Change the font size of the caption
    labelfont={bf,color=darkgray}, % Make the label (e.g., "Figure 8") bold
    textfont={color=darkgray}, % Make the text colored
    labelsep=period % Use period after the label (e.g., "Figure 8.")
}

\newcommand{\cmark}{\textcolor{CBBlue}{\ding{51}}}%
\newcommand{\xmark}{\textcolor{CBOrange}{\ding{55}}}%

% Hyperlinks setup
\hypersetup{
    colorlinks=true,
    linktocpage=true,
    urlcolor=royalblue,
    citecolor=orange,
    breaklinks=true,
    linkcolor=purple,
    %anchorcolor=magenta,
}

% Setting list of equations
\newcommand{\listequationsname}{List of Equations}
\newlistof{myequations}{equ}{\listequationsname}
\newcommand{\myequations}[1]{%
\addcontentsline{equ}{myequations}{\protect\numberline{\theequation}#1}\par}
\setlength{\cftmyequationsnumwidth}{2.3em}  % Adjust the width to accommodate three digits
\setlength{\cftmyequationsindent}{17pt}      % Match the indentation

\newcommand{\autorefpanel}[2]{\hyperref[#1]{\autoref*{#1}#2}}

\newcolumntype{M}[1]{>{\centering\arraybackslash}m{#1}}

\renewcommand{\contentsname}{List of Contents}

\begin{document}

% Roman numerals for the front matter
\pagenumbering{roman}

\begin{titlepage}

	\begin{center}
	
		\includegraphics[width=0.65\textwidth]{other/UoD_logo.pdf}\\[1.5cm]
		
		\setstretch{3}
		{\Huge \textbf{Computational methods for the characterisation and evaluation of protein-ligand binding sites}}\\[1.5cm]
		\setstretch{1.5}
		
		{\LARGE \textbf{Javier Sánchez Utgés}}\\[1.5cm]
		
		\setstretch{2}
		{\Large Thesis submitted for the degree of Doctor of Philosophy}\\[1.5cm]
		%\setstretch{1.5}
		
		{\large Primary Supervisor: Prof Geoffrey J. Barton}\\[0.25cm]

		{\large Secondary Supervisor: Prof Ulrich Zachariae}\\[0.25cm]
		
		%\vfill
		
		{\normalsize Division of Computational Biology}
		
		{\normalsize School of Life Sciences}
		
		{\normalsize University of Dundee}
		
		{\normalsize Dundee, Scotland, UK}
		
		{\normalsize March, 2025}
		
		\vfill
		
		{\small \textcopyright\ Copyright by Javier Sánchez Utgés, 2025}

	\end{center}
	
\end{titlepage}

% Add entries to the ToC for the following sections without creating new sections

\cleardoublepage
\input{Declaration}
\addcontentsline{toc}{chapter}{Declaration}

\cleardoublepage
\input{Statement}
\addcontentsline{toc}{chapter}{Statement}

\cleardoublepage
\input{Acknowledgements}
\addcontentsline{toc}{chapter}{Acknowledgements}

\cleardoublepage
\input{Dedication}
\addcontentsline{toc}{chapter}{Dedication}

\cleardoublepage
\input{Publications}
\addcontentsline{toc}{chapter}{Publications}

\cleardoublepage
\input{Abstract}
\addcontentsline{toc}{chapter}{Abstract}

%\hypersetup{linkcolor=black}
% Add Table of Contents to the ToC
\cleardoublepage
%\addtocontents{toc}{\protect\contentsname} % This forces the change in the ToC
\tableofcontents

% Add List of Figures to the ToC
\cleardoublepage
\listoffigures

% Add List of Tables to the ToC
\cleardoublepage
\listoftables

% Add List of Code Blocks to the ToC
\cleardoublepage
\renewcommand\lstlistingname{Code Block} % Change to "Code Block"
\renewcommand{\lstlistlistingname}{List of \lstlistingname s} % Correct spacing
\lstlistoflistings
\addcontentsline{toc}{chapter}{\lstlistlistingname}

% Add List of Equations to the ToC
\cleardoublepage
\phantomsection
\listofmyequations
\addcontentsline{toc}{chapter}{\listequationsname}

% Add List of Abbreviations to the ToC
\cleardoublepage
\chapter*{List of Abbreviations}

\begin{longtable}[l]{@{}p{2.5cm}p{12cm}@{}}
\textbf{017} & Darunavir \\
\textbf{3D} & Three-dimensional \\
\textbf{8PR} & Paroxetine \\
\textbf{AA} & Amino acid \\
\textbf{ADMET} & Absorption, distribution, metabolism, excretion, toxicity \\
\textbf{ADP} & Adenosine-5'-diphosphate \\
\textbf{AMI} & Adjusted mutual information \\
\textbf{ANN} & Artificial neural network \\
\textbf{AP} & Average precision \\
\textbf{ARI} & Adjusted Rand index \\
\textbf{ASA} & Accessible surface area \\
\textbf{ATP} & Adenosine-5’-triphosphate \\
\textbf{AUC} & Area under the curve \\
\textbf{BGC} & Glucose \\
\textbf{BLOSUM} & Block substitution matrix \\
\textbf{CC9} & Curcumin \\
\textbf{CCD} & Chemical component dictionary \\
\textbf{CHI} & Calinski-Harabasz index \\
\textbf{CI} & Confidence interval \\
\textbf{CLR} & Cholesterol \\
\textbf{CM} & Centre of mass \\
\textbf{CMD} & Conserved and missense depleted \\
\textbf{CME} & Conserved and missense enriched \\
\textbf{CNN} & Convolutional neural network \\
\textbf{Cryo-EM} & Cryogenic electron microscopy \\
\textbf{DBI} & Davies-Bouldin index \\
\textbf{DNA} & Deoxyribonucleic acid \\
\textbf{DBSCAN} & Density-based spatial clustering of applications with noise \\
\textbf{DCA} & Distance to closest ligand atom \\
\textbf{DCC} & Distance centroid to centroid \\
\textbf{DNN} & Deep neural network \\
\textbf{DRN} & Deep residual network \\
\textbf{DWF} & Debye-Waller factor \\
\textbf{EDO} & Etylene glycol \\
\textbf{EGNN} & Equivariant graph neural network \\
\textbf{EMA} & European Medicines Agency \\
\textbf{ENA} & European Nucleotide Archive \\
\textbf{FAD} & Flavin-adenine dinucleotide \\
\textbf{FBDD} & Fragment-based drug discovery \\
\textbf{FDA} & Food and Drug Administration \\
\textbf{FDR} & False discovery rate \\
\textbf{FMN} & Flavin mononucleotide \\
\textbf{FN} & False negative \\
\textbf{FP} & False positive \\
\textbf{FS} & Fragment screening \\
\textbf{FUC} & Fucose \\
\textbf{GAL} & Galactose \\
\textbf{GAT} & Graph attention network \\
\textbf{GNN} & Graph neural network \\
\textbf{gnomAD} & Genome aggregation database \\
\textbf{GOL} & Glycerol \\
\textbf{GPCR} & G-protein coupled receptor \\
\textbf{GSH} & Glutathione \\
\textbf{GTP} & Guanosine-5′-triphosphate \\
\textbf{HEM} & Haem \\
\textbf{HOH} & Water \\
\textbf{HPC} & High performance computing \\
\textbf{HTS} & High-throughput screening \\
\textbf{IDR} & Intrinsically disordered region \\
\textbf{IOU} & Intersection over union \\
\textbf{IQR} & Interquartile range \\
\textbf{\textit{I\textsubscript{rel}}} & Relative intersection \\
\textbf{JFP} & N-(4-methyl-1,3-thiazol-2-yl)propanamide \\
\textbf{JI} & Jaccard index \\
\textbf{LBVS} & Ligand-based virtual screening \\
\textbf{LGBM} & Light gradient boosting machine \\
\textbf{LJ} & Lennard-Jones \\
\textbf{MAN} & Mannose \\
\textbf{MCC} & Matthews correlation coefficient \\
\textbf{MCD} & Minimum centroid distance \\
\textbf{MDS} & Multi dimensional scaling \\
\textbf{MES} & Missense enrichment score \\
\textbf{MHRA} & Medicines and Healthcare products Regulatory Agency \\
\textbf{ML} & Machine learning \\
\textbf{MLP} & Multi layer perceptron \\
\textbf{MOAD} & Mother Of All Databases \\
\textbf{mRNA} & Messenger RNA \\
\textbf{MRO} & Maximum residue overlap \\
\textbf{MTD} & Maximum tolerated dose \\
\textbf{NAD} & Nicotinamide-adenine-dinucleotide \\
\textbf{NAP} & Nicotinamide-adenine-dinucleotide phosphate \\
\textbf{NME} & New molecular entity \\
\textbf{\textit{N}-mer} & Protein peptide of \textit{N} amino acids \\
\textbf{NMR} & Nuclear magnetic resonance \\
\textbf{NOAEL} & No observed adverse effect level \\
\textbf{NR} & Non-redundant \\
\textbf{OR} & Odds ratio \\
\textbf{PAM} & Point accepted mutation \\
\textbf{PanDDA} & Pan-dataset density analysis \\
\textbf{PCA} & Principal component analysis \\
\textbf{PD} & Pharmacodynamics \\ 
\textbf{PDB} & Protein Data Bank \\
\textbf{PDBe} & Protein Data Bank Europe \\
\textbf{PDBe-KB} & Protein Data Bank Europe Knowledge Base \\
\textbf{PK} & Pharmacokinetics \\
\textbf{PLP} & Vitamin B6 phosphate \\
\textbf{PR} & Precision-recall curve \\
\textbf{QSAR} & Quantitative structure-activity relationship \\
\textbf{\textit{R\textsubscript{g}}} & Radius of gyration \\
\textbf{ROC} & Receiver operating characteristic \\
\textbf{RMSD} & Root mean square deviation \\
\textbf{RNA} & Ribonucleic acid \\
\textbf{RRO} & Relative residue overlap \\
\textbf{RVO} & Relative volume overlap \\
\textbf{RSA} & Relative solvent accessibility \\
\textbf{SAH} & S-adenosyl-L-homocysteine \\
\textbf{SAS} & Solvent accessible surface \\
\textbf{SASA} & Solvent accessible surface area \\
\textbf{SBVS} & Structure-based virtual screening \\
\textbf{SD} & Standard deviation \\
\textbf{SDP} & Specificity-determining position \\
\textbf{SES} & Solvent-excluded surface \\
\textbf{SS} & Sum of squares \\
\textbf{TN} & True negative \\
\textbf{TP} & True positive \\
\textbf{TRS} & Tris buffer \\
\textbf{\textit{U\textsubscript{D}}} & Distance U \\
\textbf{\textit{U\textsubscript{max}}} & Maximum U statistic \\
\textbf{UMD} & Unconserved and missense depleted \\
\textbf{UME} & Unconserved and missense enriched \\
\textbf{UPGMA} & Unweighted pair-group method with arithmetic mean \\
\textbf{\textit{U\textsubscript{rel}}} & Relative U statistic \\
\textbf{VDW} & Van der Waals \\
\textbf{VN} & Virtual node \\
\textbf{\textit{V\textsubscript{R}}} & Volume ratio \\
\textbf{VS} & Virtual screening \\
\textbf{VUS} & Variant of uncertain significance \\
\textbf{XRC} & X-ray crystallography \\
\textbf{XYP} & Xylose \\

\end{longtable}

%\hypersetup{linkcolor=purple}
\addcontentsline{toc}{chapter}{List of Abbreviations}

% Clear the page and start Arabic numerals for the main matter
\cleardoublepage
\pagenumbering{arabic}

% Main matter chapters

\chapter{Introduction}

\section*{Preface}

%This Chapter describes the basic concepts and methodologies necessary to understand this Thesis, as well as the state of the art within the field. Additionally an overview of the results of this Thesis is provided within its context.

This Chapter introduces the fundamental concepts and methodologies essential for understanding this Thesis, along with a review of the state of the art in the field. Additionally, it provides an overview of the Thesis results in their broader scientific context.

\section*{The purpose of life}

%\begin{quote}
%``Long, long ago there was a time when nothing but mere matter existed in this world. In the teeming ooze, forms of a certain \textit{something} appeared, disappeared, and appeared again, and one of them eventually survived. We know it as \textit{life}. The reason that life ultimately survived was because it was in its nature \textit{to multiply}. Life took new forms in order to multiply, adapting to every kind of environment, and culminating in \textit{us} today. Greater numbers, greater diversity, greater abundance. This is why we say that the purpose of \textit{life} is \textit{to multiply}.'' -- Hajime Isayama, \textit{Attack on Titan} \cite{ISAYAMA_2021}
%\end{quote}

\begin{quote}
\textit{``Long, long ago there was a time when nothing but mere matter existed in this world. In the teeming ooze, forms of a certain something appeared, disappeared, and appeared again, and one of them eventually survived. We know it as life. The reason that life ultimately survived was because it was in its nature to multiply. Life took new forms in order to multiply, adapting to every kind of environment, and culminating in us today. Greater numbers, greater diversity, greater abundance. This is why we say that the purpose of life is to multiply.''} -- Hajime Isayama, \textit{Attack on Titan}  \cite{ISAYAMA_2021}
\end{quote}

\section{Central dogma of molecular biology}

It is estimated based on geological \cite{SCHIDLOWSKI_1979_LIFE}, fossil \cite{SCHOPF_2007_LIFE} and phylogenetic \cite{BETTS_2018_LIFE} analyses that the origin of life in our planet Earth dates back to 3.7-4.0 billion years ago. Since then, \textit{life} has not just survived, but adapted and evolved to give raise to a gargantuan estimated biodiversity of 8.7 million eukaryotic species \cite{MORA_2011_SPECIES} and upward of 1 trillion microbial species \cite{HUG_2016_SPECIES, LOCEY_2016_SPECIES} with the vast majority of these still to be described \cite{COSTELLO_2013_SPECIES}. Despite the immense variation across species in terms of reproductive strategies, morphological, metabolic, behavioural traits or ecological niche, there is one thing \textit{all} species have in common: nucleic acids \cite{KOONIN_2011_LIFE}. All living species rely on nucleic acids, mostly deoxyribonucleic acid (DNA), except for some viruses \cite{KOONIN_2006_VIRUS} and viroids \cite{NAVARRO_2021_VIROIDS} that use ribonucleic acid (RNA), to store their genetic information. This information flows sequentially from DNA to RNA through the process of transcription and from RNA to protein through translation. This flow of molecular information is known as the \textit{Central Dogma of Molecular Biology} \cite{CRICK_1958_DOGMA, CRICK_1970_DOGMA}.

\section{The genetic code}

DNA is a polymer composed of two polynucleotide chains that coil around each other to form a double helix \cite{WATSON_1953_DNA}. These polymer chains are formed by simpler units called nucleotides. Each nucleotide presents a common scaffold formed of a deoxyribose sugar and a phosphate and a variable nitrogen-containing nucleobase. There are four different bases: adenine (A), thymine (T), cytosine (C) and guanine (G). The information stored in DNA gets transferred to RNA through the process of transcription. RNA tends to adopt a single-stranded conformation and is also formed by nucleotides. These nucleotides differ from DNA in that they present a ribose sugar, instead of deoxyribose, and an alternative uracil (U) nucleobase instead of thymine \cite{LEVENE_1909_NUCLEICS}.

Messenger RNA (mRNA) corresponds to the sequence of a gene and is read by the ribosomal macromolecular machinery in the process of protein synthesis, or translation. In this process, a peptide chain is formed by linking amino acids in the order specified by the codons in the mRNA \cite{CRICK_1957_CODE}. A codon is a set of three nucleotides that corresponds to one of the twenty canonical amino acids. \autoref{fig:genetic_code} illustrates the equivalence between these codons and the amino acid they encode, also known as the \textit{Genetic Code} \cite{GAMOW_1954_CODE}.

\begin{figure}[htb!]
    \centering
    \includegraphics[width=\textwidth]{figures/ch_INTRO/PNG/genetic_code_CROP_JSU.png}
    \caption[Genetic code]{\textbf{Genetic code.} The genetic code illustrates the 64 codons resulting from mRNA used to synthetise proteins during translation. The chemical structure of the amino acid side chains is found next to the amino acid name. Colour indicates basic amino acids (lavender), acidic (pink), polar (green) and nonpolar (yellow). STOP codons are coloured in white. Figure adapted from Wikipedia: the free encyclopedia \cite{genetic_code_image}.}
    \label{fig:genetic_code}
\end{figure}

\section{Proteins}

Proteins are molecular machines that are involved in virtually all cellular processes including cell division, immune response or metabolism. They result from the process of translation of mRNA. Proteins are natural polymers formed of smaller monomers, called amino acids, linked to each other through peptide bonds. Proteins do not carry their function in isolation, but usually interact with other proteins, nucleic acids, ions or small molecules. Chemical compounds can be used as drugs to modulate or inhibit protein function. The amount, localisation, state and interaction with other molecules of proteins is strictly controlled by complex gene regulation networks, signalling cascades, and environment-dependent conformational changes. Any of these mechanisms failing to work correctly can result in under- or overexpression, hypo- or hypermorphism, truncated, mutated or unfolded proteins, which eventually can lead to disease. %Identifying or predicting the location of where these molecules bind to proteins and the way in which they do so is therefore critical to understand better and modulate protein function.

%\vspace{-13pt} % Adjust this value as needed

\subsection{Amino acid structure}

There are twenty canonical amino acids that are found in all protein sequences. They receive this name because of their chemical structure, which includes both an amino and carboxylic acid functional groups. \autoref{fig:amino_acid} shows the general structure of an amino acid. A carbon atom is found in the centre which binds covalently to four different groups. This carbon is known as α-carbon (CA) and is attached to the amino group (--NH2), the carboxyl group (--COOH), a hydrogen atom (H) and a side chain (R) that differs across the twenty amino acid residues.

\begin{figure}[htbp!]
    \centering
    \includegraphics[width=0.50\textwidth]{figures/ch_INTRO/PNG/amino_acid.png}
    \caption[Amino acid structure]{\textbf{Amino acid structure.} All twenty amino acids share this common structure formed by the α-carbon (CA) chemically attached to the amino (NH2) and carboxyl (COOH) groups, a hydrogen atom (H) and a side chain (R). The side chain is different and defines the amino acids.}
    \label{fig:amino_acid}
\end{figure}

%\vspace{-13pt} % Adjust this value as needed
%\vspace{-13pt} % Adjust this value as needed

\subsection{Amino acid properties}

The different side chains of amino acids confer them distinct physicochemical properties \cite{SNEATH_1966_PROPERTIES}. \autoref{fig:properties} illustrates the ten main properties, the relationship between them and which amino acids present them \cite{TAYLOR_1986_PROPERTIES}. The three most important properties setting amino acids apart are hydrophobicity, polarity and size. Hydrophobic residues present side chains that are less soluble in water and therefore tend to be located in the interior protein core, whereas hydrophilic residues are present on the surface \cite{CHOTHIA_1976_BURIEDNESS}. Polar side chains contain electronegative atoms like nitrogen (N) or oxygen (O) and favour interaction with water and other polar molecules. Size is also relevant as there is a large difference in volume between the twenty amino acids, ranging from 60 \AA{}\textsuperscript{3} (Glycine) to $>$200 \AA{}\textsuperscript{3} (Tryptophan). Within the categories defined by these three properties, other subsets can be found as aliphatic, aromatic, positively and negatively charged, tiny or proline. Proline has its own category because of its unique cyclical side chain which links back to the backbone \cite{ZVELEBIL_1987_PREDICTION}.

\begin{figure}[htb!]
    \centering
    \includegraphics[width=0.90\textwidth]{figures/ch_INTRO/PNG/properties.png}
    \caption[Amino acid properties]{\textbf{Amino acid properties.} Taylor Venn diagram illustrating the different physicochemical properties of the twenty proteinogenic amino acids. Figure adapted from the Jalview website \cite{JALVIEW}, which in turn adapted from Livingstone and Barton \cite{LIVINGSTONE_1993_MSA}.}
    \label{fig:properties}
\end{figure}

The physicochemical properties of amino acids are crucial to understand the arrangement of protein atoms in three-dimensional (3D) space. Moreover, the conservation of these properties across evolutionarily related proteins is the basis for sequence comparison, analysis as well as protein structure prediction \cite{CHOTHIA_1986_CONSERVATION}.

\subsection{Substitution matrices}

Similar or identical protein sequences carrying out related functions and displaying a comparable 3D structure can be found within a genome (\textit{paralogous} sequences) and across species (\textit{orthologous} sequences). These proteins are evolutionarily related, i.e., \textit{homologous}, and their origin can be traced back in time to a common ancestor. The comparative analysis of related sequences within protein families provides insight into its evolutionary history \cite{BARTON_1990_MSA}. Amino acid substitution matrices can be calculated by quantifying the differences between closely related sequences. These matrices indicate the likelihood of observing transitions at a given protein position between the different amino acids. Transitions between amino acids with similar physicochemical properties, e.g., aspartate $\rightarrow$ glutamate, are less likely to alter the protein structure and are therefore observed with higher frequency. The Point Accepted Mutation (PAM) \cite{DAYHOFF_1978_PAM} and Block Substitution Matrix (BLOSUM) \cite{HENIKOFF_1992_BLOSUM} are some of the more relevant substitution matrices and serve as a scoring function for constructing alignments of multiple sequences (MSA) that are related in evolution \cite{BARTON_1987_MSA}.

\subsection{Multiple sequence alignment}

In a multiple sequence alignment, sequences are arranged so that homologous positions -- those sharing a common ancestry -- appear in the same column across proteins \cite{NEEDLEMAN_1970_MSA}. Through time, sequences diverge and might undergo point mutations, insertions or deletions. To handle this, aligners introduce \textit{gaps} (--) \cite{SMITH_1981_MSA}. The distribution of amino acid residues across columns in an alignment reveals conservation patterns. These patterns emerge when residues or their physicochemical properties remain invariant across sequences. Columns that present little or no variation are called \textit{conserved} whereas columns that present a variety of amino acids with different properties are called \textit{unconserved} or \textit{divergent} \cite{LIVINGSTONE_1993_CONS}. Many methods for the alignment of multiple sequences have been developed with Clustal \cite{HIGGINS_1988_CLUSTAL, HIGGINS_1992_CLUSTALV, THOMPSON_1994_CLUSTALW, JEANMOUGIN_1998_CLUSTALX, SIEVERS_2011_CLUSTALO}, MAFFT \cite{KATOH_2002_MAFFT, KATOH_2008_MAFFT, KATOH_2013_MAFFT} and MUSCLE \cite{EDGAR_2004_MUSCLE, EDGAR_2022_MUSCLE5} among the most widely used. \autoref{fig:MSA} illustrates an MSA visualised with Jalview, an interactive application for the alignment, editing and integrative analysis of sequence alignments \cite{WATERHOUSE_2009_JALVIEW}.

\begin{figure}[htb!]
    \centering
    \includegraphics[width=0.60\textwidth]{figures/ch_INTRO/PNG/anks_msa_short.png}
    \caption[Multiple sequence alignment]{\textbf{Multiple sequence alignment.} Fragment of an alignment of 7407 ankyrin repeat protein sequences (\href{https://www.ebi.ac.uk/interpro/entry/InterPro/IPR002110/}{IPR002110}) built by Utgés \textit{et al.} \cite{UTGES_2021_ANKS}. The twenty sequences displayed on this figure all have the same length and so no gaps are observed. Alignment columns are coloured in the ClustalX colour scheme \cite{JEANMOUGIN_1998_CLUSTALX} with hydrophobic residues in blue, polar in green, glycine in orange, proline in yellow and aromatic in cyan. Additionally, columns are shaded by their conservation, so columns in darker shades are conserved through the alignment whilst those in lighter ones are divergent. The consensus sequence recapitulates the most common residues at each position. Secondary structure assignment describes the two α-helices located at columns 5-11 and 15-23 of the MSA. Strongly conserved residues include the TPLH motif at positions 4-7 as well as Ala9, Ala10, Leu21 and Leu22 which form a series of hydrophobic interactions stabilising the helices within individual repeats as well as across them. Sequence IDs are UniProt accessions numbers. Figure obtained with Jalview \cite{WATERHOUSE_2009_JALVIEW}.}
    \label{fig:MSA}
\end{figure}

\subsection{Amino acid conservation}

Amino acid conservation in MSAs is evidence of evolutionary constraint \cite{DAYHOFF_1978_PAM}. Throughout evolution, conserved positions have remained fixed due to their functional or structural relevance, while divergent positions accumulate substitutions resulting in variability in amino acid residues across proteins within the same family \cite{ZUCKERKANDL_1965_DIVERGENCE}. Several scores exploring different approaches to quantify conservation have been developed through the years \cite{VALDAR_2002_SCORES}. Some of these scores consider amino acids as symbols and use their relative frequencies \cite{WU_1970_SCORE, JORES_1990_SCORE, LOCKLESS_1999_SCORE}, or entropy \cite{SANDER_1991_SCORE, SHENKIN_1991_SCORE, GERSTEIN_1995_SCORE} to score conservation. Others focus on their stereochemical properties \cite{TAYLOR_1986_PROPERTIES, ZVELEBIL_1987_PREDICTION}, use mutation data \cite{KARLIN_1996_SCORE, THOMPSON_1997_SCORE, LANDGRAF_1999_SCORE, PILPEL_1999_SCORE, ARMON_2001_SCORE, VALDAR_2001_SCORE} or combine amino acid properties and symbol entropy \cite{WILLIAMSON_1995_SCORE, MIRNY_1999_SCORE}.

The score developed by Shenkin \textit{et al.} \cite{SHENKIN_1991_SCORE} is based on Shannon's entropy ($S$) which is calculated with \autoref{eq:entropy_shannon2} \cite{SHANNON_1948_ENTROPY}. The proportion within an alignment column of each amino acid $i$ of the $K$ = 20 naturally occurring amino acids is denoted by $p_i$. The Shenkin score, $V_{Shenkin}$, described in \autoref{eq:shenkin}, measures divergence and increases as amino acid variability grows within a column. In a fully conserved position, where all amino acids are the same, entropy is minimum ($S$ = 0) and so is divergence ($V_{Shenkin}$ = 6). Conversely, in a fully variable position, where all amino acids are equally represented, entropy is maximum ($S \approx$ 4.32) and so is divergence ($V_{Shenkin}$ = 120). Utgés \textit{et al.} \cite{UTGES_2021_ANKS} defined a version of this score, $N_{Shenkin}$ (\autoref{eq:shenkin_norm}), which normalises the original score by the minimum and maximum scores within the alignment and ranges 0-100.

\begin{equation}
S = - \sum_{i=1}^{K} p_i \log_2(p_i)
\label{eq:entropy_shannon2}
\end{equation}
\myequations{Shannon's entropy}

\vspace{-13pt} % Adjust this value as needed
\vspace{-13pt} % Adjust this value as needed

\begin{equation}
V_{Shenkin} = 2^S \times 6
\label{eq:shenkin}
\end{equation}
\myequations{Shenkin divergence score}

\vspace{-13pt} % Adjust this value as needed
\vspace{-6pt} % Adjust this value as needed

\begin{equation}
N_{Shenkin} = \frac{V_{Shenkin} - V_{Shenkin_{\text{min}}}}{V_{Shenkin_{\text{max}}} - V_{Shenkin_{\text{min}}}}
\label{eq:shenkin_norm}
\end{equation}
\myequations{Normalised Shenkin divergence score}

Beyond illuminating the evolutionary history of protein sequences, amino acid conservation patterns derived from alignments have been used to successfully predict a variety of features such as secondary structure elements \cite{ROST_1993_SSPRED}, solvent accessibility \cite{ROST_1994_RSAPRED}, protein-protein interfaces \cite{LICHTARGE_1996_PPIs}, protein-ligand binding sites \cite{GLASER_2006_PREDICTION} and inter-residue contacts \cite{MARKS_2011_CONS}, which recently has lead to a breakthrough in the prediction of protein 3D structure \cite{JUMPER_2021_ALPHAFOLD}. There is immense power in the analysis of amino acid conservation, and in this Thesis, the normalised Shenkin divergence score is employed in a systematic manner to rank ligand binding sites on likelihood of function and highlight key residues within them. 

\subsection{Protein structure}

The arrangement of protein atoms in three-dimensional space is known as protein structure. Protein structure can be defined at four different levels (\autoref{fig:protein_structure}). The primary structure of a protein corresponds to the sequence of amino acids forming the polypeptide chain from the first residue in the amino terminus (N-term) to the last one in the carboxyl terminus (C-term) (\autorefpanel{fig:protein_structure}{ A}). Protein residues adopt local sub-structures by the formation of hydrogen bond interactions between the residue backbone atoms. These local conformations are referred to as secondary structure (\autorefpanel{fig:protein_structure}{ B}). There are two main types of secondary structure: α-helix and β-sheets \cite{PAULING_1951_SS}. The absence of any of these structures could be defined as a third type of secondary structure named coil or loop. Tertiary structure is the 3D arrangement of secondary structure elements within a single chain, resulting from the process of protein folding (\autorefpanel{fig:protein_structure}{ C}). Tertiary structure is defined by the burial of hydrophobic residues in the protein core and hydrogen bonds, salt bridges and disulfide bonds ensuring a tight packing of residue side chains. Finally, quaternary structure results from the aggregation of two or more individual protein chains that come together to form the functional unit of the protein or multimer (\autorefpanel{fig:protein_structure}{ D}). These monomers are held together by the same interactions that stabilise tertiary structure. Quaternary structure can present different architectures depending on the number of copies involved (e.g., dimer for two, trimer for three, tetramer for four) and whether these copies are from the same sequence (homomers) or different ones (heteromers).

\begin{figure}[htb!]
    \centering
    \includegraphics[width=\textwidth]{figures/ch_INTRO/PNG/protein_structure.png}
    \caption[Protein structure]{\textbf{Protein structure.} Four levels of protein structure: primary (\textbf{A}); secondary (\textbf{B}); tertiary (\textbf{C}); quaternary (\textbf{D}). Blue dashed cylinders illustrate hydrogen bonds holding together the secondary structures α-helix and β-sheet. Example is PDB: \href{https://www.ebi.ac.uk/pdbe/entry/pdb/8dhv}{8DHV} \cite{LIETZAN_2023_BETAGLUCO} of β-glucuronidase of \textit{Treponema lecithinolyticum} (\href{https://www.uniprot.org/uniprotkb/A0AA82WPE8/entry}{A0AA82WPE8}). Structure visualisation with ChimeraX \cite{PETTERSEN_2021_CHIMERAX}.}
    \label{fig:protein_structure}
\end{figure}

\vspace{-13pt} % Adjust this value as needed
\vspace{-6pt} % Adjust this value as needed

\subsection{Protein structure determination}

Protein structure determination is the process of deciphering the arrangement of protein atoms in three-dimensional space. In 1958 Kendrew \textit{et al.} \cite{KENDREW_1958_MYOGLOBIN} resolved the first protein structure for sperm whale myoglobin (\href{https://www.uniprot.org/uniprotkb/P02185/entry}{P02185}) using X-ray crystallography \cite{BERNAL_1934_XRAY}. Apart from X-ray crystallography, nuclear magnetic resonance spectroscopy and more recently cryogenic electron microscopy have also been used extensively for 3D structure determination.

\subsubsection{X-ray crystallography}

The first step to resolve a protein structure using X-ray crystallography (XRC) is to obtain the protein crystal. A protein crystal is a highly ordered structure in which protein atoms are arranged in a repeating uniformly distributed pattern known as crystal lattice. Crystallising a protein can be very time consuming since the optimal conditions vary between proteins with different size, solubility or isoelectric point. Once the crystal is obtained, it is placed on an X-ray beam which will scatter the electron clouds of the atoms generating a diffraction pattern. This pattern can then be transformed to generate an electron density map revealing the position of atoms within the crystal \cite{FRIEDRICH_1913_XRAY, BRAGG_1913_XRAY}. Following this, an atomic model is fitted into the electron density map to interpret the molecular structure. X-ray crystallography provides high-resolution structural information and is accordingly the most widely used method to determine protein structure accounting for $\approx$83\% of structures deposited in the Protein Data Bank (PDB) \cite{BERMAN_2000_PDB}.

\subsubsection{Nuclear magnetic resonance spectroscopy}

Nuclear magnetic resonance spectroscopy (NMR) is a powerful technique to determine 3D structure in solution. It was first used in 1984 by Williamson \textit{et al.} \cite{WILLIAMSON_1985_NMR} to determine the structure of proteinase inhibitor IIA from bull (\href{https://www.uniprot.org/uniprotkb/P01001/entry}{P01001}). NMR does not require a protein crystal, but instead a high concentration of protein in aqueous solution \cite{WUTHRICH_1982_NMR}. NMR relies on the magnetic moment or spin of certain isotopes such as \textsuperscript{1}H, \textsuperscript{13}C or \textsuperscript{15}N. In the presence of a magnetic field, the application of radio frequency pulses to these isotopes results in a chemical shift that is diagnostic of their local electronic environment and recorded as the NMR spectrum. The chemical shifts in the spectrum are assigned to individual atoms and distance, angle and orientation restraints are derived. This information is integrated to calculate a model that is then refined to yield the final structure \cite{WUTHRICH_1984_NMR}. NMR is ideal to study the dynamics of proteins or other molecules in solution. However, this technique often results in lower structure resolution and its use is limited to smaller proteins as the spectra get more complex with increasing protein size \cite{EMWAS_2015_NMR}.

\subsubsection{Cryogenic electron microscopy}

The use of electron microscopy to determine protein structure dates back to 1975 \cite{HENDERSON_1975_EM} but modern cryogenic electron microscopy (Cryo-EM) was not used to resolve a protein structure until 1990 when Henderson \textit{et al.} \cite{HENDERSON_1990_CRYOEM} determined the structure of \textit{Halobacterium halobium} bacteriorhodopsin (\href{https://www.uniprot.org/uniprotkb/P02945/entry}{P02945}). In Cryo-EM, proteins are rapidly frozen to very low temperatures to preserve their native state. The frozen sample is then put under an electron microscope which will generate a set of two-dimensional projections from the electron beams. These projections are later integrated into a 3D model. Cryo-EM tends to provide lower resolution than XRC or NMR. However, it is the only method that can determine the structure of large macromolecular complexes such as the spliceosome \cite{CHUANGYE_2016_SPLICEOSOME} or the nucleopore \cite{KOSINSKI_2016_NUCLEOPORE}. This resolution limitation was breached in the last decade when Bartesaghi \textit{et al.} \cite{BARTESAGHI_2014_CRYOEM} reached a resolution of 3.2 \AA{} for \textit{Escherichia coli} β-galactosidase (\href{https://www.uniprot.org/uniprotkb/P00722/entry}{P00722}). While XRC has decades of advantage over Cryo-EM in terms of deposited structures, due to the rapid advances in the latter method, it is projected that the number of depositions between these two methods will coalesce by 2035 \cite{CHIU_2021_CRYOEM}.

\subsection{Protein structure characterisation}

Beyond the determination of the arrangement of atoms in 3D space, proteins can be characterised structurally in multiple ways that offer insight into their physicochemical properties, function, stability, dynamics and interaction with other molecules. These features can then be mapped onto the molecular surface of the proteins and visually analysed (\autoref{fig:protein_features}).

\begin{figure}[htb!]
    \centering
    \includegraphics[width=\textwidth]{figures/ch_INTRO/PNG/protein_features.png}
    \caption[Protein structure features]{\textbf{Protein structure features.} Protein structure features exemplified on PDB: \href{https://www.ebi.ac.uk/pdbe/entry/pdb/4c38}{4C38} \cite{COUTY_2013_ONCO} of bovine cAMP-dependent protein kinase catalytic subunit alpha (\href{https://www.uniprot.org/uniprotkb/P00517/entry}{P00517}). \textbf{(A)} Atomic displacement measured by Debye-Waller factor (DWF); \textbf{(B)} Hydrophobicity measured by molecular lipophilicity potential (MLP); \textbf{(C)} Charge measured by Coulombic electrostatic potential (ESP); \textbf{(D)} Accessibility measured by relative solvent accessibility (RSA); \textbf{(E)} Ligandability as measured by P2Rank's predicted ligandability score (LS). Structure visualisation with ChimeraX \cite{PETTERSEN_2021_CHIMERAX}.}
    \label{fig:protein_features}
\end{figure}

\subsubsection{Flexibility}

The Debye-Waller factor (DWF), or B-factor, measures the attenuation of X-ray scattering caused by thermal motion \cite{DEBYE_1913_BFACTOR, WALLER_1923_BFACTOR}. This attenuation is a decrease of intensity in diffraction caused by disorder. This disorder can be dynamic and result from the temperature-dependent vibration of the atoms, or static \cite{SUN_2019_BFACTOR}. Accordingly, low values of B-factor indicate rigid or well-ordered protein regions, while high values can identify flexible or dynamic regions in proteins such as loops or binding sites, as well as intrinsically disordered regions (IDR) (\autorefpanel{fig:protein_features}{ A}). IDRs are protein regions that lack a determined three-dimensional structure and might change conformation depending on their environmental context.

\subsubsection{Hydrophobicity}

The molecular lipophilicity potential (MLP) represents the spatial distribution of lipophilicity across a molecule’s surface, providing insight into the hydrophobicity or hydrophilicity of its different regions \cite{BROTO_1984_MLP, LAGUERRE_1997_MLP}. High MLP values correspond to lipophilic (hydrophobic) areas and lower MLP values correlate to less lipophilic (more hydrophilic) regions (\autorefpanel{fig:protein_features}{ B}). The analysis of protein lipophilicity is relevant for the identification of hydrophobic pockets where lipophilic ligands are likely to bind, allosteric sites, large hydrophobic patches prone to protein aggregation, as well as for protein and enzyme engineering \cite{EFREMOV_2007_MLP}. Analysing the lipophilicity of small molecules is also important for optimising ligand design and improving drug absorption, permeability or solubility \cite{GAILLARD_1994_MLP}.

\subsubsection{Charge}

Amino acids with charged side chains, e.g., Asp, Glu, His, Lys and Arg, play an important role in the electrostatic potential of a protein, which can be calculated using Coulomb's law \cite{COULOMB_1785_LAW}. Electrostatic potential plays a pivotal role in the field of protein analysis as it underpins processes such as protein folding, enzyme catalysis, molecular recognition and interaction with proteins, nucleic acids and small molecules, or ligands \cite{ZHOU_2018_ESP} (\autorefpanel{fig:protein_features}{ C}). Because of this, protein electrostatics analysis and fine-tuning have applications in protein design \cite{GORHAM_2011_ESP}, protein-ligand binding affinity \cite{KUKIC_2010_ELECTROSTATICS} and biocatalysis optimisation \cite{VASCON_2020_ESP}.

\subsubsection{Accessibility}

The surface area of a biomolecule that is accessible to solvent is known as accessible surface area (ASA) or solvent-accessible surface area (SASA). ASA was first described by Lee and Richards in 1971 \cite{LEE_1971_ASA} and is usually calculated using the ``rolling ball'' algorithm described by Shrake and Rupley \cite{SHRAKE_1973_ASA}. The van der Waals (VDW) surface of a molecule is defined by the VDW radii of the atoms forming it \cite{1873_VANDERWAALS_VDW}. In their algorithm, Shrake and Rupley draw a mesh of points equidistant to each atom on the molecule. These points are typically drawn at a distance of 1.4 \AA{}, emulating the radius of a water molecule, i.e., solvent. By \textit{rolling} this spherical probe over each atom, they established whether a mesh point was exposed to the solvent or buried and calculated the individual contribution of each atom or residue to the ASA of a protein (\autorefpanel{fig:protein_features}{ D}). The ASA of a molecule is then the path traced by the centre of the spherical probe rolled over the VDW surface. Years later, Richards \cite{1977_RICHARDS_SSE} defined the molecular or solvent-excluded surface (SES) which results from the trajectory of the outer edge of the sphere probe and has two components: the contact surface and the \textit{reentrant} surface. The contact surface is the part of the VDW surface that is in direct contact with the probe. The reentrant surface is an inward-facing or concave surface resulting from the contact of the probe with multiple atoms (\autoref{fig:molecular_surfaces}). Conolly was the first to implement algorithms for the analytical calculation of the SES \cite{1983_CONNOLLY_SASA, 1983_CONNOLLY_SASA2}.

\begin{figure}[htb!]
    \centering
    \includegraphics[width=\textwidth]{figures/ch_INTRO/PNG/molecular_surfaces.png}
    \caption[Molecular surfaces]{\textbf{Molecular surfaces.} Different definitions of the surfaces of a molecule including Van der Waals surface of atoms, solvent accessible surface area defined by Lee and Richards \cite{LEE_1971_ASA} and solvent-excluded surface by Richards \cite{1977_RICHARDS_SSE} with its two components: reentrant and contact surfaces. Adapted from the ChimeraX website \cite{surface_diagram}.}
    \label{fig:molecular_surfaces}
\end{figure}

The SASA measured for a given residue in a protein structure is an absolute measure and is not directly comparable across amino acids due to the different size of their side chains. To account for these differences, SASA values are often normalised. There are multiple normalisation scales, all derived from Gly-\textit{X}-Gly tripeptides, where \textit{X} represents each of the twenty amino acids \cite{1985_ROSE_MAXASA, MILLER_1987_MAXASA, TIEN_2013_RSA}. This construct is used because amino acid side chains can adopt an extended conformation and achieve their maximum ASA (MaxASA). A relative solvent accessibility (RSA) can then be obtained by dividing the ASA by the maximum allowed accessible surface area for a given residue as shown in \autoref{eq:RSA}. As the name indicates, RSA is a measure relative to each side chain and can therefore be used to compare across different amino acids.

\begin{equation}
\text{RSA} (\%) = 100 \times \text{ASA}/\text{MaxASA}
\label{eq:RSA}
\end{equation}
\myequations{Relative solvent accessibility}

\vspace{-13pt} % Adjust this value as needed

The analysis of the solvent accessibility landscape of proteins provides rich insight into protein evolution, function, stability and folding. RSA can classify residues into buried and accessible to solvent. Residues with low RSA tend to be buried in the core of the protein and form a network of hydrophobic interactions that ensure the correct packing of the protein \cite{DILL_1990_FOLDING}. Residues with high RSA are on the surface and interact with solvent and other biomolecules. Accordingly, RSA can be used to identify active sites and hotspot residues that are key contributors to the binding interaction with ligand or protein partners \cite{JONES_1997_PROTINTERS}. Active sites tend to have intermediate RSA values (20-50\%) since they need to be accessible enough to bind to their substrates but also partially protected to allow for a stable substrate binding and catalysis. Additionally, solvent accessibility is known to be correlated with evolutionary conservation \cite{GOLDMAN_1998_SS_RSA_EVO}. Residues buried in the hydrophobic core are conserved through evolution as mutations in them have a destabilising effect and can lead to protein misfolding and aggregation.

In \autoref{chap:FRAGSYS} and \autoref{chap:LIGYSIS_WEB}, ASA is calculated with DSSP \cite{KABSCH_1983_DSSP} and normalised using the method of Tien \textit{et al.} \cite{TIEN_2013_RSA} to characterise the solvent accessibility profile of ligand binding sites and predict their likelihood of function.

\subsubsection{Ligandability}

Ligandability is the conceptual feature aiming to chatacterise the ability of a residue, set of residues or protein to bind a small molecule or ligand. Ligands play a critical role in protein function acting as natural co-factors, substrates, inhibitors and drugs in disease therapy (\autoref{fig:small_molecules}). Identifying where ligands can bind to proteins is therefore of critical importance in understanding and modulating protein function. While X-ray crystallography remains the gold-standard to identify and characterise binding sites \cite{REES_2004_FBLD}, over the last three decades, significant effort has been made to develop computational methods that predict binding sites from an apo three-dimensional protein structure \cite{VOLKAMER_2010_TOPOLOGY}. Ligandability can be predicred by methods like P2Rank \cite{KRIVAK_2018_P2RANK} as the probability of an atom or residue having the ability to bind a ligand, and visualised on a protein surface (\autorefpanel{fig:protein_features}{ E}).

\begin{figure}[htb!]
    \centering
    \includegraphics[width=\textwidth]{figures/ch_INTRO/PNG/small_molecules.png}
    \caption[Protein-ligand complexes]{\textbf{Protein-ligand complexes.} Small molecule ligands interact with proteins and act as cofactors, substrates, inhibitors and drugs for therapeutic treatment. \textbf{(A)} Cytochrome P450 2C9 (\href{https://www.uniprot.org/uniprotkb/P11712/entry}{P11712}) interacting with haem (\href{https://www.ebi.ac.uk/pdbe-srv/pdbechem/chemicalCompound/show/HEM}{HEM}) as a cofactor. PDB: \href{https://www.ebi.ac.uk/pdbe/entry/pdb/7RL2}{7RL2} \cite{PARIKH_2021_CYTOP450}; \textbf{(B)} Glutathione S-transferase A3 (\href{https://www.uniprot.org/uniprotkb/Q16772/entry}{Q16772}) interacting with its substrate glutathione (\href{https://www.ebi.ac.uk/pdbe-srv/pdbechem/chemicalCompound/show/GSH}{GSH}). PDB: \href{https://www.ebi.ac.uk/pdbe/entry/pdb/1tdi}{1TDI} \cite{GU_2004_GST}; \textbf{(C)} Dual specificity tyrosine-phosphorylation-regulated kinase 2 (\href{https://www.uniprot.org/uniprotkb/Q92630/entry}{Q92630}) interacting with a natural inhibitor curcumin (\href{https://www.ebi.ac.uk/pdbe-srv/pdbechem/chemicalCompound/show/CC9}{CC9}). PDB: \href{https://www.ebi.ac.uk/pdbe/entry/pdb/6hdr}{6HDR} \cite{PDB_6HDR}; \textbf{(D)} Sodium-dependent serotonin transporter (\href{https://www.uniprot.org/uniprotkb/P31645/entry}{P31645}) binding to paroxetine (\href{https://www.ebi.ac.uk/pdbe-srv/pdbechem/chemicalCompound/show/8PR}{8PR}), which is an antidepressant. PDB: \href{https://www.ebi.ac.uk/pdbe/entry/pdb/5i6x}{5I6X} \cite{COLEMAN_2016_PAROXETIN}.}
    \label{fig:small_molecules}
\end{figure}

Ligand site prediction methods exploit a variety of techniques to suggest binding sites. Geometry-based tools like fpocket \cite{GUILLOUX_2009_FPOCKET}, Ligsite \cite{HENDLICH_1997_LIGSITE} and Surfnet \cite{LASKOWSKI_1995_SURFNET} identify cavities by analysing the geometry of the molecular surface of a protein and rely on the use of grids, gaps, spheres, or tessellation \cite{GUILLOUX_2009_FPOCKET, SCHMIDTKE_2010_FPOCKET2, WEISEL_2007_POCKETPICKER, BRADY_2000_PASS, LIANG_1998_CAVITIES, HENDLICH_1997_LIGSITE, LASKOWSKI_1995_SURFNET, KLEYWEGT_1994_CAVITIES, LEVITT_1992_POCKET}. Energy-based methods such as PocketFinder \cite{AN_2005_POCKETFINDER} rely on the interaction energy between the protein and a chemical group or probe to identify cavities \cite{AN_2005_POCKETFINDER, NGAN_2012_FTSITE, GHERSI_2009_SITEHOUND, LAURIE_2005_QSITEFINDER, AN_2004_PREDICTOR, GOODFORD_1982_PREDICTOR}. Conservation-based methods use sequence evolutionary conservation information to find patterns in multiple sequence alignments and identify conserved key residues for ligand site identification \cite{XIE_2012_CONSPRED, PUPKO_2002_RATE4SITE, ARMON_2001_SCORE}. Template-based methods rely on structural information from homologues and the assumption that structurally conserved proteins might bind ligands at a similar location \cite{ZVELEBIL_1987_PREDICTION, WASS_2010_3DLIGANDSITE, YANG_2013_COFACTOR, LEE_2013_PREDICTION, BRYLINSKI_2013_EFINDSITE, ROY_2012_COFACTOR}. Combined approaches or meta-predictors integrate multiple methods or features to infer ligand binding sites, e.g., geometric features with sequence conservation \cite{GUTTERIDGE_2003_LBSP, HUANG_2006_BU48, GLASER_2006_PREDICTION, HALGREN_2009_PREDICITON, CAPRA_2009_CONCAVITY, HUANG_2009_METAPOCKET, BRAY_2009_SITESIDENTIFY, BRYLINSKI_2009_FINDSITE}. Finally, machine learning (ML) methods utilise a wide range of techniques including random forest and deep, graph, residual, or convolutional neural networks \cite{KRIVAK_2015_PRANK, KRIVAK_2015_P2RANK, JIMENEZ_2017_DEEPSITE, KRIVAK_2018_P2RANK, JENDELE_2019_PRANKWEB, SANTANA_2020_GRaSP, KOZLOVSKII_2020_BITENET, STEPNIEWSKA_2020_KALASANTY, KANDEL_2021_PURESNET, MYOLNAS_2021_DEEPSURF, YAN_2022_POINTSITE, LI_2022_RECURPOCKET, AGGARWAL_2022_DEEPPOCKET, JAKUBEC_2022_PRANKWEB, ABDOLLAHI_2023_NODECODER, EVTEEV_2023_SITERADAR, LI_2023_GLPOCKET, ZHANG_2024_EQUIPOCKET, LIU_2023_REFINEPOCKET, SMITH_2024_GrASP, CARBERY_2024_IFSP, SESTAK_2024_VNEGNN, KANDEL_2024_PURESNET}.

\autoref{chap:LBS_COMP} and \autoref{chap:LBS_IMPROV} describe the largest benchmark of ligand binding site prediction to date by analysing the performance of thirteen methods using a series of metrics on a brand new reference dataset described in \autoref{chap:LIGYSIS_WEB}.

\subsection{Databases}

There are many databases providing relevant information for protein analysis but two of the most commonly used ones and extensively employed for the research described in this Thesis are UniProt and the Protein Data Bank (PDB).

\subsubsection{UniProt}

UniProt is a comprehensive protein sequence database including cross-references to multiple resources to provide a wealth of information about gene expression, pathogenic variation, post-translational modifications (PTMs), protein-protein interactions, domain annotations and three-dimensional structure, among others \cite{BAIROCH_2005_UNIPROT}. It is composed of two main components: SwissProt and TrEMBL. SwissProt is manually curated and includes 600,000 high-quality sequences often referred to as \textit{reviewed}. TrEMBL, on the other hand, catalogues over 250 million \textit{unreviewed} protein sequences resulting from the automatic translation of coding sequences found in the main nucleotide databases \cite{BAIROCH_2000_UNIPROT}.

\subsubsection{Protein Data Bank}

The Protein Data Bank is a worldwide repository for the three-dimensional structures of biological macromolecules such as proteins, DNA and RNA. There are currently 230,000 3D structures deposited in the archive determined mainly through X-ray crystallography, NMR spectroscopy and Cryo-EM \cite{BERMAN_2003_PDB}. Through resources such as the Protein Data Bank Europe (PDBe) knowledgebase, or PDBe-KB, three-dimensional structure is linked to functional annotations including that of domains, predicted disorder, ligand binding sites or PTMs \cite{PDBEKB_2019_PDBEKB}.

\section{Genetic variation}

Genetic variation is the difference in DNA sequence between individuals or populations of the same species. The main source of genetic variation is \textit{de novo} mutation. Mutations are changes in a genetic sequence that usually arise during DNA replication due to errors made by the imperfect replication machinery. Mutation can also occur as a result of damage to DNA, e.g., ultraviolet radiation, or during the repair process of such damage. Genetic variation can affect a single nucleotide in the sequence, i.e., a single nucleotide polymorphism (SNP), multiple nucleotides, or larger DNA regions, even entire chromosomes, e.g., insertion, deletion, translocation, or fusion. SNPs are the only type of genetic variation described in this Thesis.

\subsection{Types of genetic variation}

\subsubsection{Genomic location}

Based on genomic location, genetic variation can be classified into \textit{coding} variation if it affects the mRNA that codes for the protein sequence. Alternatively, \textit{non-coding} variants are those that affect other regions that do not code for a protein product, such as introns, intergenic regions, promoters, enhancers or other regulatory elements.

\subsubsection{Effect on coding sequence}

The genetic code is \textit{degenerate} or redundant, as there are 4 $\times$ 4 $\times$ 4 = 64 codons coding for only twenty amino acids. For this reason, a change in the coding DNA sequence is not always reflected in the protein sequence. Mutations that due to the redundancy in the genetic code do not alter the protein sequence are called synonymous or silent. Conversely, nonsynonymous mutations \textit{do} change the protein sequence and can be further classified into: missense, nonsense, stop-loss and frameshift mutations. Missense mutations are those that replace one of the twenty amino acids by a different one. They can be conservative, if the interchanged residues present similar physicochemical properties, e.g., leucine $\rightarrow$ isoleucine, or they can be non-conservative, or radical, if the exchanged amino acids are biochemically different, e.g., lysine $\rightarrow$ threonine. Nonsense mutations replace one of the twenty amino acids by one of the three STOP codons, resulting in an early termination of the peptide chain. Stop-loss variants are the exact opposite and exchange the original STOP codon by one of the twenty amino acids, thus resulting in an abnormally elongated protein. Finally, frameshift mutations result from the insertion or deletion of nucleotides that are not a multiple of three. When this happens, the frame on which the translation machinery reads the mRNA is shifted and a completely different protein product is obtained.

While missense mutations, which simply replace one amino acid by another and can be conservative, tend to have a limited effect on protein sequence and structure, nonsense, stop-loss and frameshift mutation have more drastic consequences. Because of this, missense variants tend to be more tolerated and, along with synonymous variants, are observed at higher frequencies in the general population \cite{COULTER_2004_MUTATIONS}.

\subsubsection{Impact on phenotype}

Genetic variants can also be classified based on the effect they have on the phenotype, or clinical significance, which usually corresponds to an effect on the concentration, structure, function or activity of a protein \cite{VIHINEN_2022_VARIATION}. Mutations that do not have a harmful effect on the protein are called neutral or benign. Since neutral variants have no noticeable effect on the \textit{fitness} \cite{DARWIN_1859_ORIGIN}, i.e., the ability to leave offspring, they are not under selective pressure and consequently persist in the general population \cite{KIMURA_1968_NEUTRAL}. Conversely, pathogenic variants disrupt biological processes and result in disease. Disease severity will dictate the strength with which natural selection acts upon the causing variant and therefore its frequency in the population. Mutations affecting genes needed for development and survival, or essential genes, might have lethal effects and never be observed in the population \cite{GLUECKSOHN_1963_LETHALITY}.

It is estimated that only 2\% of the more than 4 million observed human missense variants have been clinically classified as pathogenic or benign \cite{LEK_2016_EXAC}. Variants of unknown significance (VUS) therefore represent the vast majority of observed missense variants and the prediction of their effect on fitness is an ongoing challenge in human genetics \cite{MCLAREN_2016_VEP}. Several methods exploiting different technologies have been developed over the years to tackle this challenge including SIFT \cite{KUMAR_2009_SIFT}, PolyPhen \cite{ADZHUBEI_2013_POLYPHEN} and the recent AlphaMissense \cite{CHENG_2023_ALPHAMISSENSE}.

\subsection{Variation is constrained}

Since the sequencing of the first draft of the human genome in 2001 \cite{CONSORTIUM_2001_GENOME}, several massively parallel methods have been developed for the high-throughput sequencing of nucleic acids \cite{KASIANOWICZ_1996_NANOPORE, MARGULIES_2005_PYROSEQUENCING, BENTLEY_2008_ILLUMINA, EID_2009_PACBIO, ROTHBERG_2011_IONTORRENT}. The drastic reduction in both time and cost required to sequence DNA has enabled large-scale projects such as the 1000 Genomes Project \cite{AUTON_2015_1000KG} or UK Biobank \cite{BYCROFT_2018_UKBIOBANK}. The genome aggregation database (gnomAD) is a comprehensive collection of human genetic variation from over 140,000 genomes and exomes. Resources like gnomAD make it possible to carry out systematic comparative analysis to understand the distribution and constraint of genetic variation along the human genome.

In a similar way as protein sequence is constrained across species, resulting in patterns of amino acid conservation, the genomic distribution of variation within human is also restricted by factors such as protein structure and function. Several studies have demonstrated that functional elements like buried core residues, catalytic residues in enzymatic active sites and protein-protein interfaces are strongly constrained and present fewer variants than observed elsewhere in the protein \cite{GONG_2010_CONSTRAINT, BEER_2013_CONSTRAINT, DAVID_2015_CONSTRAINT, SIVLEY_2018_CONSTRAINT}. This phenomenon is a consequence of purifying or negative natural selection acting upon the population. Variants occurring at these relevant sites are likely to impair protein function and therefore removed from the gene pool. Consequently, these positions present a lower mutational burden, or \textit{depletion} in variation. By quantifying these evolutionary signals, functional constraint can be measured at the genic \cite{PETROVSKI_2013_CONSTRAINT} and domain \cite{GUSSOW_2016_CONSTRAINT} levels and used for the functional interpretation of VUS \cite{LI_2022_CONSTRAINT}. Despite the wealth of variation data available that allows for gene- and domain-level quantification of constraint, doing so at the individual residue level remains a relative challenge still.

\subsection{Missense enrichment score}

MacGowan \textit{et al.} approached this issue in their 2017 work \cite{MACGOWAN_2017_VARIANTS} by aggregating variants from human paralogous residues present in the same alignment column. Residues aligning in the same column are homologous, i.e., share a common ancestor, and therefore aggregating their variants to infer residue-level constraint is a fair assumption. They developed a missense enrichmente score (MES) to numerically quantify the evolutionary constraint acting on individual residues, or positions, by leveraging the variants found not just in a protein of interest but also in their human paralogues. For each alignment column $x$, the number of human residues mapping to it were counted ($\text{residues}_x$), as well as the number of human residues mapping to all other columns ($\text{residues}_{other}$). Additionally, the number of variants found across all human residues aligned to the column of interest ($\text{variants}_x$) and all other columns ($\text{variants}_{other}$) were obtained. With these four quantities, the MES can be computed as showed in \autoref{eq:MES}. These four values can also be arranged in a 2 $\times$ 2 contingency table and the MES understood as an odds ratio (OR) expressing the likelihood of observing variants in column $x$ relative to all other alignment columns. An OR > 1 means a column presents more variants than the average of all other columns, i.e., is \textit{enriched} in missense variants, whilst an OR < 1 indicates \textit{depletion} relative to the rest of the alignment. An OR = 1 indicates neutrality relative to the other columns, i.e., the number of variants found within column $x$ follows the same distribution as the average of the rest of the alignment. Using Fisher's exact test \cite{FISHER_1935_TEST} the significance of this MES (OR) can be assessed with a \textit{p}-value.

\begin{equation}
\text{MES} = \frac{\text{variants}_x / \text{residues}_x}{\text{variants}_{other} /\text{residues}_{other}}
\label{eq:MES}
\end{equation}
\myequations{Missense enrichment score\vspace{+9pt}}

\vspace{-13pt} % Adjust this value as needed
\vspace{-13pt} % Adjust this value as needed

\subsection{Conservation plane}

Both Shenkin and MES are measures of evolutionary constraint on protein amino acids. Nevertheless, these metrics quantify it at two completely different time scales. Amino acid divergence calculated from an MSA captures the evolutionary history of a protein family resulting of hundreds of millions of years of divergence across species originated from speciation events, large-scale genomic rearrangements and strong selective pressures, among other factors. In contrast, the missense enrichment score aims to capture the variability in \textit{our} species emerging from migration events, genetic drift and weaker selection taking place within a much shorter time scale \cite{HUBLIN_2017_HUMAN}. The stratification of conserved and divergent positions by MES yields four classifications, or four quadrants on the \textit{conservation plane} (\autoref{fig:conservation_plane}). These are conserved positions that are missense-depleted (CMD), conserved positions, yet enriched in missense variation (CME), unconserved, or divergent, positions enriched in missense variants (UME) and unconserved and missense-depleted (UMD) positions \cite{MACGOWAN_2024_VARIANTS}.

\begin{figure}[htb!]
    \centering
    \includegraphics[width=0.55\textwidth]{figures/ch_INTRO/PNG/conservation_plane.png}
    \caption[Conservation plane]{\textbf{Conservation plane.} The \textit{conservation plane} arises from the comparison of within-species constraint, as measured by the missense enrichment score (MES), and across-species constraint quantified by amino acid conservation, in this case, by a normalised Shenkin divergence score. The conservation plane can be divided in four quadrants. Positions that are conserved across species and missense-depleted in human (CMD) are found in the bottom-left corner (pink). Conserved positions that are enriched in missense variation (CME) are on the top-left quadrant (green). Unconserved or divergent positions enriched in missense variants (UME) are on the top-right (blue). Finally, unconserved and missense-depleted (UMD) positions are on the bottom-right (orange). Figure adapted from MacGowan \textit{et al.} \cite{MACGOWAN_2024_VARIANTS}.}
    \label{fig:conservation_plane}
\end{figure}

CMD positions are the most constrained both across species (conserved) and within the human population (missense-depleted). The vast majority of them are buried in the core and are critical for protein folding, packing and stability. When they are not buried, they are highly enriched in protein-protein and protein-ligand interactions \cite{UTGES_2021_ANKS}. Additionally, they are enriched in ClinVar \cite{LANDRUM_2013_CLINVAR} pathogenic variants, further emphasising their relevance. Unconserved positions are often dismissed as they appear to be mutating freely and be under no constraint, resulting in their divergence across homologues. However, MacGowan \textit{et al.} \cite{MACGOWAN_2024_VARIANTS} showed that there is a subset of unconserved positions that are strongly constrained in the human population, i.e., significantly depleted in missense variation. These positions tend to be on the surface and act as specificity-determining positions (SDP) bestowing protein domains the ability to bind a wide range of substrates. Furthermore, they are enriched in pathogenic variants relative to their missense-enriched counterpart (UMEs).

%\vspace{-13pt} % Adjust this value as needed
%\vspace{-13pt} % Adjust this value as needed

\section{Drug discovery}

Drug discovery is the process of developing a new drug. It goes from the original idea conception to the market launch of a finished product and beyond. This is an extremely complex process which can take up to 12-15 years and cost more than \$1 billion \cite{HUGHES_2011_DRUGS}. This high economic and time cost is caused by the high rate of failure that potential drug candidates experience during the development process, also known as \textit{attrition}. The drug discovery and development pipeline is illustrated in \autoref{fig:drug_discovery}. This pipeline can be divided in two stages. Stage I is drug discovery and encompasses target identification and validation, hit identification and lead optimisation. Stage II corresponds to the development of the drug and includes pre-clinical, clinical trials and drug approval.

\begin{figure}[htb!]
    \centering
    \includegraphics[width=\textwidth]{figures/ch_INTRO/PNG/drug_discovery.png}
    \caption[Drug discovery and development pipeline]{\textbf{Drug discovery and development pipeline.} The pipeline for discovering a drug can be divided in two stages. Stage I focuses on the discovery of a drug and includes target identification and validation, hit identification and lead optimisation. Stage II covers the development of the drug and includes pre-clinical, clinical trials, drug approval and pharmacovigilance. Figure adapted from Cui \textit{et al.} \cite{CUI_2020_DRUGS}.}
    \label{fig:drug_discovery}
\end{figure}

\vspace{-13pt} % Adjust this value as needed

\subsection{Target identification}

Target identification is the first step in the drug discovery pipeline and one of the most critical. Drugs often fail in the clinical stages due to two main reasons: they are not safe or they do not work. Because of this, a thorough target identification and subsequent validation is vital. The goal is to identify a biomolecule to target with a drug to treat or cure a disease. The target can either be a gene, RNA or protein and the drug is usually a small molecule, peptide or a protein, e.g., antibody. An ideal target meets a series of requirements: efficacy, safety and most importantly \textit{druggability}, among others. A \textit{druggable} target is amenable to interact with a putative drug. This interaction should trigger a biological response measurable \textit{in vitro} and \textit{in vivo} through biochemical or functional assays. The mining of available biomedical data from the literature, proteomics, 3D structure, genetic association studies, pathogenic variation or phenotypic screening are some of the most commonly used approaches for target identification \cite{SCHENONE_2013_TARGETID}.

\subsection{Target validation}

Target validation is the technical assessment of whether a target plays a critical role in a disease process and whether pharmacological modulation of the target could be effective in a particular patient population. It is predicted that a more effective target validation strategy could reduce attrition in phase II clinical trials by $\approx$24\% lowering the cost of developing a new molecular entity (NME) by $\approx$30\% \cite{PAUL_2010_RD}. Accordingly, the validation of a therapeutic target is a step of paramount importance within the discovery of new drugs. Some of the most frequently used approaches to validate a target include RNA interference, gene knockouts, the use of animal models and target druggability analysis, e.g., by ligand binding site prediction \cite{EMMERICH_2021_TARGET_VAL}.

\subsection{Hit identification}

The next step once the target has been validated is to identify \textit{hits}. Hits are compounds that bind to the target and elicit the desired biological activity in an assay. Their identification relies on a combination of experimental techniques, e.g., high-throughput (HTS) or fragment screening (FS), and computational techniques such as virtual screening (VS). In high-throughput screening, robotic automation is employed to evaluate large libraries of chemical compounds against a target in a biochemical or cell-based assay. HTS usually identifies a few compounds with the desired biological activity and high binding affinity to the target. FS is complementary to HTS and obtains high-quality information about the 3D structure of a protein-ligand complex by using X-ray crystallography. Lastly, provided the 3D structure of the target is known, virtual screening techniques can be used. VS encompasses a set of ligand-based (LBVS) and structure-based (SBVS) computational techniques, such as pharmacophore-mapping or protein-ligand docking, respectively. These techniques are able to identify hotspot residues relevant for ligand binding and guide the design of more effective compounds \cite{SINHA_2018_DD}.

\subsection{Lead optimisation}

In this phase, identified hits are refined into promising \textit{lead} compounds by optimising their properties before getting to pre-clinical drug candidates. This refinement aims to enhance pharmacokinetic (PK) properties such as potency, i.e., binding affinity to the target, as well as selectivity -- by minimising off-target effects --, solubility, permeability, stability and toxicity. Quantitative structure-activity relationship (QSAR) studies are carried out to suggest molecules with more favourable PK properties by adding or replacing functional groups of the original hit compound. Additionally, high-throughput \textit{in vitro} assays can be carried out to optimise the absorption (how it enters the bloodstream), distribution (how it travels within the body), metabolism (how it is broken down), excretion (how it is eliminated) and toxicity (ADMET) properties of the compounds \cite{SHOU_2020_ADME}.

\subsection{Pre-clinical studies}

The discovery stage concludes with the acquisition of the optimised leads and thus begins the development stage. The primary goal of pre-clinical studies is to thoroughly evaluate the safety, efficacy, pharmacokinetics and pharmacodynamics (PD) of the drug candidates before advancing to clinical trials in humans. This is achieved with a combination of \textit{in vitro} and \textit{in vivo} studies, including cell-bassed assays and animal models, respectively. ADMET properties are assessed to ensure a favourable pharmacological profile and toxicology studies are carried out to establish the no observed adverse effect level (NOAEL) and determine a safe starting dosage in human \cite{SHEGOKAR_2020_PRECLINICAL}.

\subsection{Clinical trials}

Those candidates that pass through pre-clinical development will be submitted to clinical trials in voluntary human subjects. Clinical trials are divided in three phases with different goals. Phase I focuses on establishing the maximum tolerated dose (MTD) of a drug by performing strictly calculated dose escalation in a small number (20-80) of \textit{healthy} and \textit{diseased} individuals. Phase II will aim to establish the preliminary efficacy of the drug by comparing a \textit{treatment} and a \textit{placebo}, or control, group whilst closely monitoring side effects. Usually 100-300 individuals are involved in Phase II trials. Phase III confirms the safety and efficacy of the drug by involving a larger (1000-3000) and more diverse target population whilst noting potential adverse side effects. Successful completion of clinical trials results in the submission of a comprehensive report to regulatory agencies for review, marking the final step before the drug can reach the market \cite{UMSCHEID_2011_TRIALS}.

\subsection{Drug approval}

After a drug has been approved and granted license by regulatory agencies such as the Food and Drug Administration (FDA), the European Medicines Agency (EMA) or the Medicines and Healthcare products Regulatory Agency (MHRA), it can be commercialised. Once on the market, drugs enter the post-marketing phase, also known as phase IV. In this phase, pharmacovigilance activities are conducted to monitor long-term safety and effectiveness in larger and more diverse populations. This includes the identification of rare adverse effects and potential new therapeutic uses \cite{SUVARNA_2010_PHASE4}.

\section{Fragment-based drug discovery}

Fragment-based drug discovery (FBDD), or fragment screening, is a widely used technique to identify compounds binding against a specific protein target \cite{MURRAY_2009_FBDD}. It falls within the range of tools used in the hit identification step of the drug discovery pipeline. FBDD typically uses X-ray crystallography to provide detailed information on the binding mode of small molecule fragments that bind to a protein \cite{REES_2004_FBLD}. These fragments explore the vast chemical space and usually obey the \textit{Rule of 3}: they present low molecular weight (200-500 Da), few rotatable bonds and low hydrophobicity \cite{CONGREVE_2003_RO3}. Hits tend to have low affinity (milimolar range) due to their small size. Nevertheless, they provide a good scaffold for optimisation and can be linked or grown to form more potent leads \cite{SCHIEBEL_2016_FRAGMENTS}. A typical fragment screening experiment generates a collection of three-dimensional structures with fragments bound to different regions of the protein. This is done by soaking pre-formed protein crystals in high-concentration fragment solutions, allowing the fragments to bind to the protein. After soaking, crystals are carefully washed to remove unbound fragments and cryoprotected before freezing. Once frozen, fragment-bound crystals are X-rayed, electron densities analysed and structure models obtained \cite{PATEL_2014_FS}.

While many fragments group around well understood catalytic or binding sites, fragments are also observed bound to regions of the protein where the functional significance is unclear. Such sites may be functionally irrelevant or could identify previously unknown allosteric or other functionally important sites worthy of experimental investigation. \autoref{chap:FRAGSYS} aims to address this issue by characterising fragment screening sites using a combination of structural, conservation and variation data, thus providing insight into the likelihood of function of such sites.

\section{Thesis scope}

Advances in the accurate prediction of protein 3D structure have resulted in a drastic reduction in the sequence-to-structure gap \cite{ABRAMSON_2024_ALPHAFOLD3}. The UniProt knowledgebase (UPKB) catalogues 248 million protein sequences \cite{UNIPROT_2018_UNIPROT, UNIPROT_2023_UNIPROT}, most of which have now structure models available through resources such as the AlphaFold Database (AFDB) \cite{VARADI_2022_ALPHAFOLDDB} and other providers \cite{GUEX_2009_SWISSMODEL, BEIENERT_2016_SWISSMODEL, WATERHOUSE_2018_SWISSMODEL}. However, only a minuscule fraction of these proteins present residue-level functional annotations in UniProt -- 55,000 (0.02\% of UPKB) - or include biologically relevant ligands co-crystallised in the PDB \cite{wwPDB_2019_PDB} -- 29,000 (0.01\%) (\autoref{fig:data_explosion}). The significant expense and time required for experimental validation underscores an urgent need for computational methods to characterise ligand sites systematically and highlight residues likely to be relevant to protein function.

\begin{figure}[htb!]
    \centering
    \includegraphics[width=0.55\textwidth]{figures/ch_INTRO/PNG/data_explosion.png}
    \caption[Database growth curves]{\textbf{Database growth curves.}  Growth curves for some of the most relevant nucleotide and protein sequence and structure databases from 1970 to date. The European Nucleotide Archive (ENA) catalogues nucleotide sequences \cite{LEINONEN_2010_ENA}. ModBase, ModelArchive and AlphaFold DB are some of the largest predicted protein structure resources. UniProt-KB\textsubscript{FUNC} corresponds to the subset of protein sequences with residue-level experimentally determined functional annotations in UniProt. BioLiP is a semi-curated database of biologically relevant protein-ligand complexes \cite{YANG_2013_BIOLIP}. SWISS-MODEL was not included in this graph as growth curves could not be obtained. Likewise for ENA, for which just the number of sequences in 2024 is included. Y-axis is in log\textsubscript{10} scale.}
    \label{fig:data_explosion}
\end{figure}

\newpage

Small molecule ligands are crucial for protein function and act as substrates, cofactors or drugs in therapy. Identifying the protein regions where these molecules bind, understanding the mode in which they do so and characterising that interface is therefore key to understanding and modulating protein function. \autoref{chap:FRAGSYS} describes work for the definition, characterisation and classification of likely functional class of ligand binding sites derived from fragment screening experiments. \autoref{chap:LIGYSIS_WEB} extends this approach to the entirety of the PDBe, characterising $>$65,000 biologically relevant protein-ligand binding sites using structural, divergence and human variation data. Additionally, a web server is introduced for users to explore this large dataset, named LIGYSIS, as well as analyse their own protein-ligand complexes. Finally, \autoref{chap:LBS_COMP} and \autoref{chap:LBS_IMPROV} describe the largest comparative performance assessment of ligand binding site prediction to date including thirteen canonical methods and fifteen novel variants defined in this work. Beyond ranking the methods by their prediction capability using several relevant metrics, this benchmark provides insight into the strengths and weaknesses of each method and paves the way for improvement in the field of ligand site prediction.

This Thesis aims to illuminate the nature of protein-ligand binding sites by analysing their structural features, evolutionary constraint, both within and across species, and using them to pinpoint those sites more likely to alter protein function if targeted. Additionally, a thorough benchmark is carried out, objectively quantifying the strengths and weaknesses of the state-of-the-art methods for ligand site prediction. Both of these contributions have potential applications in drug discovery and may help mitigate attrition in clinical trials, ultimately improving efficiency and resource allocation in drug discovery.

\chapter{Classification of likely functional class for ligand binding sites identified from fragment screening}

\section*{Preface}

On this chapter, a method to group ligands by protein interactions is introduced and sites clustered by their relative solvent accessibility profile. 293 unique ligand binding sites were defined from 37 fragment screening experiments and grouped into four clusters which are differentially enriched in functional sites. A multi-layer perceptron is developed to predict cluster labels with an accuracy of 96\% so allowing functional classification of sites for proteins not in this set. 

\section*{Publications}

Utgés JS, MacGowan SA, Ives CM, Barton GJ. Classification of likely functional class for ligand binding sites identified from fragment screening. \textit{Commun Biol} \textbf{7}, 320 (2024). \url{https://doi.org/10.1038/s42003-024-05970-8} \cite{UTGES_2024_FRAGSYS}

\section*{Author contributions}

G.J.B., J.S.U., S.A.M. and C.M.I. conceived, designed, and developed the research. J.S.U. and C.M.I. analysed the data. J.S.U., C.M.I. and S.A.M. developed the software. J.S.U. and G.J.B. wrote, reviewed and edited the manuscript. G.J.B. secured funding and supervised.

\section{Introduction}

Fragment-based drug discovery or fragment screening, is widely used to identify lead compounds against a specific protein target \cite{MURRAY_2009_FBDD}. Fragment screening typically uses X-ray crystallography to provide detailed information on the binding mode of small molecule fragments that bind to a target protein. Fragments can then be linked or grown to form more potent leads \cite{CONGREVE_2003_RO3, REES_2004_FBLD, SCHIEBEL_2016_FRAGMENTS}. A typical fragment screening experiment will generate a collection of three-dimensional structures with fragments bound to different regions of the protein. While many fragments group around well understood catalytic or binding sites and so provide a scaffold for drug discovery, fragments are also observed bound to regions of the protein where the functional significance is unclear. Such sites may be functionally irrelevant or could identify previously unknown allosteric or other functionally important sites worthy of experimental investigation. 

In this paper we describe a strategy to identify which fragment binding sites are most likely to be of functional importance and so prioritise sites for further investigation. The first step is to identify binding sites from the fragment data. We are not predicting ligand binding sites, as P2RANK \cite{KRIVAK_2018_P2RANK},fpocket \cite{GUILLOUX_2009_FPOCKET}, or molecular dynamics-based methods such as MixMD \cite{LEXA_2011_FLEXIBILITY, GHANAKOTA_2018_MIXMD}, MDmix \cite{ALVAREZ_2014_MIXMD}, or SILCS \cite{FALLER_2015_SILCS} do. Instead, from a set of experimentally determined three-dimensional structures of protein-ligand complexes, we define which ligands bind to the same site, based on their protein-ligand interactions.

In most previous studies the focus has been on clustering ligands by  root-mean-square deviation (RMSD) \cite{SHIN_2005_PDBLIGAND} or Euclidean distances \cite{KOZAKOV_2005_CLUSTERING} after ligand superposition. Ligand site prediction resources such as 3DLigandSite \cite{WASS_2010_3DLIGANDSITE, MCGREIG_2022_3DLIGANDSITE} also define sites based on ligand structure superposition and RMSD. Here, we describe an algorithm that defines ligand binding sites from analysis of ligand interaction residues on the protein. The method allows the extent of a fragment binding site to be described without the need for superposition. We then apply unsupervised methods to group the defined sites into four robust clusters according to their relative solvent accessibility profiles and show which clusters are enriched in functionally characterised sites. Our analysis suggests which sites in a set of 39 fragment screening experiments are most likely to be of functional significance through further stratification by evolutionary conservation and human population missense depletion \cite{MACGOWAN_2017_VARIANTS, MACGOWAN_2024_VARIANTS}. We then develop a machine learning method that takes a set of interacting residues in an experimentally determined structure or a predicted ligand binding site and identifies which of the four classes best represents the site. 

The work in this paper is likely to be of interest to groups focusing on fragment screening studies but wider applications to ligand site classification from experimentally determined or predicted structures are also discussed. 

\section{Methods}
\label{sec:ch_FRAGSYS_methods}

\subsection{Structure dataset}

The Pan-Dataset Density Analysis (PanDDA) algorithm characterises a set of related crystallographic data sets of the same crystal form and identifies binding events by isomorphous difference maps \cite{PEARCE_2017_PANDDA}.  Initially, 3,021 three-dimensional structures determined by X-ray crystallography were selected by querying the PDBe \cite{wwPDB_2019_PDB} for entries containing the string ``PanDDA'' in their title. 1,542 of the structures included bound ligands for 39 different proteins. Four proteins which were in multi-protein complexes including additional ligands were excluded to leave protein-ligand complexes coming from 35 different proteins and a total of 1,450 three-dimensional structures. The structures presented resolutions from 0.9-3.3\AA{}, with a mean resolution of $\approx$1.5\AA{}. The preferred biological assemblies, as defined by PISA \cite{KRISSINEL_2007_PISA}, were downloaded from the PDBe via ProIntVar \cite{MACGOWAN_2020_DRSASP}. 

\subsection{Binding site definition}

Ligand binding site definition or prediction approaches are usually based on the spatial superposition and clustering of the atomic coordinates of ligands according to Euclidean distances or RMSD \cite{SHIN_2005_PDBLIGAND, KOZAKOV_2005_CLUSTERING, WASS_2010_3DLIGANDSITE, MCGREIG_2022_3DLIGANDSITE}. These methods rely on structural superposition but can be computationally expensive when dealing with large numbers of structures. Here, we define sites from protein-ligand interactions without the need for superposition (\autoref{fig:fragsys_bs_def}).

\begin{figure}[ht!]
    \centering
    \includegraphics[width=\textwidth]{figures/ch_FRAGSYS/PNG/FIG8_bs_def.png}
    \caption[Ligand binding site definition algorithm]{\textbf{Ligand binding site definition algorithm.} The method defines ligand binding sites from a set of three-dimensional structures portraying the complex of a protein of interest bound to ligands. \textbf{(A)} Protein-ligand complex (P18031); \textbf{(B)} Ligand binding fingerprint, comprised by protein residue numbers interacting with ligand; \textbf{(C)} Formula of the similarity metric: relative intersection, $I_{rel}$; \textbf{(D)} Hierarchical clustering tree resulting from the similarity matrix, cut at threshold to determine distinct clusters of ligands; \textbf{(E)} Three-dimensional structure of all ligands binding to protein, coloured according to the cluster they group into. Only ligands found in clusters 1-7 are in coloured based on their cluster membership. The rest are coloured in grey. The tree on d represents only a part of the tree, showing 7/18 binding sites defined on P18031. This is represented by a dash line pointing downwards on the tree.}
    \label{fig:fragsys_bs_def}
\end{figure}

Only non-ion ligands of interest were used for the binding site definition. These do not include water molecules, nor other by-products of the experimental conditions. Ligand contacts were determined with Arpeggio \cite{JUBB_2017_ARPEGGIO}. For a given ligand, a binding fingerprint is defined as the UniProt residue numbers the ligand interacts with. For a pair of ligands $L_{A}$ and $L_{B}$, with their interaction fingerprints $A$ and $B$, their relative intersection, $I_{rel}$, is defined (\autoref{eq:relative_intersection}) by dividing the intersection of sets $A$ and $B$ by the maximum possible intersection between the two sets, given by the minimum fingerprint length (\autoref{eq:max_intersection}). $I_{rel}$ ranges from 0-1.

\begin{equation}
    I_{\mathrm{rel}} = \frac{A \cap B}{A \cap B_{\mathrm{max}}}
\label{eq:relative_intersection} % Label for the equation
\end{equation}
\myequations{Relative Intersection}
%\addcontentsline{loe}{equation}{\protect\numberline{\theequation}Relative intersection}

\begin{equation}
    A \cap B_{\mathrm{max}} = \min(\mathrm{len}(A), \mathrm{len}(B))
\label{eq:max_intersection}
\end{equation}
\myequations{Maximum Intersection}

$I_{rel}$ is thus a similarity metric that can be used to perform hierarchical clustering on the ligands. Single-linkage hierarchical clustering was performed with the OC software \cite{BARTON_1993_OC}. After exploring several threshold $I_{rel}$ values to cut the resulting tree, we settled on $I_{rel}$ = 0.66. Since this is a similarity metric, it means that a ligand shares at least two thirds of its binding residues with at least one other member of the same cluster. A total of 293 ligand binding sites across 37 protein domains were defined this way. For each protein, all structures were multiply aligned by STAMP \cite{RUSSELL_1992_STAMP}. Ligand binding sites were visualised in UCSF Chimera \cite{PETTERSEN_2004_CHIMERA}.

\subsection{Multiple sequence alignments}

Two of the 35 proteins included fragment screening experiments targeting multiple domains, or protein products, resulting in 39 protein-fragments sets. A representative sequence was selected for each of the 39 sets of structures, and used to search SwissProt \cite{BOUTET_2016_UNIPROT} for homologues with jackHMMER \cite{EDDY_1995_HMMER} with default parameters and 5 iterations to generate multiple sequence alignments. Evolutionary divergence within the alignments was quantified with the Shenkin divergence score, $V_{Shenkin}$, \cite{SHENKIN_1991_SCORE} and the normalised $N_{Shenkin}$, as defined by Utgés \textit{et al.} \cite{UTGES_2021_ANKS}.

\subsection{Human variants and enrichment}

VarAlign \cite{MACGOWAN_2020_DRSASP} was used to retrieve genetic variants from gnomAD v2.1 \cite{KARCZEWSKI_2020_GNOMAD} found in the human sequences within the multiple sequence alignment generated for each target protein. gnomAD contains exomes and genomes of 141,456 unrelated individuals with no known phenotypic conditions and is therefore a reasonable representation of the general healthy population. Variants found in the human sequences within the alignments were mapped to individual alignment columns and missense enrichment scores (MES) were calculated. MES represents the enrichment in missense variants of an alignment column relative to the average of the other columns in the alignment \cite{MACGOWAN_2017_VARIANTS, MACGOWAN_2024_VARIANTS}. 95\% confidence intervals (CI) and $p$-values were used to assess the significance of these ratios \cite{SZUMILAS_2010_ODDSRATIOS}. MES was also calculated for the defined ligand binding sites. The MES of a binding site represents the enrichment in missense variants of a binding site relative to the rest of protein residues. Alignment columns as well as binding sites were classified as enriched (MES $>$ 0), depleted (MES $<$ 0) or neutral (MES = 0). Enrichment was not calculated for two of the 39 proteins since no human homologues were identified.

\subsection{Binding site clustering}

Secondary structures were defined with DSSP \cite{KABSCH_1983_DSSP} via ProIntVar, and relative solvent accessibility (RSA) was calculated with the method of  Tien \textit{et al.} \cite{TIEN_2013_RSA}. The defined binding sites were grouped according to the pattern of RSA as follows and summarised in \autoref{fig:fragsys_bs_clust}.

\begin{figure}[ht!]
    \centering
    \includegraphics[width=\textwidth]{figures/ch_FRAGSYS/PNG/FIG9_bs_clust.png}
    \caption[Binding site clustering algorithm]{\textbf{Binding site clustering algorithm.} The method here clusters ligand binding sites defined across different proteins based on their solvent accessibility profiles. \textbf{(A)} Example of a defined ligand binding site; \textbf{(B)} Relative solvent accessibility profile of a binding site, represented by the RSA of the site residues; \textbf{(C)} Formula of our distance metric: distance U, $U_{D}$; \textbf{(D)} Multidimensional scaling (MDS) representation of binding sites coloured according to the four clusters determined by the K-means algorithm. Dashed lines represent the cluster limits.}
    \label{fig:fragsys_bs_clust}
\end{figure}

Given two binding sites, $A$ and $B$, with RSA profiles $r_{A}$ and $r_{B}$ and sizes $n_{A}$ and $n_{B}$ respectively, in amino acid residues, $U_{A}$ and $U_{B}$ can be calculated (\autoref{eq:mann_whitney_U}). The Mann-Whitney $U$ statistic \cite{MANN_WHITNEY_1947_TEST}, as implemented in SciPy \cite{VIRTANEN_2020_SCIPY}, was chosen as it has a maximum theoretical value ($U_{max}$) (\autoref{eq:maximum_U}). A relative $U$ value, $U_{rel}$, ranging 0-1 is obtained by dividing the $U$ value by $U_{max}$. The more similar $r_{A}$ and $r_{B}$ are, the bigger $U$ and $U_{rel}$ are. Thus, $U_{rel}$ is a similarity score. Subtracting $U_{rel}$ from 1 gives the $U$ distance, $U_{D}$, (\autoref{eq:relative_U}). $U_{D}$ is indicative of how different $r_{A}$ and $r_{B}$ are and can be used to cluster binding sites according to their RSA profiles.

\begin{equation}
U_A = R_A - \frac{n_A(n_A + 1)}{2}, \quad U_B = R_B - \frac{n_B(n_B + 1)}{2} \quad
\label{eq:mann_whitney_U}
\end{equation}
\myequations{Mann-Whitney's U}

\begin{equation}
U_A + U_B = n_A n_B, \quad U = \min(U_A, U_B) \rightarrow U_{\max} = \frac{n_A n_B}{2} \quad
\label{eq:maximum_U} 
\end{equation}
\myequations{Maximum U}

\begin{equation}
U_{\text{rel}} = \frac{U}{U_{\max}} \rightarrow U_D = 1 - U_{\text{rel}} \quad 
\label{eq:relative_U}
\end{equation}
\myequations{Relative U}

After calculating pairwise distances between the RSA profiles of the defined binding sites, K-means clustering \cite{LLOYD_1982_KMEANS} was performed. Several clustering algorithms were tried to realise this task, including some hierarchical, or connectivity-based, such as single and complete-linkage \cite{SORENSEN_1948_CL}, unweighted average linkage clustering (UPGMA) \cite{SOKAL_1958_UPGMA}, or Ward linkage \cite{WARD_1963_CLUSTERING}, as well as centroid-based, such as K-means. Overall, the clusters obtained by the different methods were similar. Ward linkage and K-means resulted in the most similar clusters, displaying an average similarity between clusters of 85\% (\autoref{fig:fragsys_bs_clust_ward}).

\begin{figure}[ht!]
    \centering
    \includegraphics[width=0.7\textwidth]{figures/ch_FRAGSYS/PNG/SUPP_FIG1_ward_clust.png}
    \caption[Binding sites Ward clustering]{\textbf{Binding sites Ward clustering.} Cluster map of the U distance, $U_{D}$, matrix of the 293 defined binding sites clustered by the Ward hierarchical clustering method implemented in SciPy. The tree is cut at $D_{Ward}$ = 1.8, giving four clear clusters. These clusters are labelled so they correspond to the ones obtained with K-means. Clusters in the heat map are represented by dark squares around the diagonal. $U_{D}$ is a distance; therefore, clusters include sites that are similar to each other, and present lower distances (dark colour).}
    \label{fig:fragsys_bs_clust_ward}
\end{figure}

Finally, multidimensional scaling (MDS) \cite{MEAD_1992_MDS} with $N$ = 2 dimensions was performed to visualise the clusters. We settled on K-means, as it presented better contained clusters, i.e., less overlapping between members of distinct clusters. The silhouette \cite{ROUSSEEUW_1987_SILHOUETTES}, elbow \cite{THORNDIKE_1953_ELBOW}, as well as Calinski-Harabasz index (CHI) \cite{CALINSKI_1974_CHI} and Davies-Bouldin index (DBI) \cite{DAVIES_1979_DBI} methods were used for finding optimal $K$ (\autoref{fig:kmeans_robustness}), in conjunction with the MDS, trees resulting from hierarchical clustering algorithms, and the visual representation of the RSA profiles, to decide on a final number of $K$ = 4 clusters: C1, C2, C3, and C4. Clustering was repeated 1,000 times with different random states and 289/293 (98.6\%) sites were always present in the same cluster, thus suggesting the clusters are robust.

\begin{figure}[ht!]
    \centering
    \includegraphics[width=\textwidth]{figures/ch_FRAGSYS/PNG/SUPP_FIG2_KNN_robustness.png}
    \caption[K-means clustering robustness]{\textbf{K-means clustering robustness.} Cluster analysis to assess the quality of the K-means clustering. For each $K \in$ [2, 14], clustering is bootstrapped 1,000 times with different initial random states. Error bars indicate 1 SD. \textbf{(A)} Calinski-Harabasz Index (CHI); \textbf{(B)} Davies-Bouldin Index (DBI); \textbf{(C)} Inertia; \textbf{(D)} Silhouette. All methods agree the optimal clustering of this dataset lies in $K \in$ [4, 6].}
    \label{fig:kmeans_robustness}
\end{figure}

\subsection{Binding site cluster prediction}

Two different predictive models were developed with the aim of classifying binding sites into the defined RSA-based clusters obtained with K-means, as described above. The first  uses the K-nearest neighbour (KNN) algorithm as implemented in Scikit-learn \cite{PEDREGOSA_2011_SKLEARN}, with $K$ = 3. The input for this KNN model is the rows of the $U_{D}$ matrix, containing the distances between pairs of binding site RSA profiles.

The second model is a multilayer perceptron (MLP) \cite{CYBENKO_1989_MLP}, a type of artificial neural network (ANN) constructed with Keras \cite{CHOLLET_2015_KERAS} with a single hidden fully connected layer between the input layer of 11 neurons, and the output layer of 4 neurons, one for each cluster label. RSA profiles present different lengths depending on the size (number of amino acids) of the binding site. As this input is not suitable for the neural network, binding sites were encoded as an 11-element vector. The first element of the vector encodes the size of the binding site relative to the maximum site size of 40 residues. The other 10 elements represent the proportion of residues forming the binding site with an RSA \% within a 10-unit interval: [0, 10), [10, 20), …, and [90, 100].

\subsubsection{MLP ablation}

A thorough hyperparameter optimisation was carried out by examining the effect that a series of hyperparameter changes have on the prediction accuracy relative to our current ML setup, labelled as \textit{current}. Sixty-four single-hyperparameter changes were performed, one at a time. For each variation, 100 models were trained with different seeds and the average validation accuracies compared to our current MLP. Sixty-four pairwise $t$-tests were conducted to compare the accuracy means, and Benjamini-Hochberg correction \cite{BENJAMINI_1995_FDR} applied. FDR and increment in accuracy, $\Delta_{acc}$ (\autoref{eq:delta_accuracy}) are used to describe the results, where $acc_{current}$ is the average validation accuracy of our \textit{current} ML setup across the 100 seeds, and $acc_{variant}$ is the average accuracy across 100 seeds of each one of the 64 variant models. $\Delta_{acc} <$ 0 represents a decrease in performance respect our current ML architecture, whereas $\Delta_{acc} >$ 0 corresponds to a higher accuracy. The results of this ablation study are described below and graphically represented in \autoref{fig:mlp_ablation} and \autoref{tab:MLP_ablation}.

\begin{equation}
\Delta_{acc} = acc_{variant} - acc_{current}
\label{eq:delta_accuracy}
\end{equation}
\myequations{Increment in MLP accuracy}

\paragraph{Number of layers}

Removing the single hidden layer resulted in a significant decrease in accuracy, $\Delta_{acc}$ = \textminus11\% (FDR $<$ 0.05). The addition of more layers did not improve accuracy: 2-layer  $\Delta_{acc}$ = \textminus1\% (FDR $<$ 0.05), 10-layer $\Delta_{acc}$ = \textminus8.9\% (FDR $<$ 0.05), or was not statistically different from our current setup baseline: 5-layer $\Delta_{acc}$ = \textminus0.15\% (FDR = 0.42).

\paragraph{Neurons per layer}

The addition of neurons N\textsubscript{neurons} = [11, 20, 25, 50, 100] in the single layer did not improve the current accuracy (FDR $>$ 0.05). The removal of neurons did not have an effect of performance for N\textsubscript{neurons} = [4, 5, 6, 7, 8, 9] (FDR $>$ 0.05), or a significant negative effect for 1 neuron, $\Delta_{acc}$ = \textminus15\% (FDR $<$ 0.05), 2 neurons $\Delta_{acc}$ = \textminus4\% (FDR $<$ 0.05), and 3 neurons, $\Delta_{acc}$ = \textminus1\% (FDR $<$ 0.05). This result suggests that 5 neurons on a single hidden layer might be enough to achieve a comparable accuracy to our current model.

\begin{figure}[ht!]
    \centering
    \includegraphics[width=\textwidth]{figures/ch_FRAGSYS/PNG/SUPP_FIG3_MLP_ablation.png}
    \caption[MLP ablation study]{\textbf{MLP ablation study.} Ablation study performed on the MLP. Sixty-four single hyperparameter changes are conducted one at a time to explore the hyperparameter space and the effect they have on the prediction accuracy relative to our current machine learning setup, labelled as \textit{current}. Box and whiskers represent the distribution of validation accuracy across 100 random seeds. Dashed lines mark the separation between different hyperparameters: number of layers, neurons, activation, loss functions, weight initialisers, optimisers, learning, dropout rates, and regularisation techniques.}
    \label{fig:mlp_ablation}
\end{figure}

\paragraph{Activation function}

The usage of different activation functions either negatively affected the accuracy of the MLP ($\Delta_{acc} <$ 0) or had no effect (FDR $>$ 0.05).

\paragraph{Loss function}

Different loss functions resulted in terrible loss of accuracy $\Delta_{acc} \approx$ \textminus50\% (FDR $<$ 0.05). This is expected as they are not appropriate for a multi-label classifier, unlike sparse categorical cross entropy.

\paragraph{Weight initialiser}

Most weight initialisers were tested and either negatively affected the accuracy of the MLP ($\Delta_{acc} <$ 0) or had no effect (FDR $>$ 0.05). However, \texttt{RandomNormal}, \texttt{RandomUniform}, and \texttt{TruncatedNormal} did improve the accuracy but by less than 1\%, $\Delta_{acc} <$ +1\% (FDR $<$ 0.05).

\paragraph{Optimiser}

Regarding optimisers, they either severely negatively affected accuracy $\Delta_{acc} \approx$ \textminus30\% (FDR $<$ 0.05), had no significant effect (FDR $>$ 0.05), or very slightly improved accuracy, such as \texttt{RMSProp} $\Delta_{acc} <$ +1\% (FDR $<$ 0.05).

\paragraph{Learning rate}

Extreme learning rates of 0.001 (too small), and 1.0 (too big) negatively affected prediction $\Delta_{acc} <$ \textminus5\% (FDR $<$ 0.05). Intermediate rates had either no significant effect (FDR $>$ 0.05) or a small effect $\|\Delta_{acc}\| <$ 1\%.

\paragraph{Dropout rate}

Regarding dropout rates, a rate = 75\%, negatively affected prediction  $\Delta_{acc} <$ \textminus2\%,(FDR $<$ 0.05). Lower dropout rates: 0.1, 0.25, and 0.33 did improve the accuracy, but the effect size is very small, $\Delta_{acc} <$ +1\% (FDR $<$ 0.05). This result agrees with the effect of the removal of neurons per layer and shows that fewer neurons on a single hidden layer might be enough to achieve a comparable accuracy to our current model, as dropping them out has no effect.

\paragraph{Regularisation}

Overall, implementing kernel, bias, or activity regularisation techniques did not improve prediction accuracy, but worsened it $\Delta_{acc} \ni$ [\textminus2.56, \textminus0.46] (FDR $<$ 0.05).

\begin{longtable}{|c|c|c|c|}
\hline
\textbf{Model}              & \textbf{\% Accuracy} & \textbf{\% $\Delta_{acc}$} & \textbf{FDR} \\ \hline
\endfirsthead
%
\multicolumn{4}{c}%
{{\bfseries Table \thetable\ (continued)}} \\
\hline
\textbf{Model}              & \textbf{\% Accuracy} & \textbf{\% $\Delta_{acc}$} & \textbf{FDR} \\ \hline
\endhead
%
\textit{current}            & 93.9              & -                       & -            \\ \hline
\texttt{N\textsubscript{LAYERS} = 0}                   & 82.9              &  \textminus11.0                     & 0            \\ \hline
\texttt{N\textsubscript{LAYERS} = 2}                    & 92.9              &  \textminus1.0                      & 0            \\ \hline
\texttt{N\textsubscript{LAYERS} = 5}                    & 93.8              &  \textminus0.1                   & 0.42         \\ \hline
\texttt{N\textsubscript{LAYERS} = 10}                   & 85.0              &  \textminus8.9                   & 0            \\ \hline
\texttt{N\textsubscript{NEURONS} = 1}                  & 79.0              &  \textminus14.9                  & 0            \\ \hline
\texttt{N\textsubscript{NEURONS} = 2}                  & 89.8               &  \textminus4.2                   & 0            \\ \hline
\texttt{N\textsubscript{NEURONS} = 3}                  & 92.9              &  \textminus1.1                   & 0            \\ \hline
\texttt{N\textsubscript{NEURONS} = 4}                  & 93.7              &  \textminus0.3                   & 0.24         \\ \hline
\texttt{N\textsubscript{NEURONS} = 5}                  & 93.8              &  \textminus0.1                   & 0.54         \\ \hline
\texttt{N\textsubscript{NEURONS} = 6}                  & 93.7              &  \textminus0.2                   & 0.26         \\ \hline
\texttt{N\textsubscript{NEURONS} = 7}                  & 93.6              &  \textminus0.3                   & 0.08         \\ \hline
\texttt{N\textsubscript{NEURONS} = 8}                  & 93.6              &  \textminus0.3                   & 0.08         \\ \hline
\texttt{N\textsubscript{NEURONS} = 9}                  & 93.8              &  \textminus0.1                   & 0.49         \\ \hline
\texttt{N\textsubscript{NEURONS} = 11}                 & 94.0              & +0.1                    & 0.65         \\ \hline
\texttt{N\textsubscript{NEURONS} = 20}                 & 94.0              & +0.1                    & 0.68         \\ \hline
\texttt{N\textsubscript{NEURONS} = 25}                 & 94.0              &  \textminus0.0                   & 0.83         \\ \hline
\texttt{N\textsubscript{NEURONS} = 50}                 & 93.9              &  \textminus0.0                   & 0.91         \\ \hline
\texttt{N\textsubscript{NEURONS} = 100}                & 93.8              &  \textminus0.1                   & 0.49         \\ \hline
\texttt{sigmoid}                     & 92.4              &  \textminus1.5                   & 0            \\ \hline
\texttt{elu}                         & 94.1              & +0.2                    & 0.38         \\ \hline
\texttt{selu}                        & 93.7              &  \textminus0.2                   & 0.15         \\ \hline
\texttt{exponential}                         & 92.8              &  \textminus1.1                   & 0            \\ \hline
\texttt{tanh}                        & 93.8              &  \textminus0.1                   & 0.41         \\ \hline
\texttt{softplus}                    & 93.3              &  \textminus0.6                   & 0            \\ \hline
\texttt{softmax}                     & 90.0               &  \textminus3.9                   & 0            \\ \hline
\texttt{softsign}                    & 93.5              &  \textminus0.4                   & 0.02         \\ \hline
\texttt{MSE}             & 44.1              &  \textminus49.8                  & 0            \\ \hline
\texttt{Poisson}         & 43.6              &  \textminus50.3                  & 0            \\ \hline
\texttt{KLDivergence}             & 43.9              &  \textminus50.0                     & 0            \\ \hline
\texttt{HeNormal}       & 93.7              &  \textminus0.2                    & 0.37         \\ \hline
\texttt{HeUniform}        & 93.7              &  \textminus0.2                   & 0.21         \\ \hline
\texttt{RandomNormal}     & 94.4              & +0.5                    & 0.02         \\ \hline
\texttt{RandomUniform}      & 94.4              & +0.5                    & 0.01         \\ \hline
\texttt{TruncatedNormal}    & 94.5              & +0.6                    & 0            \\ \hline
\texttt{Ones}           & 90.0               &  \textminus3.9                   & 0            \\ \hline
\texttt{Zeros}          & 42.6              &  \textminus51.3                  & 0            \\ \hline
\texttt{GlorotNormal} & 93.9              &  \textminus0.0                   & 0.79         \\ \hline
\texttt{SGD}             & 46.7              &  \textminus47.2                  & 0            \\ \hline
\texttt{RMSProp}          & 94.5              & +0.6                    & 0            \\ \hline
\texttt{Adadelta}         & 29.4              &  \textminus64.5                   & 0            \\ \hline
\texttt{Adagrad}         & 55.1              &  \textminus38.8                  & 0            \\ \hline
\texttt{Adamax}           & 94.1              & +0.2                    & 0.6          \\ \hline
\texttt{Nadam}            & 94.1              & +0.2                    & 0.48         \\ \hline
\texttt{Ftrl}             & 34.7              &  \textminus59.2                  & 0            \\ \hline
\texttt{Learning rate = 0.001}       & 89.0              &  \textminus4.9                   & 0            \\ \hline
\texttt{Learning rate = 0.005}       & 93.5              &  \textminus0.4                   & 0.04         \\ \hline
\texttt{Learning rate = 0.05}        & 93.6              &  \textminus0.3                   & 0.06         \\ \hline
\texttt{Learning rate = 0.1}         & 94.3              & +0.4                    & 0.05         \\ \hline
\texttt{Learning rate = 0.25}        & 93.1              & \textminus0.8                   & 0.27         \\ \hline
\texttt{Learning rate = 0.5}         & 88.2              &  \textminus5.7                    & 0            \\ \hline
\texttt{Learning rate = 1.0}         & 69.6               &  \textminus24.3                  & 0            \\ \hline
\texttt{Dropout rate = 0.1}                & 94.5              & +0.6                    & 0            \\ \hline
\texttt{Dropout rate = 0.25}               & 94.6              & +0.7                    & 0            \\ \hline
\texttt{Dropout rate = 0.33}               & 94.7              & +0.8                     & 0            \\ \hline
\texttt{Dropout rate = 0.5}                & 94.1              & +0.2                    & 0.54         \\ \hline
\texttt{Dropout rate = 0.75}               & 91.9              &  \textminus2.0                   & 0            \\ \hline
\texttt{L1}                 & 92.5              &  \textminus1.4                   & 0            \\ \hline
\texttt{L2}                 & 93.5              &  \textminus0.4                   & 0.04         \\ \hline
\texttt{L1L2}               & 92.0              &  \textminus1.9                   & 0            \\ \hline
\texttt{Orthogonal rows}        & 93.5              &  \textminus0.4                   & 0.02         \\ \hline
\texttt{Orthogonal columns}        & 93.9              &  \textminus0.0                   & 0.62         \\ \hline
\texttt{\textit{all} L1}            & 91.8              &  \textminus2.1                   & 0            \\ \hline
\texttt{\textit{all} L2}            & 92.2              &  \textminus1.7                    & 0            \\ \hline
\texttt{\textit{all} L1L2}          & 91.4              &  \textminus2.5                   & 0            \\ \hline
\caption[MLP ablation study]{\textbf{MLP ablation study.} Sixty-four single hyperparameter changes were conducted one at a time to explore the hyperparameter space and the effect they have on the prediction accuracy relative to our current ML setup, labelled as \textit{current}. \% Accuracy represents the validation accuracy average across 100 random seeds. \% $\Delta_{acc}$ represents the difference in accuracy between the variant MLP model and our \textit{current} setup. Negative values result from a decrease in performance, whereas positive ones mean an improvement in classification accuracy. FDR was employed to assess the significance of these differences.}
\label{tab:MLP_ablation}\\
\end{longtable}

\subsubsection{Performance evaluation}

The complete dataset ($N$ = 293) was split into a blind test set (1/11 = 27), and a training set (10/11 = 266). Ten repeats of a stratified 10-fold cross-validation were performed to assess the robustness of the ANN and compare it with the KNN model, as well as a baseline of the same models trained on randomly shuffled data and completely random label assignment ($p$ = 0.25). The reliability of the ANN predictions was assessed by means of a confidence score calculated as in \cite{CUFF_2000_PROFILES}, which represents how certain the MLP is of each individual prediction (\autoref{eq:confidence_score}). The score is based on the difference between the top- and second-class probabilities assigned by the network to each of the classes, $p_{1}$, and $p_{2}$, respectively. For example, if the output of the network were $P$ = [0.95, 0.02, 0.03, 0.0]. The probabilities would be sorted, so $p_{1}$ = 0.95, $p_{2}$ = 0.03, and a confidence score of 9 would be obtained.

\begin{equation}
\text{confidence score} = \left\lfloor 10 \times (p_1 - p_2) \right\rfloor
\label{eq:confidence_score}
\end{equation}
\myequations{Confidence Score}

The KNN is based on distances to all training data and so, as expected, consistently gives higher classification accuracy than the ANN model where sites are represented by their binned RSA profile, and are thus completely unaware of other sites, and their distances to them (\autorefpanel{fig:MLP_CV_blind_test}{A}). Both methods are significantly better than random. The average cross-validation accuracy across all repeats is of 98\%, 90\%, 33\%, 31\%, and 24\% for KNN, ANN models, their randomly trained versions, and completely random label assignment, respectively. The baseline accuracy of the randomly trained models is higher than 25\% since the dataset is unbalanced, with classes, C1 and C2 overrepresented.

\begin{figure}[ht!]
    \centering
    \includegraphics[width=\textwidth]{figures/ch_FRAGSYS/PNG/FIG10_ml_results.png}
    \caption[MLP cross-validation and blind test results]{\textbf{MLP cross-validation and blind test results.} \textbf{(A)} Average accuracy of the 10-repeat 10-fold ($N$ = 100) cross-validation of the KNN, and ANN predictive models compared to a baseline of the same models trained on randomly shuffled data, as well as complete random prediction ($p$ = 0.25). The box represents the central 50\% of the data, i.e., Q1 – median (Q2) – Q3, also known as interquartile range (IQR). Whiskers extend to 1.5 $\times$ IQR, and beyond them are the outliers; \textbf{(B)} Cross-validation accuracy and proportion of binding sites against cumulative confidence score from the trained ANN. Sites presenting a confidence score greater or equal to 5, the average accuracy is 97\%, and the percentage of sites with this score is 75\%. Predictions are for the 2,660 cross-validation data points, 10 different repeats of 10 distinct splits of 26-27 binding sites each. Accuracy error bars indicate 95\% CI of the proportion \cite{WILSON_197_PROP_CI}; \textbf{(C)} MDS representation of the 293 binding sites. Training data are coloured according to the average confidence of their prediction in the cross-validation. Test data are coloured according to whether they were correctly predicted or not. Dashed lines indicate the limits of K-means clusters.}
    \label{fig:MLP_CV_blind_test}
\end{figure}

\autorefpanel{fig:MLP_CV_blind_test}{B} shows the confidence of the ANN predictions across the 10 repeats of the 10-fold cross-validation. The overall accuracy is 90\%. Those predictions presenting a confidence score greater or equal to 5 present an accuracy of 97\% and cover 75\% of all predictions. Finally, \autorefpanel{fig:MLP_CV_blind_test}{C} shows the same two-dimensional representation of the K-means clusters found on \autorefpanel{fig:fragsys_bs_clust}{D} and demonstrates that those binding sites with lower prediction confidence are mostly located at the borders between clusters. Sites that switch cluster labels depending on the seed are also located in these regions.

Once the model hyperparameters were optimised, 50 models were trained on 10/11 of the data ($N$ = 266) for the 10 different seeds used to initialise the models. From a final pool of 500 models, the one presenting the highest validation accuracy and lowest validation loss was chosen, with a validation accuracy of 96\%. This model, as well as KNN were used to predict on the blind test set. There is no significant difference in performance of the ANN and KNN models. Accuracies are 26/27 = 0.96, 95\% CI = [0.82, 0.99], and 27/27 = 1.0, 95\% CI = [0.88, 1.0], for ANN and KNN, respectively. The adjusted Rand index (ARI) \cite{RAND_1971_ARI, HUBERT_1985_ARI}, as well as adjusted mutual information (AMI) \cite{VINH_2009_AMI, VINH_2010_AMI} were calculated. $ARI_{ANN}$ = 0.93, 95\% CI = [0.81, 1.0] \cite{STEINLEY_2016_ARI}, $AMI_{ANN}$ = 0.93, 95\% CI = [0.82, 1.0]. $ARI_{KNN}$ = 1.0, $AMI_{KNN}$ = 1.0. 95\% CI of AMI was calculated by bootstrap resampling ($N$ = 10,000). The three metrics all agree on the high performance of the MLP. \autorefpanel{fig:MLP_CV_blind_test}{C} illustrates how the binding site, which label was wrongly predicted by the ANN model is located on the limits between adjacent clusters C3 and C4. This result agrees with the K-means clustering reliability, and confidence score analysis of the cross-validation, where the same inter-cluster regions are highlighted due to their lower clustering reliability, and low confidence prediction. This suggests that the core of the clusters is stable, and that the ANN confidence score may be used to identify binding sites that are at the borders of clusters.

\subsection{Site function classification}

Ligand binding sites were divided into two groups \textit{known function} and \textit{unknown function} by searching UniProt \cite{UNIPROT_2019_UNIPROT} for feature annotations indicative of function, e.g., metal, substrate binding, active site, etc via the UniProt proteins API \cite{NIGHTINGALE_2017_API}. Seventeen out of the 35 proteins presented at least one UniProt annotated residue in one binding site. Manual curation using protein homology within the proteins in the data set added 9 more functionally annotated proteins. This gave a total of 44 sites from 26 proteins classified as of known function. All other sites were classified as unknown function.

\subsection{Statistics and reproducibility}

All data analysis was carried out primarily with  the following Python libraries: NumPy \cite{HARRIS_2020_NUMPY}, Pandas \cite{MCKINNEY_2010_PANDAS,PANDAS_2022_PANDAS} and SciPy. Keras and Scikit-learn were used for machine learning, with Matplotlib \cite{HUNTER_2007_MATPLOTLIB}, and Seaborn \cite{WASKOM_2021_SEABORN} for plotting. All statistical tests performed are two-tailed, and significance level $\alpha$ = 0.05. Sample sizes and measures of significance are reported and described in the text, figures and legends.

\subsection{Data availability}

The main summary result tables resulting from this analysis are available in the following repository: \url{https://github.com/bartongroup/FRAGSYS} (DOI: 10.5281/zenodo.10606595) \cite{UTGES_2024_FRAGSYS_ZENODO}.

\subsection{Code availability}

Software developed to carry out this analysis is also found in our GitHub repository: \url{https://github.com/bartongroup/FRAGSYS } (DOI: 10.5281/zenodo.10606595).

\section{Results}

\subsection{Defined binding sites}

The focus here is on human proteins to allow the additional information from human population variation data to be explored. For this reason, two of the 39 protein domains (products of the Replicase polyprotein 1ab from \textit{SARS-CoV-2} (P0DTD1)) were removed since they did not include any human homologues. The remaining 37 protein domains accounted for 1,309 three-dimensional structures that included interactions with 1,601 ligands of interest, of which 998 were unique. 293 ligand binding sites were defined across these domains, formed by 2,664 unique ligand binding residues. The total number of binding sites per domain ranges from 1 to 24, with 33/37 domains presenting more than one defined binding site. The median number of sites per domain is seven.

\autoref{fig:bs_def_examples} illustrates three examples of the 37 domains for which ligand binding sites were defined by the algorithm presented in this work. The grouping of the ligands into the defined sites reflects the similarity between the interaction fingerprints of the different ligands with the target protein domain.

\begin{figure}[ht!]
    \centering
    \includegraphics[width=\textwidth]{figures/ch_FRAGSYS/PNG/FIG1_bs_examples.png}
    \caption[Ligand clusters defined by the binding site definition algorithm]{\textbf{Ligand clusters defined by the binding site definition algorithm.} For simplicity, only one protein chain ribbon is shown in white for each example. Ligands are coloured according to the site they bind to. Identifiers are from UniProt. \textbf{(A)} There were 110 structures depicting human tyrosine-protein phosphatase non-receptor type 1 (\textit{PTPN1}), P18031, binding 143 ligand molecules, 104 of which were unique. 18 binding sites were defined; \textbf{(B)} The 68 ligands, 30 unique, found across 50 structures of the chestnut blight fungus endothiapepsin (\textit{EAPA}), P11838, were classified in 12 distinct binding sites; \textbf{(C)} For mouse mitogen-activated protein kinase 14 (\textit{Mapk14}), P47811, 52 structures portrayed the interaction with 53 ligand molecules, 50 unique, which clustered in 10 ligand binding sites.}
    \label{fig:bs_def_examples}
\end{figure}

\autoref{fig:bss_features} shows the 293 defined binding sites are diverse in size (number of amino acids), solvent accessibility, evolutionary divergence, and missense depletion. Binding site size ranges from 2-40 residues with a median of 9, while median site RSA ranges from 4-80\%, with a median of 30\%. For evolutionary divergence, the average site $N_{Shenkin}$ spread from 0-80, peaking at 40. Lastly, MES spans \textminus0.75 to 1.0, peaking at neutrality (MES $\approx$ 0).

\begin{figure}[ht!]
    \centering
    \includegraphics[width=0.6\textwidth]{figures/ch_FRAGSYS/PNG/FIG2_bs_feats_2x2.png}
    \caption[Variation in binding site features]{\textbf{Variation in binding site features.} Distribution of size \textbf{(A)}, median RSA \textbf{(B)} , $N_{Shenkin}$ \textbf{(C)} and MES \textbf{(D)} across the 293 binding sites defined from our data set. Black dashed lines indicate the median of each distribution.}
    \label{fig:bss_features}
\end{figure}

Despite the diversity among sites, some general trends can be observed. \autorefpanel{fig:bss_feats_corr}A shows that larger binding sites tend to be less accessible to solvent ($r$ = \textminus0.4, p $\approx$ 0). \autorefpanel{fig:bss_feats_corr}{B} illustrates that larger sites are less divergent across homologues ($r$ = \textminus0.21, $p$ = $10^{-4}$) while \autorefpanel{fig:bss_feats_corr}{C} presents how larger sites show lower enrichment in neutral missense variants within the human population, i.e., are on average more depleted in missense variants than sites of a smaller size ($r$ = \textminus0.15, $p$ = 0.008). Correlations between MES and $N_{Shenkin}$, and RSA and $N_{Shenkin}$ were not significant, i.e., 95\% CI $r \ni$ 0.

\begin{figure}[ht!]
    \centering
    \includegraphics[width=\textwidth]{figures/ch_FRAGSYS/PNG/FIG3_bs_feats_corr.png}
    \caption[Relation between different binding site properties]{\textbf{Relation between different binding site properties.} A regression line is fitted to all data points previous to binning, ($N$ = 293 binding sites), Pearson’s correlation coefficient $r$ \cite{RODGERS_1988_CORRELATION}, associated $p$-value and 95\% CI of $r$ \cite{BOWLEY_1928_R_CI}. Data points are grouped into bins according to different binding site size intervals, represented by box and swarm plots. \textbf{(A)} Median site RSA \% \textit{vs} binding site size, in amino acids; \textbf{(B)} Average $N_{Shenkin}$ \textit{vs} binding site size; \textbf{(C)} Average site MES \textit{vs} site size. Boxes represent the IQR, and whiskers extend to 1.5 $\times$ IQR.}
    \label{fig:bss_feats_corr}
\end{figure}

%\begin{figure}[ht!]
%    \centering
%    \includegraphics[width=0.85\textwidth]{figures/ch_FRAGSYS/PDF/FIG4_bs_clust_results_OPT.pdf}
%    \captionof{figure}[RSA-based binding site clusters and examples]{\textbf{RSA-based binding site clusters and examples.} \textbf{(A)} RSA profiles of the 293 binding sites that were grouped in four, C1-C4, clusters by K-means based on the difference between their RSA profiles ($U_{D}$). Each binding site is represented by a vector, plotted as a bar here. The elements of the vector represent the residues that form the binding site and are sorted according to their RSA, so buried residues are at the beginning of the vector (bottom), and more accessible residues towards the end (top). Each element of the vector, or section of the bar, is coloured according to RSA, using the matplotlib  \textit{cividis} colour palette. Within each cluster, binding sites are sorted based on the number of amino acids. Over each cluster, a line is drawn at RSA = 25\%; \textbf{(B)} Six examples of binding sites are shown in structure for each cluster. Examples were selected to represent the range of binding site sizes within each cluster. IDs are UniProt accession codes. Binding site residues are coloured according to their RSA, using the \textit{cividis} colour scheme. The rest of the protein is coloured in white. Ligands binding to the site in question are coloured in red.}
%\label{fig:bss_clust_results}
%\end{figure}

\begin{figure}[hb!]
    \centering
    \includegraphics[width=\textwidth]{figures/ch_FRAGSYS/PNG/FIG4_bs_clust_results_SPLIT_1.png}
    \captionof{figure}[Profiles of RSA-based binding site clusters]{\textbf{Profiles of RSA-based binding site clusters.} \textbf{(A)} RSA profiles of the 293 binding sites that were grouped in four, C1-C4, clusters by K-means based on the difference between their RSA profiles ($U_{D}$). Each binding site is represented by a vector, plotted as a bar here. The elements of the vector represent the residues that form the binding site and are sorted according to their RSA, so buried residues are at the beginning of the vector (bottom), and more accessible residues towards the end (top). Each element of the vector, or section of the bar, is coloured according to RSA, using the matplotlib  \textit{cividis} colour palette. Within each cluster, binding sites are sorted based on the number of amino acids. Over each cluster, a line is drawn at RSA = 25\%.}
\label{fig:bss_clust_profiles}
\end{figure}

\begin{figure}[ht!]
    \centering
    \includegraphics[width=\textwidth]{figures/ch_FRAGSYS/PNG/FIG4_bs_clust_results_SPLIT_2.png}
    \captionof{figure}[Examples of RSA-based binding site clusters]{\textbf{Examples of RSA-based binding site clusters.} \textbf{(B)} Six examples of binding sites are shown in structure for each cluster. Examples were selected to represent the range of binding site sizes within each cluster. IDs are UniProt accession codes. Binding site residues are coloured according to their RSA, using the \textit{cividis} colour scheme. The rest of the protein is coloured in white. Ligands binding to the site in question are coloured in red.}
\label{fig:bss_clusts_examples}
\end{figure}

\subsection{RSA-based binding site clustering}

\autoref{fig:bss_clust_profiles} depicts the four clusters defined by our method and the RSA profiles of the sites within them while \autoref{fig:bss_clusts_examples} illustrates six binding sites from each cluster to highlight the range of binding site size. Cluster 1 includes 46 sites, whereas 127 sites are found on C2, 91 in C3 and 29 in C4. The proportion of residues with an RSA \textless 25\% in \autoref{fig:bss_clust_profiles} follows a different profile in each cluster, which is confirmed in \autoref{fig:bs_clusts_feats_1}. C1 is the most buried with a proportion of residues with RSA \textless 25\% of 0.68, $(p_{RSA<25\%}) \approx$ 0.68), followed by C2 with $(p_{RSA<25\%}) \approx$ 0.47, then C3, $(p_{RSA<25\%}) \approx$ 0.30, and lastly C4 with $(p_{RSA<25\%}) \approx$ 0.10. \autoref{fig:bs_clusts_feats_1} displays the difference in binding site size between the clusters. There is variation within clusters in site size, but certain patterns are still apparent. C1 includes the largest sites, with an average size of $\bar{s}$ = 15 residues, followed by C2 with $\bar{s}$ = 11, then C3 with $\bar{s}$ = 8, and finally C4 with $\bar{s}$ = 5. \autorefpanel{fig:bs_clusts_feats_1}{C} shows the two-dimensional MDS representation of the binding sites, also shown in \autorefpanel{fig:fragsys_bs_clust}{D}, and \autorefpanel{fig:MLP_CV_blind_test}{C} in the Methods section. C1 and C4 are the most distinct amongst the clusters while there is some overlap between clusters. Sites near the cluster borders are those that switch groups depending on the random initialisation of the clustering. To summarise, C1 includes on average the largest, most buried sites, whereas C4 includes the smallest and most accessible. C2 and C3 are not as different as C1 and C4, but still differ in size and burial proportion with C2 including larger and overall, less accessible sites than C3.

\begin{figure}[ht!]
    \centering
    \includegraphics[width=\textwidth]{figures/ch_FRAGSYS/PNG/FIG5_bs_clust_feats_SPLIT_1.png}
    \caption[Binding site cluster features (I)]{\textbf{Binding site cluster features (I).} \textbf{(A)} Box plot of the proportion of residues with RSA \textless 25\% per binding site across the four clusters defined by K-means clustering;\textbf{(B)} Box plot of the binding site size, in amino acids, across clusters. Pairwise Mann-Whitney-Wilcoxon tests were performed to assess the differences between the clusters. Boxes represent the IQR, and whiskers extend to 1.5 $\times$ IQR. $p$-value annotation legend: ns: $p >$ 0.05, *: 0.01 $< p \leq$ 0.05, **: $10^{-2} < p \leq 10^{-3}$, ***: $10^{-4} < p \leq 10^{-3}$, ****: $p \leq 10^{-4}$; \textbf{(C)} MDS representation of the 293 binding sites on 2 dimensions. Data points represent binding sites and are coloured based on the cluster they group in.}
    \label{fig:bs_clusts_feats_1}
\end{figure}

These results support $U_{D}$ as a metric that effectively quantifies the difference between the solvent exposure and size properties of different binding sites with the four clusters encapsulating differences in RSA and binding site size. This effect might be explained by the negative correlation between solvent accessibility and binding site size shown in \autoref{fig:bss_feats_corr}.

%\begin{figure}[ht!]
%    \centering
%    \includegraphics[width=\textwidth]{figures/ch_FRAGSYS/PDF/FIG5_bs_clust_feats.pdf}
%    \caption[Binding site cluster features]{\textbf{Binding site cluster features.} \textbf{(A)} Box plot of the proportion of residues with RSA \textless 25\% per binding site across the four clusters defined by K-means clustering;\textbf{(B)} Box plot of the binding site size, in amino acids, across clusters. Pairwise Mann-Whitney-Wilcoxon tests were performed to assess the differences between the clusters. Boxes represent the IQR, and whiskers extend to 1.5 $\times$ IQR. $p$-value annotation legend: ns: $p >$ 0.05, *: 0.01 $< p \leq$ 0.05, **: $10^{-2} < p \leq 10^{-3}$, ***: $10^{-4} < p \leq 10^{-3}$, ****: $p \leq 10^{-4}$; \textbf{(C)} MDS representation of the 293 binding sites on 2 dimensions. Data points represent binding sites and are coloured based on the cluster they group in; \textbf{(D)} Histogram of RSA \% of the residues found within the ligand binding sites in each cluster; \textbf{(E)} Histogram of $N_{Shenkin}$ within cluster residues; \textbf{(F)} MES histogram plots for the 4 clusters defined.}
%    \label{fig:bs_clusts_feats}
%\end{figure}

\autorefpanel{fig:bs_clusts_feats_2}{A} shows the RSA distribution of all residues forming the binding sites within the defined clusters. This definition agrees, as expected, with \autoref{fig:bss_clust_profiles}, and  \autorefpanel{fig:bs_clusts_feats_1}{A}. C1 presents a distribution clearly different to the rest of clusters, peaking at RSA $\approx$ 5\%, indicating a high density of buried residues. C2 still presents an excess of buried residues relative to clusters 3-4, though not as high as C1. C4 presents the most different distribution to C1, peaking around RSA $\approx$ 50-70\%.

\begin{figure}[ht!]
    \centering
    \includegraphics[width=\textwidth]{figures/ch_FRAGSYS/PNG/FIG5_bs_clust_feats_SPLIT_2.png}
    \caption[Binding site cluster features (II)]{\textbf{Binding site cluster features (II).} \textbf{(A)} Histogram of RSA \% of the residues found within the ligand binding sites in each cluster; \textbf{(B)} Histogram of $N_{Shenkin}$ within cluster residues; \textbf{(C)} MES histogram plots for the 4 clusters defined.}
    \label{fig:bs_clusts_feats_2}
\end{figure}

To further characterise the defined clusters, the distributions of the normalised Shenkin divergence score ($N_{Shenkin}$) and Missense Enrichment Score (MES) of the residues found in the clusters were analysed (Figures \ref{fig:bs_clusts_feats_2}E, F). Regarding evolutionary divergence ( \autorefpanel{fig:bs_clusts_feats_2}{B}), C1 also presents a different distribution to the rest of the clusters, with a peak at $N_{Shenkin} \approx$ 5, i.e., most of the residues conforming the sites within this cluster are highly conserved. The other clusters present flatter distributions with increasing proportion of divergent residues ($N_{Shenkin} >$ 25) $p_{C2}$ = 0.55, $p_{C3}$ = 0.67, and $p_{C4}$ = 0.69. $N_{Shenkin}$ is a divergence score ranging from 0-100, therefore residues with $N_{Shenkin} <$ 25, $p_{C1}$ = 0.58, $p_{C2}$ = 0.45, $p_{C3}$ = 0.33, and $p_{C4}$ = 0.31, represent stronger residue conservation, or lower divergence, than $N_{Shenkin} >$ 25. This agrees with the pattern observed on the RSA distributions (\autoref{fig:bs_clusts_feats_1}), as buried residues tend to be evolutionarily conserved \cite{CHOTHIA_1986_CONSERVATION, RUSSELL_1994_UNCONSERVATION}. In terms of missense depletion (\autorefpanel{fig:bs_clusts_feats_2}{C}), the distribution of C1 is slightly shifted to the left, towards more negative values, i.e., more missense depleted residues, with $\overline{\text{MES}}_{C1}$ = \textminus 0.17. The distributions of C2-4 are not statistically different, but present increasing average missense enrichment scores: $\overline{\text{MES}}_{C2}$ = \textminus 0.07, $\overline{\text{MES}}_{C3}$ = \textminus 0.02, and $\overline{\text{MES}}_{C1}$ = +0.06. Once again, this pattern agrees with the ones observed with site size, solvent accessibility, and evolutionary divergence. Sites that are more buried tend to be bigger in size, more conserved across homologues, as well as depleted in missense variation in human.

\begin{figure}[htb!]
    \centering
    \includegraphics[width=0.50\textwidth]{figures/ch_FRAGSYS/PNG/FIG6_bs_clust_func_enrichment.png}
    \caption[Binding site cluster enrichment in known functional sites]{\textbf{Binding site cluster enrichment in known functional sites.} This enrichment score is an odds ratio (OR). Error bars indicate 95\% CI of the OR. Y-axis is in $\log_{10}$ scale. A pseudo-count of 1 was added to each cell of the contingency table, to be able to calculate the score.}
    \label{fig:bs_clusts_enrichment}
\end{figure}

\subsection{Clusters predict differential functional enrichment}

A key goal of this work is to identify which sites from a fragment screening experiment are most likely to be functional and so worth investigating further. \autoref{fig:bs_clusts_enrichment} shows the relative enrichment in functional sites across the four defined clusters. C1 is the most enriched in functional sites, with 17/46 sites being classed as of known function, (OR = 4.46, $p \approx$ 0). C2 was next with 21/127 (OR = 1.15, $p$ = 0.75). C3 with 6/91 is depleted relative to the other clusters, (OR = 0.33, $p$ = 0.01), and finally C4 with 0/29, (OR = 0.16,$p$ = 0.04). RSA-based defined clusters are differentially enriched in functional sites. Based on their enrichment, a binding site found in C1 is $\approx$4, $\approx$14, and $\approx$28-fold more likely to be functional than a site in C2, C3, and C4, respectively.

Functional definitions in UniProt tend to lag behind the literature. A literature search found support for 12 sites in C1 that are without UniProt annotations with two examples discussed below. We found no literature support for the remaining seventeen sites in C1 suggesting they may be novel, functionally important sites.  \autoref{tab:novel_c1_sites} shows the full list of C1 sites that are predicted to be functionally important with 2/17 examples discussed below.

\subsection{Example C1 site functional predictions supported by literature but not annotated in UniProt}

\subsubsection{NS3 protein from Zika virus -- Q32ZE1}

The Zika virus (ZIKV) genome polyprotein (Q32ZE1) is 3,419 amino acids long and codes for three structural proteins: capsid (C), envelope (E), and membrane (M) as well as seven non-structural proteins: (NS1, NS2A, NS2B, NS3, NS4A, NS4B, and NS5). NS3 is a critical serine proteinase for viral polyprotein processing and genomic regulation. It includes a protease domain at the N-terminus, and a helicase domain on the C-terminus. The helicase is responsible for RNA unwinding during replication, and thus makes an interesting drug target against ZIKV \cite{LUO_2015_FLAVIVIRUS}.

\begin{figure}[htb!]
    \centering
    \includegraphics[width=\textwidth]{figures/ch_FRAGSYS/PNG/FIG7_c1_example_1.png}
    \caption[Binding site 7 of Zika virus NS3]{\textbf{Binding site 7 of Zika virus NS3.} Non-structural protein NS3 of Zika virus (Q32ZE1) binding to N-(2-methoxy-5-methylphenyl)glycinamide, NY7 in BS7 (PDB: 5RHG) (Godoy AS, Mesquita NCMR, Oliva G). Domains I, II, and III are coloured in pink, blue, and green respectively. Binding site 7 which is in Cluster 1 is highlighted, the other 9 binding sites which fall in C2 (3), C3 (3) and C4 (3) are hidden. Ligand binding residues in red, and NY7 in yellow. Protein-ligand interactions are represented by black lines.}
    \label{fig:c1_example_1}
\end{figure}

There are 10 sites in NS3 identified from 17 structures with 17 unique ligands and all are functionally unannotated in UniProt. The analysis here shows binding site 7 (BS7) to lie in Cluster 1 and so is most likely to be functional.

The site is located between domains I-III, involving residues from $\eta$2, $\alpha$3 on domain I, and $\alpha$10, $\alpha$11 on domain III as defined in \cite{TIAN_2016_ZIKV} (\autoref{fig:c1_example_1}). Mottin \textit{et al.} \cite{MOTTIN_2017_ZIKA_HELICASE} predicted four RNA binding sites on NS3. One of them, the RNA exit crevice is located between domains I–III, and involves $\alpha$3, $\alpha$10 residues. Raubenolt \textit{et al.} \cite{RAUBENOLT_2021_ZIKA_ALLOSTERIC} probed four different allosteric sites on this protein. One of them, D3, was manually curated and included $\alpha$11, $\alpha$12, and overlapped with BS7. Later, Durgam and Guruprasad \cite{DURGAM_2022_ZIKA_ATP} stated that four of the ten residues forming this site: Ala264, Thr265, Lys537 and Asp540 bind to RNA when this is in complex with NS3. These results strongly suggest that this region plays an important role in RNA binding to NS3 and so is a site to target to modulate function. Moreover, the site is on average missense-depleted: MES =  \textminus0.28. A264 ($N_{Shenkin}$ = 18, MES = \textminus0.79), T267 ($N_{Shenkin}$ = 53, MES = \textminus0.55), and S293 ($N_{Shenkin}$ = 72, MES =  \textminus0.48) are the three key positions out of the 10 forming this binding site, as they are all constrained within the human orthologs of this protein. A264 is conserved across homologues, whereas T267 and S293 are divergent while missense depleted so could be important for binding specificity.

\subsubsection{NSP13 protein from SARS-CoV-2 -- P0DTD1}

The Severe acute respiratory syndrome coronavirus 2 (SARS-CoV-2) replicase polyprotein 1ab (P0DTD1) is 7,096 amino acids long and codes for 16 non-structural proteins \cite{NAQVI_2020_SARSCOV2}. NSP13 is a helicase that unwinds dsRNA in the 5’-3’ direction to provide a single-stranded template for viral RNA amplification \cite{YUE_2022_SARSCOV2}. NSP13 also has NTPase activity, which provides the energy for the RNA unwinding \cite{SHU_2020_SARSCOV2}. NSP13 plays a fundamental role in the replication and transcription of the SARS-CoV-2 genome and is thought to be a good drug target against SARS-CoV-2 virus infection \cite{ZENG_2021_SARSCOV2}. NSP13 has five domains. Two ``RecA like'' subdomains 1A and 2A, in charge of nucleotide binding and hydrolysis, as well as three other domains: an N-terminal zinc-binding domain, the helical ``stalk'' domain, and a beta-barrel 1B domain \cite{ROMEO_2022_SARSCOV2}. It is the most conserved protein across coronaviruses, with sequence identity $>$99\% \cite{RICCI_2022_SARSCOV2}.

\begin{figure}[htb!]
    \centering
    \includegraphics[width=\textwidth]{figures/ch_FRAGSYS/PNG/FIG7_c1_example_2.png}
    \caption[Binding site 6+16 of SARS-CoV-2 NSP13]{\textbf{Binding site 6+16 of SARS-CoV-2 NSP13.} Non-structural protein NSP13 of SARS-CoV-2 (P0DTD1) binding to 3 ligands in BS6+16 (Ribbon PDB: 5RMH) \cite{NEWMAN_2021_SARSCOV2}. 1A, 1B, 2A, stalk and zinc domains are coloured in yellow, pink, green, brown, and grey respectively. Ligand binding residues in red, and ligands in yellow. Interactions are not shown here for simplicity.}
    \label{fig:c1_example_2}
\end{figure}

Twenty-four sites are defined on the surface of NSP13. Our method identifies two binding sites: BS6, and BS16 as C1 (\autoref{fig:c1_example_2}). Visual inspection shows the two sites to be adjacent with a total of 16 residues. Three fragments bind to the site, which is located in the nucleotide and RNA binding interface of NSP13 between the 1B and 2A domains. This is the region where the 5’ end of the RNA binds \cite{YAN_2020_SARSCOV2}. This pocket is determined to be highly druggable, and drugs binding to it might be effective against other coronaviruses, due to the pocket’s high amino acid conservation \cite{NEWMAN_2021_SARSCOV2}. This agrees with our results, as this site has an average $N_{Shenkin}$ = 32, and MES = \textminus0.18. Of the 16 positions in this site, four show high conservation across homologues and missense depletion in human: P514 ($N_{Shenkin}$ = 30, MES = \textminus0.56), D534 ($N_{Shenkin}$ = 9, MES = \textminus0.56), T552 ($N_{Shenkin}$ = 48, MES = \textminus1.87), and H554 ($N_{Shenkin}$ = 36, MES = \textminus0.85). T552 shows highest conservation across species and lowest missense enrichment (\textminus1.87) and so is most likely to have a key function in this protein family.

\subsection{Examples of potentially novel C1 cluster functional predictions}

\subsubsection{Human tenascin (TN) -- P24821}

Human tenascin, is a hexameric extracellular matrix glycoprotein implicated in a variety of functions including cell migration, cell attachment, matrix assembly and proinflammatory cytokine synthesis \cite{BHATTACHARYYA_2022_TNC}. TN is known to interact with viruses and play a role in viral infections, e.g., HIV-1, and has been reported as a biomarker for disease severity \cite{ZULIANI_2023_TNC}. It also plays a key role in wound healing \cite{WANG_2022_TNC}, and is involved in diverse cardiovascular diseases \cite{KHOMTCHOUK_2022_TNC}, as well as in breast cancer \cite{LEPUCKI_2022_TNC}. For these reasons, there is considerable effort put into understanding better the function of TN and targeting it for therapeutic effect.

\begin{figure}[htb!]
    \centering
    \includegraphics[width=\textwidth]{figures/ch_FRAGSYS/PNG/FIG7_c1_example_3.png}
    \caption[Binding site 0 of human tenascin]{\textbf{Binding site 0 of human tenascin.} Human tenascin, TN, (P24821) binding to 8 ligands in BS0. (Ribbon PDB: 5R60) (Coker JA, Bezerra GA, von Delft F, Arrowsmith CH, Bountra C, Edwards AM, Yue WW, Marsden BD). A, B, and P subdomains as defined by Yee, Pratt \cite{YEE_1997_FIBRINOGEN} are coloured in blue, grey, and green respectively.}
    \label{fig:c1_example_3}
\end{figure}

The data includes 11 structures with 11 unique ligands binding to TN, grouped in four binding sites, none annotated as functional in UniProt. One of the four binding sites is in C1, and so predicted to be of functional importance. The site is found on the Fibrinogen C-terminal domain of the protein, which functions as a molecular recognition unit that interacts with either proteins or carbohydrates ( \autoref{fig:c1_example_3}). This site shows high conservation across species ($N_{Shenkin}$ = 15), as well as missense-depleted in human (MES = \textminus0.33). Accordingly, we suggest that this site in TNC is likely to be of key importance to function. Within the 15 positions forming the site, V2012 ($N_{Shenkin}$ = 5, MES = \textminus1.0), G2046 ($N_{Shenkin}$ = 0, MES = \textminus0.67), F2047 ($N_{Shenkin}$ = 0, MES = \textminus0.67), W2055 ($N_{Shenkin}$ = 0, MES = \textminus0.54), and G2057 ($N_{Shenkin}$ = 0, MES = \textminus0.83), are the most critical interacting residues, and extremely conserved across homologues.

\subsubsection{5-aminolevulinate synthase (ALAS-E) -- P22557}

\textit{ALAS2} is a gene located on the X chromosome that codes for the human mitochondrial erythroid-specific 5-aminolevulinate synthase. This dimeric enzyme carries out the first and rate-limiting step of the haem synthesis pathway: the pyridoxal 5’-phosphate (PLP)-dependent condensation of succinyl-CoA and glycine to form aminolaevulinic acid \cite{AKHTAR_1976_PORPHYRIN}. Across eukaryotes, these enzymes have developed extensions surrounding the catalytic core on both the N and C-termini \cite{MUNAKATA_1993_AMINOLEVULINATE}. The N-terminal extensions include the mitochondrial targeting sequence \cite{SRIVASTAVA_1988_AMINOLEVULINATE}, whereas the C-terminal extension (C-ext) plays an autoinhibitory role by regulating substrate binding and product release \cite{BAILEY_2020_AMINOLEVULINATE}. Mutations affecting C-ext can result in gain-of-function, such as X-linked protoporphyria \cite{WHATLEY_2008_AMINOLEVULINATE}, as well as loss-of-function disorders, e.g., X-linked sideroblastic anaemia \cite{DUCAMP_2011_SIDEROBLASTIC}. Accordingly, ALAS-E is a potential therapeutic target for the treatment of such diseases.

\begin{figure}[htb!]
    \centering
    \includegraphics[width=\textwidth]{figures/ch_FRAGSYS/PNG/FIG7_c1_example_4.png}
    \caption[Binding site 1 of human ALAS-E]{\textbf{Binding site 1 of human ALAS-E.} Human erythroid-specific mitochondrial 5-aminolevulinate synthase, ALAS-E, (P22557) binding to 7 ligands in BS1. (Ribbon PDB: 5QR0) (Bezerra GA, Foster W, Bailey H, Shrestha L, Krojer T, Talon R, Brandao-Neto J, Douangamath A, Nicola BB, von Delft F, Arrowsmith CH, Edwards A, Bountra C, Brennan PE, Yue WW). Subunits A, B, C-terminal extensions A, B, as well as PLP cofactors are coloured in grey, beige, green, orange, and purple, respectively. Ligand binding residues in red, and ligands in yellow.}
    \label{fig:c1_example_4}
\end{figure}

We considered 25 structures with 33 unique ligands binding to ALAS-E, grouped in ten binding sites, only one of which is annotated as functional in UniProt. We classify three sites as C1. Two are known to be on the interface between subunits, form key interactions to maintain the assembly and are close to the PLP binding site \cite{BAILEY_2020_AMINOLEVULINATE}. However, one (BS1) is not mentioned in the literature. This site is located on a deep pocket at the N-terminal region of the protein structure (\autoref{fig:c1_example_4}). Amino acids in this site are strongly conserved as well as depleted in missense variation: $N_{Shenkin}$ = 29, MES = \textminus0.13. Together, this suggests the site has a functional role in the protein, perhaps as an allosteric regulator, or through interaction with a partner such as succinate-CoA ligase, SCS-$\alpha$ \cite{FURUYAMA_2000_SIDEROBLASTIC}. Out of the 16 residues forming the site, K381 ($N_{Shenkin}$ = 38, MES = \textminus0.94) is the most missense-depleted position in the site and should be considered for lead optimisation of a fragment binding to this site.

\section{Discussion}

In this paper we have presented a method to identify binding sites from fragment screening data and group the sites into four robust clusters by an RSA profile metric. 29/46 sites in Cluster 1 have functional support from the literature (UniProt 17/46 -- \autoref{tab:literature_c1_sites}; Our search 12/46 -- \autoref{tab:novel_c1_sites}) 17 further sites have similar profiles, but we could not find evidence in the literature of functional significance. We show two examples from this set that have compelling support from conservation and missense depletion scores for functional significance and we list all sites in \autoref{tab:novel_c1_sites} as a resource for further experimentation on these proteins.

As a case study, we applied the method to the SARS-CoV-2 main protease, MPro (P0DTD1). Twenty-five sites were defined from 511 structures, from which 8 were classed as C1, 12 as C2, 3 as C3 and only 2 as C4. Of the 8 C1 sites, one corresponds to the active site and three to allosteric sites 1, 2, and 3 \cite{DASGUPTA_2022_ALLOSTERIC} respectively. A further C1 site  is at the dimer interface and known to be a potential allosteric site \cite{DOUANGAMATH_2020_SARSCOV2} (\autoref{fig:MPro_showcase}). The remaining three C1 sites may be important, but each binds only a single ligand and their function is currently unclear.

Here, we have focused on a small set of proteins heavily studied by fragment screening methods. However, our method can be applied to classify any ligand binding site or predicted site. Accordingly, future work will seek to classify all known ligand binding sites in the PDB and provide tools to predict likely functional sites predicted sites by tools such as P2Rank \cite{KRIVAK_2018_P2RANK}, or GRaSP \cite{SANTANA_2020_GRaSP, SANTANA_2022_GRaSP} from Alphafold2 \cite{JUMPER_2021_ALPHAFOLD, VARADI_2022_ALPHAFOLDDB} or other models.

\begin{figure}[ht!]
    \centering
    \includegraphics[width=\textwidth]{figures/ch_FRAGSYS/PNG/SUPP_FIG4_MPro_C1_sites.png}
    \caption[SARS-CoV-2 MPro fragment screening]{\textbf{SARS-CoV-2 MPro fragment screening.} \textbf{(A)} Twenty-five defined ligand binding sites on the SARS-CoV-2 main protease, MPro (P0DTD1) from 971 ligands from 511 structures; \textbf{(B)} Five of the 9 C1 sites included the known MPro active site, and four known potential allosteric sites \cite{DOUANGAMATH_2020_SARSCOV2, DASGUPTA_2022_ALLOSTERIC}.}
    \label{fig:MPro_showcase}
\end{figure}

It is natural to focus on sites that are most likely to be of functional significance and so possible targets to modulate function. However, binding sites identified here that are predicted to be least likely to have function may also be interesting as good locations for tagging proteins for degradation \cite{BEKES_2022_PROTACS}, phosphorylation \cite{SIRIWARDENA_2020_PHOSPHO}, dephosphorylation \cite{SIMPSON_2023_ADPROM}, or other modulation \cite{HEITEL_2023_PTMS, PENG_2023_PTMS}.

\begin{landscape}
\begin{longtable}{|M{22mm}|M{15mm}|M{20mm}|M{13mm}|M{13mm}|M{10mm}|M{10mm}|M{48mm}|M{48mm}|}
\hline
\textbf{UniProt ID} & \textbf{\% RSA}  & $N_{Shenkin}$ & \textbf{MES}   & $p$    & \textbf{\# aas} & \textbf{\# ligs} & \textbf{UniProt residue numbers}                                                                                       & \textbf{Literature support}                                                   \\ \hline
\endfirsthead
\multicolumn{9}{c}%
{{\bfseries Table \thetable} (continued)} \\
\hline
\textbf{UniProt ID} & \textbf{\% RSA}  & $N_{Shenkin}$ & \textbf{MES}   & $p$    & \textbf{\# aas} & \textbf{\# ligs} & \textbf{UniProt residue numbers}                                                                                       & \textbf{Literature support}
%
\endhead
%
Q32ZE1     & 17.4 & 38.4     & \textminus0.21 & 0.02 & 10          & 1          & 1762, 1763, 1765, 1766, 1769, 1791, 1991, 2034, 2035, 2038                                              & RNA binding \cite{DURGAM_2022_ZIKA_ATP}, RNA exit site \cite{MOTTIN_2017_ZIKA_HELICASE}, D3 site \cite{RAUBENOLT_2021_ZIKA_ALLOSTERIC}         \\ \hline
Q9Y2J2     & 14.6 & 38.2     & +0.01  & 0.84 & 15          & 1          & 117, 118, 119, 203, 206, 207, 210, 231, 232, 235, 236, 253, 282, 283, 286                               & GPC binding \cite{HAN_2000_CYTOSKELETON}                                                 \\ \hline
Q9Y2J2     & 13.4 & 43.3     & +0.02  & 0.7  & 21          & 4          & 154, 161, 162, 163, 164, 185, 186, 189, 208, 212, 217, 295, 297, 298, 299, 300, 301, 315, 375, 376, 379 & Calmodulin binding \cite{HAN_2000_CYTOSKELETON}                                          \\ \hline
Q8WS26     & 16.2 & 28.9     & \textminus0.22 & 0.26 & 19          & 2          & 105, 106, 107, 108, 109, 112, 151, 154, 155, 158, 159, 162, 170, 171, 173, 174, 175, 176, 179           & IPP, DMAPP binding \cite{MUNZKER_2020_FARNESYL, GABELLI_2006_FARNESYL}                                      \\ \hline
Q8WS26     & 22.1 & 31       & +0.18  & 0.58 & 8           & 2          & 308, 312, 315, 316, 320, 324, 384, 423                                                                  & IPP binding \cite{GABELLI_2006_FARNESYL}                                                 \\ \hline
P18031     & 20.8 & 33.9     & +0.05  & 0.48 & 14          & 1          & 1, 2, 3, 4, 6, 10, 19, 242, 243, 244, 245, 246, 247, 271                                                & Conformational change \cite{KEEDY_2018_PTP1B}, Cluster II \cite{CUI_2017_ALLOSTERIC}                   \\ \hline
P47811     & 17.1 & 55       & +0.08  & 0    & 19          & 10         & 191, 192, 197, 198, 232, 236, 242, 246, 249, 250, 251, 252, 255, 259, 291, 292, 293, 294, 296           & MAP insert motif, Trp197 pocket \cite{FRANCIS_2013_P38A, NICHOLS_2020_P38A} \\ \hline
Q6B0I6     & 15.8 & 41.8     & +0.12  & 0.43 & 12          & 5          & 193, 224, 225, 227, 228, 239, 240, 241, 242, 243, 277, 279                                              & Cryptic binding site \cite{PEARCE_2017_CRYPTIC}                                        \\ \hline
P0DTD1     & 12.9 & 34.3     & \textminus0.13 & 0.45 & 12          & 2          & 5501, 5503, 5809, 5810, 5811, 5838, 5839, 5840, 5841, 5856, 5858, 5878                                  & RNA binding \cite{NEWMAN_2021_SARSCOV2} \\ \hline
P0DTD1     & 22.3 & 51.5     & \textminus0.04 & 0.87 & 9           & 1          & 5806, 5809, 5810, 5811, 5839, 5874, 5876, 5878, 5879                                                    & RNA binding \cite{NEWMAN_2021_SARSCOV2} \\ \hline
P22557     & 16   & 47.8     & \textminus0.09 & 0.61 & 16          & 10         & 148, 152, 155, 267, 268, 271, 272, 409, 413, 506, 570, 572, 573, 574, 575, 576                          & Dimerisation interface \cite{BAILEY_2020_AMINOLEVULINATE} \\ \hline
P22557     & 12.7 & 53.1     & +0.08  & 0.61 & 7           & 2          & 271, 293, 294, 295, 296, 297, 575                                                                       & Conformational change, PLP binding, succinyl-CoA inhibition \cite{BAILEY_2020_AMINOLEVULINATE} \\ \hline
\caption[Literature supported C1 sites]{\textbf{Literature supported C1 sites.} These are 12 C1 sites with no functional annotations in UniProt, therefore labelled as \textit{unknown function}, for which literature has been found that support their functional relevance. UniProt ID indicates the protein UniProt accession. \% RSA is the median site RSA. $N_{Shenkin}$ is the average normalised Shenkin score for the site. MES is the average missense enrichment score for the site. $p$ is the $p$-value associated to this site MES. \# aas is the number of residues forming the site. \# ligs is the number of ligands binding to the site. UniProt residue numbers is a list of the UniProt residue numbers of the residues forming the site. Literature support contains a brief description of the literature-reported site function and references.}
\label{tab:literature_c1_sites}\\
\end{longtable}
\end{landscape}

\begin{landscape}
\begin{longtable}{|M{22mm}|M{15mm}|M{20mm}|M{13mm}|M{13mm}|M{10mm}|M{10mm}|M{96mm}|}
\hline
\textbf{UniProt ID} & \textbf{\% RSA}  & $N_{Shenkin}$ & \textbf{MES}   & $p$    & \textbf{\# aas} & \textbf{\# ligs} & \textbf{UniProt residue numbers}                                                                                                                                    \\ \hline
\endfirsthead
\multicolumn{8}{c}%
{{\bfseries Table \thetable} (continued)} \\
\hline
\textbf{UniProt ID} & \textbf{\% RSA}  & $N_{Shenkin}$ & \textbf{MES}   & $p$    & \textbf{\# aas} & \textbf{\# ligs} & \textbf{UniProt residue numbers}                                                                                      
%
\endhead
%
Q5T0W9     & 22.4 & 36.2     & \textminus0.24 & 0.08 & 12          & 10         & 149, 150, 151, 177, 233, 234, 235, 236, 270, 273, 274, 277                                         \\ \hline
Q5T0W9     & 9.7  & 38.6     & \textminus0.05 & 0.79 & 12          & 2          & 125, 126, 127, 129, 229, 255, 256, 257, 272, 275, 276, 279                                         \\ \hline
Q8WVM7     & 19.8 & 57.7     & \textminus0.23 & 0.62 & 5           & 1          & 285, 288, 322, 325, 326                                                                            \\ \hline
Q15047     & 18.1 & 12.4     & +0.08  & 0.78 & 18          & 2          & 295, 296, 297, 298, 300, 301, 302, 324, 328, 329, 330, 332, 333, 357, 389, 392, 393, 394           \\ \hline
Q8WS26     & 19.5 & 57.3     & \textminus0.11 & 0.57 & 21          & 26         & 84, 87, 88, 89, 90, 214, 217, 218, 221, 222, 225, 268, 269, 273, 277, 281, 285, 290, 295, 299, 303 \\ \hline
Q9UGL1     & 28.7 & 31.3     & \textminus0.09 & 0.66 & 10          & 1          & 53, 57, 506, 582, 583, 606, 607, 609, 610, 613                                                     \\ \hline
Q9UGL1     & 16.6 & 34       & \textminus0.01 & 1    & 12          & 3          & 658, 659, 662, 663, 666, 667, 670, 701, 736, 737, 738, 741                                         \\ \hline
P15379     & 18.3 & 19.4     & +0.09  & 0.63 & 11          & 1          & 23, 24, 40, 41, 50, 146, 148, 162, 163, 164, 165                                                   \\ \hline
Q9UJM8     & 24.3 & 42.8     & \textminus0.11 & 0.86 & 6           & 1          & 5, 11, 323, 327, 328, 331                                                                          \\ \hline
Q6B0I6     & 21.9 & 36.6     & \textminus0.15 & 0.68 & 4           & 1          & 50, 209, 265, 285                                                                                  \\ \hline
Q6B0I6     & 12.2 & 26       & \textminus0.06 & 0.84 & 7           & 1          & 44, 199, 275, 276, 297, 300, 303                                                                   \\ \hline
Q9UKK9     & 9.8  & 29.6     & \textminus0.05 & 0.73 & 15          & 1          & 65, 66, 67, 69, 75, 77, 124, 125, 145, 146, 147, 175, 200, 205, 206                                \\ \hline
Q92835     & 16.5 & 33.7     & \textminus0.05 & 0.78 & 19          & 46         & 615, 616, 617, 618, 620, 621, 622, 624, 625, 630, 631, 632, 633, 634, 635, 636, 637, 638, 674      \\ \hline
Q92835     & 12.2 & 39.4     & +0.02  & 0.92 & 12          & 1          & 560, 561, 562, 570, 571, 572, 573, 574, 578, 817, 839, 840                                         \\ \hline
Q96HY7     & 11.6 & 38.5     & +0.07  & 0.75 & 14          & 1          & 57, 58, 60, 61, 64, 105, 106, 107, 121, 122, 125, 126, 147, 151                                    \\ \hline
P22557     & 17.5 & 40.6     & +0.04  & 0.72 & 16          & 7          & 143, 145, 146, 149, 348, 349, 350, 351, 352, 353, 380, 381, 383, 402, 403, 406                     \\ \hline
P24821     & 14.2 & 24.4     & \textminus0.29 & 0    & 15          & 8          & 2010, 2011, 2012, 2025, 2045, 2046, 2047, 2048, 2049, 2050, 2054, 2055, 2056, 2057, 2060           \\ \hline

\caption[Novel C1 sites]{\textbf{Novel C1 sites.} These are 17 C1 sites with no functional annotations in UniProt, therefore labelled as \textit{unknown function}, without any literature support. These sites represent therefore novel predicted functional sites. UniProt ID indicates the protein UniProt accession. \% RSA is the median site RSA. $N_{Shenkin}$ is the average normalised Shenkin score for the site. MES is the average missense enrichment score for the site. $p$ is the $p$-value associated to this site MES. \# aas is the number of residues forming the site. \# ligs is the number of ligands binding to the site. UniProt residue numbers is a list of the UniProt residue numbers of the residues forming the site.}
\label{tab:novel_c1_sites}\\
\end{longtable}
\end{landscape}


\chapter{LIGYSIS: a dataset and resource for the analysis of protein-ligand binding sites}

This chapter will be about the LIGYSIS-web resource.

\section*{Preface}

On this chapter the approach employed on the previous chapter to build a fragment screening ligand binding site dataset is refined, extended and applied to the whole PDB. LIGYSIS comprises biologically relevant protein-ligand interactions from $\approx$30,000 proteins with experimentally determined structures across species. A web resource is presented, LIGYSIS web, to allow exploration of this vast dataset and analysis of user structure sets.

\section*{Publications}

Utgés J.S., MacGowan S. M., Barton G. J. LIGYSIS-web: a web resource for the analysis of protein-ligand binding sites. (Manuscript in preparation)

\section*{Author contributions}

G.J.B., S.A.M. and J.S.U. conceived, designed, and developed the research. J.S.U. analysed the data. J.S.U. and S.A.M. developed the software. J.S.U. and G.J.B. wrote, reviewed and edited the manuscript. G.J.B. secured funding and supervised.

\section{Introduction}

XXX.

\section{Methods}

\subsection{LIGYSIS dataset}

There are \textbf{XXX} reviewed proteins in UniProt \cite{UNIPROT_2019_UNIPROT}. For each protein with at least one experimentally resolved three-dimensional (3D) structure, a superposition matrix was obtained from the PDBe FTP site \cite{PDBE_2022_PDBEKB} at \url{http://ftp.ebi.ac.uk/pub/databases/pdbe-kb/superposition}. This represents \textbf{XXX} (\textbf{XXX \%}) of the human proteins (\textbf{XXX} structures), whereas \textbf{XXX} (\textbf{XXX \%}) present no experimentally determined structures. 

\subsection{Binding site definition}

Preferred biological assemblies, as defined by PISA \cite{KRISSINEL_2007_PISA}, were downloaded from PDBe via ProIntVar \cite{MACGOWAN_2020_DRSASP}. Protein-ligand contacts were determined with pdbe-arpeggio \cite{JUBB_2017_ARPEGGIO}. Figure 10 illustrates the ligand site definition approach used to obtain our reference dataset: LIGYSIS. For a pair of ligands, $L_{A}$, $L_{B}$, fingerprints $A$, $B$ are defined as sets containing the residue numbers of the amino acids interacting with each ligand. Relative intersection, $I_{rel}$, is a similarity metric that quantifies how similar the fingerprints are [5]. Subtracting $I_{rel}$ from 1 gives a distance, $D$ (\autoref{eq:Irel_distance}), which takes the value of 1 when $A$ and $B$ share all the binding residues and 0 when they share none. For a given protein segment, interacting with $M$ biologically meaningful ligands across $N$ chains, ligand fingerprints are clustered using average linkage with SciPy [95] and ligand sites obtained by cutting the tree at $D$ = 0.5. This resulted in \textbf{XXX} ligand binding sites.

\begin{equation}
D = 1 - I_{rel}
\label{eq:Irel_distance}
\end{equation}
\myequations{Ligand fingerprint distance}

\subsection{Multiple sequence alignments}

The representative entry of the first structure cluster for each segment was obtained from the PDBe superposition data, and its amino acid sequence retrieved using the PDBe GRAPH API domains endpoint: \url{https://www.ebi.ac.uk/pdbe/graph-api/pdbe_pages/domains}. The approach described on the previous chapter was employed here and this sequence was used as query to perform a homologue sequence search in SwissProt \cite{BOUTET_2016_UNIPROT} using jackHMMER \cite{EDDY_1995_HMMER} to generate a multiple sequence alignment (MSA). The normalised version of the Shenkin divergence score \cite{SHENKIN_1991_SCORE}, $N_{Shenkin}$ \cite{UTGES_2021_ANKS}, was employed to quantify amino acid conservation within the MSA. For \textbf{XXX} protein segments, no homologues were found. Consequently, no conservation data was obtained for these segments. 

\subsection{Human variants enrichment}

Human missense genetic variants mapping to human sequences in the MSA were retrieved from gnomAD \cite{KARCZEWSKI_2020_GNOMAD} using VarAlign \cite{MACGOWAN_2017_VARIANTS, MACGOWAN_2024_VARIANTS}, as described on the previous chapter. Missense enrichment scores, i.e., odds ratio (OR), (MES) were calculated for alignment columns and 95\% confidence intervals and $p$-values used to evaluate their significance \cite{SZUMILAS_2010_ODDSRATIOS}. MES could not be calculated for \textbf{XXX} protein segments, as no human homologues were found in their alignments, nor for \textbf{XXX} segments, which presented human homologues, yet no missense variants in gnomAD. Scores were calculated for a total of \textbf{XXX} segments.

\subsection{Binding site clustering}

XXX.

\subsection{Binding site cluster prediction}

A multilayer perceptron (MLP) was implemented with Keras \cite{CHOLLET_2015_KERAS} to predict RSA-based cluster labels on new ligand binding sites. This model is built exactly as described on the previous chapter. However, there is a considerable difference in the size of training and test sets. The human component of the LIGYSIS dataset, LIGYSIS\textsubscript{HUMAN} was employed to train, validate, and test this new model. LIGYSIS\textsubscript{HUMAN} comprises $\approx$13,000 sites coming from $\approx$3,5000 proteins. The new set is $>$40 times larger than the previous one of 293 sites across 35 proteins.

\subsection{Functional score calculation}

XXX.

\subsection{Site function classification}

UniProt features were retrieved using the UniProt proteins API \cite{NIGHTINGALE_2017_API} on the features endpoint: \url{https://www.ebi.ac.uk/proteins/api/features/}. \textbf{XXX} of the \textbf{XXX} (92.5\%) proteins presented UniProt features. Only features of the category ``DOMAINS\_AND\_SITES'' and the types ``BINDING'', ``SITE'', and ``ACT\_SITE'' were used for functional assignment of a protein, which resulted in \textbf{XXX} (\textbf{XXX\%}) proteins with such annotations. Binding sites presenting at least a functionally annotated residue were classified as known function (KF) ($N$ = \textbf{XXX}), and those with no annotations as unknown function (UF) ($N$ = \textbf{XXX}). 

\subsection{The LIGYSIS web resource}

XXX.

\subsubsection{Flask application backend}

\autoref{get_contacts_route} shows how bla, bla, bla.

\lstset{style=mystyle}

\begin{lstlisting}[language=MyPython, caption={[/get-contacts route]\textbf{/get-contacts route.} This Python code block shows an example of a Flask Web Application route that bla, bla, bla.}, label={get_contacts_route}]
@app.route('/get-contacts', methods = ['POST'])
def get_contacts():
    data = request.json
    active_model = data['modelData']
    prot_id = data['proteinId']
    seg_id = data['segmentId']
		(...)	
    response_data = {
        'contacts': json_cons,
        'ligands': struc_ligs_data,
        'protein': struc_prot_data,
    }
    return jsonify(response_data)
\end{lstlisting}

\subsubsection{HTML templates}

XXX.

\subsubsection{JavaScript frontend}

XXX.

\subsubsection{CSS stylings}

XXX.

\subsubsection{Job submission}

XXX.

\subsubsection{Deployment}

XXX.

\subsection{Data availability}

XXX.

\subsection{Code availability}

XXX.

\section{Results}

XXX.

\section{Discussion}

XXX.

\chapter{Comparative evaluation of methods for the prediction of protein-ligand binding sites}
\label{chap:LBS_COMP}

\section*{Preface}

This Chapter describes the largest benchmark of ligand binding site prediction methods to date, comparing thirteen original methods using 10 informative metrics and the LIGYSIS dataset as a reference. LIGYSIS, introduced in \autoref{chap:LIGYSIS_WEB}, is compared to widely used training and test sets and evidence shown of the advantages of using LIGYSIS over these other data sets. Finally, top-\textit{N}+2 recall is proposed as a universal benchmark metric for ligand binding site prediction, with a recommendation for open-source sharing of both methods and benchmarks. The work in this Chapter was solely carried out by me.

\section*{Publications}

Utgés, J.S. and Barton, G.J. Comparative evaluation of methods for the prediction of protein-ligand binding sites. \textit{J. Cheminform.} \textbf{16}, 126 (2024). \url{https://doi.org/10.1186/s13321-024-00923-z}.

%\section*{Author contributions}

%J.S.U. and G.J.B. conceived, designed, and developed the research. J.S.U. analysed the data. J.S.U. developed the software. J.S.U. and G.J.B. wrote, reviewed and edited the manuscript. G.J.B. secured funding and supervised.

\begin{longtable}{|c|c|c|c|c|c|c|}
\hline
\textbf{Method}      & \textbf{Source} & \textbf{Review} & \textbf{Install} & \textbf{Docs} & \textbf{Model} & \textbf{Included} \\ \hline
\endfirsthead
%
\multicolumn{7}{c}%
{{\bfseries Table \thetable} (continued)} \\
\hline
\textbf{Method}      & \textbf{Source} & \textbf{Review} & \textbf{Install} & \textbf{Docs} & \textbf{Model} & \textbf{Included} \\ \hline
\endhead
%
\textbf{VN-EGNN}     & \textbf{\cmark}      & \textbf{\cmark}      & \textbf{\cmark}       & \textbf{\cmark}    & \textbf{\cmark}     & \textbf{\cmark}        \\ \hline
\textbf{IF-SitePred} & \textbf{\cmark}      & \textbf{\cmark}      & \textbf{\cmark}       & \textbf{\cmark}    & \textbf{\cmark}     & \textbf{\cmark}        \\ \hline
\textbf{GrASP}       & \textbf{\cmark}      & \textbf{\cmark}      & \textbf{\cmark}       & \textbf{\cmark}    & \textbf{\cmark}     & \textbf{\cmark}        \\ \hline
RefinePocket         & \textbf{\cmark}      & \textbf{\cmark}      & \textbf{?}                & \textbf{\xmark}    & \textbf{\cmark}     & \textbf{\xmark}        \\ \hline
EquiPocket           & \textbf{\cmark}      & \textbf{\xmark}      & \textbf{?}                & \textbf{\xmark}    & \textbf{\cmark}     & \textbf{\xmark}        \\ \hline
GLPocket             & \textbf{\cmark}      & \textbf{\cmark}      & \textbf{?}                & \textbf{\xmark}    & \textbf{\cmark}     & \textbf{\xmark}        \\ \hline
SiteRadar            & \textbf{\xmark}      & \textbf{\cmark}      & \textbf{\xmark}       & \textbf{\xmark}    & \textbf{\xmark}     & \textbf{\xmark}        \\ \hline
NodeCoder            & \textbf{\cmark}      & \textbf{\xmark}      & \textbf{?}                & \textbf{\cmark}    & \textbf{\xmark}     & \textbf{\xmark}        \\ \hline
\textbf{DeepPocket}  & \textbf{\cmark}      & \textbf{\cmark}      & \textbf{\cmark}       & \textbf{\cmark}    & \textbf{\cmark}     & \textbf{\cmark}        \\ \hline
RecurPocket          & \textbf{\cmark}      & \textbf{\xmark}      & \textbf{?}                & \textbf{\xmark}    & \textbf{\cmark}     & \textbf{\xmark}        \\ \hline
PointSite            & \textbf{\cmark}      & \textbf{\cmark}      & \textbf{\xmark}       & \textbf{\cmark}    & \textbf{\cmark}     & \textbf{\xmark}        \\ \hline
DeepSurf             & \textbf{\cmark}      & \textbf{\cmark}      & \textbf{\xmark}       & \textbf{\cmark}    & \textbf{\cmark}     & \textbf{\xmark}        \\ \hline
\textbf{PUResNet}    & \textbf{\cmark}      & \textbf{\cmark}      & \textbf{\cmark}       & \textbf{\cmark}    & \textbf{\cmark}     & \textbf{\cmark}        \\ \hline
Kalasanty            & \textbf{\cmark}      & \textbf{\cmark}      & \textbf{\xmark}       & \textbf{\cmark}    & \textbf{\cmark}     & \textbf{\xmark}        \\ \hline
BiteNet              & \textbf{\xmark}      & \textbf{\cmark}      & \textbf{\xmark}       & \textbf{\cmark}    & \textbf{\xmark}     & \textbf{\xmark}        \\ \hline
GRaSP                & \textbf{\cmark}      & \textbf{\cmark}      & \textbf{\cmark}       & \textbf{\xmark}    & \textbf{\cmark}     & \textbf{\xmark}        \\ \hline
\textbf{P2Rank}      & \textbf{\cmark}      & \textbf{\cmark}      & \textbf{\cmark}       & \textbf{\cmark}    & \textbf{\cmark}     & \textbf{\cmark}        \\ \hline
\textbf{PRANK}       & \textbf{\cmark}      & \textbf{\cmark}      & \textbf{\cmark}       & \textbf{\cmark}    & \textbf{\cmark}     & \textbf{\cmark}        \\ \hline
DeepSite             & \textbf{\xmark}      & \textbf{\cmark}      & \textbf{\xmark}       & \textbf{\xmark}    & \textbf{\xmark}     & \textbf{\xmark}        \\ \hline
\caption[Method selection criteria]{\textbf{Method selection criteria.} These are the criteria employed to select machine learning-based methods for this benchmark. Nineteen machine learning-based methods were considered and seven were selected as all requirements were met. Source: whether the method is open source and code is publicly accessible; Review: whether the method has been published after peer-review; Install: whether installation of the method was successful; Docs: whether the method is sufficiently documented to install it and run it on an example input; Model: whether the method provides pre-trained model weights; Included: whether the method was included in this analysis. Check marks(\cmark) indicate meeting the requirement and crosses (\xmark) the opposite. Question marks (\textbf{?}) indicate uncertainty. Installation was not attempted for some methods as they already did not meet other requirements. Methods in bold font are the ones included in this work.}
\label{tab:method_selection}\\
\end{longtable}

\section{Introduction}

%Identifying where ligands can bind to proteins is of critical importance in understanding and modulating protein function. While X-ray crystallography remains the gold-standard to identify and characterise binding sites \cite{CONGREVE_2003_RO3, REES_2004_FBLD, MURRAY_2009_FBDD, SCHIEBEL_2016_FRAGMENTS, UTGES_2024_FRAGSYS}, over the last three decades, significant effort has been made to develop computational methods that predict binding sites from an apo three-dimensional protein structure \cite{VOLKAMER_2010_TOPOLOGY}.

In this Chapter, thirteen ligand binding site prediction tools are compared against the LIGYSIS reference dataset, introduced in \autoref{chap:LIGYSIS_WEB}. LIGYSIS identifies human protein-ligand binding sites for biologically relevant ligands, defined by BioLiP \cite{YANG_2013_BIOLIP}, from protein structures determined by X-ray crystallography. The methods assessed in this Chapter include geometry-based fpocket \cite{GUILLOUX_2009_FPOCKET}, Ligsite \cite{HENDLICH_1997_LIGSITE} and Surfnet \cite{LASKOWSKI_1995_SURFNET}, energy-based PocketFinder \cite{AN_2005_POCKETFINDER} and machine learning methods exemplified by PRANK \cite{KRIVAK_2015_PRANK}, P2Rank \cite{KRIVAK_2015_P2RANK, KRIVAK_2018_P2RANK}, DeepPocket \cite{AGGARWAL_2022_DEEPPOCKET}, PUResNet \cite{KANDEL_2021_PURESNET, KANDEL_2024_PURESNET}, GrASP \cite{SMITH_2024_GrASP}, IF-SitePred \cite{CARBERY_2024_IFSP} and VN-EGNN \cite{SESTAK_2024_VNEGNN}. Open source, peer-reviewed and easy-to-install methods were prioritised (\autoref{tab:method_selection}). This set of method represents the most complete and relevant set of ligand binding site prediction tools benchmarked to date and is representative of the state-of-the-art within the field.

%Prediction methods exploit a variety of different techniques to suggest binding sites. Geometry-based techniques like fpocket \cite{GUILLOUX_2009_FPOCKET}, Ligsite \cite{HENDLICH_1997_LIGSITE} and Surfnet \cite{LASKOWSKI_1995_SURFNET} identify cavities by analysing the geometry of the molecular surface of a protein and usually rely on the use of a grid, gaps, spheres, or tessellation \cite{GUILLOUX_2009_FPOCKET, LIANG_1998_CAVITIES, HENDLICH_1997_LIGSITE, LASKOWSKI_1995_SURFNET, KLEYWEGT_1994_CAVITIES, LEVITT_1992_POCKET, BRADY_2000_PASS, WEISEL_2007_POCKETPICKER}. Energy-based methods such as PocketFinder \cite{AN_2005_POCKETFINDER} rely on the calculation of interaction energies between the protein and a chemical group or probe to identify cavities \cite{AN_2005_POCKETFINDER, GOODFORD_1982_PREDICTOR, AN_2004_PREDICTOR, LAURIE_2005_QSITEFINDER, GHERSI_2009_SITEHOUND, NGAN_2012_FTSITE}. Conservation-based methods make use of sequence evolutionary conservation information to find patterns in multiple sequence alignments and identify conserved key residues for ligand site identification \cite{ARMON_2001_CONSURF, PUPKO_2002_RATE4SITE, XIE_2012_CONSPRED}. Template-based methods rely on structural information from homologues and the assumption that structurally conserved proteins might bind ligands at a similar location \cite{ZVELEBIL_1987_PREDICTION, WASS_2010_3DLIGANDSITE, ROY_2012_COFACTOR, YANG_2013_COFACTOR, LEE_2013_PREDICTION, BRYLINSKI_2013_EFINDSITE}. Combined approaches or meta-predictors combine multiple methods, or the use of multiple types of data, to infer ligand binding sites, e.g., geometric features with sequence conservation \cite{GUTTERIDGE_2003_LBSP, HUANG_2006_BU48, GLASER_2006_PREDICTION, HALGREN_2009_PREDICITON, CAPRA_2009_CONCAVITY, HUANG_2009_METAPOCKET, BRAY_2009_SITESIDENTIFY, BRYLINSKI_2009_FINDSITE}. Finally, machine learning methods utilise a wide range of machine learning techniques including random forest, as well as deep, graph, residual, or convolutional neural networks \cite{KRIVAK_2015_PRANK, KRIVAK_2015_P2RANK, JIMENEZ_2017_DEEPSITE, KRIVAK_2018_P2RANK, SANTANA_2020_GRaSP, KOZLOVSKII_2020_BITENET, STEPNIEWSKA_2020_KALASANTY, KANDEL_2021_PURESNET, MYOLNAS_2021_DEEPSURF, YAN_2022_POINTSITE, LI_2022_RECURPOCKET, AGGARWAL_2022_DEEPPOCKET, ABDOLLAHI_2023_NODECODER, EVTEEV_2023_SITERADAR, LI_2023_GLPOCKET, ZHANG_2024_EQUIPOCKET, LIU_2023_REFINEPOCKET,  SMITH_2024_GrASP, CARBERY_2024_IFSP, SESTAK_2024_VNEGNN, KANDEL_2024_PURESNET}.

\autoref{tab:methods_details_1} and \autoref{tab:methods_details_2} summarise the methods evaluated in this work, which were executed with their standard settings. VN-EGNN \cite{SESTAK_2024_VNEGNN} combines virtual nodes with equivariant graph neural networks. Virtual nodes, represented by ESM-2 embeddings \cite{RIVES_2021_EMBEDDINGS} are passed through a series of message-passing layers until they reach their final coordinates, which represent the centroid of predicted pockets. Pocket residues are not reported. IF-SitePred \cite{CARBERY_2024_IFSP} represents protein residues with ESM-IF1 embeddings \cite{HSU_2022_EMBEDDINGS} and employs 40 different light gradient boosting machine (LGBM) models \cite{KE_2017_LIGHTGBM} to classify residues as ligand-binding if all forty models return a \textit{p} $>$ 0.5. It later utilises PyMOL \cite{SCHRODINGER_2015_PYMOL} to place a series of cloud points which are clustered using DBSCAN \cite{ESTER_1996_DBSCAN}  and a threshold of 1.7 \AA{}. Pocket centroids are obtained by averaging the clustered points’ coordinates, scored and ranked based on the number of cloud points. Like VN-EGNN, no pocket residues are defined. GrASP \cite{SMITH_2024_GrASP} employs graph attention networks to perform semantic segmentation on all surface protein atoms, represented by 17 atom, residue and bond-level features, scoring which are likely part of a binding site. Atoms with a score $>$ 0.3 are clustered into binding sites using average linkage and a threshold of 15 \AA{}. Pocket scores are calculated as the sum of squares of binding site atom scores. PUResNet \cite{KANDEL_2021_PURESNET} combines deep residual and convolutional neural networks to predict ligand binding sites using an 18-element vector of atom-level features and one-hot encoding to represent grid voxels. Voxels with a score $>$ 0.34 are clustered into binding sites using DBSCAN and a threshold of 5.5 \AA{} \cite{KANDEL_2024_PURESNET}. Pockets are represented by their residues, but neither pocket centroid, nor score or ranking are reported. Similarly to PUResNet, DeepPocket \cite{AGGARWAL_2022_DEEPPOCKET} exploits convolutional neural networks on grid voxels represented by 14 atom-level features to re-score (DeepPocket\textsubscript{RESC}) and additionally extract new pocket shapes (DeepPocket\textsubscript{SEG}) from fpocket candidates. P2Rank \cite{KRIVAK_2018_P2RANK} relies on solvent accessible surface (SAS) points placed over the protein surface, represented by 35 atom and residue-level features, and a random forest classifier to score them based on their likelihood of binding a ligand. SAS points with a score $>$ 0.35 are clustered into sites using single linkage and a threshold of 3 \AA{}. P2Rank\textsubscript{CONS} \cite{JENDELE_2019_PRANKWEB} works in the same manner but considers an extra feature: amino acid conservation as measured by Jensen-Shannon divergence \cite{CAPRA_2007_JSD}. Both report residue and pocket level scores, as well as pocket centroids and rank. PocketFinder \cite{AN_2005_POCKETFINDER} uses the Lennard-Jones \cite{JONES_1924_POTENTIAL} transformation on a 1 \AA{} grid surrounding the protein surface to predict protein cavities. PocketFinder does not report pocket centroid, score or rank. Finally, geometry-based methods: fpocket \cite{GUILLOUX_2009_FPOCKET}, Ligsite \cite{HENDLICH_1997_LIGSITE} and Surfnet \cite{LASKOWSKI_1995_SURFNET} rely on the geometry of the molecular surface to find cavities. fpocket is the only one of these three methods that reports pocket centroid, score, rank and residues. Additionally, fpoket reports multiple pocket features including surface area, volume, hydrophobicity, charge or druggability. 

This Chapter compares these thirteen methods to each other and to the LIGYSIS reference dataset according to a range of metrics including the number of ligand sites, their size, shape, proximity and overlap. Additionally, this Chapter identifies the strengths and weaknesses of prediction assessment metrics and leads to guidance for developing ligand binding site prediction tools or using them to understand protein function and in drug development. This work represents the first independent ligand site prediction benchmark for over a decade, since Schmidtke \textit{et al.} \cite{SCHMIDTKE_2010_BENCHMARK} or Chen \textit{et al.} \cite{CHEN_2011_ASSESSMENT} and the largest to date in terms of dataset size (2775), methods compared (13) and metrics employed ($>$10).

\begin{landscape}
\begin{longtable}{|M{25mm}|M{29mm}|M{27mm}|M{22mm}|M{19mm}|M{19mm}|M{15mm}|M{15mm}|M{15mm}|M{16mm}|M{15mm}|M{23mm}|M{15mm}|}
\hline
\textbf{Method} & \textbf{Approach}  & \textbf{Features} & \textbf{\# Features}   & \textbf{P centroid}    & \textbf{P residues} & \textbf{P score} & \textbf{P rank} & \textbf{R score} \\ \hline
\endfirsthead
%
\endhead
%
VN-EGNN       & EGNN + VN                     & ESM-2 embeddings        & 1280       & \textbf{\cmark}         & \textbf{\xmark}         & \textbf{\cmark}      & \textbf{\cmark}        & \textbf{\xmark}      \\ \hline
IF-SitePred   & LGBM                      & ESM-IF1 embeddings      & 512         & \textbf{\cmark}        & \textbf{\xmark}         & \textbf{\cmark}      & \textbf{\cmark}        & \textbf{\xmark}      \\ \hline
GrASP         & GAT - GNN                     & Atom + residue + bond    & 17          & \textbf{\cmark}         & \textbf{\cmark}         & \textbf{\cmark}      & \textbf{\cmark}        & \textbf{\cmark}      \\ \hline
PUResNet      & DRN + 3D-CNN                  & Atom + one-hot encoding & 18          & \textbf{\xmark}         & \textbf{\cmark}         & \textbf{\xmark}      & \textbf{\xmark}        & \textbf{\xmark}      \\ \hline
DeepPocket    & fpocket + 3D-CNN                        & Atom                    & 14          & \textbf{\cmark}         & \textbf{\cmark}         & \textbf{\cmark}      & \textbf{\cmark}        & \textbf{\xmark}      \\ \hline
P2Rank\textsubscript{CONS}    & Random Forest                 & Atom + residue        & 36          & \textbf{\cmark}         & \textbf{\cmark}         & \textbf{\cmark}      & \textbf{\cmark}        & \textbf{\cmark}      \\ \hline
P2Rank        & Random Forest                 & Atom + residue        & 35          & \textbf{\cmark}         & \textbf{\cmark}         & \textbf{\cmark}      & \textbf{\cmark}        & \textbf{\cmark}      \\ \hline
fpocket\textsubscript{PRANK}       & fpocket + Random Forest & Atom + residue                       & 34           & \textbf{\xmark}         & \textbf{\cmark}         & \textbf{\cmark}      & \textbf{\cmark}        & \textbf{\xmark}      \\ \hline
fpocket       & α-spheres & \textbf{--}                       & \textbf{--}           & \textbf{\xmark}         & \textbf{\cmark}         & \textbf{\cmark}      & \textbf{\cmark}        & \textbf{\xmark}      \\ \hline
PocketFinder\textsuperscript{+} & LJ potential                  & \textbf{--}                       & \textbf{--}           & \textbf{\xmark}         & \textbf{\xmark}         & \textbf{\xmark}      & \textbf{\xmark}        & \textbf{\cmark}      \\ \hline
Ligsite\textsuperscript{+}      & Cubic grid                    & \textbf{--}                       & \textbf{--}           & \textbf{\xmark}         & \textbf{\xmark}         & \textbf{\xmark}      & \textbf{\xmark}        & \textbf{\cmark}      \\ \hline
Surfnet\textsuperscript{+}      & Gap regions                   & \textbf{--}                       & \textbf{--}           & \textbf{\xmark}         & \textbf{\xmark}         & \textbf{\xmark}      & \textbf{\xmark}        & \textbf{\cmark}      \\ \hline
%\newpage
\caption[Ligand binding site prediction methods summary (I)]{\textbf{Ligand binding site prediction methods summary (I).} All these methods were used with their default settings. Check marks (\cmark) indicate that a method provides a given output and crosses (\xmark) the contrary. Dashes (\textbf{--}) indicate a field is not applicable for a given method, e.g., features for non-machine learning-based methods. Approach: the techniques applied by the method; Features/\# Features: the features and their number if the method is machine learning-based; P centroid/P residues/P score/P rank/R score: whether the method reports the pocket centroid, pocket residues, pocket score, pocket ranking and residue \textit{ligandability} score. For example, P2Rank uses a random forest classifier on SAS points represented by 35 atom and residue features. EGNN + VN: equivariant graph neural network + virtual nodes; LGBM: Light gradient boosting machine; GAT: graph attention network; GNN: graph neural network; DRN: deep residual network; 3D-CNN: three-dimensional convolutional neural network; LJ potential: Lennard-Jones potential.}
\label{tab:methods_details_1}\\
\end{longtable}
\end{landscape}

\begin{landscape}
\begin{longtable}{|M{35mm}|M{45mm}|M{35mm}|M{35mm}|M{35mm}|}
\hline
\textbf{Method} & \textbf{R score threshold}  & \textbf{Cluster} & \textbf{Algorithm}   & \textbf{Threshold} (\AA{})\\ \hline
\endfirsthead
%
\endhead
%
VN-EGNN       & \textbf{--}                 & \textbf{--}            & \textbf{--}         & \textbf{--}         \\ \hline
IF-SitePred   & 0.50 (\textit{all} 40)      & Cloud points & DBSCAN    & 1.7       \\ \hline
GrASP         & 0.30               & Atoms        & Average   & 15        \\ \hline
PUResNet      & 0.34              & Atoms        & DBSCAN    & 5.5       \\ \hline
DeepPocket    & \textbf{--}                 & \textbf{--}            & \textbf{--}         & \textbf{--}         \\ \hline
P2Rank\textsubscript{CONS}    & 0.35              & SAS points   & Single    & 3         \\ \hline
P2Rank        & 0.35              & SAS points   & Single    & 3         \\ \hline
fpocket       & \textbf{--}                 & α-spheres   & Multiple  & 1.7, 4.5, 2.5         \\ \hline
fpocket\textsubscript{PRANK}       & \textbf{--}                 & \textbf{--}   & \textbf{--}  & \textbf{--}         \\ \hline
PocketFinder\textsuperscript{+} & \textbf{--}                 & Grid points            & \textbf{?}         & \textit{search}         \\ \hline
Ligsite\textsuperscript{+}      & \textbf{--}                 & Grid points            & \textbf{?}         & \textit{search} \\ \hline
Surfnet\textsuperscript{+}      & \textbf{--}                 & Grid points            & \textbf{?}         & \textit{search}      \\ \hline
\caption[Ligand binding site prediction methods summary (II)]{\textbf{Ligand binding site prediction methods summary (II).} All these methods were used with their default settings. Information about the clustering strategies employed by the methods. R score threshold: whether the method uses a residue ligandability threshold; Cluster: the instances they cluster to define the distinct pockets; Algorithm: the clustering algorithm used; Threshold: the distance threshold employed (\AA{}). For example, GrASP utilises average linkage clustering on atoms with predicted ligandability score $>$ 0.30 and a threshold of 15 \AA{}. A dash (\textbf{--}) indicates that the category is not applicable, i.e., VN-EGNN does not employ clustering in their prediction of ligand binding sites. Question marks (\textbf{?}) indicate variables for which values were not be found. ``\textit{search}'' represents an iterative process to find optimal clustering thresholds.}
\label{tab:methods_details_2}\\
\end{longtable}
\end{landscape}

\section{Methods}

\subsection{LIGYSIS reference dataset}

\autoref{chap:LIGYSIS_WEB} describes the LIGYSIS pipeline and ligand binding site definition approach, which groups small molecule ligands across multiple biological assemblies of the same protein. In this Chapter, the human subset of the LIGYSIS dataset is employed as a reference dataset for the benchmark of a relevant selection of tools for the prediction of ligand binding sites. There are 20,423 human reviewed proteins in UniProt \cite{UNIPROT_2019_UNIPROT}, 7640 (37.4\%) of which proteins present experimentally determined three-dimensional structures deposited in the Protein Data Bank (PDB) \cite{ARMSTRONG_2020_PDBE}. 5455 (71.4\%) proteins present at least one ligand-binding structure. After removing non-biologically relevant ligands in accordance with BioLiP \cite{YANG_2013_BIOLIP}, 3513 proteins including 4037 structural segments. A structural segment is defined by the PDBe-KB as a protein region with structural coverage that maps to a contiguous section of their corresponding UniProt sequence remained. A protein can have multiple segments. For example, each domain of a multi-domain protein for which there are independent structures would correspond to a segment. Transformation matrices were obtained from the PDBe-KB \cite{PDBE_2022_PDBEKB} and used to  structurally align a total of 64,498 protein chains across 33,715 structures for the 4037 segments. These matrices result in high quality multiple structural alignment between the different chains of the same protein across PDB structures. \autoref{fig:protein_chain_supp} illustrates this superposition process of the LIGYSIS pipelines by highlighting pancreatic alpha-amylase (\href{https://www.uniprot.org/uniprotkb/P04746/entry}{P04746}) and cAMP and cAMP-inhibited cGMP 3',5'-cyclic phosphodiesterase 10A (\href{https://www.uniprot.org/uniprotkb/P37231/entry}{P37231}). \autoref{fig:supp_ca_dists} illustrates the little variation amongst the Cα atoms of the ligand binding residues. The human subset of the LIGYSIS dataset includes 8244 ligand binding sites, i.e., sets of UniProt residues. From here on, this subset of the LIGYSIS dataset will be referred as \textit{LIGYSIS} for brevity.

\begin{figure}[htb!]
    \centering
    \includegraphics[width=\textwidth]{figures/ch_LBS_COMP/PNG/SUPP_FIG9_supp_examples.png}
    \caption[Protein chains superposition]{\textbf{Protein chains superposition.} PDBe-KB transformation matrices were utilised to structurally align protein chains. For each example, superposed chain trace (Cα atoms) are shown in sticks and coloured using the rainbow scheme from N- to C-terminus and average distance across residues from the aligned chains to the PDBe-KB-defined representative chain is reported as $\overline{d}_{C\alpha}$ (\AA{}) (left). Superposition is visualised with a heatmap (right). Protein chain residues are on the X axis and aligned protein chains on the Y axis. Protein chains are sorted by the average distance to the representative chain, so more dissimilar chains are on the bottom. Cells are coloured based on their $\overline{d}_{C\alpha}$ using the \textit{rocket} colour scheme. White cells represent residues present in the representative chain but not the aligned one, i.e., discontinuities or chain breaks. Residues with very high $>$ 20 \AA{} represent alternative locations that were not transformed correctly. \textbf{(A)} Pancreatic alpha-amylase (\href{https://www.uniprot.org/uniprotkb/P04746/entry}{P04746}) with 52 superposed chains; \textbf{(B)} Peroxisome proliferator-activated receptor gamma (\href{https://www.uniprot.org/uniprotkb/P37231/entry}{P37231}) with 443 chains.}
    \label{fig:protein_chain_supp}
\end{figure}

\begin{figure}[htb!]
    \centering
    \includegraphics[width=0.55\textwidth]{figures/ch_LBS_COMP/PNG/SUPP_FIG10_supp_ca_dists.png}
    \caption[Distance to representative chain for ligand binding residues]{\textbf{Distance to representative chain for ligand binding residues.} This histogram represents the distribution of the average Cα distance across transformed chains to the representative chain for 74,536 ligand binding residues across the 2478 segments that present more than one chain. Black dash line indicates 5 \AA{}. 95\% of ligand binding residues are within 5 \AA{} of the representative structure in average across chains. This demonstrates that the variation in the Cα trace for ligand binding residues across different structures of the same protein is very small.}
    \label{fig:supp_ca_dists}
\end{figure}

\subsection{Comparison of datasets}

Training and test datasets were downloaded for all machine learning-based methods reviewed in this Chapter. Datasets were compared to the LIGYSIS reference set, in terms of number of sites per protein, ligand-interacting chains, chain length, site size (number of amino acids), ligand composition, size and diversity. Ligand diversity was quantified by Shannon's Entropy \cite{SHANNON_1948_ENTROPY} (\autoref{eq:entropy_shannon}) where $p_i$ represents the proportion of each ligand $i$ of the $R$ ligands observed in a dataset. Ligand data was extracted from the Chemical Component Dictionary (CCD) \cite{WESTBROOK_2015_CCD}. An overlap (\%) was calculated for each dataset as the proportion of LIGYSIS binding sites that were covered by at least one ligand in a test dataset. A simplistic approach was adopted by calculating the intersection of ligand codes between LIGYSIS and each dataset. Ligand codes were defined as a string of PDB ID + ``\_'' + ligand ID, e.g., ``6GXT\_GTP'' corresponds to the guanosine-5’-triphosphate (GTP) of the PDB entry with ID: \href{https://www.ebi.ac.uk/pdbe/entry/pdb/6gx7}{6GX7} \cite{CAMPANACCI_2019_TUBULIN}.

\begin{equation}
H' = - \sum_{i=1}^{R} p_i \ln(p_i)
\label{eq:entropy_shannon}
\end{equation}
\myequations{Shannon's entropy}

\subsection{Training datasets}

VN-EGNN is trained on a subset \cite{KANDEL_2021_PURESNET} of the sc-PDB (v2017) \cite{PAUL_2004_SCPDB, KELLENBERGER_2006_SCPDB, MESLAMANI_2011_SCPDB, DESAPHY_2015_SCPDB} (sc-PDB\textsubscript{SUB}). sc-PDB is a comprehensive database of pharmacological ligand-protein complexes. The database is composed of proteins in complex with buried, biologically relevant synthetic or natural ligands deposited in the PDB. sc-PDB contains unique non-repeating protein-ligand pairs, meaning that only one ligand is considered per PDB entry. Smith \textit{et al.} \cite{SMITH_2024_GrASP} enriched this dataset with 9000 extra ligands resulting in a version of sc-PDB referred to here as sc-PDB\textsubscript{RICH}, which GrASP trained on. This dataset is not publicly accessible and therefore not considered in this analysis. DeepPocket used the full sc-PDB set to train on, sc-PDB\textsubscript{FULL}. IF-SitePred used a sequence identity-filtered version of the non-redundant subset of the binding Mother Of All Databases (MOAD) \cite{HU_2005_BMOAD, BENSON_2008_BMOAD, AHMED_2015_BMOAD, SMITH_2019_BMOAD}, which considers only protein family leaders. The binding MOAD, here referred to as bMOAD\textsubscript{SUB}, is a large collection of crystal structures with clearly identifies biologically relevant ligands with binding data extracted from the literature. Finally, PRANK and P2Rank used the CHEN11 dataset to train, which aimed to cover all SCOP \cite{HUBBARD_1997_SCOP, HUBBARD_1998_SCOP, LOCONTE_2000_SCOP} families of ligand binding proteins in a non-redundant manner \cite{CHEN_2011_ASSESSMENT}. P2Rank utilised the JOINED dataset for validation. CHEN11 not only considers the ligands in each structure but is enriched with ligands binding to homologous structures. JOINED is a combined dataset formed by other smaller datasets: ASTEX \cite{HARTSHORN_2007_ASTEX}, UB48 \cite{HUANG_2006_BU48}, DT198 \cite{ZHANG_2011_METAPOCKET} and MP210 \cite{HUANG_2009_METAPOCKET}, which represent diverse collections of protein-ligand complexes, including bound/unbound states, drug-target complexes and other ligand site predictor benchmark sets.

\subsection{Test datasets}

The majority of ligand binding site predictors published since 2018 have been using two datasets that were first presented by Krivák \textit{et al.} \cite{KRIVAK_2018_P2RANK}: COACH420 and HOLO4K, or subsets of them. COACH420 is comprised by a set of 420 single-chain structures binding a mix of drug-like molecules and naturally occurring ligands which is disjunct with the CHEN11 and JOINED datasets. COACH420 is a modified version of the original COACH test set \cite{ROY_2012_COFACTOR, YANG_2013_COFACTOR}. HOLO4K is a larger set, \textit{N} $\approx$ 4000, based on the list by Schmidtke \textit{et al.} \cite{SCHMIDTKE_2010_BENCHMARK}, which includes a mix of single- and multi-chain complexes, also disjunct with P2Rank training (CHEN11) and validation (JOINED) datasets. PRANK employed the small datasets comprising the JOINED set for testing. VN-EGNN, DeepPocket and GrASP use the Mlig and Mlig+ subsets of the COACH and HOLO4K datasets, which include strictly biologically relevant ligands as defined by the binding MOAD. IF-SitePred tested on the HOLO4K-AlphaFold2 Paired (HAP) and HAP-small sets. HAP is a subset of the HOLO4K dataset which presents high quality models in the AlphaFold database \cite{VARADI_2022_ALPHAFOLDDB}. HAP-small is a smaller subset of HAP that only contains proteins with sequence identity lower than 25\% to proteins in the P2Rank training set. VN-EGNN uses the refined version of PDBbind (v2020), referred here as PDBbind\textsubscript{REF}, \cite{WANG_2004_PDBBIND, WANG_2005_PDBBIND, CHENG_2009_PDBBIND, LI_2014_PDBBIND, LIU_2015_PDBBIND, LIU_2017_PDBBIND} as a third test set. Like binding MOAD, the PDBbind database provides a comprehensive collection of experimentally measured binding affinity data for macromolecular complexes. Specifically, the refined set includes those protein-ligand complexes for which binding data was obtained with the literature and met certain experimental quality thresholds. Lastly, SC6K is a dataset presented by Aggarwal \textit{et al.} \cite{AGGARWAL_2022_DEEPPOCKET} containing 6000 protein-ligand pairs from PDB entries submitted from 01/01/2018 – 28/02/2020.

\subsection{Protein chain alignment}

For each protein chain, atomic coordinates were translated to be centred at the origin, \textit{O} = (0, 0, 0), and rotated using a rotation matrix, $R$. The two principal components of the coordinate space $pc_{1}$ and $pc_{2}$ were obtained using principal component analysis (PCA) \cite{HOTELLING_1933_PCA}. A third component, $pc_{\perp}$, was obtained with the cross-product of the other two, to ensure orthogonality. A rotation matrix $P$ was constructed from these vectors (\autoref{eq:pca_components}). By placing the main component $pc_{1}$ on the second row of $P$, the Y axis was fixed as the major axis, representing the height of the protein chain. The second largest axis is the X axis, representing the width of the protein, and lastly the depth is represented by the smaller magnitude of the Z axis. The final rotation matrix $R$ was obtained by multiplying $P$ by the negative identity matrix $NI$ (\autoref{eq:NI_matrix} and \autoref{eq:R_matrix}). This was done to maintain the left-handedness of the protein chains whilst ensuring a consistent alignment on the major axes.

\begin{equation}
pc_{\perp} = pc_{1} \times pc_{2} \quad \rightarrow \quad P = \begin{bmatrix}
pc_{2} \\
pc_{1} \\
pc_{\perp}
\end{bmatrix}
\label{eq:pca_components}
\end{equation}
\myequations{Principal components vector}

\begin{equation}
NI = -1 \cdot I_3 = -1 \cdot \begin{bmatrix}
1 & 0 & 0 \\
0 & 1 & 0 \\
0 & 0 & 1 
\end{bmatrix} = \begin{bmatrix}
-1 & 0 & 0 \\
0 & -1 & 0 \\
0 & 0 & -1 
\end{bmatrix}
\label{eq:NI_matrix}
\end{equation}
\myequations{Negative identity matrix}

\begin{equation}
R = P \cdot NI
\label{eq:R_matrix}
\end{equation}
\myequations{Rotation matrix}

\vspace{-12pt} % Adjust this value as needed
\vspace{-12pt} % Adjust this value as needed
\vspace{-12pt} % Adjust this value as needed

\subsection{Protein chain characterisation}

For a protein chain with \textit{N} amino acid residues, the centre of mass (CM) was calculated by averaging the coordinates, $r_{i}$, of all atoms (\autoref{eq:centre_of_mass}), and from it, the radius of gyration, $R_{g}$, was derived (\autoref{eq:radius_of_gyration}) \cite{FIXMAN_1962_ROG}. Since the protein chains were already aligned on the axis and centred on the origin, \textit{O}, the dimensions of the protein chain could be obtained as the magnitude of the PCA components or \textit{eigenvectors}, i.e., the \textit{eigenvalues}. The dimensions represent width, height, and depth for the X, Y and Z axes, respectively.

\begin{equation}
\text{CM} = \frac{1}{n} \sum_{i=1}^{n} r_i \rightarrow \text{CM} = \text{O} = (0,0,0)
\label{eq:centre_of_mass}
\end{equation}
\myequations{Centre of mass}

\vspace{-12pt} % Adjust this value as needed
\vspace{-6pt} % Adjust this value as needed

\begin{equation}
R_g = \sqrt{\frac{1}{n} \sum_{i=1}^{n} (r_i - \text{CM})^2} = \sqrt{\frac{1}{n} \sum_{i=1}^{n} (r_i - \text{O})^2} \rightarrow R_g = \sqrt{\frac{1}{n} \sum_{i=1}^{n} r_i^2}
\label{eq:radius_of_gyration}
\end{equation}
\myequations{Radius of gyration}

\begin{figure}[htb!]
    \centering
    \includegraphics[width=\textwidth]{figures/ch_LBS_COMP/PNG/FIG11_PROTEIN_SHAPE_APPROACH_3_SPLIT1.png}
    \caption[Protein chain shape and size classification approach]{\textbf{Protein chain shape and size classification approach.} The volume of the sphere enclosing the protein chain as well as the protein chain volumes were calculated, and their ratio obtained (\textit{V\textsubscript{R}}). Globular proteins present more spherical shapes and therefore occupy a higher portion of the sphere volume, resulting in higher volume ratios. Non-globular, elongated or fibrous proteins on the other hand do not and present lower volume ratios. After extensive visual examination, a threshold was established at $V_R$ = 0.08, and so proteins classified in these two groups. Proteins were classified as ``tiny'' if their chain was $\leq$ 100 amino acids. Examples for each class are from left to right: \href{https://www.uniprot.org/uniprotkb/Q9Y5G1/entry}{Q9Y5G1} -- PDB: \href{https://www.ebi.ac.uk/pdbe/entry/pdb/6mer}{6MER} \cite{NICLOLUDIS_2019_CADH}, chain: A; \href{https://www.uniprot.org/uniprotkb/P05412/entry}{P05412} -- PDB: \href{https://www.ebi.ac.uk/pdbe/entry/pdb/5t01}{5T01} \cite{HONG_2017_EPSTEINBARR}, chain: A; \href{https://www.uniprot.org/uniprotkb/Q12884/entry}{Q12884} -- PDB: \href{https://www.ebi.ac.uk/pdbe/entry/pdb/6Y0F}{6Y0F} \cite{PDB_6Y0F}, chain: A; \href{https://www.uniprot.org/uniprotkb/Q8NE86/entry}{Q8NE86} -- PDB: \href{https://www.ebi.ac.uk/pdbe/entry/pdb/5KUJ}{5KUJ} \cite{LEE_2016_CALCIUM}. chain: A.}
    \label{fig:protein_class_approach}
\end{figure}

\begin{figure}[htb!]
    \centering
    \includegraphics[width=\textwidth]{figures/ch_LBS_COMP/PNG/FIG11_PROTEIN_SHAPE_APPROACH_2_SPLIT2.png}
    \caption[Protein shape class examples]{\textbf{Protein shape class examples.} Four examples of each protein chain group to illustrate the outcome of the approach. Elongated: \href{https://www.uniprot.org/uniprotkb/Q8NEZ3/entry}{Q8NEZ3} -- PDB: \href{https://www.ebi.ac.uk/pdbe/entry/pdb/8FGW}{8FGW} \cite{JIANG_2023_IFTA}, chain: C; \href{https://www.uniprot.org/uniprotkb/P02679/entry}{P02679} -- PDB: \href{https://www.ebi.ac.uk/pdbe/entry/pdb/3GHG}{3GHG} \cite{KOLLMAN_2009_FIBRINOGEN}, chain: C; \href{https://www.uniprot.org/uniprotkb/Q14126/entry}{Q14126} -- PDB: \href{https://www.ebi.ac.uk/pdbe/entry/pdb/7A7D}{7A7D} \cite{SIKORA_DESMOSOME}, chain: A; \href{https://www.uniprot.org/uniprotkb/Q08554/entry}{Q08554} -- PDB: \href{https://www.ebi.ac.uk/pdbe/entry/pdb/5IRY}{5IRY} \cite{HARRISON_2016_DESMOCOLLINS}, chain: A. Elongated + tiny: \href{https://www.uniprot.org/uniprotkb/Q9H2S9/entry}{Q9H2S9} -- PDB: \href{https://www.ebi.ac.uk/pdbe/entry/pdb/2MA7}{2MA7} \cite{PDB_2MA7}, chain: A; \href{https://www.uniprot.org/uniprotkb/Q9BV73/entry}{Q9BV73} -- PDB: \href{https://www.ebi.ac.uk/pdbe/entry/pdb/6OQA}{6OQA} \cite{SHIGDEL_2020_SMALLMOL}, chain: H; \href{https://www.uniprot.org/uniprotkb/Q8IYW5/entry}{Q8IYW5} -- PDB: \href{https://www.ebi.ac.uk/pdbe/entry/pdb/5YDK}{5YDK} \cite{TAKAHASHI_2018_RNF168}, chain: F; \href{https://www.uniprot.org/uniprotkb/P60880/entry}{P60880} -- PDB: \href{https://www.ebi.ac.uk/pdbe/entry/pdb/3rk2}{3RK2} \cite{KUMMEL_2011_SNARES}, chain: G. Globular: \href{https://www.uniprot.org/uniprotkb/O43451/entry}{O43451} -- PDB: \href{https://www.ebi.ac.uk/pdbe/entry/pdb/3TOP}{3TOP} \cite{REN_2011_MALTASE}, chain: A; \href{https://www.uniprot.org/uniprotkb/P21399/entry}{P21399} -- PDB: \href{https://www.ebi.ac.uk/pdbe/entry/pdb/2B3Y}{2B3Y} \cite{DUPUY_2006_ACONITASE}, chain: A; \href{https://www.uniprot.org/uniprotkb/Q9UI17/entry}{Q9UI17} -- PDB: \href{https://www.ebi.ac.uk/pdbe/entry/pdb/5L46}{5L46} \cite{AUGUSTIN_2016_DEHYDRO}, chain: B; \href{https://www.uniprot.org/uniprotkb/P27487/entry}{P27487} -- PDB: \href{https://www.ebi.ac.uk/pdbe/entry/pdb/3VJM}{3VJM} \cite{YOSHIDA_2012_DPP4}, chain: A. Globular + tiny: \href{https://www.uniprot.org/uniprotkb/Q9UN19/entry}{Q9UN19} -- PDB: \href{https://www.ebi.ac.uk/pdbe/entry/pdb/1FAO}{1FAO} \cite{FERGUSON_2000_PLECKSTRIN}, chain: A; \href{https://www.uniprot.org/uniprotkb/Q12923/entry}{Q12923} -- PDB: \href{https://www.ebi.ac.uk/pdbe/entry/pdb/1D5G}{1D5G} \cite{KOZLOV_2002_PDZ}, chain: A; \href{https://www.uniprot.org/uniprotkb/P42566/entry}{P42566} -- PDB: \href{https://www.ebi.ac.uk/pdbe/entry/pdb/1C07}{1C07} \cite{ENMON_2000_EPS15}, chain: A; \href{https://www.uniprot.org/uniprotkb/P42566/entry}{P42566} PDB: \href{https://www.ebi.ac.uk/pdbe/entry/pdb/1EH2}{1EH2} \cite{BEER_1998_EPS15}, chain: A.}
    \label{fig:protein_class_examples}
\end{figure}

Protein chain volumes were calculated using ProteinVolume \cite{CHEN_2015_PROTEINVOLUME}. A sphere enclosing the protein and centred on the protein centre of mass was obtained. The radius of this sphere was the maximum Euclidean distance between the protein atoms and the CM (\autoref{eq:radius_protein}). The volume of the sphere is calculated using \autoref{eq:volume_sphere}. Proteins were classified into four different groups based on their shape and size. Protein chains with $\leq$ 100 amino acids were classified as ``tiny''. Regarding the shape, protein chains were classified into ``elongated'' if their protein to sphere volume ratio $\leq$ 0.08 ($V_R$) (\autoref{eq:volume_ratio}), i.e., the protein volume contains no more than 8\% of the sphere volume. This threshold was derived empirically by the visual inspection of all 3448 protein chains on the LIGYSIS set. Otherwise, proteins were considered globular (\autoref{fig:protein_class_approach}). In this manner, protein chains were classified into \textit{globular} (\textit{N} = 2104; 61\%), \textit{elongated} \textit{N} = 670; 19\%), \textit{elongated tiny} (\textit{N} = 341; 10\%) and \textit{globular tiny} (\textit{N} = 333; 10\%).

\begin{equation}
R = \max \| r_i - \text{CM} \|
\label{eq:radius_protein}
\end{equation}
\myequations{Protein radius}

\vspace{-12pt} % Adjust this value as needed

\begin{equation}
V_{S} = \frac{4}{3} \pi R^3
\label{eq:volume_sphere}
\end{equation}
\myequations{Sphere volume}

\vspace{-12pt} % Adjust this value as needed

\begin{equation}
V_R = \frac{V_{P}}{V_{S}}
\label{eq:volume_ratio}
\end{equation}
\myequations{Volume ratio}

%\FloatBarrier

\subsection{Ligand binding site prediction}

For each segment in the LIGYSIS dataset, the representative chain as defined in the PDBe-KB was selected. Structures were cleaned using the \textit{clean\_pdb.py} script \cite{JUBB_2019_PDBTOOLS}. Eleven different ligand binding site prediction tools were used to predict on the 3448 representative chains: VN-EGNN \cite{SESTAK_2024_VNEGNN}, IF-SitePred \cite{CARBERY_2024_IFSP}, GrASP \cite{SMITH_2024_GrASP}, PUResNet \cite{KANDEL_2021_PURESNET, KANDEL_2024_PURESNET}, DeepPocket \cite{AGGARWAL_2022_DEEPPOCKET}, P2Rank \cite{KRIVAK_2015_P2RANK, KRIVAK_2018_P2RANK}, PRANK \cite{KRIVAK_2015_PRANK}, fpocket \cite{GUILLOUX_2009_FPOCKET, SCHMIDTKE_2010_FPOCKET2}, PocketFinder\textsuperscript{+} \cite{AN_2005_POCKETFINDER}, Ligsite\textsuperscript{+} \cite{HENDLICH_1997_LIGSITE}, and Surfnet\textsuperscript{+} \cite{LASKOWSKI_1995_SURFNET}. Conservation scores for P2Rank were obtained from PrankWeb: \url{https://prankweb.cz/} and used for further prediction. This variant of P2Rank employing amino acid conservation is referred to as P2Rank\textsubscript{CONS} \cite{JENDELE_2019_PRANKWEB, JAKUBEC_2022_PRANKWEB}. When running DeepPocket, the \textit{-r} threshold was removed and so all fpocket candidates were passed to the CNN-based segmentation module for pocket shape estimation. fpocket predictions re-scored by DeepPocket will be referred as DeepPocket\textsubscript{RESC}, whereas pockets extracted by the segmentation module of DeepPocket will be referred as DeepPocket\textsubscript{SEG}. fpocket predictions were also re-scored with PRANK \cite{KRIVAK_2015_PRANK} (fpocket\textsubscript{PRANK}), as introduced in previous studies \cite{KRIVAK_2015_PRANK, KRIVAK_2015_P2RANK, KRIVAK_2018_P2RANK, COMAJUNCOSA_2024_POCKETS}. Re-implementations of Capra \textit{et al.} \cite{CAPRA_2009_CONCAVITY} were used for PocketFinder, Ligsite and Surfnet, indicated by the ``+'' superscript. VN-EGNN, IF-SitePred, PocketFinder\textsuperscript{+}, Ligsite\textsuperscript{+} and Surfnet\textsuperscript{+} do not provide a list of residues for each pocket, but a list of centroids and their scores for the first two, and a list of grid points for each predicted pocket for the last three. For VN-EGNN, residues within 6 \AA{} of the virtual nodes were considered pocket residues. For 429 predicted pockets ($\approx$3\%) no residues were found within this threshold. For IF-SitePred, residues within 6 \AA{} of the clustered cloud points that resulted on a predicted pocket centroid were considered as pocket residues. Pocket residues were obtained in a similar manner for PocketFinder\textsuperscript{+}, Ligsite\textsuperscript{+} and Surfnet\textsuperscript{+}, by taking those residues within 6 \AA{} of the pocket grid points. In total, thirteen methods are considered in this analysis: VN-EGNN, IF-SitePred, GrASP, PUResNet, DeepPocket\textsubscript{RESC}, DeepPocket\textsubscript{SEG}, P2Rank\textsubscript{CONS}, P2Rank, fpocket\textsubscript{PRANK}, fpocket, PocketFin-der\textsuperscript{+}, Ligsite\textsuperscript{+} and Surfnet\textsuperscript{+}.

Seven of the considered methods provide residue \textit{ligandability} scores. P2Rank and P2Rank\textsubscript{CONS} report calibrated probabilities of residues being ligand-binding. Similarly, GrASP predicts the likelihood for any given heavy atom to be part of a binding site. A residue-level score was obtained for GrASP predictions by taking the maximum score of the residue atoms. For IF-SitePred, a residue ligandability score (LS) was computed by averaging the 40 independently predicted probabilities of a residue being ligand-binding (\autoref{eq:IFSP_score}). Though calculated in a different way, these three scores range 0-1, represent the likelihood of a residue binding a ligand and can therefore be compared. PocketFinder\textsuperscript{+}, Ligsite\textsuperscript{+}, and Surfnet\textsuperscript{+} also provide residue scores which maximum value can be $>$ 1.

\begin{equation}
LS = \frac{1}{40} \sum_{i=1}^{40} p_i
\label{eq:IFSP_score}
\end{equation}
\myequations{IF-SitePred ligandability score}

VN-EGNN, PUResNet, DeepPocket\textsubscript{RESC}, DeepPocket\textsubscript{SEG}, fpocket\textsubscript{PRANK} and fpocket do not report residue-level scores. However, binary labels represent whether a residue is part of a pocket (1) or not (0), in the same manner as all other methods. Throughout this Chapter, the terms ``site'' and ``pocket'' are used indistinctly. Methods are sorted in chronological order across all figures, tables and legends.

\subsection{Binding site characterisation}

Radius of gyration was calculated for pockets as it was done for whole protein chains (\autoref{eq:radius_of_gyration}). Distance between pockets was calculated as the Euclidean distance between their centroids and overlap between pocket residues with the Jaccard Index (JI), or intersection over union (IOU) (\autoref{eq:jaccard_index}) \cite{JACCARD_1901_INDEX, JACCARD_1912_INDEX}. POVME 2.0 was employed for pocket volume calculation \cite{DURRANT_2011_POVME, DURRANT_2014_POVME2, WAGNER_2017_POVME3}. A single inclusion region was used for each pocket. This region was defined by the smallest rectangular prism containing all pocket atoms. The prism was centred on the pocket centroid and its dimensions were determined by the distance between the two farthest atomic coordinates on each axis. No exclusion regions were used. Points outside the convex hull were deleted. A contiguous-points region was defined as a 5 \AA{}-radius sphere on the pocket centroid (\autoref{fig:protein_volume_approach}).

\begin{equation}
JI(A, B) = \frac{|A \cap B|}{|A \cup B|}
\label{eq:jaccard_index}
\end{equation}
\myequations{Jaccard index}

\begin{figure}[h]
    \centering
    \includegraphics[width=\textwidth]{figures/ch_LBS_COMP/PNG/FIG12_POCKET_VOLUME_APPROACH.png}
    \caption[Pocket volume calculation algorithm]{\textbf{Pocket volume calculation algorithm.} \textbf{(A)} PUResNet predicted pocket for PDB: \href{https://www.ebi.ac.uk/pdbe/entry/pdb/4PX2}{4PX2} \cite{PDB_4PX2}. Pocket residues are coloured in blue and have their side chains displayed; \textbf{(B)} An inclusion region is determined using the coordinates of the pocket residue atoms; \textbf{(C)} POVME 2.0 calculates the shape of the pocket defined by the residues and contained within the inclusion region; \textbf{(D)} The pocket shape is defined by a series of unit-volume (1 \AA{}\textsuperscript{3}) spheres. The volume of the pocket is calculated as the addition of the sphere volumes or the number of spheres within the pocket. Structure visualisation with PyMOL v2.5.2 \cite{SCHRODINGER_2015_PYMOL}.}
    \label{fig:protein_volume_approach}
\end{figure}

\vspace{-12pt} % Adjust this value as needed
\vspace{-12pt} % Adjust this value as needed

\subsubsection{Determination of DCC threshold}

Most methods employ distance to closest ligand atom  (DCA) and a threshold of 4 \AA{} to consider a prediction as correct. Because of the way the LIGYSIS dataset has been curated, it is easier to use DCC, since binding sites result of the clustering of multiple ligands, and not just a single ligand binding a protein. Despite DCC and DCA being different metrics, the same threshold of 4 \AA{} is used for both when benchmarking methods \cite{AGGARWAL_2022_DEEPPOCKET, SESTAK_2024_VNEGNN, KANDEL_2021_PURESNET}. \autorefpanel{fig:irel_vs_dcc}{ A} shows the relation between DCC, and pocket residue overlap for the best pocket predictions, i.e., minimum Euclidean distance between predicted and observed pocket centroid, for each observed pocket for each method. Across all methods, there are more than 15,000 predicted pockets with a DCC $>$ 4 \AA{} and a residue overlap $\geq$ 0.5. Setting the DCC threshold at 4 \AA{} would result in the wrong labelling of these predictions as ``false positives''. For this reason, a more meaningful DCC threshold was empirically established through the thorough visual inspection of predicted-observed pocket pairs. \autorefpanel{fig:irel_vs_dcc}{ B} suggests this threshold might be somewhere in between 10-15 \AA{}, where the proportion of pockets with \textit{I\textsubscript{rel}} $\geq$ 0.5 decreases until reaching 0.

\begin{figure}[htb!]
    \centering
    \includegraphics[width=\textwidth]{figures/ch_LBS_COMP/PNG/SUPP_FIG10_IREL_vs_DCC.png}
    \caption[\textit{I\textsubscript{rel}} \textit{vs} DCC]{\textbf{\textit{I\textsubscript{rel}} \textit{vs} DCC.} \textbf{(A)} Hexagonal binned plot of \textit{I\textsubscript{rel}} (Y) \textit{vs} DCC (X). Data points are grouped into hexagonal bins which are coloured by the number of data points within each bin using the \textit{viridis} colour palette. Colour bar axis is in log\textsubscript{10} scale. Black dashed lines indicate the literature consensus DCC = 4 \AA{} threshold and an arbitrary \textit{I\textsubscript{rel}} threshold of 0.5, i.e., coverage of half of the observed ligand-binding residues by the predicted pocket.  Red lines delimit the likely location of a potentially more informative DCC thresholds; \textbf{(B)} Cumulative proportion of predicted pockets with \textit{I\textsubscript{rel}} $\geq$ 0.5 for each DCC 1 \AA{} interval. The commonly used threshold of DCC = 4 \AA{} would label $>$15,000 predictions with \textit{I\textsubscript{rel}} $\geq$ 0.5 as false. Error bars indicate 95\% CI of the proportion.}
    \label{fig:irel_vs_dcc}
\end{figure}

\begin{figure}[p]
    \centering
    \includegraphics[width=\textwidth]{figures/ch_LBS_COMP/PNG/SUPP_FIG11_DETERMINING_DCC_THRESHOLD_SPLIT1.png}
    \caption[Determination of DCC threshold (I)]{\textbf{Determination of DCC threshold (I).} Highest and lowest-residue overlap predictions for each 2 \AA{} DCC unit interval. Observed LIGYSIS sites are coloured in green, predicted pockets in other colours. \textit{D} represents DCC and \textit{I\textsubscript{rel}} the relative intersection between predicted and observed pocket residues, i.e., proportion of observed site residues covered by predicted pocket residues. ``YES'' or ``NO'' labels indicate whether a prediction was considered correct upon visual inspection. ``?'' at DCC = 12 \AA{} illustrates the inflection point between 10-12 \AA{}, where it is no longer clear whether predicted pockets within this DCC interval and \textit{I\textsubscript{rel}} $\approx$ 0 agree with the observed pockets. To facilitate the visualisation of the observed pocket, this one is coloured after the predicted one. Otherwise, for cases where \textit{I\textsubscript{rel}} = 1 only the predicted pocket would be shown. Despite 1 \AA{} intervals were inspected, only representatives of 2 \AA{} intervals are shown here for simplicity. Examples from left to right and top to bottom: \href{https://www.uniprot.org/uniprotkb/O75417/entry}{O75417} -- PDB: \href{https://www.ebi.ac.uk/pdbe/entry/pdb/5A9J}{5A9J} \cite{NEWMAN_2015_POLYMERASE}, chain: D; \href{https://www.uniprot.org/uniprotkb/P01574/entry}{P01574} -- PDB: \href{https://www.ebi.ac.uk/pdbe/entry/pdb/1AU1}{1AU1} \cite{KARPUSAS_1997_INTERFERON}, chain: A; \href{https://www.uniprot.org/uniprotkb/Q01118/entry}{Q01118} -- PDB: \href{https://www.ebi.ac.uk/pdbe/entry/pdb/7TJ8}{7TJ8} \cite{NOLAND_2022_NAX}, chain: A; \href{https://www.uniprot.org/uniprotkb/Q92847/entry}{Q92847} -- PDB: \href{https://www.ebi.ac.uk/pdbe/entry/pdb/7W2Z}{7W2Z} \cite{QIN_2022_GHRELIN}, chain: R; \href{https://www.uniprot.org/uniprotkb/Q9BY49/entry}{Q9BY49} -- PDB: \href{https://www.ebi.ac.uk/pdbe/entry/pdb/1YXM}{1YXM} \cite{PDB_1YXM}, chain: A; \href{https://www.uniprot.org/uniprotkb/O15178/entry}{O15178} -- PDB: \href{https://www.ebi.ac.uk/pdbe/entry/pdb/8FMU}{8FMU} \cite{CHASE_2024_BRACHYURY}, chain: B; \href{https://www.uniprot.org/uniprotkb/Q14534/entry}{Q14534} -- PDB: \href{https://www.ebi.ac.uk/pdbe/entry/pdb/6C6P}{6C6P} \cite{PADYANA_2019_EPOXIDASE}, chain: A; \href{https://www.uniprot.org/uniprotkb/P21728/entry}{P21728} -- PDB: \href{https://www.ebi.ac.uk/pdbe/entry/pdb/8IRR}{8IRR} \cite{XU_2023_DOPAMINE}, chain: R; \href{https://www.uniprot.org/uniprotkb/Q9P2W7/entry}{Q9P2W7} -- PDB: \href{https://www.ebi.ac.uk/pdbe/entry/pdb/1V84}{1V84} \cite{KAKUDA_2004_GLCATP}, chain: A; \href{https://www.uniprot.org/uniprotkb/Q9P0M2/entry}{Q9P0M2} -- PDB: \href{https://www.ebi.ac.uk/pdbe/entry/pdb/5JJ2}{5JJ2} \cite{BJERREGAARD_2016_AKAP18}, chain: A; \href{https://www.uniprot.org/uniprotkb/Q8N695/entry}{Q8N695} -- PDB: \href{https://www.ebi.ac.uk/pdbe/entry/pdb/7SL9}{7SL9} \cite{HAN_2022_SGLT}, chain: A; \href{https://www.uniprot.org/uniprotkb/P41145/entry}{P41145} -- PDB: \href{https://www.ebi.ac.uk/pdbe/entry/pdb/6VI4}{6VI4} \cite{CHE_2020_NANOBODY}, chain: A.}
    \label{fig:DCC_determination1}
\end{figure}
%\FloatBarrier

\begin{figure}[p]
    \centering
    \includegraphics[width=\textwidth]{figures/ch_LBS_COMP/PNG/SUPP_FIG11_DETERMINING_DCC_THRESHOLD_SPLIT2.png}
    \caption[Determination of DCC threshold (II)]{\textbf{Determination of DCC threshold (II).} Highest and lowest-residue overlap predictions for each 2 \AA{} DCC unit interval. Observed LIGYSIS sites are coloured in green, predicted pockets in other colours. \textit{D} represents DCC and \textit{I\textsubscript{rel}} the relative intersection between predicted and observed pocket residues, i.e., proportion of observed site residues covered by predicted pocket residues. ``YES'' indicates that a prediction was considered correct upon visual inspection. To facilitate the visualisation of the observed pocket, this one is coloured after the predicted one. Otherwise, for cases where \textit{I\textsubscript{rel}} = 1 only the predicted pocket would be shown. Despite 1 \AA{} intervals were inspected, only representatives of 2 \AA{} intervals are shown here for simplicity. Examples from left to right and top to bottom: \href{https://www.uniprot.org/uniprotkb/Q9UDR5/entry}{Q9UDR5} -- PDB: \href{https://www.ebi.ac.uk/pdbe/entry/pdb/5L78}{5L78} \cite{PDB_5L78}, chain: B; \href{https://www.uniprot.org/uniprotkb/P16473/entry}{P16473} -- PDB: \href{https://www.ebi.ac.uk/pdbe/entry/pdb/7XW7}{7XW7} \cite{DUAN_2022_THYROTROPIN}, chain: R; \href{https://www.uniprot.org/uniprotkb/Q09013/entry}{Q09013} -- PDB: \href{https://www.ebi.ac.uk/pdbe/entry/pdb/2VD5}{2VD5} \cite{ELKINS_2009_KINASE}, chain: B; \href{https://www.uniprot.org/uniprotkb/Q15047/entry}{Q15047} -- PDB: \href{https://www.ebi.ac.uk/pdbe/entry/pdb/6BHD}{6BHD} \cite{JURKOWSKA_2017_H3K9}, chain: A; \href{https://www.uniprot.org/uniprotkb/Q9BXT4/entry}{Q9BXT4} -- PDB: \href{https://www.ebi.ac.uk/pdbe/entry/pdb/5M9N}{5M9N} \cite{PDB_5M9N}, chain: A; \href{https://www.uniprot.org/uniprotkb/Q9UGM1/entry}{Q9UGM1} -- PDB: \href{https://www.ebi.ac.uk/pdbe/entry/pdb/4UY2}{4UY2} \cite{ZOURIDAKIS_2014_NICOTINIC}, chain: B; \href{https://www.uniprot.org/uniprotkb/Q9UHV8/entry}{Q9UHV8} -- PDB: \href{https://www.ebi.ac.uk/pdbe/entry/pdb/5XG7}{5XG7} \cite{SU_2018_GALECTIN}, chain: A; \href{https://www.uniprot.org/uniprotkb/Q15303/entry}{Q15303} -- PDB: \href{https://www.ebi.ac.uk/pdbe/entry/pdb/3BCE}{3BCE} \cite{QIU_2008_HER4}, chain: A; \href{https://www.uniprot.org/uniprotkb/Q8IVW4/entry}{Q8IVW4} -- PDB: \href{https://www.ebi.ac.uk/pdbe/entry/pdb/3ZDU}{3ZDU} \cite{CANNING_2018_CDKL}, chain: A; \href{https://www.uniprot.org/uniprotkb/P49760/entry}{P49760} -- PDB: \href{https://www.ebi.ac.uk/pdbe/entry/pdb/6KHE}{6KHE} \cite{LEE_2019_CDC2}, chain: A.}
    \label{fig:DCC_determination2}
\end{figure}
%\FloatBarrier

\begin{figure}[p]
    \centering
    \includegraphics[width=\textwidth]{figures/ch_LBS_COMP/PNG/SUPP_FIG12_EXAMPLES_DCC_10_NEW.png}
    \caption[Predicted-observed pocket pairs at DCC = 10 \AA{} and \textit{I\textsubscript{rel}} $<$ 0.25]{\textbf{Predicted-observed pocket pairs at DCC = 10 \AA{} and \textit{I\textsubscript{rel}} $<$ 0.25.} 94/100 visually inspected examples were considered as correct predictions on the bases that the predicted and observed pockets are adjacent, i.e., their surface area is in contact, and it is therefore easy to imagine a ligand that would bind to this region. LIGYSIS observed sites are coloured in green, and predicted pockets in other colours. Examples from left to right and top to bottom: \href{https://www.uniprot.org/uniprotkb/P22303/entry}{P22303} -- PDB: \href{https://www.ebi.ac.uk/pdbe/entry/pdb/5HQ3}{5HQ3} \cite{GOLDENZWEIG_2016_PROTSTABILITY}, chain: B; \href{https://www.uniprot.org/uniprotkb/O76003/entry}{O76003} -- PDB: \href{https://www.ebi.ac.uk/pdbe/entry/pdb/2YAN}{2YAN} \cite{PDB_2YAN}, chain: A; \href{https://www.uniprot.org/uniprotkb/P13498/entry}{P13498} -- PDB: \href{https://www.ebi.ac.uk/pdbe/entry/pdb/8WEJ}{8WEJ} \cite{LIU_2024_NADPH}, chain: A; \href{https://www.uniprot.org/uniprotkb/Q9HD26/entry}{Q9HD26} -- PDB: \href{https://www.ebi.ac.uk/pdbe/entry/pdb/2LOB}{2LOB} \cite{PISERCHIO_2005_CAL}, chain: A; \href{https://www.uniprot.org/uniprotkb/P35247/entry}{P35247} -- PDB: \href{https://www.ebi.ac.uk/pdbe/entry/pdb/5OXS}{5OXS} \cite{LITTLEJOHN_2018_HSPD}, chain: C; \href{https://www.uniprot.org/uniprotkb/Q96TC7/entry}{Q96TC7} -- PDB: \href{https://www.ebi.ac.uk/pdbe/entry/pdb/7CC7}{7CC7} \cite{YEO_2021_PTPIP51}, chain: A; \href{https://www.uniprot.org/uniprotkb/P10828/entry}{P10828} -- PDB: \href{https://www.ebi.ac.uk/pdbe/entry/pdb/1NQ1}{1NQ1} \cite{HUIBER_2003_THYROID}, chain: A; \href{https://www.uniprot.org/uniprotkb/Q9BXJ8/entry}{Q9BXJ8} -- PDB: \href{https://www.ebi.ac.uk/pdbe/entry/pdb/7F3U}{7F3U} \cite{RONG_2021_TMEM120A}, chain: B; \href{https://www.uniprot.org/uniprotkb/P21728/entry}{P21728} -- PDB: \href{https://www.ebi.ac.uk/pdbe/entry/pdb/8IRR}{8IRR} \cite{XU_2023_DOPAMINE}, chain: R; \href{https://www.uniprot.org/uniprotkb/O00213/entry}{O00213} -- PDB: \href{https://www.ebi.ac.uk/pdbe/entry/pdb/3D8E}{3D8E} \cite{RADZIMANOWSKI_2008_PTB1}, chain: C; \href{https://www.uniprot.org/uniprotkb/P21728/entry}{P21728} -- PDB: \href{https://www.ebi.ac.uk/pdbe/entry/pdb/8IRR}{8IRR}, chain: B; \href{https://www.uniprot.org/uniprotkb/P48546/entry}{P48546} -- PDB: \href{https://www.ebi.ac.uk/pdbe/entry/pdb/7DTY}{7DTY} \cite{ZHAO_2021_RECEPTOR}, chain: R; \href{https://www.uniprot.org/uniprotkb/Q9H3H5/entry}{Q9H3H5} -- PDB: \href{https://www.ebi.ac.uk/pdbe/entry/pdb/6BW6}{6BW6} \cite{YOO_2018_GLCNAC}, chain: B; \href{https://www.uniprot.org/uniprotkb/O95278/entry}{O95278} -- PDB: \href{https://www.ebi.ac.uk/pdbe/entry/pdb/4RKK}{4RKK} \cite{RATHTHAGALA_2015_LAFORA}, chain: A; \href{https://www.uniprot.org/uniprotkb/Q96BI3/entry}{Q96BI3} -- PDB: \href{https://www.ebi.ac.uk/pdbe/entry/pdb/5FN5}{5FN5} \cite{BAI_2015_secretase}, chain: C; \href{https://www.uniprot.org/uniprotkb/Q93096/entry}{Q93096} -- PDB: \href{https://www.ebi.ac.uk/pdbe/entry/pdb/5BX1}{5BX1} \cite{PDB_5BX1}, chain: A;}
    \label{fig:DCC_10_examples}
\end{figure}
%\FloatBarrier

\begin{figure}[p]
    \centering
    \includegraphics[width=\textwidth]{figures/ch_LBS_COMP/PNG/SUPP_FIG13_EXAMPLES_DCC_11_NEW.png}
    \caption[Predicted-observed pocket pairs at DCC = 11 \AA{} and \textit{I\textsubscript{rel}} $<$ 0.25]{\textbf{Predicted-observed pocket pairs at DCC = 11 \AA{} and \textit{I\textsubscript{rel}} $<$ 0.25.} 86/100 visually inspected examples were considered as correct predictions. LIGYSIS observed sites are coloured in green, and predicted pockets in other colours. Examples from left to right and top to bottom: \href{https://www.uniprot.org/uniprotkb/O43451/entry}{O43451} -- PDB: \href{https://www.ebi.ac.uk/pdbe/entry/pdb/3TOP}{3TOP} \cite{REN_2011_MALTASE}, chain: A; \href{https://www.uniprot.org/uniprotkb/P04180/entry}{P04180} -- PDB: \href{https://www.ebi.ac.uk/pdbe/entry/pdb/5TXF}{5TXF} \cite{MANTHEI_2017_APOLIPOPROTEIN}, chain: B; \href{https://www.uniprot.org/uniprotkb/P49190/entry}{P49190} -- PDB: \href{https://www.ebi.ac.uk/pdbe/entry/pdb/7F16}{7F16} \cite{WANG_2021_PHR2}, chain: R; \href{https://www.uniprot.org/uniprotkb/P21728/entry}{P21728} -- PDB: \href{https://www.ebi.ac.uk/pdbe/entry/pdb/8IRR}{8IRR} \cite{XU_2023_DOPAMINE}, chain: R; \href{https://www.uniprot.org/uniprotkb/Q16394/entry}{Q16394} -- PDB: \href{https://www.ebi.ac.uk/pdbe/entry/pdb/7SCH}{7SCH} \cite{LI_2023_EXT12}, chain: A; \href{https://www.uniprot.org/uniprotkb/Q8IU60/entry}{Q8IU60} -- PDB: \href{https://www.ebi.ac.uk/pdbe/entry/pdb/5MP0}{5MP0} \cite{PDB_5MP0}, chain: D; \href{https://www.uniprot.org/uniprotkb/P56192/entry}{P56192} -- PDB: \href{https://www.ebi.ac.uk/pdbe/entry/pdb/5Y6L}{5Y6L} \cite{PDB_5Y6L}, chain: A; \href{https://www.uniprot.org/uniprotkb/Q5VSL9/entry}{Q5VSL9} -- PDB: \href{https://www.ebi.ac.uk/pdbe/entry/pdb/7K36}{7K36} \cite{JEONG_2021_STRIPAK}, chain: I; \href{https://www.uniprot.org/uniprotkb/P20160/entry}{P20160} -- PDB: \href{https://www.ebi.ac.uk/pdbe/entry/pdb/1FY1}{1FY1} \cite{KASTRUP_2001_CAP37}, chain: A; \href{https://www.uniprot.org/uniprotkb/P06280/entry}{P06280} -- PDB: \href{https://www.ebi.ac.uk/pdbe/entry/pdb/4NXS}{4NXS} \cite{YU_2014_FABRY}, chain: B; \href{https://www.uniprot.org/uniprotkb/Q9H490/entry}{Q9H490} -- PDB: \href{https://www.ebi.ac.uk/pdbe/entry/pdb/7WLD}{7WLD} \cite{XU_2022_GLYCOSYL}, chain: U; \href{https://www.uniprot.org/uniprotkb/O95631/entry}{O95631} -- PDB: \href{https://www.ebi.ac.uk/pdbe/entry/pdb/7NDG}{7NDG} \cite{ROBINSON_2021_NET1}, chain: G; \href{https://www.uniprot.org/uniprotkb/Q00975/entry}{Q00975} -- PDB: \href{https://www.ebi.ac.uk/pdbe/entry/pdb/7MIX}{7MIX} \cite{GGAO_2021_CAV2}, chain: A; \href{https://www.uniprot.org/uniprotkb/O15496/entry}{O15496} -- PDB: \href{https://www.ebi.ac.uk/pdbe/entry/pdb/5G3M}{5G3M} \cite{GIORDANETTO_2016_SPLA2}, chain: B; \href{https://www.uniprot.org/uniprotkb/Q8TCJ2/entry}{Q8TCJ2} -- PDB: \href{https://www.ebi.ac.uk/pdbe/entry/pdb/6S7T}{6S7T} \cite{RAMIREZ_2019_OSTA}, chain: A; \href{https://www.uniprot.org/uniprotkb/P41145/entry}{P41145} -- PDB: \href{https://www.ebi.ac.uk/pdbe/entry/pdb/6VI4}{6VI4} \cite{CHE_2020_NANOBODY}, chain: A.}
    \label{fig:DCC_11_examples}
\end{figure}
%\FloatBarrier

\begin{figure}[p]
    \centering
    \includegraphics[width=\textwidth]{figures/ch_LBS_COMP/PNG/SUPP_FIG14_EXAMPLES_DCC_12_NEW.png}
    \caption[Predicted-observed pocket pairs at DCC = 12 \AA{} and \textit{I\textsubscript{rel}} $<$ 0.25]{\textbf{Predicted-observed pocket pairs at DCC = 12 \AA{} and \textit{I\textsubscript{rel}} $<$ 0.25.} 85/100 visually inspected examples were considered as correct predictions. LIGYSIS observed sites are coloured in green, and predicted pockets in other colours. Examples from left to right and top to bottom: \href{https://www.uniprot.org/uniprotkb/P50616/entry}{P50616} -- PDB: \href{https://www.ebi.ac.uk/pdbe/entry/pdb/2Z15}{2Z15} \cite{PDB_2Z15}, chain: C; \href{https://www.uniprot.org/uniprotkb/Q9BXT4/entry}{Q9BXT4} -- PDB: \href{https://www.ebi.ac.uk/pdbe/entry/pdb/5M9N}{5M9N} \cite{PDB_5M9N}, chain: B; \href{https://www.uniprot.org/uniprotkb/Q9P0X4/entry}{Q9P0X4} -- PDB: \href{https://www.ebi.ac.uk/pdbe/entry/pdb/7WLK}{7WLK} \cite{HE_2022_CAV3}, chain: A; \href{https://www.uniprot.org/uniprotkb/P08648/entry}{P08648} -- PDB: \href{https://www.ebi.ac.uk/pdbe/entry/pdb/3VI3}{3VI3} \cite{NAGAE_2012_INTEGRIN}, chain: A; \href{https://www.uniprot.org/uniprotkb/P22102/entry}{P22102} -- PDB: \href{https://www.ebi.ac.uk/pdbe/entry/pdb/2QK4}{2QK4} \cite{WELIN_2020_GART}, chain: B; \href{https://www.uniprot.org/uniprotkb/P48546/entry}{P48546} -- PDB: \href{https://www.ebi.ac.uk/pdbe/entry/pdb/7DTY}{7DTY} \cite{ZHAO_2021_RECEPTOR}, chain: R; \href{https://www.uniprot.org/uniprotkb/Q99496/entry}{Q99496} -- PDB: \href{https://www.ebi.ac.uk/pdbe/entry/pdb/4S3O}{4S3O} \cite{TAHERBHOY_2015_RING}, chain: B; \href{https://www.uniprot.org/uniprotkb/O15496/entry}{O15496} -- PDB: \href{https://www.ebi.ac.uk/pdbe/entry/pdb/5G3M}{5G3M} \cite{GIORDANETTO_2016_SPLA2}, chain: B; \href{https://www.uniprot.org/uniprotkb/P16471/entry}{P16471} -- PDB: \href{https://www.ebi.ac.uk/pdbe/entry/pdb/3MZG}{3MZG} \cite{KULKARNI_2010_HISTIDINES}, chain: B; \href{https://www.uniprot.org/uniprotkb/P35247/entry}{P35247} -- PDB: \href{https://www.ebi.ac.uk/pdbe/entry/pdb/5OXS}{5OXS} \cite{LITTLEJOHN_2018_HSPD}, chain: C; \href{https://www.uniprot.org/uniprotkb/Q00688/entry}{Q00688} -- PDB: \href{https://www.ebi.ac.uk/pdbe/entry/pdb/2MPH}{2MPH} \cite{PRAKASH_2016_RECOGNITION}, chain: A; \href{https://www.uniprot.org/uniprotkb/Q6PL18/entry}{Q6PL18} -- PDB: \href{https://www.ebi.ac.uk/pdbe/entry/pdb/7M98}{7M98} \cite{EVANS_2021_ATAD2}, chain: A; \href{https://www.uniprot.org/uniprotkb/Q9H082/entry}{Q9H082} -- PDB: \href{https://www.ebi.ac.uk/pdbe/entry/pdb/6ZAY}{6ZAY} \cite{PANTOOM_2021_RAB33B}, chain: A; \href{https://www.uniprot.org/uniprotkb/P00156/entry}{P00156} -- PDB: \href{https://www.ebi.ac.uk/pdbe/entry/pdb/5XTE}{5XTE} \cite{GUO_2017_CYTOCHROME}, chain: J; \href{https://www.uniprot.org/uniprotkb/P00374/entry}{P00374} -- PDB: \href{https://www.ebi.ac.uk/pdbe/entry/pdb/1DRF}{1DRF} \cite{OEFNER_1988_FOLATE}, chain: A; \href{https://www.uniprot.org/uniprotkb/P02746/entry}{P02746} -- PDB: \href{https://www.ebi.ac.uk/pdbe/entry/pdb/2JG9}{2JG9} \cite{PAIDASSI_2008_C1Q}, chain: F.}
    \label{fig:DCC_12_examples}
\end{figure}
%\FloatBarrier

A hard threshold was set at \textit{D} = 20 \AA{} and a decision made so that based purely on distance, pockets with DCC $>$ 20 would not be considered as correct predictions. For each DCC interval of 1 \AA{}, the pocket with the highest and lowest \textit{I\textsubscript{rel}} were inspected (\autoref{fig:DCC_determination1} and \autoref{fig:DCC_determination2}). This initial visual inspection supported the hypothesis that a more meaningful DCC threshold is between 10-14 \AA{}. For the next step, only predicted-observed pocket pairs with minimal overlap (\textit{I\textsubscript{rel}} $<$ 0.25) were considered. Starting at DCC = 10 \AA{}, and using unit (1 \AA{}) intervals, the 100 farthest pockets were inspected for each interval, and the proportion of correct predictions was calculated as the number of pockets labelled as ``correct'' upon visual inspection divided by 100, i.e., \%. For \textit{D} = 10 \AA{}, 94\% of pockets were correct (\autoref{fig:DCC_10_examples}), 86\% for \textit{D} = 11 \AA{} (\autoref{fig:DCC_11_examples}), 85\% for \textit{D} = 12 \AA{} (\autoref{fig:DCC_12_examples}) and 66\% for \textit{D} = 13 \AA{}. Due to the considerable drop of correct pockets at \textit{D} = 13 \AA{}, the final distance threshold was set at \textit{D} = 12\AA{}. Accordingly, predictions were considered as true positives if DCC $\leq$ 12 \AA{} (\autoref{eq:DCC_threshold}).

\begin{equation}
\textit{True Positive} \iff (\text{DCC} \leq 12\text{\AA{}})
\label{eq:DCC_threshold}
\end{equation}
\myequations{Empirically determined DCC threshold}

\vspace{-12pt} % Adjust this value as needed
\vspace{-12pt} % Adjust this value as needed
\vspace{-12pt} % Adjust this value as needed

\subsection{Prediction evaluation}

LIGYSIS binding sites consist of sets of UniProt residue numbers to which ligands bind across the multiple structures of a protein. The thirteen ligand binding site predictors benchmarked in this work predict only on the representative chains for each protein. These representative structures are defined in the PDBe-KB based on three criteria: data quality, sequence coverage and resolution \cite{ELLAWAY_2024_CONFORMATIONS}. Despite this, representative chains might still be missing some residues present in other structures. To compare LIGYSIS binding sites to predicted sites on the representative chains, UniProt sequence mappings are needed for each residue in the LIGYSIS-defined sites. For this reason, LIGYSIS entries with ligand-binding residues missing UniProt residue mappings on the protein's representative chain were discarded, resulting
in a set of 3048 human proteins, including 3448 segments. After predicting on these 3448 LIGYSIS chains, only chains where all residues across all predicted sites presented UniProt residue mapping were kept. This resulted in a final set of 2775 protein chains which was employed to assess the performance of the methods.

The performance of ligand binding site prediction methods can be evaluated at two different levels: \textit{residue} level, and \textit{pocket} level. Prediction at the residue level involves the discrimination of those residues that are likely to interact with a ligand, whereas the aim of pocket-level prediction is to define distinct regions on a protein, i.e., pockets, where \textit{a} ligand is likely to bind. This region can either be defined by a centroid, a group of cloud/grid points, a set of residues, or a combination of these. Some methods are \textit{residue}-centric, and first predict at the residue-level, use a threshold to select high-probability ligand-binding residues, and then cluster them into pockets. Residue-centric methods include IF-SitePred, or GrASP. Other (\textit{pocket}-centric) methods directly predict the location or shape of the pocket, without the need of predicting at the residue level first. Some of these methods can use their pocket-level prediction to report residue ligandability scores, e.g., P2Rank\textsubscript{CONS}, P2Rank, PocketFinder\textsuperscript{+}, Ligsite\textsuperscript{+} or Surfnet\textsuperscript{+}, and others, such as VN-EGNN, PUResNet, DeepPocket, or fpocket do not report residue ligandability scores.

\FloatBarrier

\subsubsection{Residue-level predictions}

GrASP, P2RANK\textsubscript{CONS}, P2Rank, PocketFinder\textsuperscript{+}, Ligsite\textsuperscript{+} and Surfnet\textsuperscript{+} all offer residue ligandability scores. Additionally, a ligandability score was derived for IF-SitePred using \autoref{eq:IFSP_score}. Prediction at the residue level is a binary classification problem: binding (1) or non-binding (0). Given a ligandability threshold $t_{LS}$, a residue $i$ with a ligandability score $LS_{i}$ is classified as ``positive'' if  $LS_{i} > t_{LS}$. Conversely, the residue is classed as ``negative'' if $LS_{i} \leq t_{LS}$. Further stratification results from comparing the predictions to the LIGYSIS reference dataset.

\begin{itemize}
\item True Positive (TP): residue classified as positive that binds a ligand according to the reference.
\item False Positive (FP): residue classified as positive that does not bind a ligand in the reference.
\item True Negative (TN): residue classified as negative that does not bind a ligand.
\item False Negative (FN): residue classified as negative but is known to bind a ligand according to the reference.
\end{itemize}

With these four classes, true positive rate (TPR) (\autoref{eq:TPR}, false positive rate (FPR) (\autoref{eq:FPR}), precision (\autoref{eq:precision}) and recall (\autoref{eq:recall}) can be calculated and receiver operating characteristic (ROC) and precision-recall (PR) curves plotted. ROC and PR curves were obtained for each of the LIGYSIS protein chains. Using these curves, mean ROC and PR curves, representative of the variation across proteins for these metrics were obtained by taking the mean TPR and FPR (ROC curve) and mean precision and recall (PR curve) at each score interval. Mean area under the curve (AUC) for ROC and average precision (AP) were calculated by averaging the areas and precisions across curves. Baselines for these are 50\% and the proportion of true binding residues (10\%), respectively. ROC and AUC can't be calculated for VN-EGNN, PUResNet, DeepPocket, fpocket\textsubscript{PRANK} and fpocket as these methods do not provide residue ligandability scores.

\begin{equation}
\text{TPR (\%)} = 100 \times \frac{\text{TP}}{\text{TP} + \text{FN}}
\label{eq:TPR}
\end{equation}
\myequations{True positive rate}

\vspace{-12pt} % Adjust this value as needed

\begin{equation}
\text{FPR (\%)} = 100 \times \frac{\mathrm{FP}}{\mathrm{FP} + \mathrm{TN}}
\label{eq:FPR}
\end{equation}
\myequations{False positive rate}

\vspace{-12pt} % Adjust this value as needed

\begin{equation}
\text{Precision (\%)} = 100 \times \frac{\text{TP}}{\text{TP} + \text{FP}}
\label{eq:precision}
\end{equation}
\myequations{Precision}

\vspace{-12pt} % Adjust this value as needed

\begin{equation}
\text{Recall (\%)} = 100 \times \frac{\text{TP}}{\text{TP} + \text{FN}}
\label{eq:recall}
\end{equation}
\myequations{Recall}

\vspace{-12pt} % Adjust this value as needed

Pocket binary labels (0: no pocket residue; 1: pocket residue) can determine TP, FP, TN and FN for each residue in a protein chain $P_i$. VN-EGNN, IF-SitePred, PocketFinder\textsuperscript{+}, Ligsite\textsuperscript{+} and Surfnet\textsuperscript{+} do not report pocket residues. For these methods, residues within 6 \AA{} of the pocket centroid, cloud points and grid points (3$\times$), respectively, were labelled as pocket residues (1). All other residues in $P_i$ were labelled as non-binding (0). Across all residues in $P_i$, an F1 score was computed, which combines precision and recall into a unified metric, capturing the accuracy and completeness of predictions at the residue level (\autoref{eq:F1_score}). The Matthews Correlation Coefficient (MCC) \cite{MATTHEWS_1975_MCC} (\autoref{eq:MCC}) was also calculated. The median F1 score and MCC across the dataset proteins is reported for each method.

\begin{equation}
\text{F1} = \frac{2 \times \text{Precision} \times \text{Recall}}{\text{Precision} + \text{Recall}}
\label{eq:F1_score}
\end{equation}
\myequations{F1 score}

\begin{equation}
\text{MCC} = \frac{\text{TP} \times \text{TN} - \text{FP} \times \text{FN}}{\sqrt{(\text{TP} + \text{FP})(\text{TP} + \text{FN})(\text{TN} + \text{FP})(\text{TN} + \text{FN})}}
\label{eq:MCC}
\end{equation}
\myequations{Matthews correlation coefficient}

\vspace{-12pt} % Adjust this value as needed
\vspace{-12pt} % Adjust this value as needed

\subsubsection{Pocket-level predictions}
\label{subsub:pocket_level_metrics}

Ligand binding site prediction at the pocket level is a multi-instance prediction problem. There are no \textit{negatives} predicted at the pocket level of ligand binding site prediction, only \textit{positives}. A positive is a predicted pocket, which will be true (TP) or false (FP) depending on whether it is observed in the reference data. False negatives are those pockets observed in the reference data that are not predicted. They are the pockets the method fails to predict, and therefore, are not scored. A true negative would be a ``non-pocket'' that is \textit{not} predicted. This can't be quantified easily and even if it was, it would not be scored by the method, as it is not predicted. For this reason, in this context, neither TPR, nor FPR can be calculated. Consequently, ROC/AUC can't be utilised to assess ligand binding site prediction at the pocket level. False negatives are known, but not scored, and therefore PR/AUC is not an option either. What can be calculated is the recall given a certain criterion. In this case, because of the nature of the LIGYSIS dataset, where defined sites result from the clustering of multiple ligands, the distance between the predicted pocket centroid to the observed binding site (DCC) was chosen.

For each observed binding site in the LIGYSIS reference, the ``best'' prediction for each method was chosen. This is defined as the prediction with the minimum Euclidean distance to the observed pocket centroid or DCC. Once the observed-predicted pairs were obtained, only those with DCC $\leq$ 12 \AA{} were considered as correct predictions. A threshold of 12 \AA{} was chosen as 4 \AA{} is too strict a threshold when using DCC. A threshold of 4 \AA{} works well for the distance to closest ligand atom (DCA) but does not for DCC. The top-\textit{N} and \textit{N}+2 ranking predictions were considered to calculate success rate, or recall (\autoref{eq:success_rate}), and maximum recall was calculated by considering all predictions, regardless of their score or rank. \textit{N} represents the number of observed sites for a given protein.

\begin{equation}
\text{Success rate (\%)} = 100 \times \frac{\text{observed sites with predicted site DCC} \leq 12 \text{Å}}{\text{observed sites}}
\label{eq:success_rate}
\end{equation}
\myequations{Success rate}

\vspace{-12pt} % Adjust this value as needed

Additionally, instead of conventional ROC, ROC100 \cite{WEBBER_2003_ROC100, SCOTT_2007_ROC100} was used to measure the predictive performance of the methods. To do this, for each method, all predictions across dataset proteins were ranked based on pocket score and cumulative true positives were plotted against cumulative false positives until 100 false positives were reached. In a similar way, a precision curve can be calculated by taking the top-\textit{N}, in this case \textit{N} = 1000, predictions. This curve measures how precision changes as more predictions with lower scores are considered. This is indicative of how informative pocket scores are.

Precision and recall are key measures for evaluating the performance of ligand binding site prediction methods. However, these indicators are calculated and interpreted slightly differently depending on the context a prediction is analysed, i.e., pocket \textit{vs} residue level, as well as the metric employed, e.g., F1 score, MCC, ROC or PR curves. At the residue level, the prediction is a binary classification task, where each residue is classified as binding (1) or non-binding (0). Here, precision reflects the proportion of residues predicted as binding that are true, i.e., observed in the reference data. Recall measures the proportion of true binding residues that are correctly identified. For the calculation of F1 and MCC, a residue is labelled ``positive'' or ``negative'' depending on whether it is part of a predicted pocket. However, for ROC and PR curves, the positive and negative labels are derived based on a ligandability threshold, $t_{LS}$. Prediction at the pocket level represents a multi-instance prediction task. Precision indicates the proportion of predicted pockets that are observed in the reference data whilst recall represents the propostion of true binding pockets that are correctly predicted. It is important to  keep this in mind to correctly interpret precision and recall across different contexts.

To measure the similarity in shape and residue membership between predicted and observed pockets, relative residue overlap (RRO) and relative volume overlap (RVO) were employed. For an observed-predicted pocket pair, RRO represents the proportion of observed ligand-binding residues ($R_{o}$) that are covered by the predicted pocket residues ($R_{p}$) (\autoref{eq:RRO}). The POVME output was used for the calculation of RVO (\autoref{fig:protein_RVO_approach}). POVME defines the volume of a pocket as a series of equidistantly spaced spheres of unit volume. As predictions by the different methods were on the same coordinate reference, these pocket volume spheres were already aligned, and the volume overlap was calculated simply as the proportion of spheres in the observed pocket ($V_{o}$) that overlap with the predicted pocket spheres ($V_{p}$) (\autoref{eq:RVO}).

\begin{equation}
\text{RRO (\%)} = 100 \times \frac{|R_p \cap R_o|}{R_o}
\label{eq:RRO}
\end{equation}
\myequations{Relative residue overlap}

\vspace{-12pt} % Adjust this value as needed

\begin{equation}
\text{RVO (\%)} = 100 \times \frac{|V_p \cap V_o|}{V_o}
\label{eq:RVO}
\end{equation}
\myequations{Relative volume overlap}

\begin{figure}[p]
    \centering
    \includegraphics[width=0.9\textwidth]{figures/ch_LBS_COMP/PNG/FIG13_VOLUME_OVERLAP_APPROACH.png}
    \caption[Relative Volume Overlap (RVO) calculation]{\textbf{Relative Volume Overlap (RVO) calculation.} \textbf{(A)} Example of two very accurate predictions by PUResNet and P2Rank on PDB: \href{https://www.ebi.ac.uk/pdbe/entry/pdb/4px2}{4PX2} \cite{PDB_4PX2}. Pocket volumes were calculated with POVME 2.0 and represented by coloured surfaces. These volumes result from the addition of unit volume spheres on a grid. To obtain the RVO, the intersection of these spheres between predicted and observed site was divided by the number of observed pocket spheres. Both predictions cover the entirety of the observed pocket volume; \textbf{(B)} GrASP and VN-EGNN predictions of a site on PDB: \href{https://www.ebi.ac.uk/pdbe/entry/pdb/2ZOX}{2ZOX} \cite{NOGUCHI_2008_STRUCTURE}. Th volumes of these predicted sites overlap less with the observed site: RVO = 67\% for GrASP and RVO = 11\% for VN-EGNN.}
    \label{fig:protein_RVO_approach}
\end{figure}
%\FloatBarrier

\vspace{-12pt} % Adjust this value as needed
\vspace{-12pt} % Adjust this value as needed

\subsection{Statistics and reproducibility}

VN-EGNN was installed and run locally from \url{https://github.com/ml-jku/vnegnn}. Likewise, for IF-SitePred: \url{https://github.com/annacarbery/binding-sites}. GrASP was obtained from \url{https://github.com/tiwarylab/GrASP} and predictions generated using the Google Colab Notebook. PUResNet predictions were obtained through the PUResNet v2.0 web server: \url{https://nsclbio.jbnu.ac.kr/tools/jmol}. DeepPocket was installed and executed locally: \url{https://github.com/devalab/DeepPocket}. P2Rank v2.4.2 was used to run all predictions as well as PRANK re-scoring: \url{https://github.com/rdk/p2rank}. fpocket v4.0 was installed via Conda: \url{https://anaconda.org/conda-forge/fpocket}. For PocketFinder, Ligsite and Surfnet, the ConCavity v0.1 ``+'' re-implementations were employed: \url{https://compbio.cs.princeton.edu/concavity/}.

Other recent methods including RefinePocket \cite{LIU_2023_REFINEPOCKET}, EquiPocket \cite{ZHANG_2024_EQUIPOCKET}, GLPocket \cite{LI_2023_GLPOCKET}, SiteRadar \cite{EVTEEV_2023_SITERADAR}, NodeCoder \cite{ABDOLLAHI_2023_NODECODER}, RecurPocket \cite{LI_2022_RECURPOCKET}, PointSite \cite{YAN_2022_POINTSITE}, DeepSurf \cite{MYOLNAS_2021_DEEPSURF}, Kalasanty \cite{STEPNIEWSKA_2020_KALASANTY}, BiteNet \cite{KOZLOVSKII_2020_BITENET}, GRaSP \cite{SANTANA_2020_GRaSP} or DeepSite \cite{JIMENEZ_2017_DEEPSITE} were not included in this analysis due to technical reasons. Peer-reviewed, open-source methods with publicly accessible code, clear installation instructions, well defined dependencies, and accessible command line interfaces were prioritised for this benchmark. This set of thirteen methods, counting \textsubscript{RESC} and \textsubscript{SEG} modes of DeepPocket is representative of the state-of-the-art within the field.

ChimeraX v1.7.1 \cite{PETTERSEN_2021_CHIMERAX} was used for structural visualisation in all figures unless otherwise stated, in which case PyMOL v2.5.2 was employed \cite{SCHRODINGER_2015_PYMOL}.

\subsection{Data and code availability}

The main results tables and files necessary to replicate the analysis described in this paper can be found here: \url{https://doi.org/10.5281/zenodo.13121414} \cite{UTGES_2024_LBSCOMP_ZENODO}. Software developed to carry out this analysis is found in this GitHub repository: \url{https://github.com/bartongroup/LBS-comparison} \cite{UTGES_2024_LBSCOMP_REPO}.

\section{Results}

\subsection{The LIGYSIS dataset}

\begin{figure}[htb!]
    \centering
    \includegraphics[width=\textwidth]{figures/ch_LBS_COMP/PNG/FIG1_REDUNDANT_PLIS_SPLIT_1.png}
    \caption[Redundancy in protein-ligand interfaces (I)]{\textbf{Redundancy in protein-ligand interfaces (I).} For PDB: \href{https://www.ebi.ac.uk/pdbe/entry/pdb/1jqy}{1JQY}, the asymmetric unit comprises three copies of a homo-pentamer, whereas the biologically functional assembly is a single pentamer. An \href{https://www.ebi.ac.uk/pdbe-srv/pdbechem/chemicalCompound/show/A32}{A32} ligand molecule binds to each copy, except for one, of each of the three pentamers. This results in the same protein-ligand interface repeated 14 times, i.e., 14$\times$ redundancy. Dashed rectangles indicate the asymmetric and biological units.}
    \label{fig:redundant_plis_1}
\end{figure}

The human subset of the LIGYSIS dataset, LIGYSIS\textsubscript{HUMAN}, comprises protein-ligand complexes for 3448 human proteins. For each protein, biologically relevant protein-ligand interactions, in accordance with BioLiP \cite{YANG_2013_BIOLIP}, are considered across the PISA-defined \cite{KRISSINEL_2007_PISA} biological assemblies of the multiple entries deposited in the PDBe \cite{ARMSTRONG_2020_PDBE}. Ligands are clustered using their protein interaction fingerprint to identify ligand binding sites as described in \autoref{chap:LIGYSIS_WEB}. The full LIGYSIS dataset includes $\approx$30,000 proteins with known ligand-bound complexes. Here, the human subset of LIGYSIS is employed as a manageable set to run all prediction methods on and referred to as \textit{LIGYSIS} for brevity.

\begin{figure}[htb!]
    \centering
    \includegraphics[width=\textwidth]{figures/ch_LBS_COMP/PNG/FIG1_REDUNDANT_PLIS_SPLIT_2.png}
    \caption[Redundancy in protein-ligand interfaces (II)]{\textbf{Redundancy in protein-ligand interfaces (II).} For PDB: \href{https://www.ebi.ac.uk/pdbe/entry/pdb/1PPR}{1PPR} both the asymmetric and biological units are a homo-trimer. Different molecules of the same ligands are binding to the same interfaces across the three copies of the trimer, i.e., 3$\times$ redundancy. Dashed rectangle indicate the asymmetric and biological units.}
    \label{fig:redundant_plis_2}
\end{figure}

The LIGYSIS dataset differs from previous train and test sets for ligand binding sites by considering biological units, aggregating multiple structures of the same protein and removing redundant protein-ligand interfaces. The asymmetric unit is the smallest portion of a crystal structure that can reproduce the complete unit cell through a series of symmetry operations. The asymmetric unit often does not correspond to the biological assembly or unit and relying on it can lead to artificial crystal contacts or redundant protein-ligand interfaces. The biological unit is the biologically relevant and functional macromolecular assembly for a given structure and might be formed by one, multiple copies or a portion of the asymmetric unit \cite{XU_2019_ASSEMBLIES}. LIGYSIS consistently considers biological units, which is key in any analysis that delves into molecular interactions at residue or atomistic level. An example of this illustrated in \autoref{fig:redundant_plis_1} is PDB: \href{https://www.ebi.ac.uk/pdbe/entry/pdb/1jqy}{1JQY} \cite{PICKENS_2002_ANCHOR}, part of the HOLO4K dataset, where the asymmetric unit is formed by three copies of a homo-pentamer, whereas the biological unit comprises a single pentamer. In this structure, 14 molecules of BMSC-0010 (\href{https://www.rcsb.org/ligand/A32}{A32}) interact with 14 copies of \textit{Escherichia coli} heat-labile enterotoxin B chain (\href{https://www.uniprot.org/uniprotkb/P32890/entry}{P32890}). This protein-ligand interface is the same repeated 14 times.

Protein-ligand interface redundancy can also be an issue when the asymmetric unit equals the biological assembly (\autoref{fig:redundant_plis_2}). In PDB: \href{https://www.ebi.ac.uk/pdbe/entry/pdb/1PPR}{1PPR} \cite{HOFMANN_1996_CAROTENOID}, also in HOLO4K, molecules of chlorophyll A (\href{https://www.ebi.ac.uk/pdbe-srv/pdbechem/chemicalCompound/show/CLA}{CLA}), peridinin (\href{https://www.ebi.ac.uk/pdbe-srv/pdbechem/chemicalCompound/show/PID}{PID}) and digalactosyl diacyl glycerol (\href{https://www.ebi.ac.uk/pdbe-srv/pdbechem/chemicalCompound/show/DGD}{DGD}) bind to the three copies of a peridinin-chlorophyll a-binding protein 1, chloroplastic, PCP, (\href{https://www.uniprot.org/uniprotkb/P80484/entry}{P80484}) trimer, resulting in a redundancy of 3$\times$. To account for this, LIGYSIS considers unique non-redundant protein-ligand interfaces by retrieving the UniProt sequence numbers of the residues the ligands interact with, so 1/14 interfaces would be retrieved for PDB: \href{https://www.ebi.ac.uk/pdbe/entry/pdb/1JQY}{1JQY} and 12/36 for PDB: \href{https://www.ebi.ac.uk/pdbe/entry/pdb/1PPR}{1PPR}. Finally, unique ligand interactions are aggregated across different structures for the same protein and ligand sites defined.

\begin{figure}[htb!]
    \centering
    \includegraphics[width=0.9\textwidth]{figures/ch_LBS_COMP/PNG/FIG2_LIGYSIS_vs_PDBbind.png}
    \caption[Comparison of PDBbind and LIGYSIS]{\textbf{Comparison of PDBbind and LIGYSIS.} PDBbind is comprised by complexes between a protein and the most biologically relevant ligand in a structure. For PDB: \href{https://www.ebi.ac.uk/pdbe/entry/pdb/4GQQ}{4GQQ}, this is ethyl caffeate (\href{https://www.ebi.ac.uk/pdbe-srv/pdbechem/chemicalCompound/show/0XR}{0XR}). LIGYSIS considers all unique biologically relevant protein-ligand interactions across all the structures for a given protein. For human pancreatic alpha-amylase (\href{https://www.uniprot.org/uniprotkb/P04746/entry}{P04746}), which representative structure is \href{https://www.ebi.ac.uk/pdbe/entry/pdb/4GQQ}{4GQQ}, 13 ligand binding sites are defined from 195 ligands across 51 structures. LIGYSIS provides a better representation of the ligand-binding capabilities of a protein than a single protein-ligand complex and constitutes therefore a better benchmark for ligand binding site prediction tools.}
    \label{fig:PDBbind_VS_LIGYSIS}
\end{figure}

\autoref{fig:PDBbind_VS_LIGYSIS} shows the comparison between PDB: \href{https://www.ebi.ac.uk/pdbe/entry/pdb/4GQQ}{4GQQ} \cite{WILLIAMS_2012_AMYLASE}, present in the PDBbind dataset, and the LIGYSIS entry for human pancreatic alpha-amylase (\href{https://www.uniprot.org/uniprotkb/P04746/entry}{P04746}), which representative structure is also \href{https://www.ebi.ac.uk/pdbe/entry/pdb/4GQQ}{4GQQ}. The entry in PDBbind represents a single protein-ligand complex, whereas LIGYSIS makes use of 51 structures, 195 ligands to define 13 different ligand binding sites. LIGYSIS aggregates all unique biologically relevant protein-ligand interactions for a protein in a non-redundant manner, thus representing the most complete and integrative protein-ligand binding dataset up to date. For this reason, LIGYSIS is proposed as a new benchmark dataset for the prediction of ligand binding sites and used in this Chapter to evaluate a set of thirteen ligand binding site prediction and cavity identification tools.

\begin{landscape}
\begin{longtable}{|M{24mm}|M{20mm}|M{27mm}|M{20mm}|M{25mm}|M{25mm}|M{58mm}|}
\hline
\textbf{Dataset}    & \textbf{Type}  & \textbf{\# Structures} & \textbf{\# Sites} & \textbf{\# Ligands} & \textbf{Overlap} (\%) & \textbf{Methods}                                      \\ \hline
\endfirsthead
%
\endhead
%
LIGYSIS    & NEW   & 3448         & 8244    & \textbf{\textcolor{CBBlue}{65,116}}     & --          & --                                            \\ \hline
LIGYSIS\textsubscript{NI}  & NEW   & 2275         & 4572    & 38,595     & --          & --                                            \\ \hline
sc-PDB\textsubscript{FULL} & TRAIN & \textbf{\textcolor{CBBlue}{17,594}}        & \textbf{\textcolor{CBBlue}{17,594}}   & 17,594     & \textbf{\textcolor{CBOrange}{801 (9.7)}}        & VN-EGNN, GrASP, PUResNet, DeepPocket         \\ \hline
bMOAD\textsubscript{SUB}   & TRAIN & 5899         & 11,184   & 11,184     & 606 (7.6)        & IF-SitePred                                  \\ \hline
CHEN11     & TRAIN & \textbf{\textcolor{CBOrange}{244}}           & \textbf{\textcolor{CBOrange}{479}}      & \textbf{\textcolor{CBOrange}{479}}        & \textbf{\textcolor{CBBlue}{40 (0.5)}}        & PRANK, P2Rank                                       \\ \hline
PDBbind\textsubscript{REF} & TEST  & 5316         & 5316    & 5316      & 310 (3.8)        & VN-EGNN                                      \\ \hline
SC6K       & TEST  & 6147         & 6147    & 6147      & 259 (3.1)        & DeepPocket                                   \\ \hline
HOLO4K     & TEST  & 4009         & 10,175   & 10,175     & 207 (2.5)        & \textit{ALL*}                                         \\ \hline
COACH420   & TEST  & 413           & 624      & 624        & 41 (0.5)        & VN-EGNN, GrASP, DeepPocket, PUResNet, P2Rank \\ \hline
JOINED   & TEST  & 557           & 752      & 752        & 110 (1.3)        & PRANK \\ \hline
\caption[Datasets summary statistics]{\textbf{Datasets summary statistics.} \# Structures, \# Sites and \# Ligands represent the number of PDB structures, ligand sites and total number of ligands for each dataset. For LIGYSIS and LIGYSIS\textsubscript{NI}, 3448 and 2775, are the number of structural segments, each represented by a single chain. For each segment, biologically relevant ligands across structures were considered: \textit{N} = 23,321 (LIGYSIS) and \textit{N} = 19,012 (LIGYSIS\textsubscript{NI}). The number of ligands is not equal to the number of sites for LIGYSIS, as ligands from multiple structures of the same protein are aggregated into unique sites. Overlap is the number of LIGYSIS binding sites represented by at least one protein-ligand complex for a given dataset. Percentage relative to LIGYSIS also reported. Methods represents the ligand site predictors that use these datasets for training or test. For \# Structures, \# Sites and \# Ligands, highest values are coloured in bold blue font and lowest in orange. This is the other way around for Overlap.}
\label{tab:datasets_comp}\\
\end{longtable}
\end{landscape}

\subsection{Comparison of datasets}

To assess the scope and limitations of the predictive methods surveyed in this Chapter, their training and test sets were compared with LIGYSIS by number of sites per protein, number of interacting protein chains per ligand site, ligand size, ligand site size, and ligand composition. sc-PDB\textsubscript{FULL} represents the full sc-PDB dataset used for training by DeepPocket, bMOAD\textsubscript{SUB} the subset of binding MOAD used for training by IF-SitePred and PDBbind\textsubscript{REF} the reference subset of PDBbind which VN-EGNN uses for testing. Only original versions of each dataset are considered in this analysis e.g., HOLO4K, but not HOLO4K\textsubscript{Mlig}, nor HOLO4K\textsubscript{Mlig+} HAP, or HAP-small. The same goes for Mlig and Mlig+ versions of COACH420, sc-PDB\textsubscript{SUB} and sc-PDB\textsubscript{RICH}. \textit{ALL*} represents all methods in this work except for fpocket, PocketFinder\textsuperscript{+}, Ligsite\textsuperscript{+} and Surfnet\textsuperscript{+}. \autoref{tab:datasets_comp} summarises the size of the datasets, which methods employ them and their overlap with the LIGYSIS set. LIGYSIS differs from all other datasets since biologically relevant ions are considered, comprising $\approx$ 40\% of the ligand sites. For additional reference, LIGYSIS\textsubscript{NI}, a subset of LIGYSIS without ions, is also included in this analysis. \autorefpanel{fig:dataset_comp_1}{ A} shows the number of binding sites per entry across datasets. sc-PDB\textsubscript{FULL}, PDBbind\textsubscript{REF} and SC6K only consider the most relevant ligand for each entry. COACH420 and JOINED mostly present single-ligand entries ($\approx$70\%). bMOAD\textsubscript{SUB} (46\%) and CHEN11 (58\%) present more similar distributions to LIGYSIS, where 54\% of the protein chains present more than one binding site. This percentage decreases for LIGYSIS\textsubscript{NI} (38\%) as ion sites are removed. HOLO4K presents the highest proportion (62\%) of multi-ligand entries. Both HOLO4K and COACH420 are based on asymmetric units, and not biological assemblies. For HOLO4K, 1811 (40\%) of structures present different numbers of chains between the asymmetric and biological units. This is even more frequent in COACH420: 234 (56\%). Moreover, multimeric complexes might present the same protein-ligand interface repeated across the copies of the complex (\autoref{fig:redundant_plis_1}, \autoref{fig:redundant_plis_2}). Considering predictions of these interfaces as independent can lead to overestimating the performance of a predictor. Regarding the number of chains interacting with a given ligand (\autorefpanel{fig:dataset_comp_1}{ B}), CHEN11 and COACH420 present the smaller fraction of multimeric protein-ligand interactions: 3\% and 6\%, respectively, whereas SC6K presents the highest (44\%). The rest of the methods range between 20-30\%. There are no striking differences regarding the size of the interacting protein chains, represented by the number of residues (\autorefpanel{fig:dataset_comp_1}{ C}).

\begin{figure}[ht!]
    \centering
    \includegraphics[width=\textwidth]{figures/ch_LBS_COMP/PNG/FIG3_DATASET_COMPARISON_SPLIT_1.png}
    \caption[Comparison of datasets (I)]{\textbf{Comparison of datasets (I).} Panels A, B, C, E and F plot the frequencies (\%) of binned intervals of a discrete variable, coloured using the \textit{cividis} palette. Interval ranges were selected to facilitate data interpretation. \textbf{(A)} Number of ligand binding sites per dataset entry; \textbf{(B)} Number of ligand-interacting protein chains. This represents whether the ligand interacts with a single protein chain or more; \textbf{(C)} Length of ligand-interacting protein chains (number of amino acids); \textbf{(D)} Ligand molecule type frequency as described in the CCD; \textbf{(E)} Number of ligand atoms; \textbf{(F)} Binding site size, i.e., number of ligand-interacting protein residues. Dashed lines drawn at frequency = 50\%. A subset of LIGYSIS with no ions (NI), LIGYSIS\textsubscript{NI}, is included in this analysis, as most training and test datasets do not consider ions.}
    \label{fig:dataset_comp_1}
\end{figure}

\autorefpanel{fig:dataset_comp_1}{ D} represents the ligand type composition of the datasets. Non-polymer ligands dominate all datasets ($>$66\%), and the proportion of peptides and nucleic acids differ across datasets, with JOINED and LIGYSIS presenting fewer ligands of these types (0.9\% and 1.6\%). sc-PDB\textsubscript{FULL} and SC6K are depleted in saccharides ($<$1\%). \autorefpanel{fig:dataset_comp_1}{ E} depicts the difference in the number of atoms of the ligands in each dataset. LIGYSIS is, as expected, different due to its ion ligand content, however, there is no difference between LIGYSIS\textsubscript{NI} and the other datasets. \autorefpanel{fig:dataset_comp_1}{ F} conveys the difference in the number of ligand-interacting residues. LIGYSIS has the largest proportion of small sites, 1-10 residues, (56\%). This is directly related to the prominent ion component, and the frequency decreases when ions are removed (LIGYSIS\textsubscript{NI}: 36\%). CHEN11, COACH420, JOINED and HOLO4K are more similar to LIGYSIS\textsubscript{NI}, whereas sc-PDB\textsubscript{FULL}, PDBbind\textsubscript{REF} and SC6K are clearly different and present almost exclusively large sites, larger than 20 amino acids, ($>$90\%).

\begin{figure}[ht!]
    \centering
    \includegraphics[width=\textwidth]{figures/ch_LBS_COMP/PNG/FIG3_DATASET_COMPARISON_SPLIT_2.png}
    \caption[Comparison of datasets (II)]{\textbf{Comparison of datasets (II).} Five most frequent ligands per dataset. Error bars represent 95\% confidence interval of the proportion \cite{WILSON_197_PROP_CI}. Ligands of similar type are coloured in shades of the same colour: greens for ions, reds for co-factors, blues for energy-carrier molecules, yellows for sugars, grey for peptides and white for other non-polymeric ligands. Above the bars, Shannon's Entropy and the proportion of all ligands in each set covered by these top-5 can be found. Both are measures of ligand diversity within each dataset. LIGYSIS\textsubscript{NI}, a subset of LIGYSIS without ions is included in this analysis, as most training and test datasets do not consider ions. \href{https://www.ebi.ac.uk/pdbe-srv/pdbechem/chemicalCompound/show/017}{017}: Darunavir; \href{https://www.ebi.ac.uk/pdbe-srv/pdbechem/chemicalCompound/show/ADE}{ADE}: adenine; \href{https://www.ebi.ac.uk/pdbe-srv/pdbechem/chemicalCompound/show/BGC}{BGC}: glucose; \href{https://www.ebi.ac.uk/pdbe-srv/pdbechem/chemicalCompound/show/CLR}{CLR}: cholesterol; \href{https://www.ebi.ac.uk/pdbe-srv/pdbechem/chemicalCompound/show/GAI}{GAI}: guanidine; \href{https://www.ebi.ac.uk/pdbe-srv/pdbechem/chemicalCompound/show/GSH}{GSH}: glutathione; \href{https://www.ebi.ac.uk/pdbe-srv/pdbechem/chemicalCompound/show/MAN}{MAN}: mannose; \href{https://www.ebi.ac.uk/pdbe-srv/pdbechem/chemicalCompound/show/FUC}{FUC}: fucose; \href{https://www.ebi.ac.uk/pdbe-srv/pdbechem/chemicalCompound/show/NAP}{NAP}: nicotinamide-adenine-dinucleotide phosphate; \href{https://www.ebi.ac.uk/pdbe-srv/pdbechem/chemicalCompound/show/SAH}{SAH}: S-Adenosyl-L-homocysteine; \href{https://www.ebi.ac.uk/pdbe-srv/pdbechem/chemicalCompound/show/FMN}{FMN}: Flavin mononucleotide; \href{https://www.ebi.ac.uk/pdbe-srv/pdbechem/chemicalCompound/show/GAL}{GAL}: galactose; \textit{N}-mer: protein peptides of \textit{N} amino acids; \href{https://www.ebi.ac.uk/pdbe-srv/pdbechem/chemicalCompound/show/PLP}{PLP}: Vitamin B6 phosphate; \href{https://www.ebi.ac.uk/pdbe-srv/pdbechem/chemicalCompound/show/XYP}{XYP}: xylose.}
    \label{fig:dataset_comp_2}
\end{figure}

\autoref{fig:dataset_comp_2} explores the ligand diversity on each dataset by showing the top-5 most frequent ligands per dataset, the percentage of the total number of ligands they represent, as well as Shannon's Entropy, \textit{H'}. Shannon's Entropy is a measure of diversity. Larger numbers indicate a more evenly spread distribution of a larger number of different molecules, whereas small numbers indicate higher frequency of a few ligands. While four out of the top-5 ligands of LIGYSIS are ions -- Zn\textsuperscript{+2} (\href{https://www.ebi.ac.uk/pdbe-srv/pdbechem/chemicalCompound/show/ZN}{ZN}), Ca\textsuperscript{+2} (\href{https://www.ebi.ac.uk/pdbe-srv/pdbechem/chemicalCompound/show/CA}{CA}), Mg\textsuperscript{+2} (\href{https://www.ebi.ac.uk/pdbe-srv/pdbechem/chemicalCompound/show/MG}{MG}), Mn\textsuperscript{+2} (\href{https://www.ebi.ac.uk/pdbe-srv/pdbechem/chemicalCompound/show/MN}{MN}) -- and represent 19.2\% of all ligands, its diverse composition is comparable to that of PDBbind\textsubscript{REF}. Removing ions, LIGYSIS\textsubscript{NI} becomes the most diverse dataset with \textit{H'} = 8.8 and its top-5 ligands only covering 5.3\% of all ligands in the set. SC6K is the least diverse with its top-5 most frequent ligands covering 33\% of all ligands. All datasets, except for LIGYSIS, LIGYSIS\textsubscript{NI}, and PDBbind\textsubscript{REF}, are dominated by co-factor ligands, such as flavin-adenine dinucleotide (\href{https://www.ebi.ac.uk/pdbe-srv/pdbechem/chemicalCompound/show/FAD}{FAD}), nicotinamide-adenine dinucleotide (\href{https://www.ebi.ac.uk/pdbe-srv/pdbechem/chemicalCompound/show/NAD}{NAD}), and haem (\href{https://www.ebi.ac.uk/pdbe-srv/pdbechem/chemicalCompound/show/HEM}{HEM}) or energy carrier molecules such as adenine tri-, di- and monophosphate (\href{https://www.ebi.ac.uk/pdbe-srv/pdbechem/chemicalCompound/show/ATP}{ATP}, \href{https://www.ebi.ac.uk/pdbe-srv/pdbechem/chemicalCompound/show/ADP}{ADP}, \href{https://www.ebi.ac.uk/pdbe-srv/pdbechem/chemicalCompound/show/AMP}{AMP}). Short peptides ($<$10 aas) are the most common ligands in PDBbind\textsubscript{REF} (5\%), and energy carriers represent $<$2\% of the top-5 ligands. Cholesterol, mannose and fucose are some of the most common ligands in LIGYSIS\textsubscript{NI}. For the pocket characterisation and performance evaluation analyses, LIGYSIS (including ions) was utilised.

\subsection{Binding pocket characterisation}

After removing backbone-only chains and those with missing residue mapping to UniProt, the final LIGYSIS set which was employed for the benchmark of the methods comprises 2,775 representative chains. Not all methods predict pockets on all the chains. VN-EGNN, GrASP, fpocket, PocketFinder\textsuperscript{+}, Ligsite\textsuperscript{+}, and Surfnet\textsuperscript{+} predict in $>$99\% of the chains, P2Rank\textsubscript{CONS} on 93\%, followed by P2Rank on 86\%, PUResNet and DeepPocket\textsubscript{SEG} (85\%), and finally IF-SitePred only predicts pockets on 75\% of the chains. PUResNet, DeepPocket, P2Rank\textsubscript{CONS} and P2Rank often don't predict on smaller proteins ($<$100 amino acids) as well as non-globular or elongated proteins, representing 60-80\% of proteins with no predicted pockets. However, for IF-SitePred larger globular proteins represent $\approx$50\% of all proteins where this method fails to predict a pocket (\autoref{fig:missed_preds_prot_class}). Predicted residue ligandability scores for P2Rank\textsubscript{CONS}, P2Rank and IF-SitePred (which we derived in this work), were examined for the proteins with no predicted pockets. \autoref{fig:ifsp_missed_preds} illustrates 8 examples of proteins where residues with IF-SitePred (\autoref{eq:IFSP_score}) high ligandability scores cluster in space into clear binding sites that are not reported as predictions by this method. This suggests that IF-SitePred is too strict in selecting only those residues predicted as ligand-binding by all 40 models. The cloud point selection clustering approach or threshold in this method may also play a role in this.

\begin{figure}[ht!]
    \centering
    \includegraphics[width=0.60\textwidth]{figures/ch_LBS_COMP/PNG/SUPP_FIG1_MISSED_PREDS_PROT_CLASS.png}
    \caption[Where methods do not predict any sites]{\textbf{Where methods do not predict any sites.} IF-SitePred does not predict any ligand binding sites on 700 of the 2775 protein chains in the LIGYSIS set (25\%), PUResNet on 415 (15\%), DeepPocket\textsubscript{SEG} (426; 15\%), P2Rank\textsubscript{CONS} (196; 7\%) and P2Rank (373; 13\%). All methods struggle to predict on elongated proteins, regardless of their size, as well as on tiny globular proteins. Globular proteins comprise the most common group amongst the proteins with no predictions for IF-SitePred (53\%). Dashed line indicates frequency of 50\%.}
    \label{fig:missed_preds_prot_class}
\end{figure}

\begin{figure}[ht!]
    \centering
    \includegraphics[width=\textwidth]{figures/ch_LBS_COMP/PNG/FIG4_IFSP_MISSED_PREDICTIONS.png}
    \caption[IF-SitePred ``missed'' predictions]{\textbf{IF-SitePred ``missed'' predictions.} Eight examples of human protein chains where IF-SitePred does not report any predicted ligand binding sites. Predictions are made on ligand-stripped chains. Ligand molecules, in orange, are superposed to illustrate how the ligandability scores recapitulate the observed binding site. These are protein representative chains and ligand molecules might not be observed in the same entry; \textbf{(A)} GDP-fucose protein O-fucosyltransferase 2, \href{https://www.uniprot.org/uniprotkb/Q9Y2G5/entry}{Q9Y2G5}, with \href{https://www.ebi.ac.uk/pdbe-srv/pdbechem/chemicalCompound/show/GFB}{GFB} superimposed (PDB: \href{https://www.ebi.ac.uk/pdbe/entry/pdb/4ap6}{4AP6}) \cite{CHEN_2012_POFUT2}; \textbf{(B)} tRNA (cytosine(72)-C(5))-methyltransferase NSUN6, \href{https://www.uniprot.org/uniprotkb/Q8TEA1/entry}{Q8TEA1}, (PDB: \href{https://www.ebi.ac.uk/pdbe/entry/pdb/5WWT}{5WWT}) \cite{LIU_2017_NSUN6} with superposed \href{https://www.ebi.ac.uk/pdbe-srv/pdbechem/chemicalCompound/show/SFG}{SFG} (PDB: \href{https://www.ebi.ac.uk/pdbe/entry/pdb/5WWR}{5WWR}) \cite{LIU_2017_NSUN6}; \textbf{(C)} Tubulin beta-2B chain, \href{https://www.uniprot.org/uniprotkb/Q9BVA1/entry}{Q9BVA1}, with \href{https://www.ebi.ac.uk/pdbe-srv/pdbechem/chemicalCompound/show/G2P}{G2P} (PDB: \href{https://www.ebi.ac.uk/pdbe/entry/pdb/7ZCW}{7ZCW}) \cite{RAMIREZ_2023_VASH2}; \textbf{(D)} Cyclic GMP-AMP phosphodiesterase SMPDL3A, \href{https://www.uniprot.org/uniprotkb/Q92484/entry}{Q92484}, with \href{https://www.ebi.ac.uk/pdbe-srv/pdbechem/chemicalCompound/show/C5P}{C5P} (PDB: \href{https://www.ebi.ac.uk/pdbe/entry/pdb/5EBE}{5EBE}) \cite{LIM_2016_SPHINGOMYELIN}; \textbf{(E)} tRNA (adenine(58)-N(1))-methyltransferase catalytic subunit, \href{https://www.uniprot.org/uniprotkb/Q96FX7/entry}{Q96FX7}, with \href{https://www.ebi.ac.uk/pdbe-srv/pdbechem/chemicalCompound/show/SAH}{SAH} (PDB: \href{https://www.ebi.ac.uk/pdbe/entry/pdb/5CCB}{5CCB}) \cite{FINER_2015_tRNA}; \textbf{(F)} Chronophin, \href{https://www.uniprot.org/uniprotkb/Q96GD0/entry}{Q96GD0}, (PDB: \href{https://www.ebi.ac.uk/pdbe/entry/pdb/5gyn}{5GYN}) \cite{PDB_5GYN} with \href{https://www.ebi.ac.uk/pdbe-srv/pdbechem/chemicalCompound/show/PLP}{PLP} (PDB: \href{https://www.ebi.ac.uk/pdbe/entry/pdb/2FCT}{2FCT}) \cite{BLASIAK_2006_SYRB2}; \textbf{(G)} Mitochondrial Methylmalonic aciduria type A protein, \href{https://www.uniprot.org/uniprotkb/Q8IVH4/entry}{Q8IVH4}, with \href{https://www.ebi.ac.uk/pdbe-srv/pdbechem/chemicalCompound/show/GDP}{GDP} (PDB: \href{https://www.ebi.ac.uk/pdbe/entry/pdb/8GJU}{8GJU}) \cite{MASCARENHAS_2023_GPROTEIN}; \textbf{(H)} Renalase, \href{https://www.uniprot.org/uniprotkb/Q5VYX0/entry}{Q5VYX0}, (PDB: \href{https://www.ebi.ac.uk/pdbe/entry/pdb/3QJ4}{3QJ4}) with \href{https://www.ebi.ac.uk/pdbe-srv/pdbechem/chemicalCompound/show/FAD}{FAD} \cite{MILANI_2011_NADP}. Residues are coloured based on the ligandability score calculated by averaging the probabilities predicted by each of the 40 IF-SitePred prediction models. This is a score ranging 0-1 which is indicative of the likelihood of a given residue binding a ligand. Clear pockets can be observed formed by residues with high ligandability scores (darker blue colour), which agree with the sites where ligands bind.}
    \label{fig:ifsp_missed_preds}
\end{figure}

\autoref{tab:pocket_features_stats} summarises the ligand site characterisation analysis. fpocket predicts the most sites out of all the methods, with 57,859, followed by IF-SitePred (44,948), DeepPo-cket\textsubscript{SEG} (21,718), VN-EGNN (13,582), P2Rank (12,412), P2Rank\textsubscript{CONS} (10,180), Surfnet\textsuperscript{+} (9043), PocketFinder\textsuperscript{+} (8913), Ligsite\textsuperscript{+} (6903), GrASP (4694) and PUResNet, which predicts fewest sites (2621). LIGYSIS defines 6882 binding sites from experimental data. Relative to LIGYSIS, the prediction methods have ratios of predicted/defined sites ranging from 8.4 (fpocket) to 0.4 (PUResNet) with P2Rank\textsubscript{CONS} in the middle, predicting 1.5 pockets per observed reference site. IF-SitePred, DeepPocket\textsubscript{SEG} as well as P2Rank and fpocket predict more pockets on larger protein chains, whereas the rest of methods do not (\autoref{fig:sites_vs_prot_size}). This effect is most clear with fpocket, which predicts 350 pockets for chain A of PDB: \href{https://www.ebi.ac.uk/pdbe/entry/pdb/7SUD}{7SUD} \cite{LIU_2022_DNAPK}, a structure of the DNA-dependent protein kinase catalytic subunit, DNPK1, (\href{https://www.uniprot.org/uniprotkb/P78527/entry}{P78527}) with 3736 amino acid residues. In contrast, VN-EGNN, which initially places \textit{K} = 8 virtual nodes, results in a maximum of 8 predicted pockets, regardless of protein chain size, and PUResNet predicts a single pocket in 90\% of the protein chains. 

\begin{figure}[ht!]
    \centering
    \includegraphics[width=\textwidth]{figures/ch_LBS_COMP/PNG/SUPP_FIG2_POCKETS_VS_PROTEIN_SIZE.png}
    \caption[Number of pockets \textit{vs} protein size]{\textbf{Number of pockets \textit{vs} protein size.} Number of defined (LIGYSIS) and predicted sites against protein chain size, i.e., number of amino acid residues. Number of residues has been discretised into intervals of 50 until 650, and larger intervals until the maximum, $\approx$3800. Error bars represent one standard deviation (SD).}
    \label{fig:sites_vs_prot_size}
\end{figure}

\begin{figure}[ht!]
    \centering
    \includegraphics[width=\textwidth]{figures/ch_LBS_COMP/PNG/FIG5_BINDING_POCKET_FEATURES_SPLIT_1.png}
    \caption[Binding pocket characterisation (I)]{\textbf{Binding pocket characterisation (I).} Violin plots show the main distributions and swarm plots are used to show outliers. Data points farther than four standard deviations (SD) from the mean are considered outliers. The limit of the Y axis is the maximum non-outlier value plus a buffer value. This way, only the most extreme outliers are hidden, which maximises visual interpretation of the data whilst minimising the number of data points not shown. Within the violin plots are box plots representing the underlying distribution. Line represents the median, box contains the interquartile range (IQR), and whiskers extend to 1.5 $\times$ IQR. \textbf{(A)} Number of pockets per protein; \textbf{(B)} Pocket radius of gyration, \textit{R\textsubscript{g}}, (\AA{}).}
    \label{fig:pocket_features_1}
\end{figure}

\autoref{fig:pocket_features_1} and \autoref{fig:pocket_features_2} represent how the eleven sets of unique ligand site predictors compare to each other as well to LIGYSIS, which \textit{defines} ligand sites from experimentally determined biologically relevant protein-ligand complexes. There are eleven unique sets of predictions since DeepPocket\textsubscript{RESC} and fpocket\textsubscript{PRANK} do not predict, their own pockets, but re-score and re-rank original fpocket predictions. DeepPocket\textsubscript{SEG} predictions are different as new pocket shapes are extracted by the CNN segmentation module. \autorefpanel{fig:pocket_features_2}{ A} shows how PUResNet, VN-EGNN and GrASP differ from the other methods with a maximum of 4, 7 and 12 predicted pockets, respectively. PocketFinder\textsuperscript{+}, Ligsite\textsuperscript{+} and Surfnet\textsuperscript{+} present narrow distributions like LIGYSIS and with medians of 1-3 pockets per protein. P2Rank\textsubscript{CONS} and P2Rank also present a median of 3 pockets per protein but display wider distributions as they can predict up to 60 and 80 pockets per protein, respectively. Overall, P2Rank\textsubscript{CONS} predicts fewer pockets than P2Rank. DeepPocket\textsubscript{SEG}, fpocket and IF-SitePred follow, with a median of 6, 17 and 20 pockets. The difference in number of pockets between DeepPocket\textsubscript{SEG} and DeepPocket\textsubscript{RESC} or fpocket is because 60\% of fpocket candidates are not extracted by the CNN segmentation module implemented in DeepPocket.

\autorefpanel{fig:pocket_features_1}{ B} shows the distribution of pocket radius of gyration, \textit{R\textsubscript{g}}. VN-EGNN and IF-SitePred differ from the rest of methods with narrow distributions and medians around 6 \AA{}. These two methods do not report pocket residues. Instead, they were obtained using a distance threshold of 6 \AA{} from the centroid, for VN-EGNN, and cloud points, for IF-SitePred. This is reflected by examining the percentage of pockets with \textit{R\textsubscript{g}} $>$ 10 \AA{} which is 0\% and 0.1\% for VN-EGNN and IF-SitePred. This is a striking difference compared to the LIGYSIS reference and other methods: 1.8\% (fpocket), 4.8\% (P2Rank), 5.7\% (DeepPocket\textsubscript{SEG}), 6.4\% (GrASP), 6.5\% (P2Rank\textsubscript{CONS}), 11.6\% (PUResNet), 12.6\% (LIGYSIS), 16\% (Surfnet\textsuperscript{+}), 21.4\% (PocketFinder\textsuperscript{+}) and 33.5\% (Ligsite\textsuperscript{+}). The latter three predict the sites with largest median \textit{R\textsubscript{g}} $\approx$ 9 \AA{}. VN-EGNN, GrASP, PUResNet and DeepPocket\textsubscript{SEG} predict sites with \textit{R\textsubscript{g}} = 0 \AA{}. This is rather infrequent (7.8\% GrASP) and $<$3\% for the other three. These examples correspond to singletons, i.e., pockets formed by only one amino acid.

\begin{figure}[ht!]
    \centering
    \includegraphics[width=\textwidth]{figures/ch_LBS_COMP/PNG/FIG5_BINDING_POCKET_FEATURES_SPLIT_2.png}
    \caption[Binding pocket characterisation (II)]{\textbf{Binding pocket characterisation (II).} Violin plots show the main distributions and swarm plots are used to show outliers. Data points farther than four standard deviations (SD) from the mean are considered outliers. The limit of the Y axis is the maximum non-outlier value plus a buffer value. This way, only the most extreme outliers are hidden, which maximises visual interpretation of the data whilst minimising the number of data points not shown. Within the violin plots are box plots representing the underlying distribution. Line represents the median, box contains the interquartile range (IQR), and whiskers extend to 1.5 $\times$ IQR. \textbf{(A)} Minimum inter-pocket centroid distance (MCD) (\AA{}). This is a measure of how close predicted pockets are to each other within a protein; \textbf{(B)} Maximum inter-pocket residue overlap (MRO). Residue overlap was calculated as Jaccard Index. This is a measure of how much the pockets overlap in terms of binding residues.}
    \label{fig:pocket_features_2}
\end{figure}

\autorefpanel{fig:pocket_features_2}{ A} illustrates how close predicted sites are to each other within a protein chain. Pairwise distances between the centroids of all ligand site pairs for a protein are calculated, and for each site, the minimum distance is taken. Sites predicted by VN-EGNN, IF-SitePred and DeepPocketSEG are very close to each other, with median distances ($\tilde{\textit{d}}$) of 1.1, 3.4 and 4.6 \AA{}, respectively. fpocket follows with $\tilde{\textit{d}}$ = 9.7 \AA{}. The rest of the methods and LIGYSIS (reference) present median distances ranging between 13-18 \AA{} (LIGYSIS, P2Rank\textsubscript{CONS}, P2Rank, PocketFinder\textsuperscript{+}, Ligsite\textsuperscript{+}, Surfnet\textsuperscript{+}), and finally GrASP ($\tilde{\textit{d}}$ = 21.7 \AA{}) and PUResNet ($\tilde{\textit{d}}$ = 27 \AA{}). Both versions of P2Rank present the most similar distribution to what is observed on LIGYSIS.

\autorefpanel{fig:pocket_features_2}{ B} depicts the overlap existing between residues that form the predicted pockets within a protein. All pairwise overlaps, i.e., Jaccard Index, are calculated between pockets in a chain, and for each pocket, the maximum is taken. This is a measure of how much predicted pockets overlap with each other. This is directly related to how close pockets are, and so VN-EGNN, IF-SitePred and DeepPocket\textsubscript{SEG} present very high overlaps $\tilde{\textit{o}}$ = 0.85, $\tilde{\textit{o}}$ = 0.55 and $\tilde{\textit{o}}$ = 0.4, respectively. fpocket follows with $\tilde{\textit{o}}$ = 0.15, Ligsite\textsuperscript{+} ($\tilde{\textit{o}}$ = 0.09), Surfnet\textsuperscript{+} ($\tilde{\textit{o}}$ = 0.07), P2Rank\textsubscript{CONS}, P2Rank and PocketFinder\textsuperscript{+} ($\tilde{\textit{o}}$ = 0.05), and finally LIGYSIS, GrASP and PUResNet with $\tilde{\textit{o}}$ = 0. GrASP is the only method of the thirteen presented here that clusters atoms directly, and as a result, overlap between pockets is minimised. Other methods cluster cloud points (IF-SitePred), SAS points (P2Ranks), voxels (PUResNet, DeepPocket), alpha spheres (fpocket), or grid points (PocketFinder\textsuperscript{+}, Ligsite\textsuperscript{+}, Surfnet\textsuperscript{+}) but not residues, resulting consequently in higher overlapping.

Proximity in space between predicted sites as well as residue overlap are indicators of redundant ligand binding site prediction, i.e., duplicate predictions of a unique observed ligand site. This is the case for VN-EGNN, IF-SitePred and DeepPocket\textsubscript{SEG}. This phenomenon could negatively impact the precision and recall of these methods. Accordingly, correcting for redundancy should have a significant impact on the performance of these methods. In contrast, GrASP and PUResNet which predict a small can of pockets show low proximity and overlap of predicted sites and so redundancy is not an issue.

% Please add the following required packages to your document preamble:
% \usepackage{lscape}
% \usepackage{longtable}
% Note: It may be necessary to compile the document several times to get a multi-page table to line up properly
\begin{landscape}
\begin{longtable}[c]{|M{28mm}|M{29mm}|M{29mm}|M{41mm}|M{26mm}|M{26mm}|M{20mm}|}
\hline
\textbf{Method}        & \textbf{Coverage} (\%)    & \textbf{\# Total Pockets}  & \textbf{\# Pockets per protein} & \textbf{\textit{R\textsubscript{g}}} (\AA{}) & \textbf{MCD} (\AA{}) & \textbf{MRO}  \\ \hline
\endfirsthead
%
%\multicolumn{7}{c}%
%{{\bfseries Table \thetable\ continued from previous page}} \\
%\hline
%\textbf{Method}        & \textbf{\% Coverage}    & \textbf{Total Pockets}  & \textbf{Pockets per protein} & \textbf{$R_{g}$ (\AA{})} & \textbf{MCD (\AA{})} & \textbf{MRO}  \\ \hline
\endhead
%
LIGYSIS       & 2775          & 6882          & 1, 1, 27            & 5.9                              & 14.1                         & 0    \\ \hline
VN-EGNN       & 2764 (99.6\%)   & 13,582 ($\times$2.0) & 1, 5, 7             & 5.9                              & \textbf{1.1}                          & \textbf{0.85} \\ \hline
IF-SitePred   & \textbf{2075 (74.8\%}) & 44,948 ($\times$6.5) & 1, \textbf{20}, 129          & 5.9                              & 3.4                          & 0.55 \\ \hline
GrASP         & 2771 (99.9\%) & 4694 ($\times$0.7)  & 1, 1, 12            & 7.9                              & 21.4                         & 0    \\ \hline
PUResNet      & 2360 (85.1\%) & 2621 ($\times$0.4)  & 1, 1, 4             & 8.1                              & 27                           & 0    \\ \hline
DeepPocket\textsubscript{SEG} & 2349 (84.7\%) & 21,718 ($\times$3.2) & 1, 6, 196           & 7.7                              & 4.6                          & 0.4  \\ \hline
P2Rank\textsubscript{CONS}    & 2759 (92.9\%) & 12,412($\times$1.8)  & 1, 3, 57            & 7.1                              & 13.9                         & 0.05 \\ \hline
P2Rank        & 2402 (86.6\%) & 10,180 ($\times$1.5) & 1, 3, 85            & 7.1                              & 13.8                         & 0.05 \\ \hline
fpocket       & 2759 (99.4\%) & \textbf{57,859 ($\times$8.4}) & 1, 17, \textbf{349}          & 6.3                              & 9.7                          & 0.15 \\ \hline
PocketFinder\textsuperscript{+} & 2775 (100\%)   & 8913 ($\times$1.3)  & 1, 3, 23            & 8.6                              & 18.7                         & 0.05 \\ \hline
Ligsite\textsuperscript{+}      & 2775 (100\%)   & 6903 ($\times$1.0)  & 1, 2, 12            & \textbf{9.1} & 16.7                         & 0.09 \\ \hline
Surfnet\textsuperscript{+}      & 2775 (100\%)   & 9043 ($\times$1.3)  & 1, 3, 40            & 8.4                              & 17.2                         & 0.07 \\ \hline
\caption[Ligand site characterisation]{\textbf{Ligand site characterisation.} LIGYSIS is not a ligand site predictor, but a reference dataset derived from experimentally determined protein-ligand complexes. These predictions result from the default prediction of the methods, indicated by \textbf{(d)} preceding method names. Coverage is the number of chains where methods predict at least one pocket. Percentage is relative to number of LIGYSIS chains. Total number of pockets and ratio of predicted pockets per reference site in parenthesis, e.g., for each LIGYSIS site, fpocket predicts 8.4 pockets on average; Minimum, median and maximum number of predicted pockets per chain; Median pocket radius of gyration, \textit{R\textsubscript{g}}, (\AA{}); Minimum centroid distance (MCD) (\AA{}) measures how close predicted pockets are; Maximum residue overlap (MRO) measures residue overlap between pockets, e.g., the median overlap between VN-EGNN predicted pockets is 85\%. Bold font indicates the most extreme values within each column.}
\label{tab:pocket_features_stats}\\
\end{longtable}
\end{landscape}

\begingroup
\captionsetup{belowskip=0pt,aboveskip=9pt} % or some smaller space 
\begin{landscape}
\begin{longtable}[c]{|M{29mm}|M{29mm}|M{31mm}|M{27mm}|M{30mm}|M{19mm}|M{18mm}|M{18mm}|}
\hline
\textbf{Method}         & \textbf{Recall\textsubscript{top-\textit{N}}} (\%) & \textbf{Recall\textsubscript{top-\textit{N}+2}} (\%) & \textbf{Recall\textsubscript{max}} (\%) & \textbf{Precision\textsubscript{1K}} (\%) & \textbf{\# TP\textsubscript{100 FP}} & \textbf{RRO} (\%) & \textbf{RVO} (\%) \\ \hline
\endfirsthead
%
\footnotesize{(\textbf{d})} VN-EGNN        & 27.5 (\#11)           & 40.9 (\#12)             & 49.3 (\#10)         & \textbf{\textcolor{CBBlue}{92.5 (\#1)}}                   & \textbf{\textcolor{CBBlue}{1301 (\#1)}}               & \textbf{\textcolor{CBOrange}{32.8 (\#12)}}             & \textbf{\textcolor{CBOrange}{27.6 (\#11)}}             \\ \hline
\footnotesize{(\textbf{d})} IF-SitePred    & \textbf{\textcolor{CBOrange}{19.8 (\#12) }}           & \textbf{\textcolor{CBOrange}{25.7 (\#13)}}             & 52.1 (\#6)         & 91.0 (\#2)             & 961 (\#3)         & 46.5 (\#11)     & 40.4 (\#9)     \\ \hline
\footnotesize{(\textbf{d})} GrASP          & 48.0 (\#2)              & 49.9 (\#5)             & 50.0 (\#8)           & \textbf{\textcolor{CBBlue}{92.5 (\#1)}}                   & 1017 (\#2)       & 54.5 (\#7)     & 59.8 (\#6)     \\ \hline
\footnotesize{(\textbf{d})} PUResNet       & 40.6 (\#6)            & 41.1 (\#11)             & \textbf{\textcolor{CBOrange}{41.1 (\#12)}}         & 81.6 (\#6)           & 534 (\#8)         & 61.0 (\#4)     & 63.9 (\#4)     \\ \hline
\footnotesize{(\textbf{d})} DeepPocket\textsubscript{SEG}  & 35.4 (\#10)            & 43.8 (\#10)             & 56.5 (\#5)         & 82.6 (\#4)           & 670 (\#5)         & 57.5 (\#5)     & 60.3 (\#5)     \\ \hline
\footnotesize{(\textbf{d})} DeepPocket\textsubscript{RESC} & 46.6 (\#4)            & 58.1 (\#2)                     & 89.3 (\#2)         & 81.7 (\#5)          & 637 (\#6)         & 53.1 (\#9)     & 38.2 (\#10)     \\ \hline
\footnotesize{(\textbf{d})} P2Rank\textsubscript{CONS}     & \textbf{\textcolor{CBBlue}{48.8 (\#1)}}           & 53.9 (\#3)             & 57.0 (\#4)           & 90.7 (\#3)           & 932 (\#4)         & 56.4 (\#6)     & 43.8 (\#8)     \\ \hline
\footnotesize{(\textbf{d})} P2Rank         & 46.7 (\#3)            & 51.9 (\#4)             & 57.0 (\#3)           & 79.2 (\#7)           & 586 (\#7)         & 54.4 (\#8)     & 58.2 (\#7)   \\ \hline
\footnotesize{(\textbf{d})} fpocket\textsubscript{PRANK}        & \textbf{\textcolor{CBBlue}{48.8 (\#1)}}           & \textbf{\textcolor{CBBlue}{60.4 (\#1)}}             & \textbf{\textcolor{CBBlue}{91.3 (\#1)}}         & 81.7 (\#5)           & 526 (\#9)          & 52.6 (\#10)     & 38.2 (\#10)     \\ \hline
\footnotesize{(\textbf{d})} fpocket        & 38.8 (\#8)           & 46.5 (\#8)             & \textbf{\textcolor{CBBlue}{91.3 (\#1)}}         & 47.3 (\#9)           & 94 (\#11)          & 52.6 (\#10)     & 38.2 (\#10)     \\ \hline
\footnotesize{(\textbf{d})} PocketFinder\textsuperscript{+}  & 39.2 (\#7)           & 47.8 (\#7)             & 50.5 (\#7)         & 42.0 (\#10)             & 64 (\#12)          & 72.3 (\#2)     & 75.9 (\#2)     \\ \hline
\footnotesize{(\textbf{d})} Ligsite\textsuperscript{+}       & 41.3 (\#5)           & 48.4 (\#6)             & 49.7 (\#9)         & 52.3 (\#8)           & 115 (\#10)         & \textbf{\textcolor{CBBlue}{77.6 (\#1)}}             & \textbf{\textcolor{CBBlue}{77.0 (\#1)}}             \\ \hline
\footnotesize{(\textbf{d})} Surfnet\textsuperscript{+}       & 37.7 (\#9)           & 45.8 (\#9)             & 48.9 (\#11)         & \textbf{\textcolor{CBOrange}{39.5 (\#11)}}           & \textbf{\textcolor{CBOrange}{61 (\#13)}}                  & 71.7 (\#3)     & 72.0 (\#3)     \\ \hline
\caption[Pocket level evaluation]{\textbf{Pocket level evaluation.} This table illustrates the performance of default methods indicated by (\textbf{d}) preceding method names. Recall considering top-\textit{N}, \textit{N}+2 and \textit{all} predictions (max) regardless of rank, i.e., maximum recall. Precision for the top-1000 scored predictions. Number of TP reached for the first 100 FP (\# TP\textsubscript{100 FP}). Mean relative residue overlap (RRO) for those sites correctly predicted and relative volume overlap (RVO) for sites that have a volume, i.e., are pockets or cavities, and not exposed sites, which do not have a volume. RRO and RVO represent the overlap in residues and volume relative to the observed site. See \autoref{subsub:pocket_level_metrics} for definitions of RRO and RVO. Bold font indicates the best (blue) and worst (orange) performing methods for each metric.}
\label{tab:pocket_level_benchmark}\\
\end{longtable}
\end{landscape}
\endgroup

\subsection{Evaluation of predictive performance}

\subsubsection{Pocket level evaluation}

The ideal ligand binding site predictor would have a high precision, i.e., most of the predictions it makes are correct, whilst maintaining a high recall, i.e., recapitulating most of the observed sites. Moreover, the ideal predictor returns predictions that are non-redundant, i.e., it does not predict the same pocket multiple times. Additionally, pockets are ranked in a systematic manner according to a strong and meaningful pocket scoring scheme which captures well the nature of existing ligand binding sites and therefore ranks the predicted pockets from more likely (high score, top) to least likely (low score, bottom). A good predictor would also perform well at the residue level. This means it is able of capturing the likelihood for a residue binding a ligand. This can be done by means of a residue ligandability score, which additionally might highlight key residues, the more ligandable within a binding site. Ligand site prediction methods were benchmarked with these criteria in mind.

\autorefpanel{fig:pocket_level_benchmark_OG}{ A} illustrates the recall curve for top-\textit{N}+2 pockets for each method, where \textit{N} is the number of observed sites for a target protein. Reported recall is obtained using DCC = 12 \AA{}. Re-scored fpocket predictions by PRANK (fpocket\textsubscript{PRANK}) and DeepPocket (DeepPocket\textsubscript{RESC}) yield the highest recall with 60.4\% and 58.1\%, closely followed by P2Rank\textsubscript{CONS} (53.9\%) and P2Rank (51.9\%). The rest of the methods present recall $<$50\% with PUResNet, VN-EGNN and IF-SitePred presenting the lowest recalls of 41.1\%, 40.9\% and 25.7\%, respectively (\autoref{tab:pocket_level_benchmark}). \autorefpanel{fig:pocket_level_benchmark_OG}{ B} shows the recall curve considering different top-\textit{N}+\textit{X} predictions. Most methods reach a plateau by top-\textit{N}+5, as they do not predict that many pockets. However, methods that predict more pockets per protein, such as IF-SitePred or fpocket, fpocket\textsubscript{PRANK}, DeepPocket\textsubscript{SEG} and DeepPocket\textsubscript{RESC}, which take fpocket predictions as a base, increase their recall as more predictions are considered. fpocket, fpocket\textsubscript{PRANK} and DeepPocket\textsubscript{RESC} reach a maximum recall of $\approx$90\% when \textit{all} predictions are considered, regardless of rank. Other methods present maximum recall of $\approx$50-60\%. \autorefpanel{fig:pocket_level_benchmark_OG}{ C} depicts the recall curve if residue overlap was used instead of DCC as a criterion. In this case, Ligsite\textsuperscript{+}, PocketFinder\textsuperscript{+}, and Surfnet\textsuperscript{+} come on top with recall $\approx$45\% at \textit{I\textsubscript{rel}} $\geq$ 0.5. This is explained by their prediction of massive cavities, that while often fully contain or overlap with the observed pocket, do not meet the DCC criterion, as their centroids are farther than 12 \AA{} from the observed site.

\begin{figure}[ht!]
    \centering
    \includegraphics[width=\textwidth]{figures/ch_LBS_COMP/PNG/FIG7_POCKET_LEVEL_BENCHMARK_OG.png}
    \caption[Ligand binding site prediction benchmark at the pocket level]{\textbf{Ligand binding site prediction benchmark at the pocket level.} These curves correspond to the default predictions of the thirteen methods,
indicated by (\textbf{d}) preceding their names. \textbf{(A)} Recall, percentage of observed sites that are correctly predicted by a method within the top-\textit{N}+2 predictions according to a DCC = 12 \AA{} threshold; \textbf{(B)} Recall using DCC = 12 \AA{} but considering increasing rank thresholds, i.e., top-\textit{N}, \textit{N}+1, \textit{N}+2, etc. ``\textit{all}'' represents the maximum recall of a method, obtained by considering all predictions, regardless of their rank or score; \textbf{(C)} Recall curve for top-\textit{N}+2 predictions using \textit{I\textsubscript{rel}} as a criterion; \textbf{(D)} ROC100 curve (cumulative \#TP against cumulative FP until 100 FP are reached); \textbf{(E)} Precision curve for the top-1000 predictions of each method across the LIGYSIS dataset, Precision\textsubscript{1K}. Error bars represent 95\% CI of the recall \textbf{(A-C)} and precision \textbf{(E)}, which is 100 $\times$ proportion. Numbers at the right of the panels indicate groups or blocks of methods that perform similarly for each metric. Stars (*) indicate outlier methods, or methods that perform very differently than the rest.}
    \label{fig:pocket_level_benchmark_OG}
\end{figure}

\autorefpanel{fig:pocket_level_benchmark_OG}{ D} represents the cumulative number of TP against FP when predictions across the proteins in the reference dataset are sorted by score. This shows how effective the scoring scheme of each method is in ranking their predictions to reflect the nature of ligand binding sites. At 100 FP the \#TP fall into three different blocks and one outlier: Ligsite\textsuperscript{+}, fpocket, Surfnet\textsuperscript{+} and PocketFinder\textsuperscript{+} are at the bottom with \#TP $\ni$ (60, 120). Secondly, fpocket\textsubscript{PRANK}, DeepPocket\textsubscript{SEG}, DeepPocket\textsubscript{RESC}, P2Rank, and PUResNet follow with \#TP $\ni$ (530, 670). Re-scoring fpocket predictions with PRANK or DeepPocket results in up to +500 TP. GrASP, IF-SitePred and P2Rank\textsubscript{CONS}, present a high \#TP ranging 900-1000 at 100 FP. Finally, VN-EGNN sits at the top with 1301 TP. However, this number might not be representative, as the \#TP could be inflated due to the redundancy in the predictions of VN-EGNN. This is the same for IF-SitePred and DeepPocket\textsubscript{SEG}. Redundant correct predictions of the same pocket will count as multiple TPs, whereas they should only count as 1 TP. Newer methods, e.g., GrASP, P2Rank\textsubscript{CONS}, and VN-EGNN and IF-SitePred, despite redundancy in their prediction for the latter two, are better at ranking their predicted pockets, presenting up to 1000 more TP for 100 FP than earlier methods. This means their scoring schemes are significantly better at capturing the essence of a ligand binding site. Including evolutionary conservation in P2Rank (P2Rank\textsubscript{CONS}) results in an increase of +346 TP relative to default P2Rank, indicating that the fewer predicted pockets, and their scores are a more faithful representation of the observed LIGYSIS dataset.

\autorefpanel{fig:pocket_level_benchmark_OG}{ E} provides insight into the precision of the methods by examining how this metric changes as more predictions are considered. In the same manner as for \autorefpanel{fig:pocket_level_benchmark_OG}{ D}, predictions across proteins in the LIGYSIS dataset are sorted and cumulative precision is plotted for the top-1000 scoring predictions. Methods group into two clear blocks. Newer (machine learning-based) methods VN-EGNN, GrASP, IF-SitePred, P2Rank\textsubscript{CONS}, DeepPocket\textsubscript{SEG}, , fpocket\textsubscript{PRANK}, DeepPocket\textsubscript{RESC} and PUResNet present a Precision\textsubscript{1K} of 80-95\%. Earlier (geometry/energy-based) methods Ligsite\textsuperscript{+}, fpocket, Pocket-Finder\textsuperscript{+} and Surfnet\textsuperscript{+} present lower Precision\textsubscript{1K} of 40-50\%. fpocket\textsubscript{PRANK} and DeepPocket\textsubscript{RESC} take fpocket (geometry-based) predictions as a starting point and achieve much higher \#TP\textsubscript{100FP} (+500) as well as Precision\textsubscript{1K} (+30\%). This is further evidence that performance can be boosted with a solid scoring scheme and agrees with previous studies \cite{KRIVAK_2015_PRANK, KRIVAK_2015_P2RANK, KRIVAK_2018_P2RANK, COMAJUNCOSA_2024_POCKETS}.

\autoref{tab:pocket_level_benchmark} summarises these results and shows the mean relative residue overlap (RRO) and relative volume overlap (RVO), which measure how well predicted sites align with observed ones in shape. VN-EGNN and IF-SitePred present the smallest RRO and RVO, but it is important to note that these methods do not report pocket residues and so residues were taken within 6 \AA{} of their centroid, or pocket spheres. PocketFinder\textsuperscript{+}, Ligsite\textsuperscript{+}, and Surfnet\textsuperscript{+} present unusually high RRO and RVO. This is a consequence of the massive size of their predicted cavities, that rather than overlap with the observed site, fully contain and are much larger than it. This might not be convenient in the context of pocket finding for drug discovery where more clearly defined drug-like sites might be of interest. GrASP, PUResNet and DeepPocket\textsubscript{SEG} present high values of RRO $\approx$ 60\% and RVO $\approx$60\% whilst presenting a size distribution more like LIGYSIS (\autorefpanel{fig:pocket_features_1}{ B}) and provide the best representation of the observed sites in terms of shape similarity.

\subsubsection{Residue level evaluation}

Ligand binding site prediction tools can also be evaluated at the residue level. F1 Score as well as Matthews correlation coefficient (MCC) were utilised to do so. For each protein chain, F1 and MCC were calculated, distributions graphed and means reported (\autoref{tab:residue_level_benchmark}). Binary labels are employed to calculate these scores, 1 if the residue is found in a pocket and 0 otherwise, and compared to the ground truth, i.e., whether a residue binds a ligand in the LIGYSIS set. For VN-EGNN, IF-SitePred, PocketFinder\textsuperscript{+}, Ligsite\textsuperscript{+}, and Surfnet\textsuperscript{+}, which do not report pocket residues \autoref{tab:methods_details_1}, pocket residues were obtained by considering those residues within 6 \AA{} of the pocket centroid, cloud, or grid points, respectively. DeepPocket\textsubscript{RESC} was not considered for this analysis since the predictions are re-scored and re-ranked fpocket predictions.

\autorefpanel{fig:residue_level_benchmark}{ A-B} illustrate the distributions of the F1 score and MCC for each method on the 2775 protein chains of the final LIGYSIS set. Both metrics agree that PUResNet (F1 = 0.41, MCC = 0.39), GrASP (F1 = 0.39, MCC = 0.33) and P2Rank\textsubscript{CONS} (F1 = 0.36, MCC = 0.30) are the top-3 performing methods in this task of binary classification into pocket (1) and non-pocket residues (0). fpocket presents the lowest F1 = 0.23 and MCC = 0.12 since it predicts many unobserved pockets (residues) that will count as FP here.

IF-SitePred does not report a residue ligandability score beyond a binary label (0, 1). Nevertheless, in this Chapter, a score was computed by utilising the scores returned by the 40 prediction models of IF-SitePred. These scores range 0-1 and can be averaged as probabilities (\autoref{eq:IFSP_score}). This will now be referenced as IF-SitePred ligandability score. For IF-SitePred, GrASP, P2Rank\textsubscript{CONS}, P2Rank, PocketFinder\textsuperscript{+}, Ligsite\textsuperscript{+}, and Surfnet\textsuperscript{+}, which report a residue level score (beyond a binary label), ROC and PR curves were plotted (\autorefpanel{fig:residue_level_benchmark}{ C}) and mean area under the curve (AUC) and mean average precision (AP) reported. This was not possible for VN-EGNN, PUResNet, DeepPocket\textsubscript{SEG}, DeepPocket\textsubscript{RESC}, fpocket\textsubscript{PRANK} and fpocket as they do not report residue ligandability scores. \autoref{fig:ROC_variation} illustrates the variation in ROC/AUC for each method across the 2775 chains in the LIGYSIS set. P2Rank\textsubscript{CONS} and IF-SitePred, with the ligandability score calculated in this work (\autoref{eq:IFSP_score}), present the highest mean AUC = 76\%, closely followed by P2Rank (AUC = 74\%). Surfnet\textsuperscript{+} presents the lowest AUC = 68\%. \autorefpanel{fig:residue_level_benchmark}{ D} shows the mean PR curves, which agree with ROC AUC and highlight P2Rank\textsubscript{CONS} as the method with the highest average precision = 46\%, followed by IF-SitePred (with \autoref{eq:IFSP_score} scoring) with AP = 45\% and PocketFinder\textsuperscript{+} the lowest with (AP = 34\%). \autoref{fig:PR_variation} displays the variability across LIGYSIS proteins for PR curve and AP. \autorefpanel{fig:residue_level_benchmark}{ E} shows IF-SitePred presenting a different residue ligandability score distribution to GrASP, P2Rank\textsubscript{CONS}, and P2Rank. The IF-SitePred ligandability score, resulting from averaging the scores from the 40 IF-SitePred models, is the most ``generous'' with $\approx$20\% of the residues presenting a score $>$ 0.5, in contrast with GrASP which residue scoring is very strict \textit{P}(\textit{LS} $\geq$ 0.5) = 0.75\% and P2Ranks ($\approx$3\%). This difference, combined with the mean ROC and PR curves, further supports the use of the IF-SitePred ligandability score proposed in this Chapter to define the predicted binding sites for this method. It also suggests that GrASP might benefit from a less strict residue level scoring scheme. PocketFinder\textsuperscript{+}, Ligsite\textsuperscript{+} and Surfnet\textsuperscript{+} were not included in this analysis as their scores do not range 0-1 and very high scores ($>$25) can be obtained.

\begin{figure}[htb!]
    \centering
    \includegraphics[width=\textwidth]{figures/ch_LBS_COMP/PNG/FIG8_RESIDUE_LEVEL_BENCHMARK.png}
    \caption[Ligand binding site prediction benchmark at the residue level]{\textbf{Ligand binding site prediction benchmark at the residue level.} DeepPocket\textsubscript{RESC} predictions are not included in F1 and MCC analyses as these are re-scored and re-ranked fpocket predictions and the results would be the same as for fpocket. \textbf{(A)} F1 score distributions; \textbf{(B)} MCC distributions. In both panels, each data point corresponds to the score obtained from all residues in a protein chain; \textbf{(C)} Mean ROC curve for methods that report a residue score. Dashed line represents the baseline, 1 FP for each TP, i.e., diagonal and AUC = 50\%; \textbf{(D)} Mean PR curve. Dashed line represents the baseline, i.e., percentage of observed binding residues (true positives) = 10\%; \textbf{(E)} Distribution of residue ligandability scores for IF-SitePred, GrASP, P2Rank\textsubscript{CONS} and P2Rank. PocketFinder\textsuperscript{+}, Ligsite\textsuperscript{+} and Surfnet\textsuperscript{+} are not included as their scores do not range 0-1, and a small number of scores can reach values $>$ 25. These predictions are from the original methods run with default parameters (\textbf{d}).} 
    \label{fig:residue_level_benchmark}
\end{figure}

\begin{figure}[htb!]
    \centering
    \includegraphics[width=\textwidth]{figures/ch_LBS_COMP/PNG/SUPP_FIG_X_ROC_variation.png}
    \caption[Variation in ROC curve and AUC across LIGYSIS proteins]{\textbf{Variation in ROC curve and AUC across LIGYSIS proteins.} For each of the methods that report or for which residue ligandability scores were calculated, a ROC curve was calculated for each of the 2775 protein chains in the LIGYSIS set and AUC calculated. Plotted curves represent the mean ROC curve for each method. These are obtained by averaging the TPR for each FPR interval across proteins. Shaded area represents one standard deviation (1 SD) from the mean ROC curve. Reported AUC is the mean AUC calculated by averaging the AUC for the 2775 ROC curves obtained. Baseline AUC is random chance (AUC = 50\%). \textbf{(A)} IF-SitePred; \textbf{(B)} GrASP; \textbf{(C)} P2Rank\textsubscript{CONS}; \textbf{(D)} P2Rank; \textbf{(E)} PocketFinder\textsuperscript{+}; \textbf{(F)} Ligsite\textsuperscript{+}; \textbf{(G)} Surfnet\textsuperscript{+}. These results originate from default methods, indicated by (\textbf{d}) preceding method names.}
    \label{fig:ROC_variation}
\end{figure}

\begin{figure}[htb!]
    \centering
    \includegraphics[width=\textwidth]{figures/ch_LBS_COMP/PNG/SUPP_FIG_X_PR_variation.png}
    \caption[Variation in PR curve and AP across LIGYSIS proteins]{\textbf{Variation in PR curve and AP across LIGYSIS proteins.} For each of the methods that report or for which residue ligandability scores were calculated, a precision-recall (PR) curve was calculated for each of the 2775 protein chains in the LIGYSIS set and average precision (AP) calculated. Plotted curves represent the mean PR curve for each method. These are obtained by averaging the precision for each recall interval across proteins. Shaded area represents one standard deviation (1 SD) from the mean PR curve. Reported AP is the mean AP calculated by averaging the AP for the 2775 PR curves obtained. Baseline AP is the percentage of observed ligand-binding residues (AP = 10\%). \textbf{(A)} IF-SitePred; \textbf{(B)} GrASP; \textbf{(C)} P2Rank\textsubscript{CONS}; \textbf{(D)} P2Rank; \textbf{(E)} PocketFinder\textsuperscript{+}; \textbf{(F)} Ligsite\textsuperscript{+}; \textbf{(G)} Surfnet\textsuperscript{+}. These results originate from default methods, indicated by (\textbf{d}) preceding method names.}
    \label{fig:PR_variation}
\end{figure}

\FloatBarrier

\begin{longtable}[c]{|c|c|c|c|c|}
\hline
\textbf{Method}         & \textbf{F1}   & \textbf{MCC}  & \textbf{AUC} (\%)  & \textbf{AP} (\%)   \\ \hline
\endfirsthead
%
\footnotesize{(\textbf{d})} VN-EGNN        & 0.29 (\#8) & 0.26 (\#4) & --    & --    \\ \hline
\footnotesize{(\textbf{d})} IF-SitePred    & 0.29 (\#9) & 0.24 (\#6) & 76 (\#2)  & 45 (\#2) \\ \hline
\footnotesize{(\textbf{d})} GrASP          & 0.39 (\#2) & 0.34 (\#2) & 70 (\#4) & 42 (\#3) \\ \hline
\footnotesize{(\textbf{d})} PUResNet       & \textbf{\textcolor{CBBlue}{0.41 (\#1)}} & \textbf{\textcolor{CBBlue}{0.39 (\#1)}} & --    & --    \\ \hline
\footnotesize{(\textbf{d})} DeepPocket\textsubscript{SEG}  & 0.27 (\#10) & 0.21 (\#9) & --    & --    \\ \hline
\footnotesize{(\textbf{d})} P2Rank\textsubscript{CONS}     & 0.36 (\#3) & 0.30 (\#3)  & \textbf{\textcolor{CBBlue}{76 (\#1)}} & \textbf{\textcolor{CBBlue}{46 (\#1)}} \\ \hline
\footnotesize{(\textbf{d})} P2Rank         & 0.31 (\#4) & 0.26 (\#5) & 74 (\#3) & 42 (\#3) \\ \hline
\footnotesize{(\textbf{d})} fpocket        & \textbf{\textcolor{CBOrange}{0.23 (\#11)}} & \textbf{\textcolor{CBOrange}{0.12 (\#11)}} & --    & --    \\ \hline
\footnotesize{(\textbf{d})} PocketFinder\textsuperscript{+}  & 0.31 (\#5) & 0.22 (\#7) & 68 (\#6) & \textbf{\textcolor{CBOrange}{34 (\#6)}} \\ \hline
\footnotesize{(\textbf{d})} Ligsite\textsuperscript{+}       & 0.31 (\#6) & 0.21 (\#8) & 70 (\#5) & 38 (\#4) \\ \hline
\footnotesize{(\textbf{d})} Surfnet\textsuperscript{+}       & 0.29 (\#7) & 0.20 (\#10)  & \textbf{\textcolor{CBOrange}{68 (\#7)}} & 35 (\#5) \\ \hline
\caption[Residue level evaluation]{\textbf{Residue level evaluation.} These results come from default predictions, indicated by (\textbf{d}). DeepPocket\textsubscript{RESC} was not considered in this analysis as their predictions are re-scored and re-ranked fpocket's. Ligand binding site prediction benchmark at the residue level was calculated from 1775 protein chains in the LIGYSIS set. Mean F1 score,mean Matthews correlation coefficient (MCC), mean ROC area under the curve (AUC) and mean precision recall (PR) curve average precision (AP). Numbers following a hash (\#) indicate how methods rank for each metric. Bold font indicates the best (blue) and worst (orange) performing methods. Pocket binary labels (0, 1) were employed for the calculation of F1 and MCC and obtained form predicted pockets. Residue ligandability scores were employed to calculate ROC/AUC and PR/AP. Reported AUC and AP are means resulting from the average across the 2775 LIGYSIS chains. This was not possible for VN-EGNN, PUResNet, DeepPocket\textsubscript{SEG} and fpocket as these methods do not provide such scores, indicated by a dash (--).}
\label{tab:residue_level_benchmark}\\
\end{longtable}

\vspace{-12pt} % Adjust this value as needed
\vspace{-12pt} % Adjust this value as needed

\section{Discussion}

This Chapter describes the most complete comparative analysis of ligand binding site prediction methods to date, spanning three decades of methods development. Firstly, predictions from the thirteen methods as well as observed sites from the new reference dataset introduced here, LIGYSIS, were compared in terms of the number of proteins methods predict on, the number of predicted sites per protein, their size, distance and overlap between the sites. This analysis provides insight into how the different methods work and hints at potential limitations or room for improvement, e.g., the prediction of a fixed number of sites per protein, or considerable proximity and overlap between the predictions. Secondly, predictions from thirteen canonical ligand binding site prediction methods are objectively evaluated using the LIGYSIS set. This evaluation considers prediction at the residue level by F1 score, MSCC, ROC/AUC and PR/AP, as well as the pocket level by recall for top-\textit{N}, \textit{N}+2 and \textit{all} predictions, precision\textsubscript{1K}, \# TP\textsubscript{100 FP}, RRO and RVO. This is the first independent ligand site prediction benchmark since Schmidtke \textit{et al.} \cite{SCHMIDTKE_2010_BENCHMARK} in 2010 and Chen \textit{et al.} \cite{CHEN_2011_ASSESSMENT} shortly after in 2011 and the largest to date both in terms of reference dataset size (2775), methods compared (10) and metrics employed ($>$10).

Recall (\% of observed sites that are correctly predicted) is more informative than precision (\% of predictions that are correct), particularly, recall considering top-\textit{N}+2 ranked predictions. In most cases, not all the existing binding pockets are observed with a ligand bound. In other words, the reference data are incomplete, with 33-50\% of existing sites yet to be observed with ligands bound in a structure, as conjectured by Krivák and Hoksza \cite{KRIVAK_2018_P2RANK}. Considering only the top-\textit{N} predicted pockets assumes that there are exactly \textit{N} real pockets for a given protein, which might not be the case. A method could predict a \textit{real} pocket that is yet to be observed and rank it before other predicted and observed pockets. By considering the top-\textit{N}+2 pockets, the noise in the reference data is controlled for to some extent and a more accurate representation of the method performance is obtained. In a context of discovery, where the true ligand binding sites of the target are unknown, it is more useful to have multiple predictions that might or might not correspond to real sites (lower precision), rather than a single or few predictions that are very precise but are missing other likely sites (lower recall). Most methods do well in predicting the most obvious (orthosteric) site. This site, however, might not be available for therapeutic targeting and it is convenient to predict other sites that could modulate function acting as allosteric sites. Precision, though a metric that provides valuable insight, and covered in this work, must always be contextualised with recall. This Chapter shows that the most precise methods do not correspond to higher recalling methods. A method predicting the most obvious site, that could be identified by eye, might be 90\% precise, but present a lower recall, e.g., 30\%. This said, methods predicting fewer pockets with higher precision might prove more advantageous when users aim to study a particular region of interest in a protein, a few high-priority sites are needed for experimental validation or false predictions are costly in downstream analysis. 

Some methods define ``success rate'' as the precision of the top-1 or top-3 scoring predictions, which is not a very representative performance assessment metric. For this reason, \textit{we} encourage method developers not only to share the code of their approach, but also of the benchmarking analysis. Furthermore, the definition of success rate must be standardised as recall, as some methods use recall, whereas others sue precision, both under the name of \textit{success rate}. This can be confusing when comparing the results from different analyses. Moreover, due to the inherent noise in the reference data, i.e., not all existing pockets are known, recall considering top-\textit{N}+2 is more informative than taking top-1, top-3 or top-\textit{N} predictions. In any case, success rate must be clearly defined, so readers can fully understand the implications of the metric employed in a benchmark.

It is clear from the results described in this Chapter that a DCC threshold of 4 \AA{} is too conservative, and a more flexible DCC threshold of 10-12 \AA{} should be used for comparable performance with DCA = 4 \AA{}. According to this work and the LIGYSIS reference, predictions with DCC 4-12 \AA{} overlap or are adjacent to observed sites and should be considered as correct predictions. The reason for this is the inherent noise in the ground truth, i.e., a ligand binding to a cavity might not be representative of all ligands that could bind to it. For most proteins, not all existing ligand sites are characterised and as different ligands can bind to the same region, it is unrealistic to use such a small DCC threshold. Our results show several examples of predictions of an observed cavity where DCC $>$ 4 \AA{}.

Re-scoring of original fpocket predictions by PRANK (fpocket\textsubscript{PRANK}) and DeepPocket (DeepPocket\textsubscript{RESC}) present the highest recall considering top-\textit{N}+2 predictions of 60.4\% and 58.1\%, respectively. P2Rank\textsubscript{CONS} and P2Rank follow closely with 53.9\% and 51.9\% recall, then GrASP (49.9\%). All other methods range between 40-49\% and IF-SitePred presents the lowest recall (25.7\%). fpocket and methods that re-score its predictions, i.e., fpocket\textsubscript{PRANK} and DeepPocket\textsubscript{RESC} predict the most pockets per protein, reaching a maximum recall between 80-90\% (considering all pockets regardless of the rank). DeepPocket\textsubscript{SEG}, P2Rank\textsubscript{CONS} and P2Rank follow with a maximum recall of $\approx$60\%. The rest of the methods range 40-55\%. This indicates that while there are still some pockets un-predicted by fpocket (10-20\%), the maximum recall of this method is 20-30\% higher than any other method. However, considering top-\textit{N}+2 pockets, fpocket only recalls 47\% of the observed pockets. fpocket\textsubscript{PRANK} and DeepPocket\textsubscript{RESC} gain $>$10\% in recall by simply re-scoring those predictions. This highlights the paramount importance of a robust scoring scheme, which captures well the nature of binding sites and places those with a higher probability of being real binding sites at the top of the ranking.

Newer methods like VN-EGNN, IF-SitePred and GrASP are the most precise methods, however because of redundancy in predictions (VN-EGNN, IF-SitePred), or low number of predicted pockets per protein (VN-EGNN and GrASP) are limited in their recall. Their high precision indicates that their models learn and capture well the nature of ligand binding sites and so they represent a great venue to pursue in the field of ligand binding site prediction. This is also reflected when looking at \# TP\textsubscript{100 FP}.

The usefulness of residue-level metrics like F1 score and MCC is limited, as methods that predict the clearest sites, e.g., PUResNet (high precision, low recall) perform better on these metrics, while methods that predict more pockets, e.g., fpocket and re-scored versions (lower precision, higher recall) obtain worse results. Pocket-level metrics, such as top-\textit{N}+2 recall, are more representative of the ability to predict ligand binding sites.

\begin{figure}[htb!]
    \centering
    \includegraphics[width=\textwidth]{figures/ch_LBS_COMP/PNG/SUPP_FIG_FULL_VS_NI_SPLIT1.png}
    \caption[Change in top-\textit{N}+2 recall for LIGYSIS \textit{vs} LIGYSIS\textsubscript{NI} (I)]{\textbf{Change in top-\textit{N}+2 recall for LIGYSIS \textit{vs} LIGYSIS\textsubscript{NI} (I).} Recall is calculated considering top-\textit{N}+2 pockets at DCC = 12 \AA{} and default methods (\textbf{d}) on a set of LIGYSIS binding sites containing at least one non-ion ligand, \textit{N} = 4141/6882 (60\%). Solid lines indicate recall curve on LIGYSIS and dashed lines for LIGYSIS\textsubscript{NI}. The relative change in recall and rank are indicated by Δ\textsubscript{Recall} and Δ\textsubscript{Rank}. These changes are relative to performance on LIGYSIS (including ions). All machine learning-based methods present an increase in recall when removing ion binding sites. This is expected ad none of the methods are trained on ion sites. However, ion sites were kept on the main benchmark to challenge and test the limits of the methods. \textbf{(A)} VN-EGNN; \textbf{(B)} IF-SitePred; \textbf{(C)} GrASP; \textbf{(D)} PUResNet; \textbf{(E)} DeepPocket\textsubscript{SEG}; \textbf{(F)} DeepPocket\textsubscript{RESC}; \textbf{(G)} P2Rank\textsubscript{CONS}; \textbf{(H)} P2Rank; \textbf{(I)} fpocket\textsubscript{PRANK}.}
    \label{fig:LIGYSIS_VS_LIGYSISNI_1}
\end{figure}

\begin{figure}[htb!]
    \centering
    \includegraphics[width=\textwidth]{figures/ch_LBS_COMP/PNG/SUPP_FIG_FULL_VS_NI_SPLIT2.png}
   \caption[Change in top-\textit{N}+2 recall for LIGYSIS \textit{vs} LIGYSIS\textsubscript{NI} (II)]{\textbf{Change in top-\textit{N}+2 recall for LIGYSIS \textit{vs} LIGYSIS\textsubscript{NI} (II).} Recall is calculated considering top-\textit{N}+2 pockets at DCC = 12 \AA{} and default methods (\textbf{d}) on a set of LIGYSIS binding sites containing at least one non-ion ligand, \textit{N} = 4141/6882 (60\%). Solid lines indicate recall curve on LIGYSIS and dashed lines for LIGYSIS\textsubscript{NI}. The relative change in recall and rank are indicated by Δ\textsubscript{Recall} and Δ\textsubscript{Rank}. These changes are relative to performance on LIGYSIS (including ions). Geometry and energy-based methods also present an increase in recall when removing ion binding sites. This is expected ad none of the methods are trained on ion sites. However, ion sites were kept on the main benchmark to challenge and test the limits of the methods. fpocket is an exception to this, suggesting this method \textit{does} not struggle with ion binding sites as much as other methods do.}
    \label{fig:LIGYSIS_VS_LIGYSISNI_2}
\end{figure}

While datasets like PDBbind, binding MOAD or the brand new PLINDER \cite{DURAIRAJ_2024_PLINDER} (will) prove extremely useful to train, validate and test deep learning models tackling problems such as rigid body docking \cite{STARK_2022_EQUIBIND}, flexible pocket docking \cite{QIAO_2024_DGN}, or pocket-conditioned ligand generation \cite{SCHNEUING_2023_DIFFUSION}, they might not be ideal as a test set for ligand binding site prediction. LIGYSIS analyses all unique, biologically relevant protein-ligand interfaces, including ions, across the biological assembly from multiple experimentally determined structures of a given protein. It then clusters these ligands based on their interactions with the protein, resulting in the observed binding sites. Beyond considering biological assemblies and unique protein-ligand interfaces, the greatest innovation in LIGYSIS is leveraging the extensive structural data on the PDBe-KB to aggregate ligand-binding interactions across different structures of the same protein, thus better capturing the ligand-binding capabilities than just taking a single structure of a protein-ligand complex. In doing so, LIGYSIS represents the most complete and non-redundant protein-ligand complex dataset to date for the prediction of protein-ligand binding sites.

The benchmark is performed on LIGYSIS, which includes ion binding sites. When these are removed, all methods, except for fpocket, experience an increase in (top-\textit{N}+2) recall of 5-10\% and the overall ranking of methods does not change (\autoref{fig:LIGYSIS_VS_LIGYSISNI_1} and \autoref{fig:LIGYSIS_VS_LIGYSISNI_2}. Due to its integrative approach, features, diversity and size (covering $>$30\$ of PDB and $>$20\% of BioLiP), LIGYSIS is the most inclusive and representative dataset of protein-ligand interactions.

Aggregating protein-ligand interactions across structures of the same protein is likely to be beneficial not only for testing, but also when training these methods. Most current methods train on datasets where a protein is represented by a single structure interacting with a single ligand. For example, in 100\% of sc-PDB and 50\% of entries for binding MOAD training sets. Methods consider as ligand binding, and therefore TP, those residues within a certain distance of the ligand and TN all other residues. In doing so, residues of the same protein that bind ligands on other structures, but not the one present on these sets, will be incorrectly labelled as ``non-ligand-binding'' (FN). This mislabelling of residues could lead to a lower prediction performance. This issue is to a certain extent approached by P2Rank and GrASP, which enriched their training datasets by including ligands from other chains, or homologous structures. This noise in the training dataset might be more prevalent for DeepPocket, PUResNet and VN-EGNN, which seem to rely fully on 1:1 interactions. The usage of LIGYSIS, or any other data set that aggregates ligand interactions across structures, might alleviate this issue and hints at potential room for improvement in the field of ligand binding site prediction.

\section{Conclusions}

The conclusions resulting from the analysis described in this Chapter are as follows:

\begin{itemize}

\item Ligand binding site prediction methods differ significantly in the number of predicted sites, their size, proximity and overlap, which offers insight into how the methods work.

\item Recall is a more informative measure of the performance of a ligand site prediction tool, rather than precision and so it must be reported. Precision, though a useful metric, should always be contextualised with recall.

\item LIGYSIS aggregates non-redundant biologically relevant protein-ligand interactions across multiple structures for a protein and sets a new test set standard for the benchmark of ligand binding site prediction tools. 

\item All authors of ligand site prediction tools should use top-\textit{N}+2 recall as ``success rate'' for consistency. Benchmarking code should also be shared by the authors for the sake of reproducibility.

\item Pocket-level metrics (recall, precision) are a more adequate representation of the capabilities of ligand site prediction methods than residue-level metrics (F1, MCC).

\item A DCC threshold of 4 \AA{} is too conservative, and to obtain comparable results between DCA and DCC recall, a threshold of DCC of 10-12 \AA{} should be employed.

\item Re-scoring of fpocket predictions, like fpocket\textsubscript{PRANK} or DeepPocket\textsubscript{RESC} present the highest (top-\textit{N}+2) recall (60\%) among the methods reviewed in this Chapter.

\item Methods that systematically predict a low number of pockets, e.g., VN-EGNN, GrASP or PUResNet, are very precise ($>$90\%), however their recall is low, and might not be as useful in a discovery context.

\item The IF-SitePred ligandability scored introduced in this work correctly recapitulates observed ligand binding sites and suggests IF-SitePred could benefit greatly from using it in their prediction.

\item The use of duplicated protein-ligand interfaces in asymmetric units results in an overestimate of both precision and recall when benchmarking ligand site predictors. Only unique protein-ligand interfaces in biological units should be considered for a more accurate benchmark of the performance of these methods.

\item The work presented in this Chapter objectively evaluates the performance of thirteen canonical ligand binding site prediction methods and represents the largest benchmark of ligand site prediction tools to date.

\end{itemize}


\chapter{Improvement on methods for the prediction of protein-ligand binding sites}
\label{chap:LBS_IMPROV}

\section*{Preface}

This Chapter explores in detail fifteen non-redundant and scoring variants of the thirteen ligand binding site prediction methods evaluated in \autoref{chap:LBS_COMP}. The negative effect of feeble pocket scoring schemes and redundancy in ligand site prediction is demonstrated through the performance evaluation of these variants relative to their default modes using seven informative metrics. The work in this Chapter was solely carried out by \textit{me}.

\section*{Publications}

Utgés, J.S. and Barton, G.J. Comparative evaluation of methods for the prediction of protein-ligand binding sites. \textit{J. Cheminform.} \textbf{16}, 126 (2024). \url{https://doi.org/10.1186/s13321-024-00923-z}.

%\section*{Author contributions}

%J.S.U. and G.J.B. conceived, designed, and developed the research. J.S.U. analysed the data. J.S.U. developed the software. J.S.U. and G.J.B. wrote, reviewed and edited the manuscript. G.J.B. secured funding and supervised.

\section{Introduction}

In \autoref{chap:LBS_COMP} the human component of the LIGYSIS dataset was employed to carry out the largest critical assessment of ligand binding site prediction tools to date \cite{UTGES_2024_LBSCOMP}. This evaluation included a set of thirteen methods combining the latest machine learning methods such as VN-EGNN \cite{SESTAK_2024_VNEGNN}, IF-SitePred \cite{CARBERY_2024_IFSP} or GrASP \cite{SMITH_2024_GrASP}, established methods like P2Rank \cite{KRIVAK_2015_P2RANK, KRIVAK_2018_P2RANK}, PRANK \cite{KRIVAK_2015_PRANK} or fpocket \cite{GUILLOUX_2009_FPOCKET, SCHMIDTKE_2010_FPOCKET2} and earlier geometry/energy-based methods such as PocketFinder\textsuperscript{+} \cite{AN_2005_POCKETFINDER}, Ligsite\textsuperscript{+} \cite{HENDLICH_1997_LIGSITE} and Surfnet\textsuperscript{+} \cite{LASKOWSKI_1995_SURFNET}. These methods were thoroughly evaluated at the residue and pocket level using more than ten different metrics.

Beyond ranking the methods by several metrics, \autoref{chap:LBS_COMP} identified VN-EGNN, IF-SitePred and DeepPocket\textsubscript{SEG} as having predicted pockets within very high spatial proximity ($<$5 \AA{}) and significant residue overlap (\textit{I\textsubscript{rel}} $>$ 0.5). Both of these features hint at redundancy in pocket prediction. Additionally, PUResNet, PocketFinder\textsuperscript{+}, Ligsite\textsuperscript{+} or Surfnet\textsuperscript{+} were highlighted as they do not report scores, nor explicit rank for their predicted pockets. Both of these issues, pocket prediction redundancy and lack of scoring scheme, are likely to have a considerable negative effect on the methods' performance. This Chapter explores in detail both of these aspects and evaluates the performance of fifteen novel scoring and non-redundant variants of the thirteen methods evaluated in \autoref{chap:LBS_COMP}.

\begin{figure}[htb!]
    \centering
    \includegraphics[width=\textwidth]{figures/ch_LBS_IMPROV/PNG/FIG6_PREDICTION_REDUNDANCY_SPLIT_1.png}
    \caption[The issue of redundancy in ligand binding site prediction]{\textbf{The issue of redundancy in ligand binding site prediction.} \textbf{(A)} A set of predictions where 6/10 (60\%) predictions are redundant, resulting in a low recall of 1/5 (20\%) and inflated precision of 7/7 (100\%); \textbf{(B)} When redundancy is removed, only four predictions remain and recall increases to 3/5 (60\%) and precision decreases to 3/4 (75\%).}
    \label{fig:prediction_redundancy}
\end{figure}

Pocket prediction redundancy is defined here as the prediction of pockets with centroids very close in space (\textit{D} $\leq$ 5\AA{}) or with overlapping residues (\textit{I\textsubscript{rel}} $\geq$ 0.75). This indicates multiple predictions of the same potential ligand binding site. Most ligand site prediction tools predict not only the location of the pocket by means of a centroid or pocket residues, but also a pocket confidence, and an associated rank among all the predicted pockets. Ligand site predictors tend to be evaluated by considering the top-\textit{N}, or top-\textit{N}+2 ranking pockets, where \textit{N} is the number of observed sites for a given protein. The redundant prediction of pockets can result in a sub-optimal ranking, thus negatively affecting the performance of the predictors.

\newpage

\begin{figure}[htb!]
    \centering
    \includegraphics[width=\textwidth]{figures/ch_LBS_IMPROV/PNG/FIG6_PREDICTION_REDUNDANCY_SPLIT_2.png}
    \caption[Example of redundant predictions]{\textbf{Example of redundant predictions.} Predictions by VN-EGNN, IF-SitePred and PUResNet, on chain D of PDB: \href{https://www.ebi.ac.uk/pdbe/entry/pdb/4z9m}{4Z9M} \cite{PDB_4Z9M} of human creatine kinase (\href{https://www.uniprot.org/uniprotkb/P17540/entry}{P17540}) where \href{https://www.ebi.ac.uk/pdbe-srv/pdbechem/chemicalCompound/show/ADP}{ADP} binds. For this \href{https://www.ebi.ac.uk/pdbe-srv/pdbechem/chemicalCompound/show/ADP}{ADP} binding site, VN-EGNN reports 7 predictions, IF-SitePred 33 and PUResNet a single prediction. These three methods correctly predict this site. However, VN-EGNN and IF-SitePred report redundant pocket predictions, which centroids are very close ($\leq$ 5 \AA{}) in space and residues overlap ($\geq$ 0.75).}
    \label{fig:prediction_redundancy_examples}
\end{figure}

\autoref{fig:prediction_redundancy} shows an example protein with \textit{N} = 5 observed pockets. A ligand site predictor returns 10 predictions, but the top-7 are all within 3 \AA{} of one of the observed pockets, and $>$12 \AA{} from any of the other four observed pockets. If the top-\textit{N}+2 (top-7) predictions were considered, this would only recall a single unique pocket, as six of the top-7 predictions are redundant. Top-\textit{N}+2 recall would then be 20\% (1/5). Precision, however, within this top-7 would be 100\% (7/7), as the seven predictions are correctly recalling an observed pocket (which happens to be the same). In this case, both the low recall and the high precision are artefacts resulting of the redundancy (\autorefpanel{fig:prediction_redundancy}{ A}). Redundancy in prediction can often result in the overestimate of the precision and the underestimate of the recall. \autorefpanel{fig:prediction_redundancy}{ B} illustrates what happens when redundant predictions are removed, keeping always higher-scoring predictions. When the six redundant predictions (blue stars) are removed, the other three predictions, which are of different pockets, are considered as now fall within the top-\textit{N}+2 predictions. This increases the recall to 60\%, as 3/5 observed pockets are now correctly predicted. However, precision decreases, as only three out of the four predictions made overlap with an observed pocket. Pocket rank \#2 has a high score but is not observed. This is a \textit{false positive} in this context, however it might be a candidate pocket yet to be resolved and could prove interesting as a drug target.

\autoref{fig:prediction_redundancy_examples} showcases PDB: \href{https://www.ebi.ac.uk/pdbe/entry/pdb/4z9m}{4Z9M} of human creatine kinase S-type, mitochondrial (\href{https://www.uniprot.org/uniprotkb/P17540/entry}{P17540}) as an example of this phenomenon, where VN-EGNN and IF-SitePred redundantly predict the same pocket 7 and 33 times, whereas PUResNet returns a single prediction. All three methods correctly predict the site, just the difference is in the number of returned predictions. These redundant predictions of the same observed site would count as 7 and 33 TPs when calculating precision, leading to an inflated artificial precision.

\section{Methods}

\subsection{Generation of non-redundant sets of predictions}

\autorefpanel{fig:pocket_features_2}{ B} shows that prediction redundancy is an issue particularly for VN-EGNN, IF-SitePred, and to a lesser extent, DeepPocket\textsubscript{SEG}. To assess the effect that redundancy had on the performance of these methods, non-redundant subsets of predictions were obtained and labelled with the subscript ``NR''. A predicted pocket $i$ is considered redundant if there exists a pocket $j \neq i$ so that the distance between their centroids $D_{i,j} \leq$ 5 \AA{} or their residue overlap $JI_{i,j} >$ 0.75, i.e., they share at least 3/4 (75\%) of their residues. Refer to \autoref{fig:closest_pred_pockets} for the closest predicted sites for each method. Redundancy filtering was carried out for each method keeping always the higher-scoring pocket. Redundancy (\%) was calculated as the proportion of redundant pockets relative to the original total number of pockets. VN-EGNN presents the highest percentage of redundant pockets with 9066/13,582 (67\%) redundant pockets, followed by IF-SitePred with 22,232/44,948 (49\%) and DeepPocket\textsubscript{SEG} with 6744/21,718 (31\%). For other methods, redundancy was minimal ($<$1\%).

\begin{figure}[ht!]
    \centering
    \includegraphics[width=\textwidth]{figures/ch_LBS_IMPROV/PNG/SUPP_FIG9_CLOSEST_PREDICTED_POCKET.png}
    \caption[Closest predicted pockets for each methods]{\textbf{Closest predicted pockets for each method.} LIGYSIS is a reference dataset, not a prediction method. For each method, the two closest predicted pockets across all protein chains are shown. This is the pair of pockets with the minimum Euclidean distance between their centroids. Protein surface is coloured in tan. The larger pocket (more residues) and centroid is coloured in the method colour and the other pocket's in grey. A distance threshold of \textit{D} = 5 \AA{} was selected to determine whether a pocket prediction was redundant. VN-EGNN, IF-SitePred and DeepPocket\textsubscript{SEG} clearly differ from other methods presenting distances $<$ 1 \AA{}. Examples for each method are from top to bottom and left to right: \href{https://www.uniprot.org/uniprotkb/P00492/entry}{P00492} -- PDB: \href{https://www.ebi.ac.uk/pdbe/entry/pdb/3gep}{3GEP} \cite{KEOUGH_2009_HYPOXAN}, chain: B; \href{https://www.uniprot.org/uniprotkb/Q96KS0/entry}{Q96KS0} -- PDB: \href{https://www.ebi.ac.uk/pdbe/entry/pdb/5v1b}{5V1B} \cite{AHMED_2017_HIF}, chain: A; \href{https://www.uniprot.org/uniprotkb/P31645/entry}{P31645} -- PDB: \href{https://www.ebi.ac.uk/pdbe/entry/pdb/5i73}{5I73} \cite{COLEMAN_2016_SEROTONIN}, chain: A; \href{https://www.uniprot.org/uniprotkb/Q04724/entry}{Q04724} -- PDB: \href{https://www.ebi.ac.uk/pdbe/entry/pdb/1gxr}{1GXR} \cite{PICKLES_2002_WD40}, chain: A; \href{https://www.uniprot.org/uniprotkb/Q5W0Z9/entry}{Q5W0Z9} -- PDB: \href{https://www.ebi.ac.uk/pdbe/entry/pdb/7khm}{7KHM} \cite{CHULJIN_2022_ACYLCOA}, chain: B; \href{https://www.uniprot.org/uniprotkb/Q06187/entry}{Q06187} -- PDB: \href{https://www.ebi.ac.uk/pdbe/entry/pdb/1b55}{1B55} \cite{BARALDI_1999_PH}, chain: B; \href{https://www.uniprot.org/uniprotkb/Q9UQG0/entry}{Q9UQG0} -- PDB: \href{https://www.ebi.ac.uk/pdbe/entry/pdb/7sr6}{7SR6} \cite{BALDWIN_2022_HERVK}, chain: G; \href{https://www.uniprot.org/uniprotkb/P13866/entry}{P13866} -- PDB: \href{https://www.ebi.ac.uk/pdbe/entry/pdb/7yni}{7YNI} \cite{CUI_2023_SGLT}, chain: A; \href{https://www.uniprot.org/uniprotkb/Q14534/entry}{Q14534} -- PDB: \href{https://www.ebi.ac.uk/pdbe/entry/pdb/6c6p}{6C6P} \cite{PADYANA_2019_EPOXIDASE}, chain: A; \href{https://www.uniprot.org/uniprotkb/P31321/entry}{P31321} -- PDB: \href{https://www.ebi.ac.uk/pdbe/entry/pdb/4din}{4DIN} \cite{ILOUZ_2012_PKA}, chain: B; \href{https://www.uniprot.org/uniprotkb/P78527/entry}{P78527} -- PDB: \href{https://www.ebi.ac.uk/pdbe/entry/pdb/7sud}{7SUD} \cite{LIU_2022_DNAPK}, chain: A; \href{https://www.uniprot.org/uniprotkb/Q14416/entry}{Q14416} -- PDB: \href{https://www.ebi.ac.uk/pdbe/entry/pdb/7epb}{7EPB} \cite{DU_2021_MGLU7}, chain: A.}
    \label{fig:closest_pred_pockets}
\end{figure}

\subsection{Pocket re-scoring strategies}

PUResNet, PocketFinder\textsuperscript{+}, Ligsite\textsuperscript{+} and Surfnet\textsuperscript{+} do not score, nor explicitly rank their pockets, and so pockets were taken in the order given by their ID. This means that when sorting across the dataset, the order of all pockets with the same rank is arbitrary. Multiple strategies were employed to obtain scores for these pockets. Firstly, a score was obtained as the number of pocket amino acids, resulting in variants PUResNet\textsubscript{AA}, PocketFinder\textsuperscript{+}\textsubscript{AA}, Ligsite\textsuperscript{+}\textsubscript{AA} and Surfnet\textsuperscript{+}\textsubscript{AA}. Secondly, PRANK pocket scoring was employed, resulting in variants PUResNet\textsubscript{PRANK}, PocketFinder\textsuperscript{+}\textsubscript{PRANK}, Ligsite\textsuperscript{+}\textsubscript{PRANK} and Surfnet\textsuperscript{+}\textsubscript{PRANK}. IF-SitePred uses a simple pocket scoring scheme, which assigns to each centroid the number of clustered cloud points it results from. In this Chapter, newly defined IF-SitePred pocket scores were calculated as the sum of squares (SS) of the ligandability scores ($LS_{i}$), calculated with \autoref{eq:IFSP_score}, of the \textit{K} residues on a site (\autoref{eq:IFSP_pocket_score}) resulting in IF-SitePred\textsubscript{RESC}. The same was done for PocketFinder\textsuperscript{+}, Ligsite\textsuperscript{+} and Surfnet\textsuperscript{+}, but instead of residue scores, grid point scores ($GS_{i}$) were used (\autoref{eq:leagcy_methos_pocket_score}). This resulted in further variants PocketFinder\textsuperscript{+}\textsubscript{SS}, Ligsite\textsuperscript{+}\textsubscript{SS} and Surfnet\textsuperscript{+}\textsubscript{SS}. This is the same approach introduced by Krivák \textit{et al.} \cite{KRIVAK_2015_P2RANK} and later adopted by Smith \textit{et al.} \cite{SMITH_2024_GrASP}.

\begin{equation}
SS_{\text{IF--SitePred}} = \sum_{i=1}^{K} LS_i^2
\label{eq:IFSP_pocket_score}
\end{equation}

\begin{equation}
SS_{\text{PocketFinder}^+} = SS_{\text{Ligsite}^+} = SS_{\text{Surfnet}^+} = \sum_{i=1}^{K} GS_i^2
\label{eq:leagcy_methos_pocket_score}
\end{equation}

\subsection{Performance evaluation}

This Chapter evaluates the performance of fifteen novel non-redundant and scoring variants of the thirteen canonical ligand binding site prediction methods surveyed in \autoref{chap:LBS_COMP}. These variants do not affect prediction at the residue level, so performance is only assessed at the pocket level.

Mainly three metrics are discussed in this Chapter. Recall (\autoref{eq:recall}) is the percentage of observed binding sites in the reference data that are correctly predicted by a given method for a given DCC, rank or \textit{I\textsubscript{rel}} threshold. Precision (\autoref{eq:precision}) is the percentage of predicted sites that are correct, i.e., match a pocket in the reference. In this case, Precision\textsubscript{1K} is reported, where all predictions by a method across the LIGYSIS reference set are sorted by score and precision reported as predictions are considered up to the 1000\textsuperscript{th} highest scoring prediction. In a similar way, ROC\textsubscript{100} \cite{WEBBER_2003_ROC100} reports cumulative TP \textit{vs} cumulative FP until 100 FP are reached. Finally, relative residue overlap (\autoref{eq:RRO}) and relative volume overlap (\autoref{eq:RVO}) represent how well predicted sites match the observed site in terms of residue overlap and shape (\%). Refer to \autoref{subsub:pocket_level_metrics} for more details.

%\FloatBarrier

\subsection{Statistics and reproducibility}

ChimeraX v1.7.1 \cite{PETTERSEN_2021_CHIMERAX} was used for structural visualisation. Performed statistical tests were two-tailed and α = 0.05. Sample sizes and measures of significance are reported in text, figures and legends. For more details on method selection and execution, refer to \autoref{subsub:stats_repro}.

\subsection{Data and code availability}

The main results tables and files necessary to replicate the analysis described in this Chapter can be found here: \url{https://doi.org/10.5281/zenodo.13121414} \cite{UTGES_2024_LBSCOMP_ZENODO}. Software developed to carry out this analysis is found in this GitHub repository: \url{https://github.com/bartongroup/LBS-comparison} \cite{UTGES_2024_LBSCOMP_REPO}.

\section{Results}

\subsection{Effect of redundancy and pocket score on ranking}

Since VN-EGNN uses \textit{K} = 8 virtual nodes by default, a maximum of eight predicted pockets are possible. However, only seven are observed in our dataset, i.e., in all cases at least one virtual node gets clustered with another, resulting in seven ``unique'' predictions. \autorefpanel{fig:pocket_score_vs_rank1}{ A} illustrates the issue of prediction redundancy and how it affects the scoring and ranking of the pockets. Predictions of the same pocket are reported multiple times as distinct virtual nodes or pocket centroids. These nodes are very close to each other and present similar scores. This is why there is no apparent difference in the distribution of scores across the pocket ranks for VN-EGNN, unlike all other methods. This is no longer the case after removing redundancy and obtaining VN-EGNN\textsubscript{NR} (\autorefpanel{fig:pocket_score_vs_rank1}{ B}). IF-SitePred predictions are also highly redundant. However, these pockets, despite being close to each other, present different scores (number of points). That is why higher ranks (1, 2, 3...) present higher scores (\autorefpanel{fig:pocket_score_vs_rank1}{ C}). Redundancy removal can be observed in \autorefpanel{fig:pocket_score_vs_rank1}{ D} as the scatter plot is less crowded and the maximum rank across the dataset is 60 as opposed to 120. \autorefpanel{fig:pocket_score_vs_rank1}{ E} shows the non-redundant set of re-scored IF-SitePred predictions, IF-SitePred\textsubscript{RESC-NR}. This distribution is wider, i.e., scores take values from a larger value distribution, which might yield a better scoring scheme.

\begin{figure}[htb!]
    \centering
    \includegraphics[width=\textwidth]{figures/ch_LBS_IMPROV/PNG/SUPP_FIG3_SCORE_vs_RANK_SPLIT1.png}
    \caption[Pocket score \textit{vs} pocket ranking (I)]{\textbf{Pocket score \textit{vs} pocket ranking (I).} \textbf{(A)} VN-EGNN reported pocket scores; \textbf{(B)} Non-redundant VN-EGNN predictions (VN-EGNN\textsubscript{NR}); \textbf{(C)} Default IF-SitePred predictions are ranked based on the number of pocket cloud points; \textbf{(D)} Non-redundant variant of IF-SitePred (IF-SitePred\textsubscript{NR}); \textbf{(E)} Re-scored non-redundant IF-SitePred predictions (IF-SitePred\textsubscript{RESC-NR}). Score is calculated as sum of squares of residue ligandability scores (\autoref{eq:IFSP_pocket_score}); \textbf{(F)} GrASP; \textbf{(G)} PUResNet does not score its pockets. PUResNet\textsubscript{AA} uses the number of pocket amino acids as a score; \textbf{(H)} PRANK-scored PUResNet pockets; \textbf{(I)} DeepPocket\textsubscript{SEG}; \textbf{(J)} Non-redundant DeepPocket\textsubscript{SEG} predictions (DeepPocket\textsubscript{SEG-NR}); \textbf{(K)} DeepPocket\textsubscript{RESC}; \textbf{(L)} P2Rank\textsubscript{CONS}. (\textbf{d}) and (\textit{v}) indicate whether methods are default or a variant generated in this work.}
    \label{fig:pocket_score_vs_rank1}
\end{figure}

\begin{figure}[htb!]
    \centering
    \includegraphics[width=\textwidth]{figures/ch_LBS_IMPROV/PNG/SUPP_FIG3_SCORE_vs_RANK_SPLIT2.png}
    \caption[Pocket score \textit{vs} pocket ranking (II)]{\textbf{Pocket score \textit{vs} pocket ranking (II).} \textbf{(A)} P2Rank; \textbf{(B)} fpocket\textsubscript{PRANK}; \textbf{(C)} fpocket. This distribution differs massively from the re-scored fpocket\textsubscript{PRANK} one; \textbf{(D)} PocketFinder\textsuperscript{+} does not report pocket scores, so the number of pocket residues is displayed for the PocketFinder\textsuperscript{+}\textsubscript{AA} variant; \textbf{(E)} PocketFinder\textsuperscript{+}\textsubscript{PRANK}; \textbf{(F)} PocketFinder\textsuperscript{+}\textsubscript{SS}. This variant uses the pocket grid points’ scores to calculate a pocket score by summing the squared scores (\autoref{eq:leagcy_methos_pocket_score}); \textbf{(G)} Just like PocketFinder\textsuperscript{+}, Ligsite\textsuperscript{+} does not score pockets, Y-axis is number of pocket residues (Ligsite\textsuperscript{+}\textsubscript{AA}); \textbf{(H)} Ligsite\textsuperscript{+}\textsubscript{PRANK}; \textbf{(I)} Ligsite\textsuperscript{+}\textsubscript{SS}; \textbf{(J)} Surfnet\textsuperscript{+}\textsubscript{AA}; \textbf{(K)} Surfnet\textsuperscript{+}\textsubscript{PRANK}; \textbf{(L)} Surfnet\textsuperscript{+}\textsubscript{SS}. (\textbf{d}) and (\textit{v}) indicate whether methods are default or a variant generated in this work.}
    \label{fig:pocket_score_vs_rank2}
\end{figure}

There is no clear difference between \autorefpanel{fig:pocket_score_vs_rank1}{ G-H}, meaning that using PRANK to score PUResNet predictions does not alter the overall ranking of the predictions made within a protein. This makes sense, as only 10\% of proteins present $>$1 predicted pocket by this method. Nevertheless, this new score could help in the ranking of pockets across the dataset and not just within a protein. The distribution of scores does not change when removing the redundancy from Deep-Pocket\textsubscript{SEG} predictions (\autorefpanel{fig:pocket_score_vs_rank1}{ I-J}), but the maximum rank goes from 200 to 140, indicating the decrease in total predictions. The score distributions of fpocket\textsubscript{PRANK} (\autorefpanel{fig:pocket_score_vs_rank2}{ B}) and fpocket (\autorefpanel{fig:pocket_score_vs_rank2}{ C}) are completely different. This makes sense, since the ranking of pockets, recall and precision of these two pocket scoring schemes differ considerably as shown in \autoref{chap:LBS_COMP}.

The score distributions of ``\textsubscript{AA}'', ``\textsubscript{SS}'' and ``\textsubscript{PRANK}'' variants of PocketFinder\textsuperscript{+}, Ligsite\textsuperscript{+} and Surfnet\textsuperscript{+} are similar, suggesting that the number of pocket amino acids might dictate the order in which these pockets are reported and that re-scoring predictions by these methods might not have an effect on their performance (\autorefpanel{fig:pocket_score_vs_rank2}{ D-L}).

\subsection{Effect of redundancy and pocket score on recall}

\autoref{fig:pocket_score_vs_rank1} and \autoref{fig:pocket_score_vs_rank2} demonstrate the drastic effect of redundancy removal in pocket ranking, with VN-EGNN as the clearest example. The following analysis explored the effect of redundancy removal and different pocket scoring schemes in recall for PUResNet, PocketFinder\textsuperscript{+}, Ligsite\textsuperscript{+} and Surfnet\textsuperscript{+}, which do not report pocket scores.

\autoref{fig:pocket_score_vs_rank1} and \autoref{fig:pocket_score_vs_rank2} demonstrate how removing redundancy from predictions can have a drastic effect in the ranking of the predictions, with VN-EGNN being the clearest example. The following analysis explored the effect that redundancy removal and different pocket scoring schemes have on recall for PUResNet, PocketFinder\textsuperscript{+}, Ligsite\textsuperscript{+} and Surfnet\textsuperscript{+}, which do not report pocket scores.

\autorefpanel{fig:pocker_recall_variants1}{ A-B} shows a significant +5.2\% increase in recall after removing redundancy for VN-EGNN predictions (Recall = 46.1\%). This increase corresponds to 346 extra predictions that fall within the top-\textit{N}+2 after redundancy removal. An even stronger improvement can be observed for IF-SitePred (\autorefpanel{fig:pocker_recall_variants1}{ C-D}), where a combination of redundancy removal and pocket re-scoring (\autoref{eq:IFSP_pocket_score}) results in a significant increase of +13.4\% (Recall = 39.1\%), corresponding to 901 extra predictions within the top-\textit{N}+2.

\begin{figure}[htbp!]
    \centering
    \includegraphics[width=\textwidth]{figures/ch_LBS_IMPROV/PNG/SUPP_FIG5_RECALL_VARIANTS_1.png}
    \caption[Recall curves for method variants (I)]{\textbf{Recall curves for method variants (I).} Recall curves for different scoring and ranking variants for VN-EGNN \textbf{(A-C)}, IF-SitePred \textbf{(D-F)}, PUResNet \textbf{(G-I)} and DeepPocket\textsubscript{SEG} \textbf{(J-L)}. For each method, panels illustrate how recall changes as DCC, rank and \textit{I\textsubscript{rel}} thresholds vary. In this last one, \textit{I\textsubscript{rel}} is the criterion used to classify predictions. Dashed lines indicate the thresholds used as reference in this work: DCC = 12 \AA{}, rank = top-\textit{N}+2, and \textit{I\textsubscript{rel}} = 0.5. (\textbf{d}) and (\textit{v}) indicate whether methods are default or variants.}
    \label{fig:pocker_recall_variants1}
\end{figure}

\begin{figure}[htb!]
    \centering
    \includegraphics[width=\textwidth]{figures/ch_LBS_IMPROV/PNG/SUPP_FIG5_RECALL_VARIANTS_2.png}
    \caption[Recall curves for method variants (II)]{\textbf{Recall curves for method variants (II).} Recall curves for different scoring and ranking variants for PocketFinder\textsuperscript{+} \textbf{(A-C)}, Ligsite\textsuperscript{+} \textbf{(D-F)} and Surfnet\textsuperscript{+} \textbf{(G-I)}. For each method, panels illustrate how recall changes as DCC, rank and \textit{I\textsubscript{rel}} thresholds vary. In this last one, \textit{I\textsubscript{rel}} is the criterion used to classify predictions. Dashed lines indicate the thresholds used as reference in this work: DCC = 12 \AA{}, rank = top-\textit{N}+2, and \textit{I\textsubscript{rel}} = 0.5. (\textbf{d}) and (\textit{v}) indicate whether methods are default or variants.}
    \label{fig:pocker_recall_variants2}
\end{figure}

\noindent
Most of this change is due to the redundancy removal, as can be seen by the higher recall of IF-SitePred\textsubscript{NR}. Scoring of PUResNet predictions using the number of pocket amino acids (PUResNet\textsubscript{AA}) or PRANK (PUResNet\textsubscript{PRANK}) had no effect on the recall. This was expected as PUResNet predicts a single pocket in 90\% of the cases. Consequently, there is no strong need for a score to sort predictions within a protein (\autorefpanel{fig:pocker_recall_variants1}{ G-H}). Just like VN-EGNN and IF-SitePred, the recall of DeepPocket\textsubscript{SEG} benefits from redundancy removal, increasing by +5.6\% (\autorefpanel{fig:pocker_recall_variants1}{ J-K}). For PocketFinder\textsuperscript{+}, Ligsite\textsuperscript{+} and Surfnet\textsuperscript{+}, none of the variants had a significant improvement in the recall (\autorefpanel{fig:pocker_recall_variants2}{ A-I}). This is expected as these predict only a few non-redundant sites per protein, with medians ranging 1-3 pockets per protein. Additionally, these pockets might already be sorted by number of amino acids as suggested by \autorefpanel{fig:pocket_score_vs_rank2}{ D-L}.

\subsection{Effect of redundancy and pocket score on \# TP\textsubscript{100 FP}}

There are no negative predictions, either true (TN) or false (FN) in the context of ligand binding site prediction at the pocket level. Accordingly, standard ROC/AUC curves cannot be obtained. Only positives are predicted (pockets). FN can be obtained by examining the observed pockets that are not predicted, but there are not scores for them. ROC\textsubscript{100} curves provide an alternative to observe the relationship between true (TP) and false positives (FP). Predictions for each method across the whole reference dataset, LIGYSIS, were sorted based on pocket score, and cumulative TP and FPs were counted until a certain number of FP was reached, in this case, 100. This visualisation provides insight into how well high-scoring predictions match the ground truth. A higher number of TP at FP = 100 indicates that the high-scoring pockets recapitulate well the ground truth, whereas a low number indicates that pockets scoring high do not match the observed data, given the used threshold of DCC $\leq$ 12 \AA{}. It is important to understand that FPs in this context do not always represent wrong predictions, but could be binding sites that are not considered in the ground truth dataset, composed by biologically relevant protein-ligand interactions \cite{YANG_2013_BIOLIP}. They could also be relevant sites that simply have not been experimentally determined yet. It is also important to contextualise this metric with success rate, or recall, i.e., how many of the observed sites are predicted by each method given the above-mentioned DCC threshold, as well as a rank threshold: top-\textit{N}+2. A method might present a high number of TP within the first 100 FP, yet have a low recall overall. \autoref{fig:pocket_ROC100_variants} explores how ROC\textsubscript{100} changes for the non-redundant ``\textsubscript{NR}'' and re-scored ``\textsubscript{AA}'', ``\textsubscript{PRANK}'', ``\textsubscript{SS}'' and ``\textsubscript{RESC}'' sets of VN-EGNN, IF-SitePred, PUResNet, DeepPocket\textsubscript{SEG}, PocketFinder\textsuperscript{+}, Ligsite\textsuperscript{+} and Surfnet\textsuperscript{+}.

\begin{figure}[htbp!]
    \centering
    \includegraphics[width=\textwidth]{figures/ch_LBS_IMPROV/PNG/SUPP_FIG6_TP100FP_VARIANTS.png}
    \caption[ROC\textsubscript{100} curves for non-redundant and re-scored variants]{\textbf{ROC\textsubscript{100} curves for non-redundant and re-scored variants.} For each method, predicted pockets across the whole dataset, i.e., all LIGYSIS protein chains, were ranked by their score. This way, pockets with the highest scores were at the top of the list, whereas pockets with the lowest scores located at the bottom. This ranking does not correspond to ranking pockets across proteins by their rank, as a pocket ranked \#2, \#3 or lower could have a higher score than a pocket \#1 on a different protein. Each method has a colour assigned and each variant resulting of redundancy removal or pocket (re-)scoring a different line style. \textbf{(A)} VN-EGNN and ``\textsubscript{NR}'' variant; \textbf{(B)} IF-SitePred, non-redundant (``\textsubscript{NR}'') and non-redundant re-scored (``\textsubscript{RESC-NR}'') variants; \textbf{(C)} PUResNet, ``\textsubscript{AA}'' and ``\textsubscript{PRANK}'' variants; \textbf{(D)} DeepPocket\textsubscript{SEG} and ``\textsubscript{NR}'' variant; \textbf{(E)} PocketFinder\textsuperscript{+}, ``\textsubscript{AA}'', ``\textsubscript{PRANK}'' and ``\textsubscript{SS}'' variants; \textbf{(F)} Ligsite\textsuperscript{+} and variants; \textbf{(G)} Surfnet\textsuperscript{+} and variants.}
    \label{fig:pocket_ROC100_variants}
\end{figure}

%\FloatBarrier

\autorefpanel{fig:pocket_ROC100_variants}{ A} illustrates how redundancy can be misleading and overestimate the performance of VN-EGNN. Removing redundancy results in Δ\textsubscript{TP} = \textminus273 (TP = 1028). This is because redundant predictions by VN-EGNN are very close in space and present very similar scores (\autorefpanel{fig:pocket_score_vs_rank1}{ A}). Because of this, in the redundant default set of predictions, multiple TP counts were being added for predictions of the same observed pocket. Even with redundancy removed, VN-EGNN reached 1028 TP for the first 100 FP, indicating that the non-redundant higher-scoring pockets recapitulate well the observed data.

There was no difference between IF-SitePred and IF-SitePred\textsubscript{NR} (curves overlap completely), which indicates that despite the redundancy in predictions by this method, its scoring scheme can sort sites in a meaningful manner. Considering multiple proteins with redundant predictions for IF-SitePred: the scoring scheme allows for the top-1 site of each of these proteins to rank above any of the other redundant predictions of the other proteins. The re-scored and non-redundant set of IF-SitePred predictions, IF-SitePred\textsubscript{RESC-NR}, results in a Δ\textsubscript{TP} = +285 (TP = 1246), indicating that IF-SitePred could benefit from a more sophisticated scoring scheme, rather than the number of cloud points per binding site (\autorefpanel{fig:pocket_ROC100_variants}{ B}).

\autorefpanel{fig:pocket_ROC100_variants}{ C} is a perfect example of the importance of scoring pocket predictions. PUResNet does not score its predictions. For this reason, within a protein, pockets were ranked based on the order they are reported, i.e., on their identifier. When sorting across the whole dataset, pockets with the same ID or rank were randomly shuffled. A massive increase in TP was observed when sorting simply by the number of pocket residues. Using PRANK to score this pockets provided an even larger increment in TP of Δ\textsubscript{TP} = +563 (TP = 1097). An application of this could be running PUResNet on a list of potential drug target proteins. Ranking pocket predictions across different proteins could be a useful criterion to prioritise more druggable targets.

The curve did not change much for DeepPocket\textsubscript{SEG} with Δ\textsubscript{TP} = \textminus27 (TP = 643). This indicates that despite the overlap in pockets resulting from DeepPocket's segmentation module, the method's scoring scheme is robust. It is important to consider that this method's pocket scores come from the re-scoring the fpocket candidates, which are not redundant. The redundancy in DeepPocket\textsubscript{SEG} is therefore unrelated to its scoring scheme, but instead a direct consequence of the shape extraction segmentation module. This suggests that there is a big difference between fpocket candidates (which result in DeepPocket's scores) and the extracted shapes by DeepPocket\textsubscript{SEG}. This difference raises the question of whether it is technically accurate to consider the DeepPocket\textsubscript{RESC} score for the newly segmented pockets by DeepPocket\textsubscript{SEG} (\autorefpanel{fig:pocket_ROC100_variants}{ D}).

For the last three methods, earlier and geometry/energy-based PocketFinder\textsuperscript{+}, Ligsite\textsuperscript{+} and Surfnet\textsuperscript{+}, the results agree in that simply using the number of pocket amino acids results in the maximum TP for 100 FP: Δ\textsubscript{TP} = +114 (TP = 178) (\autorefpanel{fig:pocket_ROC100_variants}{ F}), Δ\textsubscript{TP} = +44 (TP = 159) (\autorefpanel{fig:pocket_ROC100_variants}{ G}) and Δ\textsubscript{TP} = +247 (TP = 308) (\autorefpanel{fig:pocket_ROC100_variants}{ H}) for PocketFinder\textsuperscript{+}, Ligsite\textsuperscript{+} and Surfnet\textsuperscript{+}, respectively. This is surprising, as sum of squares ``\textsubscript{SS}'' and ``\textsubscript{PRANK}'' scoring schemes worked better for other methods. This result might be related to the fact that pockets predicted by these three methods tend to be larger than those predicted by other methods.

\begin{figure}[ht!]
    \centering
    \includegraphics[width=\textwidth]{figures/ch_LBS_IMPROV/PNG/SUPP_FIG7_PRECISION_VARIANTS.png}
    \caption[Precision\textsubscript{1K} curves for non-redundant and re-scored variants]{\textbf{Precision\textsubscript{1K} curves for non-redundant and re-scored variants.}  Precision\textsubscript{1K} represents the precision (\%) calculated for the top-scoring 1000 predictions. For each method, predicted pockets across the whole LIGYSIS set were ranked by score. This way, pockets with the highest scores were at the top of the list, whereas pockets with the lowest scores at the bottom. Each method has a colour assigned and each scoring variant its own line style. Δ\textsubscript{Precision} indicates the difference in precision between the selected method variant and the default one (\%).  \textbf{(A)} VN-EGNN and ``\textsubscript{NR}'' variant; \textbf{(B)} IF-SitePred, ``\textsubscript{NR}'' and ``\textsubscript{RESC-NR}'' variants; \textbf{(C)} PUResNet, ``\textsubscript{AA}'' and ``\textsubscript{PRANK}'' variants; \textbf{(D)} DeepPocket\textsubscript{SEG} and ``\textsubscript{NR}'' variant; \textbf{(E)} PocketFinder\textsuperscript{+}, ``\textsubscript{AA}'', ``\textsubscript{PRANK}'' and ``\textsubscript{SS}'' variants; \textbf{(F)} Ligsite\textsuperscript{+} and variants; \textbf{(G)} Surfnet\textsuperscript{+} and variants. Error bars indicate 95\% CI of the precision (100 $\times$ proportion) and are displayed every 100 predictions.}
    \label{fig:pocket_precision_variants}
\end{figure}

\subsection{Effect of redundancy and pocket score on precision}

Precision-recall curves cannot be calculated for ligand site prediction at the pocket level for the same reason as ROC/AUC: false negatives (FN) are not predicted, and therefore not scored. Nevertheless, precision can be measured. For this, as it was done for ROC\textsubscript{100}, all predictions for a method were sorted by pocket score and precision calculated as more predictions with lower scores were considered.

\autoref{fig:pocket_precision_variants} portrays the precision curve for the top-1000 predictions for the non-redund-ant and re-scored variants for VN-EGNN, IF-SitePred, PUResNet, DeepPocket\textsubscript{SEG}, Pocket-Finder\textsuperscript{+}, Ligsite\textsuperscript{+} and Surfnet\textsuperscript{+}. There was no significant (\textit{p} $>$ 0.05) change in precision between VN-EGNN and VN-EGNN\textsubscript{NR} within the first 1000 predictions, Precision\textsubscript{1K} = 91.5\% (\autorefpanel{fig:pocket_precision_variants}{ A}). The same can be said for IF-SitePred with a Precision\textsubscript{1K} = 94.3\% (\autorefpanel{fig:pocket_precision_variants}{ B}). Using PRANK to score PUResNet pockets resulted in a significant +11.7\% increase with precision = 93.3\% (\autorefpanel{fig:pocket_precision_variants}{ C}). DeepPocket\textsubscript{SEG-NR}, as the other redundant methods, did not experience a significant change in precision as redundancy is removed: precision = 81.6\% (\autorefpanel{fig:pocket_precision_variants}{ D}). Using the number of pocket amino acids as a score (``\textsubscript{AA}'') resulted in a precision increase of +23.3\% (Precision\textsubscript{1K} = 65.3\%), +16.5\% (Precision\textsubscript{1K} = 68.8\%) and 29.1\% (Precision\textsubscript{1K} = 68.8\%) for PocketFinder\textsuperscript{+} (\autorefpanel{fig:pocket_precision_variants}{ E}), Ligsite\textsuperscript{+} (\autorefpanel{fig:pocket_precision_variants}{ F}) and Surfnet\textsuperscript{+} (\autorefpanel{fig:pocket_precision_variants}{ G}), respectively.

\subsection{Evaluation of predictive performance}

\autoref{fig:pocket_level_benchmark_variants} and \autoref{tab:pocket_level_benchmark_variants} compare the performance of the thirteen methods evaluated in this Chapter, which now include six canonical methods (\textbf{d}) and seven novel variants (\textit{v}) first introduced in this Chapter. In terms of recall, the ranking of the methods does not change much. The increase obtained by the re-scoring and non-redundant variants, though considerable (+13.4\% for IF-SitePred) is not enough to reach the recall achieved by the top-performing original methods (\autorefpanel{fig:pocket_level_benchmark_variants}{ A-C}). However, a shift in the ranking can be observed in Precision\textsubscript{1K} and \# TP\textsubscript{100 FP}. PUResNet, PocketFinder\textsuperscript{+}, Ligsite\textsuperscript{+} and Surfnet\textsuperscript{+}, which originally did not score their predictions, benefit greatly of scoring their predictions by simply using the number of amino acids of predicted pockets as a score. This results in increases of up to $\approx$560 TP at 100 FP for PUResNet\textsubscript{PRANK} and 29.1\% in precision for Surfnet\textsuperscript{+}\textsubscript{AA}. The improvement of the best seven novel variants relative to their default counterparts is summarised in \autoref{tab:methods_improvement_summary}.

\begin{figure}[htb!]
    \centering
    \includegraphics[width=\textwidth]{figures/ch_LBS_IMPROV/PNG/SUPP_FIG8_POCKET_LEVEL_BENCHMARK_IMPROVED.png}
    \caption[Ligand binding site prediction at the pocket level (\textit{best} variants)]{\textbf{Ligand binding site prediction at the pocket level (\textit{best} variants).} Only the top-performing, i.e., highest top-\textit{N}+2 recall, variant of each method is drawn on this figure, e.g., IF-SitePred\textsubscript{RESC-NR} or VN-EGNN\textsubscript{NR} instead of their default modes. \textbf{(A)} Top-\textit{N}+2 recall according to a DCC threshold. Reported recall on \autoref{tab:pocket_level_benchmark_variants} corresponds to DCC = 12 \AA{}; \textbf{(B)} Recall using DCC = 12 \AA{} but considering increasing rank thresholds. \textit{all} represents the maximum recall of a method, obtained by considering all predictions, regardless of their rank or score; \textbf{(C)} Recall curve using \textit{I\textsubscript{rel}} as a criterion; \textbf{(D)} ROC\textsubscript{100} curve (cumulative TP against cumulative FP until 100 FP are reached); \textbf{(E)} Precision curve for the top-1000 predictions of each method across the LIGYSIS dataset. Error bars represent 95\% CI of the recall \textbf{(A-C)} and precision \textbf{(E)}, which are 100 $\times$ proportion. Numbers at the right of the panels indicate groups or blocks of methods that perform similarly for each metric. Stars ``*'' indicate outlier methods, or methods that perform very differently than the rest. (\textbf{d}) and (\textit{v}) indicate whether methods are default or variant, respectively.}
    \label{fig:pocket_level_benchmark_variants}
\end{figure}

\begingroup
\captionsetup{belowskip=0pt,aboveskip=9pt} % or some smaller space ,aboveskip=4pt
\begin{landscape}
\begin{longtable}[c]{|M{33mm}|M{28mm}|M{31mm}|M{26mm}|M{29mm}|M{19mm}|M{18mm}|M{18mm}|}
\hline
\textbf{Method}         & \textbf{Recall\textsubscript{top-\textit{N}}} (\%) & \textbf{Recall\textsubscript{top-\textit{N}+2}} (\%) & \textbf{Recall\textsubscript{max}} (\%) & \textbf{Precision\textsubscript{1K}} (\%) & \textbf{\# TP\textsubscript{100 FP}} & \textbf{RRO} (\%) & \textbf{RVO} (\%) \\ \hline
\endfirsthead
%
\footnotesize{(\textit{v})} VN-EGNN\textsubscript{NR}          & 44.5 (\#7)           & 46.1 (\#11)             & 46.3 (\#11)         & 91.5 (\#4)           & 1028 (\#3)       & \textbf{\textcolor{CBBlue}{31.6 (\#11)}}     & \textbf{\textcolor{CBBlue}{26.7 (\#11)}}     \\ \hline
\footnotesize{(\textit{v})} IF-SitePred\textsubscript{RESC-NR} & \textbf{\textcolor{CBOrange}{29.7 (\#12)}}           & \textbf{\textcolor{CBOrange}{39.1 (\#13)}}             & 51.6 (\#12)           & \textbf{\textcolor{CBBlue}{94.3 (\#1)}}           & \textbf{\textcolor{CBBlue}{1246 (\#1)}}       & 49.3 (\#10)     & 43.7 (\#9)     \\ \hline
\footnotesize{(\textbf{d})} GrASP              & 48.0 (\#2)             & 49.9 (\#5)             & 50.0 (\#8)           & 92.5 (\#3)           & 1017 (\#4)       & 54.5 (\#7)     & 59.8 (\#6)     \\ \hline
\footnotesize{(\textit{v})} PUResNet\textsubscript{PRANK}      & 40.8 (\#10)           & 41.1 (\#12)             & \textbf{\textcolor{CBOrange}{41.1 (\#12)}}         & 93.3 (\#2)           & 1092 (\#2)       & 61.0 (\#4)     & 63.9 (\#4)     \\ \hline
\footnotesize{(\textit{v})} DeepPocket\textsubscript{SEG-NR}   & 43.4 (\#8)          & 49.4 (\#6)            & 55.4 (\#5)        & 81.6 (\#7)          & 643 (\#6)        & 58.4 (\#5)     & 61.3 (\#5)     \\ \hline
\footnotesize{(\textbf{d})} DeepPocket\textsubscript{RESC}     & 46.6 (\#4)           & 58.1 (\#2)            & 89.3 (\#2)        & 81.7 (\#6)          & 637 (\#7)        & 52.6 (\#9)     & 38.2 (\#10)     \\ \hline
\footnotesize{(\textbf{d})} P2Rank\textsubscript{CONS}         & \textbf{\textcolor{CBBlue}{48.8 (\#1)}}           & 53.9 (\#3)             & 57.0 (\#3)           & 90.7 (\#5)           & 932 (\#5)         & 56.4 (\#6)     & 43.8 (\#8)     \\ \hline
\footnotesize{(\textbf{d})} P2Rank             & 46.7 (\#3)           & 51.9 (\#4)             & 57.0 (\#4)           & 79.2 (\#8)           & 586 (\#8)         & 54.4 (\#8)     & 58.2 (\#7)   \\ \hline
\footnotesize{(\textbf{d})} fpocket\textsubscript{PRANK}       & \textbf{\textcolor{CBBlue}{48.8 (\#1)}}           & \textbf{\textcolor{CBBlue}{60.4 (\#1)}} & \textbf{\textcolor{CBBlue}{91.3 (\#1)}}         & 81.7 (\#6)           & 526 (\#9)        & 52.6 (\#9)     & 38.2 (\#10)     \\ \hline
\footnotesize{(\textbf{d})} fpocket        & 38.8 (\#11)           & 46.5 (\#10)             & \textbf{\textcolor{CBBlue}{91.3 (\#1)}}         & \textbf{\textcolor{CBOrange}{47.3 (\#12)}}          & \textbf{\textcolor{CBOrange}{94 (\#13)}}          & 52.6 (\#9)     & 38.2 (\#10)     \\ \hline
\footnotesize{(\textit{v})} PocketFinder\textsuperscript{+}\textsubscript{AA}    & 44.5 (\#6)          & 48.9 (\#8)            & 50.5 (\#7)        & 65.3 (\#11)           & 178 (\#11)         & 72.3 (\#2)     & 75.9 (\#2)     \\ \hline
\footnotesize{(\textit{v})} Ligsite\textsuperscript{+}\textsubscript{AA}         & 44.9 (\#5)          & 49.0 (\#7)              & 49.7 (\#9)        & 68.8 (\#9)          & 159 (\#12)         & \textbf{\textcolor{CBBlue}{77.6 (\#1)}}     & \textbf{\textcolor{CBBlue}{77.0 (\#1)}}     \\ \hline
\footnotesize{(\textit{v})} Surfnet\textsuperscript{+}\textsubscript{AA}         & 43.3 (\#9)          & 47.4 (\#9)            & 48.9 (\#10)        & 68.6 (\#10)          & 308 (\#10)         & 71.7 (\#3)     & 72.0 (\#3)     \\ \hline
\caption[Pocket level evaluation (\textit{best} variants)]{\textbf{Pocket level evaluation (\textit{best} variants).} Only the top-performing, i.e., highest top-\textit{N}+2 recall, variant of each method is present on this table. Recall (\%) for each method considering top-\textit{N}, \textit{N}+2 and \textit{all} predictions (max), i.e., maximum recall. Precision (\%) of the method for the top-1000 scored predictions. Number of TP reached for the first 100 FP (\# TP\textsubscript{100 FP}). Mean relative residue overlap (RRO) for correctly predicted sites and relative volume overlap (RVO) only for sites that have a volume, i.e., are pockets or cavities, and not fully exposed sites, which do not have a volume. RRO and RVO represent the overlap in residues and volume relative to the observed site. Bold font indicates the best (blue) and worst (orange) performing methods for each metric. (\textbf{d}) and (\textit{v}) indicate whether methods are default or variant, respectively.}
\label{tab:pocket_level_benchmark_variants}\\
\end{longtable}
\end{landscape}
\endgroup

\begin{landscape}
\begin{longtable}[c]{|c|c|c|c|c|c|c|}
\hline
\textbf{Default method \footnotesize{(d)}} & \textbf{Best variant \footnotesize{(\textit{v})}} & \textbf{\textDelta\textsubscript{Recall\textsubscript{top-\textit{N}}}} (\%) & \textbf{\textDelta\textsubscript{Recall\textsubscript{top-\textit{N}+2}}} (\%) & \textbf{\textDelta\textsubscript{Recall\textsubscript{max}}} (\%) & \textbf{\textDelta\textsubscript{Precision\textsubscript{1K}}} (\%) & \textbf{\textDelta\textsubscript{\# TP\textsubscript{100 FP}}} \\ \hline
\endfirsthead
%
\endhead
%
VN-EGNN                     & VN-EGNN\textsubscript{NR}              & +17.0                          & +5.2                           & \textminus3.0                        & \textminus1.0                          & \textminus273                     \\ \hline
IF-SitePred                 & IF-SitePred\textsubscript{RESC-NR}      & +9.9                         & +13.4                          & \textminus0.5                      & +3.3                         & +285                      \\ \hline
PUResNet                    & PUResNet\textsubscript{PRANK}          & +0.2                         & 0.0                             & 0.0                         & +11.7                        & +558                      \\ \hline
DeepPocket\textsubscript{SEG}              & DeepPocket\textsubscript{SEG-NR}       & +8.0                           & +5.6                           & \textminus1.1                      & \textminus1.0                          & \textminus27                      \\ \hline
PocketFinder\textsuperscript{+}                & PocketFinder\textsuperscript{+}\textsubscript{AA}         & +5.3                         & +1.1                           & 0.0                         & +23.3                        & +114                      \\ \hline
Ligsite\textsuperscript{+}                     & Ligsite\textsuperscript{+}\textsubscript{AA}              & +3.6                         & +0.6                           & 0.0                         & +16.5                        & +44                       \\ \hline
Surfnet\textsuperscript{+}                     & Surfnet\textsuperscript{+}\textsubscript{AA}              & +6.0                           & +1.6                           & 0.0                         & +29.1                        & +247                      \\ \hline
\caption[Methods improvement summary]{\textbf{Methods improvement summary.} Summary of the performance improvement for the seven methods for which non-redundant or re-scoring variants were generated in this Chapter: VN-EGNN, IF-SitePred, PUResNet, DeepPocket\textsubscript{SEG}, PocketFinder\textsuperscript{+}, Ligsite\textsuperscript{+} and Surfnet\textsuperscript{+}. Performance increase is calculated for each metric from the \textit{best} variant (\textit{v}), i.e., highest top-\textit{N}+2 recall, relative to the default original method (\textbf{d}). Maximum recall is reduced for VN-EGNN, IF-SitePred and DeepPocket\textsubscript{SEG} as their non-redundant variants present fewer pockets. The same happens for Precision\textsubscript{1K} and \# TP\textsubscript{100 FP} for VN-EGNN. Overall, the method variants introduced in this work have a significant positive effect on performance.}
\label{tab:methods_improvement_summary}\\
\end{longtable}
\end{landscape}

\section{Discussion}

This Chapter shows that prediction redundancy underestimates recall and overestimates precision, therefore providing a misleading performance assessment. Redundancy removal and subsequent pocket re-ranking can yield a significant increase in recall. A robust pocket scoring scheme can also have a major impact in performance, both in recall and precision and emphasis should be put into this area. Even if a single site is predicted per protein -- as it is the case for PUResNet -- a pocket score can be highly useful when ranking predicted pockets across different proteins or conformations, e.g., when having a list of potential drug targets and deciding which protein might be best to target therapeutically. \autoref{tab:methods_improvement_summary} summarises the performance improvements accomplished in this Chapter by removing redundant predictions (VN-EGNN, IF-SitePred and DeepPocket\textsubscript{SEG}) as well as using more sophisticated pocket scoring schemes (IF-SitePred, PUResNet, PocketFinder\textsuperscript{+}, Ligsite\textsuperscript{+} and Surfnet\textsuperscript{+}). The magnitude of these improvements is notable: increase in top-\textit{N}+2 Recall by $>$15\%, Precision\textsubscript{1K} by $\approx$30\% and \# TP\textsubscript{100 FP} by $>$500. Nevertheless, the overall ranking of the methods after including novel variants (\autoref{tab:pocket_level_benchmark_variants}) does not change much relative to the benchmark of the default methods (\autoref{tab:pocket_level_benchmark}). This is due to the fact that the difference in Recall\textsubscript{top-\textit{N}+2} between the top-performing methods (fpocket and re-scored predictions) and the ones for which variants were generated (VN-EGNN, IF-SitePred and DeepPocket\textsubscript{SEG}) was larger than the increase in recall resulting from the variants. While removing redundancy post-prediction has a significant improvement in performance (VN-EGNN\textsubscript{NR} and IF-SitePred\textsubscript{NR}), approaching this issue before prediction would be better. For VN-EGNN, which predicts a maximum of eight sites, ensuring all of these predictions are non-redundant is more convenient than removing redundant ones ending up with 1/8 predictions. The same applies to IF-SitePred. This method would also benefit from more sophisticated residue and pocket scoring schemes, as well as a different cloud point clustering and site definition algorithm. It is likely that after tweaking their algorithm and approaching the issues highlighted in this and \autoref{chap:LBS_COMP}, these methods could perform better and perhaps even out-perform fpocket and P2Rank.

\section{Conclusions}

The conclusions resulting from the work presented in this Chapter are as follows:

\begin{itemize}

\item Redundancy in ligand binding site prediction leads to the underestimate of recall and the overestimate of precision. The removal of such redundancy and subsequent re-ranking of the remaining pockets results in a drastic increase in recall.

\item A robust pocket scoring scheme is crucial for the correct ranking and prioritisation of predicted sites in downstream analysis, e.g., docking, simulation. Additionally, it has a significant positive effect on both precision and recall.

\item IF-SitePred benefits significantly from pocket re-scoring, which suggests that protein inverse folding embeddings, which are not directly dependent on the input structure, represent great promise in the field of ligand site prediction.

\end{itemize}



\chapter{Conclusions}
These will be the main conclusions of the thesis.

\cleardoublepage
\addcontentsline{toc}{chapter}{Bibliography}
\printbibliography

\end{document}